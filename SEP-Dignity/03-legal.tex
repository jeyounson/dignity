\section{법에서의 인간존엄}

법적 담론에서의 인간존엄의 사용을 본고는 크게 인권 담론에서의 사용과, 헌법 담론에서의 사용, 그리고 생명윤리 담론에서의 사용으로 나누어 각각의 특징들을 살펴보고자 한다. 먼저 인권 담론에서의 인간존엄의 사용을 살펴보자.

\subsection{인권 담론에서 인간존엄}

현대의 법적 논의에 활용되고 있는 가장 영향력있는 `인간존엄'의 사용의 맥락은 홀로코스트라는 참혹한 사건으로 대표되는 제2차 세계대전시기의 나치 이데올로기에 대한 반성으로 형성된 국제사회의 이해방식에 있다고 할 수 있다. 인간존엄 등의 개념을 자신의 철학의 중심으로 삼았던 카톨릭 철학자 자크 마리탱은,\footnote{마리탱이 이러한 철학을 가질 수 있었던 배경에는 당시 카톨릭교회의 분위기가 결정적이었다. 맥크루든에 의하면, ``마르크스에 의한 공산주의의 발달과 급진적 재분배, 계급투쟁, 전체주의의 공포 등을 가지고 사회주의가 제기한 것으로 여겨졌던 위협은 모든 전반적인 카톨릭적 사회적 교리의 중심으로 존엄을 채택하는 데 공헌했다. \ldots{} 이 맥락에서 발전된 {[}카톨릭의{]} 존엄에의 접근방식은, 완전한 범위에서 인간의 복지에 필요한 것을 포착할 수 있는 가능성에 있어서의 권리들의 한계들, 갈등하는 정치학의 위험성들, 그리고 사회에서 서로 다른 이익에 있어서의 연대의 필요성을 강조하고, 인간존엄의 보다 공동체주의적 이해방식으로 귀결되었다. 하지만 존엄은 단순히 정치적 사회적 동물로서의 인간의 이해방식은 아니었고, 신의 형상에서의 인간의 창조는 그 정식화와 이해에 있어 핵심 요소로 남아있었다.''고 말한다. Christopher McCrudden, ``Human Dignity and Judicial Interpretation of Human Rights'', 662면} 존엄을 인간의 본성이라는 단순한 철학적 탐구의 맥락이 아니라 인간 관계와 정치적 삶을 바라보는 관점의 중심에 두기 시작했다.\footnote{예를 들어 Jacques Maritain, \emph{Man and the State}, (CUA Press, 1998).} 당시 그는 단지 학자가 아니라 국제정세가로서도 위상이 높았다.\footnote{Christopher McCrudden, ``Human Dignity and Judicial Interpretation of Human Rights'', 662면.} 마리탱 등의 정치적 영향으로 인해 전후 국제정치, 특히 UN에 있어 인간존엄의 개념은 중요한 역할을 수행하기 시작했다. UN은 그 헌장의 서문에 존엄 개념을 편입시켰고, 이러한 UN헌장에의 편입은 나아가 세계인권선언(The Universal Declaration of Human Rights, 이하 UDHR)에 인간존엄이 중심적 개념이 되도록 하는 초석이 된다.\footnote{이러한 편입 과정에 대해서는 Christopher McCrudden, ``Human Dignity and Judicial Interpretation of Human Rights'', 675-677면 참조.}

맥크루든은 국제사회의 인간존엄을 바라보는 다양한 방식 사이에 세 가지 정도의 ``가족유사성''을 발견할 수 있다고 한다. 하나는 단지 인간이기 때문에 보유하는 내적가치라는 점이고, 둘째는, 이것이 타인에 의해 존중되어야 하는 것이라는 점, 셋째, 개인의 내적가치의 인정이 의미하는 것은 국가는 개인을 위해 존재하는 것이고 개인은 국가를 위해 존재하는 것은 아니라는 것이다.\footnote{Ibid., 679-680면.} 이와 같은 인식은 인권담론에서 인간존엄이 어떤 권리들, 특히 인권들을 도출하는 기초 혹은 정초가 된다는 생각으로 발전한다. 다음에서는 국제문서의 이러한 인식들을 살펴보고자 한다.

\subsubsection{국제연합헌장, 세계인권선언, ICCPR/ICESCR}

국제연합헌장과 세계인권선언과 같은 대표적인 UN문서들은 인간의 존엄에 대해 다음과 같이 서술하고 있다.

\begin{displayquote}
국제연합헌장(1945)

우리 연합국 국민들은 {[}\ldots{]} 기본적 인권, 인간의 존엄 및 가치, 남녀 및 대소 각국의 평등권에 대한 신념을 재확인하며, {[}\ldots{]} 그리고 이러한 목적을 위하여 {[}\ldots{]} 이러한 목적을 달성하기 위하여 우리의 노력을 결집할 것을 결정하였다.\footnote{Charter of the United Nations (1946). New York: United Nations Publications, p. 2., 번역: 법제처 국가법령정보센터(http://www.law.go.kr/조약/국제연합헌장\%20및\%20국제사법재판소규정)}

세계인권선언(1949)

인류 가족 모든 구성원의 본래적 존엄과 동등하고 양도할 수 없는 권리를 인정하는 것이 세계의 자유, 정의 및 평화의 기초가 됨을 인정하며,{[}\ldots{]} 따라서 이제 국제연합 총회는 {[}\ldots{]} 모든 인민들과 국가가 성취해야 할 공통의 기준으로서 이 세계인권선언을 선포한다.\footnote{Universal Declaration of Human Rights (1949). New York: United Nations Publications, `Preamble'. 제임스 니켈(조국 역), 인권의 좌표 (명인문화사, 2010)에 수록된 부록1의 번역에 따름.}

시민적·정치적 권리에 관한 국제규약(ICCPR, 1966), 경제적·사회적 및 문화적 권리에 관한 국제규약(ICESCR, 1966)

이 규약의 당사국은, 국제연합 헌장에 천명된 원칙에 따라 인류 가족 모든 구성원의 본래적 존엄과 동등하고 양도할 수 없는 권리를 인정하는 것이 세계의 자유, 정의 및 평화의 기초가 됨을 고려하고,

이러한 권리는 인간의 본래적 존엄으로부터 유래함을 인정하며,{[}\ldots{]}

이하의 조문에 합의한다. {[}\ldots{]}\footnote{International Covenants on Human Rights (1967). New York: United Nations Office of Public Information, `Preamble'. 제임스 니켈(조국 역), 인권의 좌표 (명인문화사, 2010)에 수록된 부록 3, 4의 번역에 따름. 두 규약의 전문은 동일하다.}
\end{displayquote}

국제연합헌장에 `존엄'의 용어가 구체적으로 편입되게 된 배경은 그 사료의 부족으로 인해 정확히 알 수 없으나,\footnote{국제연합헌장에서 `존엄'의 용어는 잔 크리스찬 스머츠(Jan Christian Smuts)가 대다수 작성한 서문 초안에는 등장하지 않았고, 몇 차례의 위원회 토론 이후에 나타나기 시작했다고 한다. Christopher McCrudden, `Human Dignity and Judicial Interpretation of Human Rights', the European Journal of International Law, 2008, 676면.} 다만 당시 국제노동기구(ILO)가 1944년 필라델피아 선언에서 이미 존엄의 언어를 편입시켰던 사정,\footnote{Declaration concerning the aims and purposes of the International Labour Organisation (Declaration of Philadelphia), May 1944.} 이어서 몇 달 뒤 유네스코가 그 설립조약의 서문에 편입시켰던 사정\footnote{Constitution of the UN Educational, Scientific and Cultural Organization (UNESCO), adopted in London on 16 Nov. 1945, 3 Bevans 1311.} 등으로 볼 때, 당시 국제기구를 고안하는 사람들 사이에서 일반적으로 사용되고 있었다고 미루어 짐작해 볼 수 있을 것이다.\footnote{Christopher McCrudden, ``Human Dignity and Judicial Interpretation of Human Rights'', 676면.}

이에 비해 세계인권선언에의 편입배경은 상대적으로 명확하다. 이 선언의 초안 작업에서 르네 카슨과 존 험프리의 논쟁이 있었는데,\footnote{Johannes Morsink, \emph{The Universal Declaration of Human Rights: Origins, Drafting, and Intent} (University of Pennsylvania Press, 2000).} 존 험프리는 존엄의 언급이 그의 초안에 더하는 바가 전혀 없으며 선언의 1조에 편입하는 것은 단순한 수사에 불과하다고 주장했고,\footnote{John Peters Humphrey, \emph{Human Rights and the United Nations: a Great Adventure} (Transnational Publishers, 1984), 44면.} 이에 반하여 존엄 개념의 편입을 주장한 것은 르네 카슨이었다. 다른 이들도 존엄의 편입을 옹호했는데, 예를 들어, 이 용어에 대해 의문이 제기되었을 때 엘리너 루즈벨트는 ``존엄은 `모든 인간이 존중받을 가치가 있다는 것을 강조하기 위해서' 포함되었으며 `\ldots{} 그것은 왜 인간이 우선적으로 권리들을 가지는지 이유를 설명하기 위해 의도되었다'\,''고 주장했다.\footnote{Mary Ann Glendon, \emph{A World Made New: Eleanor Roosevelt and the Universal Declaration of Human Rights} (Random House, 2001), 146면.}

하지만, 이러한 문헌들에서 `존엄'의 용어가 지칭하는 바는 명확하지 않다. 이러한 불명확성은 때로 의도적인 것이라고 주장하는 경우도 있다. 올리버 센슨은 UN문서들에 대하여 ``애매성의 대가로 매우 많은 서로다른 당파들 사이에서 어떤 협약을 보호할 수 있기 때문''에 ``이와 같은 문서들에는 핵심 용어들이 고의적으로 모호함을 유지한다''고 생각한다.\footnote{Mary Ann Glendon, ``Foundations of Human Rights: The Unfinished Business'', \emph{The American Journal of Jurisprudence} 44, (Oxford, 1999), 10면; Knut Ipsen, \emph{Völkerrecht. Ein Studienbuch}. (München: C. H. Beck'sche Verlagsbuchhandlung, 1990), 642면; Franz Josef Wetz, \emph{Die Würde der Menschen ist antastbar} (Stuttgart: Klett-Cotta, 1998), 75 -76면 참조.} ``만약 인간존엄의 의미와 기초하는 힘을 구체화하려 한다면, 이는 일부 당파들의 깊게 뿌리박힌 의견과 믿음들과 불화하게될 지도 모르고'', ``이러한 경우 전체 프로젝트가 실패할 수도 있''기에 ``이 문서들에 인간존엄을 분명히하거나 정당화하는 명백한 시도는 없''다는 것이다.\footnote{Oliver Sensen, \emph{Kant on Human Dignity}, 149면.} 유사하게, 크리스토퍼 맥크루든은 세계인권선언이 그 성공적 합의를 위해 임시대리기호(placeholder)로서 존엄을 도입했다고 주장한다.

초안의 역사를 넘어서서 좀 더 폭넓게 생각해보면, 우리는 인간존엄의 중요성이 UN헌장과 UDHR의 시점에 (그리고 그 때 이래로 다른 인권 협정들의 초안에서) 다른 공감대의 기초가 없는 인권운동을 위한 이론적 기초를 공급하는 것이었다는 것을 알 수 있다. 우리는 UDHR이 협상되고 있었던 국제적 맥락을 기억할 필요가 있다. 성공적 결과를 내기 위해, 매우 다른 이데올로기적 색깔의 국가들을 설득할 필요가 있었다.\footnote{Christopher McCrudden, ``Human Dignity and Judicial Interpretation of Human Rights'', 677면.}

예를 들어 고문이 금지되어야 한다는데 많은 국가들이 동의했지만, 왜 고문이 잘못되었는지에 대해서는 국가들 간의 입장이 일치하지 않을 수 있었다. 그 근거에 대해서까지 공감을 형성하려고 시도하는 것은 의견의 일치를 구하지 못하게 하는 것이었다. 그런데, 국제연합헌장과 세계인권선언을 제정할 당시에 수많은 다른 이론들에 비하여 존엄의 개념을 사용하는 이해방식이야말로 충분한 공감대를 가질 수 있는 방식이었다는 것이다. 이미 유네스코가 이러한 이데올로기적 공감대를 성취하기 위해 노력할 때 자크 마리탱의 역할이 컸으며 그는 그의 카톨릭 철학의 배경에도 불구하고 누구나 근거할 수 있는 개념을 도입하고자 노력했었는데, 이 때 인간존엄은 인권의 이론들이 충돌하고 논쟁하는 지점에서 그들 자신의 이론을 삽입하는 것을 가능하게 해 주는 유용한 개념이었다.\footnote{Ibid., 678면.}

그럼에도 불구하고 UN문서들에서 사용된 인간존엄의 언어는 인권의 정초라는 중대한 특징을 공유한다. 올리버 센슨은 UN문서의 변화를 다음과 같이 분석하고 있다. 첫째로, 각국에 대한 구속력이 약한 UN헌장(1945)과 세계인권선언(1948)에는, 존엄과 인권이 나란히 병렬되어 있기에 인권의 기초로서의 존엄의 사용이 다소 불분명하게 되어있긴 하지만, UN헌장에서도 일단 존엄과 권리의 모종의 관계가 암시되어 있고, UN헌장에서는 `존엄'이 `권리'보다 뒤에 나타나던 것이 세계인권선언에서는 `권리'보다 앞에 등장하고 있으며, 헌장에서는 단순히 ``확신을 표현''하던 것이, 인권선언에서는 ``인식가능한'', ``고유한'' 속성으로 드러나기 시작한다. 둘째, 비록 처벌규정은 없을지라도 각국에 요청들을 부과하는 정도의 구속력이 있는 시민적·정치적 권리에 관한 국제규약(the International Covenant on Civil and Political Rights(ICCPR, 1966))과 경제적·사회적 및 문화적 권리에 관한 국제규약(the International Covenant on Economic, Social and Cultural Rights(ICESCR, 1966))의 경우에는 존엄이 인권의 ``기초''라는 점이 명백해졌다는 것이다.\footnote{Oliver Sensen, \emph{Kant on Human Dignity}, 149-152면.}

\subsubsection{인권의 정초로서의 인간존엄}

인권담론에서 인간존엄은 구체적으로 명시하지 않은 권리들을 정초한다고 주장한다. 그리고 이러한 권리는 인간존엄의 개념으로부터 연역이나 구체화의 방법에 의해 도출될 수 있다는 신념을 불러일으킨다(강건한 정초주의). 이러한 태도는 앞에서 살펴본 ICCPR 과 같은 국제인권문서들이 인권들이 인간존엄에서 ``유래한다''거나 많은 문헌에서 인간존엄이 인권들의 ``기초''라고 확언하는 것으로부터 더 강한 확신을 얻는다.

그러나 한편에서는 인간존엄의 인권에 대한 이러한 `도출', `유래', `기초', `정초'의 의미를 1) 축소하려고 하거나, 2) 보다 동적이고 현실적인 대응으로 전환하려고 노력한다. 이러한 개념들은 ``연역''이나 ``구체화''와 같은 역할을 불러일으키는 것처럼 보일 수 있고, 이에 따라 인권을 구체화하는 것은 인간존엄이라는 개념에서 매우 포괄적이고 다양한 인권을 도출할 만큼 이론적으로 충분히 열심히 생각하는 이론철학자에 의해 가장 잘 결정될 수 있는 어떤 정적인 생산인 것처럼 생각할 수 있는데, ``이는 잘못''이라는 것이다.\footnote{Marcus Düwell, ``Human dignity: concepts, discussions, philosophical perspectives'', \emph{The Cambridge Handbook of Human Dignity} (Cambridge, 2014), 39면.} 1)의 입장에서 인간존엄의 인권에 대한 정초 기능은, 우리가 이미 대상으로 삼고 있는 권리들을 넘어서는 권리 목록의 확장을 허용하는 방식으로 권리들의 핵심을 이해할 필요는 없으며, 새로운 권리들을 도출하는 기초가 아니라 이미 존재하는 권리들을 더 자세히 이해하기 위한 해석적 근거로서의 기능으로 충분히 해명된다고 한다.\footnote{Jeremy Waldron, ``Is Dignity the Foundation of Human Rights?'', \emph{Philosophical Foundations of Human Rights} (Oxford, 2015), 131면.} 2)의 입장에서는 인권에 대한 이해와 구체화는 가능한 또는 있음직한 맥락-관련적 위협에 대한 대응으로 이루어져야 하므로, 인권은 시간이 지남에 따라 변할 수 있고, 변할 것이고, 변해야만 하는 개념으로 이해되어야 한다는 것이다.\footnote{Marcus Düwell, ``Human dignity: concepts, discussions, philosophical perspectives'', 39면.} 그렇다면 인간존엄이 인권을 정초한다는 말은 어떻게 이해되어야 하는가? 다음에서 그 가능성들을 살펴보려고 한다.

\paragraph{`인권의 정초'의 개념의 다양한 이해}

우리는 ``개념 α로부터 개념 β가 도출된다''거나 ``개념 α는 개념 β의 정초이다''라는 말을 사용한다. 우리는 이런 말들을 어떻게 이해할 수 있을까? 월드론은 가능한 설명을 네 가지로 제시한다.

\begin{displayquote}
(i) 역사나 계통의 문제로서, β는 α로부터 일어난 것이다

(ii) ``하나의 법명제의 적용은 다른 법명제의 효력(validity)의 연원일 수 있다''는 식으로, α는 β의 연원이다

(iii) 연역적으로 혹은 경험적 전제들의 도움으로, β는 α로부터 논리적으로 도출될 수 있다

(iv) α는 β의 해결에 어떤 필수불가결한 실마리를 던져준다. 즉, β의 해석에 있어 도움을 준다.\footnote{Jeremy Waldron, ``Is Dignity the Foundation of Human Rights?'', 125면.}
\end{displayquote}

또한 뒤벨은 존엄으로부터의 인권에 대한 이해와 구체화는 가능한 또는 있음직한, 그리고 또한 시간이 지남에 따라 변하는 맥락-관련적 위협에 대한 대응으로 이루어져야 하므로, 인권은 변할 수 있고, 변할 것이고, 변해야만 하는 개념으로 이해되어야 한다고도 주장했다.\footnote{Marcus Düwell, ``Human dignity: concepts, discussions, philosophical perspectives'', 39면.} 이를 추가해 보면 다음과 같을 것이다.

\begin{displayquote}
(v) β는 α로부터 맥락-관련적으로 도출될 수 있다.
\end{displayquote}

이러한 맥락에서 ``인간존엄이 권리들을 정초한다''는 말 역시 여러가지로 이해될 수 있다. 다음에서는 월드론의 4가지 제안에 덧붙여, 인간존엄의 개념이 인권에 대해 가지는 정초적 역할로서 뒤벨이 제시한 `맥락-관련적 도출'이라는 제안을 다섯번째로 추가하여 ``인간존엄이 권리들을 정초한다''의 의미에 대한 다섯 가지 해석가능성들을 자세히 분석해보고자 한다.

\subparagraph{기원과 계통의 문제}

한 가지 설명방법은, 인권들이 인간존엄으로부터 나온다고 말할 때 이를 이해하는 한 가지 방식은 ``인권의 담론은 인간존엄에 대한 선재하는 담론으로부터 자라났다''고 이해하는 것이다.\footnote{Jeremy Waldron, ``Is Dignity the Foundation of Human Rights?'', 126면.} 즉 인권에 비해 인간존엄이 역사적 혹은 계통적인 의미에서 먼저 있었고 후자로부터 전자가 발생했다는 것이다. ``인간존엄을 자연권의 매우 초기 역사에 관계짓는 것, 그리고 자연권의 관념이 백년이상 많은 학파들에서 불신당할 때 왜 이 관념이 이러한(존엄이라는) 새로운 명찰을 가지고 쉽게 재생되었는지 설명하는 것은 흥미로운 과제''이다.\footnote{Ibid.} 즉 인간존엄에 대한 이야기의 보급과 힘은, 역사적인 방법으로, 인권에 대한 우리의 생각이 어디서 왔는지 우리에게 설명하는 것을 돕는다.

그러나 오스카 샤흐터의 말처럼 근대적 인권 담론이 있기 이전에 인간존엄 담론이 먼저 있었다는 것은 사실이지만, 인권들이 존엄의 담론으로부터 나왔다고 추측하는 것은 그럴듯하지 않다. 샤흐터는 헬싱키 최종의정서(Helsinki Final Act)가 모든 인권이 인간의 고유한 존엄에서 도출되었다고 선언하고 있다는 것을 역사적 의미가 아니라 철학적인 의미에서 이해해야 한다고 말한다. 존엄은 ``기본적 권리들과 자유들의 사회역사적 이해방식을 반영하는 것이지, 그것들을 생산해낸 것은 아니''라는 것이다.\footnote{Oscar Schachter, ``Human Dignity as a Normative Concept'', \emph{The American Journal of International Law} Vol. 77, No. 4 (Cambridge, 1983), 853면.} 오히려 현대적인 존엄 담론은 2차대전 이후 발생한 인권담론에 의존하고 있다고 보는 것이 더 그럴듯해 보인다. 즉 역사적인 의미에서 인간존엄의 담론을 인권 담론의 기원으로 보기는 어렵다.

\subparagraph{연원(source)과 정당성(legitimacy)}

켈젠은 헌법이하 하위규범들의 효력을 근본규범(Grundnorm)의 존재를 통해서 정당화했다.\footnote{Hans Kelsen, \emph{Pure Theory of Law} 2nd Edition, translated by Max Knight, (University of California Press, Berkeley, 1970(1960)). 변종필 역, 순수법학 (길안사, 1999), 30면 참조.} 여기서 근본규범은 그 하위 규범에 대해 실질적 내용에 대해 설명하는 바가 없는 형식적 규범으로 보통 알려져 있다.

켈젠은 법의 당위를 사실로부터 끌어내는 자연주의의 오류를 피하기 위해, 규범의 효력근거를 오로지 다른 규범에만 의존하도록 하는 피라미드와 같은 계층적인 구조로 법체계 전체를 파악하고자 시도했다. 법 피라미드의 가장 상위에는 헌법이 존재할 것이고, 논리적으로 이러한 헌법 역시 상위의 규범으로부터 그 규범력을 확보해야 할 것이다. 이 때, 그는 법체계의 최종적 근거, 모든 법규범들에 공통된 최종적 효력원천인 근본규범을 발견한다. 이러한 근본규범의 문제는 헌법에 구속력을 부여해 줄 뿐 내용이 없는 규범이며 단지 논리적으로 전제된 규범이라는 점이다.\footnote{이상영⋅김도균, 법철학, (한국방송통신대학교출판부, 2011), 72-76면 참조.}

이와 유사하게 개별 인권들과 존엄의 관계를 하위규범과 근본규범의 관계처럼 바라보는 견해가 있다. 클라우스 디케는 그의 기능적 분석을 통해 ``인간의 존엄은 인권 주장들을 정당화하는 형식적 초월적 규범''이라고 주장한다. ``존엄은 실체적으로 정의될 수 있고 개별 인권 주장들이 연역에 의해 즉시 도출될 수 있는 실질적 규범이 아니라''는 것이다.\footnote{Klaus Dicke, ``The Founding Function of Human Dignity in the Universal Declaration of Human Rights'' in David Kretzmer and Eckart Klein (eds), \emph{The Concept of Human Dignity in Human Rights Discourse} (Martinus Nijhoff Publishers, 2002), 118면.}

이러한 주장에는 이미 우리가 인권의 역사에서 임시대리기호(placeholder)로서 살펴본 것과 유사한, 존엄에 대한 다음과 같은 관찰이 전제되어 있다. 세계인권선언을 비롯한 많은 법적 문언들이 존엄을 선언하면서도 존엄의 실질적 내용을 정의하는 것은 삼가는 경향이 있다는 사실, 이들이 다양한 사상들의 전통을 언급하면서도 특정 전통의 존엄을 확정하지 않으려 한다는 사실 말이다.\footnote{Ibid.} 이렇게 정의되기 어려운 개념이 인권을 도출하는 척도와 규범을 확립할 수 있다는 사실을 설명하기 위해서는 근본규범과 같은 형식적 규범의 개념을 도입하지 않고서는 그 설명이 어렵다는 것이다.

물론 디케는 수많은 국가의 서명과 비준에 의해 효력있는 인권협약이 법의 연원이 전혀 아니라고 주장하려는 것은 아닐 것이다. 그가 부정하려는 것은 그 협약들에 의해 사람들이 권리를 가지는 이유가 단 하나, 오직 다자간의 조약에 의해 ICCPR이 입법되었다는 사실 뿐이라는 것이다. 디케의 설명에 의하면 ICCPR은 인권들을 창조하는 것이 아니라 이미 있는 인권들을 확인하고 선언하는 것이다. 여기서 인간존엄은 왜 인간이 실정법적 선언 이전에 이와 독립적으로 해당 권리들을 가지는 지를 설명하는 인간의 특별한 속성, 고유한 가치를 지칭한다.

앞서 역사적 설명방식을 비판한 샤흐터도 소극적이기는 하지만 비슷한 것을 말했다고 볼 수 있다. 그는 고유한 인간의 존엄으로부터 권리가 도출된다는 명제를 역사적인 진술이 아니라 철학적 진술로 이해했다. 이것이 의미하는 또 다른 한 가지는 인권들은 국가나 다른 외적 권위에서 도출되는 것이 아니라는 것이다. 샤흐터 역시 실정법의 권위를 부정한 것은 아니지만, 법이 이러한 권리들을 인식하는데 초실정적 요소가 배경에 있음을 주장하고 있다. 왜 우리가 인권의 보편성, 양도불가능성, 박탈불가능성과 같은 인권의 중요성을 주장하는지에 대한 초실정적 설명이 존재한다는 것이다.\footnote{Oscar Schachter, ``Human Dignity as a Normative Concept'', 853면.}

그럼에도 불구하고 인권 규범들에 대한 법실증주의적 입장을 견지하는 월드론은 이러한 초실정적 설명을 더 높은 자연법의 이해방식으로 끌고가는 것을 경계한다. 인권 규범들의 효력이나 정당성을 어떤 신법이나 자연법과 같은 비실정적 법으로 소급시키는 것보다는, 이러한 협약들이 초실정적 관념들에 대한 실정법적 반영이라고 생각하는 것이 더 낫다는 것이다. 월드론은 ``정당성(legitimacy)''의 용어를 매우 유연한 것으로 바라보면서, 이는 법적 효력을 의미할 수도 있지만, 느슨하게는 대중의 승인, 더 느슨하게는 도덕적 호소력을 말할 수도 있다고 보았다. 정당성이 만약 도덕적 호소력을 의미하는 것이라면 인권의 정당성이 인간존엄의 관념에 호소하고 있다고 말하는 것도 충분히 수긍할 수 있다는 것이다.\footnote{Jeremy Waldron, ``Is Dignity the Foundation of Human Rights?'', 128면.} 이러한 주장은 여전히 인권들의 법적 효력의 연원으로서의 존엄을 형식적이고 전제에 의존하는 단순히 상위입법권으로서의 권한부여적 관념으로 이해한다는 한계를 가진다. 보다 구체적인 내용적 근거를 부여하는 관념으로서의 정초의 이해는 다음의 강건한 정초주의적 입장에서 드러난다.

\subparagraph{진정한 논리적 도출의 기초(강건한 정초주의)}

보다 실질적인 기능을 하는 정초의 의미는, 권리들의 정초가 무엇인지 알게되면 우리가 (때로 새로운) 인권주장들을 발생시키거나 도출하는 것을 가능하게 해 주는 것이라고 할 수 있다. 이는 사람들이 인권에 대해 말하는 것을 평가하는 리트머스 시험지를 제공하기도 한다. 그리핀이 존엄과 인권의 관계에 대해 설명하는 것을 예로 들어보자.

제임스 그리핀은 인권을 우리의 규범적 주체성(normative agency)의 보호장치로 이해한다. 규범적 주체성은 세 국면을 가지고 있는데 하나는 더 나은 삶에 대한 (다소 불완전한) 이해방식을 형성하는 자율(autonomy), 두번째는 이를 지원하는데 필요한 건강수준과 육체적 정신적 능력과 교육과 같은 최소조건을 일컫는 복지(welfare), 그리고 타인이 이러한 추구를 방해하는 것을 금지하는 자유(liberty)가 그것이다. 그리핀에 의하면 이 세 가지가 가장 상위의 인권의 삼중주를 형성한다고 한다.\footnote{James Griffin, \emph{On Human Rights} (Oxford, 2008), 150-152면: ``{[}Autonomy's{]} value, on my account, is related to its being a constituent of the dignity of the human person.''}

여기서 그리핀은 자율과 자유를 구분하고 있다. 그가 생각하는 자율은 진실된 정보를 통해 자신에게 가치있는 것을 결정할 능력을 말한다. 반면 자유는 이렇게 결정된 것을 실제로 수행함으로써 성취하는 것을 의미한다. 자율의 적은 세뇌, 정보의 조작이나 미성숙함 등이지만, 자유의 적은 강요나 구속, 삶에 있어서 선택지의 부족과 같은 것들이다. 따라서 어떤 사람은 자유롭지만 자율적이지 않을 수 있다. 자율성은 사람들이 각 개인의 삶을 잘 살게 만드는 목록을 각자 채택하는 것을 의미하고, 따라서 자기가 선택한 것이 아니면 어떤 좋은 것도 성취가 되지 않게 하는 그런 가치를 지칭하기 때문이다. 그리고 나서 제임스 그리핀은, 자율의 가치를 설명하면서, 이 가치는 인간존엄의 구성요소이기 때문에 가치있는 것이라 설명한다.\footnote{Ibid.}

월드론은 이와 같은 그리핀의 설명을 다음과 같이 단계화한다. 인간의 존엄은 `규범적주체성의 중요성'이라는 용어로 가장 잘 이해되고, 이러한 규범적 주체성의 가치는 자율성 안에서 발현되며, 이 자율성 능력은 자유를 요청한다는 것이다. 이는 단계적 연역에 의해 분석적으로 확립되는데, 존엄으로부터 규범적 주체성이, 이로부터 자율성이, 이로부터 소극적 그리고 적극적 자유가 확립된다는 것이다.\footnote{Jeremy Waldron, ``Is Dignity the Foundation of Human Rights?'', 129면.} 그리고 복지권 등과 관계된 이러한 연역들의 일부는 선험적 전제들에서만 도출되는 것이 아니라 경험적 전제들에 의해 중개되기도 한다. 우리는 이러한 정초에 대한 이해방식을 선형적 이해방식이라고 할 수 있으며, 이 때 존엄은 모든 인권들의 텔로스와 같은 것이 된다. 이러한 이해방식의 특징은 우리가 이미 대상으로 삼고 있는 권리들을 넘어서 다른 권리들을 도출하여 권리목록을 확장할 수 있는 가능성을 열어준다는 것이다.

\subparagraph{권리해석의 기준}

강건한 정초주의가 연역적, 하향식 접근이라면, 여기서 살펴볼 접근방법은 귀납적, 상향식 접근방법에 가깝다고 할 수 있다. 월드론에 의하면 우리가 무슨 권리들을 가지고 있고 왜 그들을 가지는지에 대한 좋은 원리적 설명을 부여하는 것은 (iii)의 모델에서처럼 선형적 도출을 포함할 필요는 없다고 한다. 왜 우리가 인권을 가지는지를 이해하는 것은 우리가 가진 권리들을 핵심을 이해하는 것을 포함하지만, 권리들의 핵심이 엄격하게 목적론적인 의미---텔로스의 진술로부터 다른 권리들의 도출을 허가한다는 의미---로 이해될 필요는 없기 때문이라는 것이다. 이러한 방식으로 월드론은 우리가 이미 대상으로 삼고 있는 권리들을 넘어서는 권리 목록의 확장을 허용하는 방식으로 권리들의 핵심을 이해할 필요는 없다고 말한다. 즉, 그는 새로운 권리들을 도출하는 기초가 아니라 이미 존재하는 권리들을 더 자세히 이해하기 위한 해석적 근거로서 정초의 개념을 이해하려고 한다. 이는 인권이 법적 권리로 이해될 때 특히 타당하다고 한다.\footnote{Ibid., 130-131면.}

월드론에 의하면 우리는 어떤 것이 법이라는 것을 보여줄 때, 그 법이 다른 법명제들을 이해하는데 필요해 보이는 것으로부터 도출될 수 있음을 항상 보여줄 수 있는 것은 아니다. 그렇지만 권리들이 헌법이나 인권법에서 법적으로 현존하고 있다는 사실이 더 깊은 이해의 필요성을 제거하지 않는다. 가장 명백히 확립된 권리들조차 당혹스러울 때가 많기 때문이다.\footnote{Ibid., 131면.}

우리가 때로 정초라고 부르는 것은 권리의 핵심을 이해하는 방식으로서 이는 1) 우리가 특정한 권리 조항들을 해석하는 것을 돕고, 2) 권리기반 주장들을 심화하는 데 있어 진행해야 할 정신을 결정하는 것을 돕고, 3) 권리들의 가능한 충돌과 그 한계의 문제를 다루는 방식을 결정하는 것을 돕는다.

월드론은 존엄을 이러한 의미에서 정초로 다루면, 존엄을 강하게 혹은 약하게 해석할 때마다 그 영향이 달라질 수 있다고 이야기한다. 예를 들어 독일의 항공기 사례에서 독일연방헌법재판소는 911테러와 같은 상황에서 승객이 탑승한 항공기를 군이 격추할 권한을 부여한 제정법의 맥락 속에서 생명권을 고려했는데, 이 때 재판소는 강한 칸트적 의미의 존엄개념을 사용하였기 때문에 무고한 승객을 단순히 훨씬 더 많은 다른 무고한 생명을 구하기 위해 죽일 수 없다고 판단했다는 것이다. 하지만 덜 강한 의미의 존엄을 사용한다면, 이러한 접근방식은 단지 권리를 해석하는 데 있어 개인들과 그들의 자율성을 심각하게 고려하고, 공동선의 향상을 위한 경험적인 발견법(heuristic)으로서 그들을 대해서는 안된다는 것을 지시하는 것 뿐이라고 한다.\footnote{Ibid., 131-132면.}

인간존엄의 인권에 대한 정초 기능에 대해 이러한 입장을 취하게 되면 인간존엄의 가치는 권리를 창조하는 역할을 하지 못하고, 단지 기존에 존재하는 규범의 이해를 돕는 해석적 기능만을 수행하게 된다.

\subparagraph{맥락-관련적 도출}

또 다른 입장은 인간존엄의 인권에 대한 정초 개념을 보다 동적이고 현실적인 대응으로 전환하려고 노력한다. 인권을 구체화하는 것은 인간존엄이라는 개념에서 매우 포괄적이고 다양한 인권을 도출할 만큼 이론적으로 충분히 열심히 생각하는 이론철학자에 의해 가장 잘 결정될 수 있는 어떤 정적인 생산이 아니다. 인권에 대한 이해와 구체화는 가능한 또는 있음직한 맥락-관련적 위협에 대한 대응으로 이루어져야 하므로, 인권은 시간이 지남에 따라 변할 수 있고, 변할 것이고, 변해야만 하는 개념으로 이해되어야 한다는 것이다. \footnote{Marcus Düwell, ``Human dignity: concepts, discussions, philosophical perspectives'', 39면}

이와 같이, 인권에 대한 이해와 구체화는 가능한 또는 있음직한 맥락-관련적 위협에 대한 대응으로 이루어져야 하며, 인간존엄의 개념을 통해 시간이 지남에 따라 변할 수 있고, 변할 것이고, 변해야만 하는 개념으로 인권을 구성하고자 하는 마커스 뒤벨은, \footnote{Ibid.} 새로운 권리들을 결정할 때에는 적어도 다음과 같은 5가지 요소를 잘 고려해야 한다고 한다.

\begin{displayquote}
(1) 인간존엄의 존중과 보호는 전체 인권 체제의 규범적 기초를 형성한다.

(2) 인간의 필요, 욕구 및 취약성의 변화는 그들의 존엄을 침해 할 수있는 관점에 영향을 미칠 수있는 요인이다. 내 자신을 통제 할 수있는 행위주체로 보는 나의 가능성은 다양한 문화적 요인에 의존할 것이다.

(3) 존엄한 삶을 사는 것에 대한 새로운 위협이 있을 수 있다. 기후 변화 또는 디지털 세계에서의 가능성들은 삶의 조건을 매우 심각하게 변화시켜 존엄한 삶이 새로운 방식으로 위협받을 수 있다. 기후 변화는 지구 전체에서 자율적 삶을 살 가능성을 위험에 빠뜨릴 수 있다.

(4) 또한 인간존엄을 보호할 수 있는 새로운 형태가있다. 인터넷의 가능성은 (예를 들어, 사람들이 더 유능한 행위자가 될 수 있게 함으로써) 권리 실현을 위한 새로운 기회를 창출 할 수 있다.

(5) 가능한 의무-담지자는 예를 들어 인간을 보호 할 수 있는 새로운 능력을 가진 새로운 (국제적 또는 세계적) 기관의 진화에 의해 변할 수 있다.\footnote{Ibid., 39-40면.}
\end{displayquote}

인권의 목록은 도덕적 의미에서든 법적 의미에서든 결코 완성될 성질의 것이 아니며, 경제질서나 자연환경의 변화 등에 의한 위협은 계속해서 새로운 보호방법을 강구할 것을 요청한다. 그럼에도 불구하고 이에 대한 새로운 대응을 하지 않는 것이야말로 부도덕한 일이 된다. 인권의 권위를 유지하기 위해서는 이러한 변화를 반영하여 인간존엄에 대한 보호를 해석하고 새로운 형식의 인권을 창조할 필요가 있다는 것이 뒤벨의 주장이다.\footnote{Ibid., 40면.}

\subsubsection{강건한 맥락-관련적 정초 개념의 요청}

현대 인권 담론이 인간존엄의 이해방식의 확립에 거는 기대는 아마도 위에서 언급한 강건한 정초주의의 입장에 가깝다고 할 수 있을 것이다. 많은 이들은 인간존엄이 인권들을 산출하는 진정한 논리적 기초가 되면서도 뒤벨이 우려한 바와 같은 맥락-관련적 위협에 대응할 수 있는 유연하고 다양한 권리들을 산출하는 근거가 되어주기를 바란다.

그러나 논의의 한 축에서는 인간존엄 개념의 지나친 추상성과 개방성이 권리의 개념을 형해화시키고 기존의 논의를 잠식시키는 논의종결자(conversation stopper)로 이용되는 것을 경계한다. 이러한 입장에서 인간존엄의 의미를 최대화하는 해법은, 인간존엄이 새로운 권리를 증가시키는 것이 아니라 다만 기존에 있는 권리를 보다 풍부하게 만들어 주는 것으로 이해하는 것이다. 앞서 살펴보았듯 월드론은 권리들이 헌법이나 인권법에서 법적으로 현존하고 있다는 사실이 더 깊은 이해의 필요성을 제거하지 않으며,\footnote{Jeremy Waldron, ``Is Dignity the Foundation of Human Rights?'', 131면 .} 존엄은 우리가 특정한 권리 조항들을 해석하는 것을 돕고, 권리기반 주장들을 심화하는 데 있어 진행해야 할 정신을 결정하는 것을 돕고, 권리들의 가능한 충돌과 그 한계의 문제를 다루는 방식을 결정하는 것을 돕는다고 말한다. 이것이 우리가 인간존엄이 권리들의 정초라고 이야기할 때 `정초'를 이해하는 방식이어야 한다는 것이다.

그러나 만약 월드론의 입장과 마찬가지로 인간존엄이 권리를 창조하는 역할을 하지 못하고, 단지 기존에 존재하는 규범의 이해를 돕는 해석적 기능만을 수행하게 된다면, 기존 권리의 해석만으로는 도출하기 어려운 인간성의 핵심적 부분들을 보호할 권리의 누락이 발생하는 것을 피할 수 없게 된다. 그리고 무엇보다도 이는 인간존엄을 (실정적) 권리를 가질 권리 혹은 지위로 이해하는 것을 전제하고 있는데, 이러한 이해는 2차대전의 경험과 생명공학의 발전 이후에 우리가 인간존엄의 개념에 투영시키고자 하는 어떤 개인의 내적가치나 특별히 보호해야할 인간성으로서의 이해방식에 위배된다.\footnote{Christopher McCrudden, ``Human Dignity and Judicial Interpretation of Human Rights'', 679-680면 참조.}

본고는 인간존엄의 개념이 인권담론에서 수행해야 할 강건한 맥락-관련적요청을 수행할 수 있는 개념을 확립하고자 한다. 이에 대해서는 제3부의 제2장에서 제시할 것이다.

\subsection{헌법 담론에서 인간존엄}

앞서 살펴본 세계인권선언을 비롯한 국제문서들을 통해 관찰할 수 있는20세기 후반의 국제사회의 인간존엄에 대한 이해방식은 각국의 법문언에도 영향을 끼치게 된다. 우리나라 헌법이나 독일 기본법에서도, 인간존엄은 무엇이고 그에 대한 침해는 어떤 태양을 말하는지, 그 침해에 대한 적절한 법률효과는 무엇인지에 대한 침묵하는 편이지만, 독일기본법이 인간존엄을 ``건드릴 수 없는(unantastbar)'' 어떤 것으로, 대한민국 헌법이 인간존엄을 인간이 ``가지는'' 어떤 것으로 표현하고 있으며, 이에 기반한 사법적 판단들은 다양한 사안에서 인간존엄의 침해를 자주 인정하고, 이에 대한 모종의 구체적 법률효과와 국가의 보호의무가 있는 것으로, 특히 이로부터 어떤 권리들이 침해되는 것으로 파악하고 있다. 이러한 점으로 미루어 볼때, 대표적인 각국의 헌법 문서가 예정하고 있는 인간존엄 역시 인간이 가지는 어떤 인식가능한 고유한 속성으로, 또한 어떤 침해가능한 권리들이 도출되는 기초로 파악하는 것으로 이해될 수 있다.

\subsubsection{기능론: 법적 가치와 권리로서의 인간존엄}

대한민국 헌법은 총 세 군데에서 `존엄' 혹은 `존엄성'을 언급한다. 먼저 제10조에서 ``모든 국민은 인간으로서의 존엄과 가치를 가지며, 행복을 추구할 권리를 가진다. 국가는 개인이 가지는 불가침의 기본적 인권을 확인하고 이를 보장할 의무를 진다.''라고 선언하고 있다. 또한 제32조 3항에서 ``근로조건의 기준은 인간의 존엄성을 보장하도록 법률로 정한다.''고 하고 있으며, 제36조 제1항에서 ``혼인과 가족생활은 개인의 존엄과 양성의 평등을 기초로 성립되고 유지되어야 하며, 국가는 이를 보장한다.''고 선언하고 있다. 이 가운데에서 가장 중요하고 또한 논란이 되는 조항은 모든 기본권에 앞서 ``모든 국민은 인간으로서의 존엄{[}\ldots{]}를 가지며,{[}\ldots{]}''라고 규정하고 있는 제10조라고 할 것이다. 제32조 제3항과 제36조 제1항의 ``인간의 존엄성''과 ``개인의 존엄''의 의미는 가장 포괄적인 표현을 하고 있는 제10조의 ``인간으로서의 존엄''의 의미와 그 기능을 탐구하는 과정에서 자연스럽게 밝혀진다.

\subparagraph{헌법적 가치로서의 인간존엄}

그렇다면 헌법 제10조의 ``인간으로서의 존엄'', 즉 대한민국 헌법에서 인간존엄은 어떤 기능을 하고 있다고 보아야 하는가? 먼저, 그 규정형식 이전에 학자들은 인간존엄이 ``최고규범으로서 모든 법령의 효력과 내용을 해석하는 기준이 되는 근본원리''라고 하거나,\footnote{김철수, 헌법학신론 제21전정신판(박영사, 2013), 420-424면.} ``기본권 보장의 이념적 기초'' 혹은 ``헌법상의 이념 또는 원리'' \footnote{정종섭, 헌법학원론 제11판(박영사, 2016), 410면.} 등으로 기능할 수 있다고 생각한다. 이를 아론 바락은 이와 같은 기능을 헌법이 지향하여야 할 ``가치''(constitutional value)\footnote{Aharon Barak, \emph{Human Dignity: The Constitutional Value and the Constitutional Right} (Cambridge University Press, 2015), 제6장 참조.}로서의 인간존엄으로 설명한다.

바락에 의하면 헌법적 가치로서의 인간존엄은 ``인권들을 하나의 전체로 통합하는 요소'', 즉 ``인권들의 규범적 통합을 가능하게'' 하는 것이다.\footnote{Aharon Barak, \emph{Human Dignity}, 103면.} 이는 세 가지 방식으로 작동하는데, 첫째는 ``헌법적 권리들의 규범적 기초''로서, 둘째는 ``인간존엄권을 포함한 헌법적 권리의 범위를 결정하는 해석적 원리''로서, 세 번째는 ``헌법적 권리들을 제한하는 제정법의 비례성을 결정하는 데 있어서 중요한 역할''로서 작동하는 것이다.\footnote{Aharon Barak, \emph{Human Dignity}, 103-104면.}

본고는 이에서 더 나아가 인간존엄이 단지 권리규범 뿐만 아니라 권리를 제한하는 규범을 포함한 전체 헌법규범의 통합을 가능하게 하는 요소로 파악한다. 즉, 권리규범 및 권리제한규범을 모두 포괄하는 전체 헌법규범의 범위를 결정하는 해석적 원리로서 인간존엄의 역할을 확장하여 이해하고자 한다.

첫 번째로 바락이 언급한 ``헌법적 권리들의 규범적 기초''로서의 인간존엄의 헌법적 가치는 본고가 앞서 제1장에서 살펴본 인권의 정초로서의 인간존엄을 이야기하고 있는 것과 같은 취지의 서술이다.\footnote{바락은 여기서 그가 말하는 `권리들의 기초'라는 말의 의미가 헌법의 구속력을 설명하는 켈젠의 근본규범을 의도하고 있는 것이 아니라는 점을 분명히 하고 있다. Aharon Barak, \emph{Human Dignity}, 105면.} 그에 따르면 인간존엄은 인권의 존재를 증명하는 ``중심적 논변''이고 모든 인권들의 ``합리적 정당화의 근거''이다.\footnote{Aharon Barak, \emph{Human Dignity}, 104면.} 이러한 역할, 즉 정초 개념에 대해 제기되는 추가적인 의문에 대해서는 해당 장에서 이미 설명하였다. 본고의 확장적 관점에서 바라보면, 인간존엄은 헌법의 권리규범 뿐만 아니라, 권리를 제한하는 규범을 비롯한 전체 헌법규범 모두의 중심적 논변이자 합리적 정당화의 근거라고 할 수 있다.

두 번째 ``인간존엄권을 포함한 헌법적 권리의 범위를 결정하는 해석적 원리''로서의 인간존엄은 보다 본격적으로 제2장에서 다루고자 하는 인간존엄의 헌법적 가치로서의 역할이다. 헌법적 가치로서의 인간존엄은 ``법체계의 규범에 의미를 공급''한다.\footnote{Aharon Barak, \emph{Human Dignity}, 105면.} 바락에 의하면 헌법의 ``모든'' 규정, ``특히 권리장전의 모든 권리들''은 ``인간존엄의 측면에서 해석된다.''\footnote{Aharon Barak, \emph{Human Dignity}, 105-6면.}

이를 또한 본고의 확장적 관점에서 바라보면, 바락은 주로 권리들의 관점에서 해석적 원리로서의 인간존엄을 설명하고 있지만, 거꾸로 인간존엄은 헌법적 권리들을 제한하는 규정을 해석하는데 있어서도 사용된다. 대표적으로 프랑스의 난쟁이 던지기 사건에서 직업과 활동의 자유를 행정권력이 제한할 수 있는 규범의 토대가 되는 `공공질서(l\textquotesingle ordre public)' 개념의 해석적 기초로서 인간존엄이 사용된 바 있다.

세 번째 ``헌법적 권리들을 제한하는 제정법의 비례성을 결정하는 데 있어서의 중요한 역할''로서의 인간존엄은 헌법적 권리들을 제한하는데 있어 한계를 설정하는 문제에 관여한다.\footnote{Aharon Barak, \emph{Human Dignity}, 103-104면.} 대부분의 헌법적 권리들은 상대적 권리이기 때문에 행사에 제한이 따를 수 있다. 하지만 이러한 제한은 비례적이어야 한다는 것이 통설적 견해라 할 수 있다. 이 비례성을 구체화하는데 있어 인간존엄은 중요한 역할을 하게 된다.

본고의 확장적 관점에서 인간존엄은 단지 인간존엄의 기본권 제한의 한계를 설정하는 좁은 의미의 비례성 원칙에서의 형량에서만 고려되는 것이 아니다. 인간존엄은 모든 해석이 전체 법체계의 관점에서 지향해야 할 최종적 목적으로서 기본권 규정의 이익형량이 필요한 모든 지점, 예를 들어 기본권 간의 충돌문제에서의 이익형량에 있어서도 중요한 고려사항이 되어야 한다.

\subparagraph{주관적 권리로서의 인간존엄}

다음으로, 학자들은 인간존엄이 국민 혹은 인간이 가지는 주관적 권리이기도 하다고 주장한다.\footnote{김철수, 헌법학신론, 420-424면; Aharon Barak, \emph{Human Dignity: The Constitutional Value and the Constitutional Right}.} 그런데 인간존엄의 주관적 권리성의 의미와 내용에 대해서는 학설이 다양하게 대립한다. 물론, 인간존엄의 권리성을 인정하지 않는 부정설도 존재한다. 부정설은, 인간존엄 자체의 권리성은 부정하지만 인간존엄의 가치로부터 권리들이 도출될 수 있다고 보는 견해와,\footnote{정종섭, 헌법학원론, 412면.} 인간존엄은 단지 이미 실정화된 기본권의 이념적 출발점, 혹은 배경적 관념(underlying idea)일 뿐이어서 이로부터 어떤 새로운 법적 권리들이 도출되는 근거가 되지 않는다는 견해\footnote{Jeremy Waldron, ``Is Dignity the Foundation of Human Rights?'', 130-131면 참조. 다만 이 견해는 권리를 가지는 지위로서의 인간존엄을 인정한다.}로 나눌 수 있다.

제정법의 확인 이전에 존재하는 고유한 인권의 존재를 인정하더라도, 법이 보호해야 할 가치의 존재만으로 그러한 가치를 성취하도록 해 주는 개인이 행사할 수 있는 주관적 권리의 존재를 함축하는 것은 아니다. 여러 법적 가치들은 비개인적이고 공동체적인 가치인 경우도 있고, 설사 개인적 가치라 하더라도 타인의 권리행사를 수인해야 할 상대방의 의무를 인정하기 어려운 경우도 많기 때문이다. 따라서 인간존엄의 주관적 권리를 인정하기 위해서는 추가적인 논증이 필요하다.

인간존엄의 주관적 권리의 존재를 인정하는 경우에도, 그 권리의 내용에 대해서는 또한 많은 논란이 있다. 인간존엄이 그 자체로 주관적 권리라고 하거나, 인간존엄의 가치로부터 권리들이 도출될 수 있다고 보거나, 제10조 ``조항''이 기본권을 설시한 조항이라고 간주하는 경우,\footnote{한수웅, 헌법학 제6판(법문사, 2016), 525-526면.} 공히 그 `인간존엄의 권리', `인간존엄으로부터의 권리', 혹은 `인간존엄조항에 설시된 권리', 다시 말해 ``존엄을 존중 및 보호받을 권리''\footnote{남아프리카공화국헌법 제10조.}의 내용은 구체적으로 무엇인가? 특정한 가치를 보호하는 권리인가? 다른 권리들을 파생시키는 포괄적 권리인가?

인간존엄의 주관적 권리성을 인정하고, 또한 그 내용이 어느 정도 확정되면, 다른 권리들과의 관계가 문제된다. 인간존엄을 일종의 포괄적 권리라고 한다면, 인간존엄을 고려하는 경우 다른 권리는 고려하지 않아도 좋은가? 아니면 인간존엄의 권리는 다른 모든 권리들을 고려하고 남는 영역만을 관장하는가? 인간존엄의 권리가 다른 권리와 충돌하는 경우는 어떻게 이해해야 하는가? 이와 같이 인간존엄의 권리성을 이해하기 위해서는 다양한 문제들을 해명해야 할 것이다.

다음에서는 인간존엄의 헌법적 가치로서의 기능과 헌법적 권리로서의 기능을 보다 세부적으로 분석하면서 위에서 제기된 질문에 대한 해답을 구해보고자 한다. 그런데 우리나라 헌법 제10조의 기술형식은 인간존엄이 헌법적 가치 혹은 권리인지에 관한 규범적 기능에 대해 침묵하고 있고, 따라서 인간존엄의 규범적 기능을 풍부하게 만들고자 하는 근거들이 모두 이론에 기대고 있다. 이러한 침묵은 인간존엄 규범을 정교화하고 세분화하려는 시도를 위축시키거나 특정한 방향없이 난립시키는 경향이 있다. 독일에서 독일기본법 제1조로부터의 인간존엄의 논의도 마찬가지의 현상이 발생하고 있다. 이와 대조적으로 남아프리카공화국의 헌법은 인간존엄의 규범적 기능을 가치와 권리로서 세부적으로 분류하고 있다. 다음에서 인간존엄의 헌법적 기능에 대한 논의를 진행하는데 있어, 남아프리카공화국헌법은 인간존엄의 법제도화의 대표적인 실례를 잘 제시해주고 있다.

\paragraph{헌법적 가치로서의 인간존엄: 권리 또는 법률 해석의 근거}

\subparagraph{헌법적 권리의 내용을 결정하는 해석적 근거}

법에서 인간존엄이 어떤 역할을 할 때, 인간존엄이 가치로서 사용되었다고 볼 수 있는가? 이러한 사용으로 자주 예시되는 것은 소위 ``권리 해석의 기초''로서의 기능이다. 예를 들어 인간존엄이 법에서 증진시키고자 하는 모종의 구체적인 가치라고 해 보자. 이 때, 이러한 가치는 그러한 가치를 증진하기 위해서 규정된 법규범의 내용, 특히 권리규범의 내용을 명확히 하기 위한 해석의 근거로서 사용될 수 있다. 이 때 법적 가치는 주로 법적 권리가 증진시켜야 할 대상을 분명히 하는 개념이다.

인간존엄의 가치와 이를 증진시키는 규범의 관계를 재산적 가치와 이를 증진시키고자 규정된 재산을 보호하기 위한 규범들과 재산적 권리들에 비유해보자. 때로 재산권의 내용이 불명확 할 때, 이를 명확히 하기 위해서는 재산적 가치가 무엇인지를 분명히 할 필요성이 있다. 이는 사용, 수익, 처분의 이익이라고 말할 수 있다. 따라서 어떤 재산권을 보호하기 위한 규정이 모호한 경우, 재산권자의 사용, 수익, 처분의 이익을 보호하는 방향으로 해석할 수 있다. 마찬가지로 많은 혹은 모든 헌법적 권리의 내용은 인간존엄이며, 이러한 권리들은 인간존엄을 보호하거나 증진시키기 위한 수단이라고 생각할 수 있다. 이 때, 인간존엄의 헌법적 가치는 이러한 헌법적 권리들의 내용을 결정하는 해석적 근거가 된다. 이와 같이 헌법적 권리 내용을 결정하는 해석적 근거로서 인간존엄의 헌법적 가치가 사용된 예를 바락은 다음과 같이 평등권의 해석과 형벌, 그리고 사회경제적 권리와 관련한 사례에서 찾고 있다.

사례1. 평등권해석에 있어서 인간존엄의 헌법적 가치

많은 국가들의 헌법에는 평등권 혹은 차별금지의 조항이 있다. 예를 들어 남아프리카공화국헌법 제9조는 제3항에서 ``국가는 인종, 성별, 임신, 혼인 상태, 민족적 또는 사회적 출신, 피부색, 성적 지향, 연령, 장애, 종교, 양심, 신념, 문화, 언어 및 태생을 포함한 하나 이상의 사유를 근거로 하여 누군가를 직간접적으로 부당하게 차별해서는 안 된다''고 규정하고 있다. 이 때, 인종, 성별, 임신, 혼인 상태, 민족적 또는 사회적 출신, 피부색, 성적 지향, 연령, 장애, 종교, 양심, 신념, 문화, 언어 및 태생은 한정적 열거가 아니다. 따라서 금지되는 차별을 확인하기 위해서는 어떤 구별적 취급이 차별이 되는지에 대한 기준이 요구된다. 이 때, 남아프리카공화국 대법원이 그 기준을 인간존엄에서 찾고 있는 것처럼,\footnote{\emph{Prinsloo v. Van der Linde}, 1997 (3) SA 101 (CC); \emph{President of the Republic of South Africa v. Hugo}, 1997 (4) SA 1 (CC).} 인간존엄의 헌법적 가치는 평등권의 내용을 결정하는 해석적 근거로서 사용될 수 있다.\footnote{Aharon Barak, \emph{Human Dignity}, 108-109면.}

사례2. 형벌에 있어서 인간존엄의 헌법적 가치

바락이 헌법적 권리 내용을 결정하는 해석적 근거로서 인간존엄의 헌법적 가치가 사용된 것으로 보고 있는 두 번째 사례는 헌법에 위반되는 형벌을 확인하는 사례다. 많은 국가들의 헌법은 잔인하고 비인간적인 형벌을 금지하는 조항을 두고 있는데, 예를 들어 미국의 수정헌법 제8조는 ``잔인하고 기이한 처벌''을 금지하고 있다. 이에 관하여 미국연방대법원은 ``수정헌법 제8조는 모든 인간의 존엄을 존중할 정부의 의무를 재확인한다''거나,\footnote{\emph{Roper v. Simmons}, 534 US 551, 560 (2005).} ``제8조에 깔려있는 기본 개념은 인간의 존엄에 다름아니다''라고 설시하는 등,\footnote{\emph{Trop v. Dulles}, 356 US 86, 100 (1958).} 어떤 형벌이 위헌적 형벌인지에 대한 기준으로 인간존엄의 헌법적 가치가 사용될 수 있다.\footnote{Aharon Barak, \emph{Human Dignity}, 109-110면.}

사례3. 사회경제적 권리 해석에 있어서 인간존엄의 헌법적 가치

바락이 헌법적 권리 내용을 결정하는 해석적 근거로서 인간존엄의 헌법적 가치가 사용된 것으로 보고 있는 마지막 사례는 사회경제적 권리의 내용을 결정하는 데 있어서 보여지는 남아프리카공화국 헌법재판소 판례의 태도이다.\footnote{Ibid., 110면.} 예를 들어 남아프리카공화국헌법은 주거권과 관련하여 제26조 제2항에서 ``국가는 가용 자원의 범위 내에서 이러한 권리를 점진적으로 실현하기 위해 합리적인 입법 조치 및 기타 조치를 취해야 한다.''고 하고 있는데, 이 때, ``합리적 {[}\ldots{]} 조치''를 해석함에 있어서, 인간존엄의 가치가 고려되어야 한다는 것이다.\footnote{\emph{Government of the Republic of South Africa and Others v. Grootboom}, 2001 (1) SA 46 (CC), paras. 38 and 41.}

\subparagraph{권리제한 규범의 내용을 결정하는 해석적 근거}

한편으로 인간존엄은 권리를 제한하는 규범을 해석하는 근거로서도 사용된다. 대한민국 헌법 제37조 2항은 ``국민의 모든 자유와 권리는 국가안전보장·질서유지 또는 공공복리를 위하여 필요한 경우에 한하여 법률로써 제한할 수 있으며, 제한하는 경우에도 자유와 권리의 본질적인 내용을 침해할 수 없다''고 하여, 국가안전보장·질서유지 또는 공공복리의 목적으로 법률로써 권리를 제한할 수 있도록 규정하고 있다. 이 때 국가안전보장·질서유지 또는 공공복리를 해석하는 기준으로 해당 개념이 지향하는 ``안전''이나 ``공동체적 이익''과 같은 법적 가치를 인간존엄의 관점에서 이해해 볼 수 있다.

사례. 프랑스의 난쟁이 던지기 사건

법적 가치로서, 특히 권리제한 규범의 내용을 결정하는 해석적 근거로서 인간존엄의 보호를 시도한 대표적 판례는, 프랑스 꽁세유데따(Conseil d\textquotesingle État)의 난쟁이 던지기 판례다.\footnote{에네뜨-보쉐(Stephanie Hennette-vauchez)에 의하면 프랑스에서 인간존엄은 크게 두 가지 의미로 사용되고 있으며, 그중 하나는 사회권을 도출하는 근거로 사용하는 것, 다른 하나는 인간존엄의 원리를 공리주의나 인권도출의 차원과는 상반되는 것으로 이해하는 존엄주의적으로 해석하는 것이었다. 둘 중 여기서 주목하고자 하는 것은 후자의 존엄주의적 해석을 반영한 것으로 보이는 대표적 사례인 꽁세유데따(Conseil d\textquotesingle État)의 난쟁이 던지기 금지 판결이다. Stephanie Hennette-vauchez, ``Human Dignity in French law'', \emph{The Cambridge Handbook of Human Dignity} (Cambridge, 2014).} 이 판결에서 꽁세유데따는, 추상적 목적으로서의 인간존엄의 보호가 유효하게 시민적 자유의 사용에 대한 행정적 제약의 우선적인 근거가 될 수 있다고 보았다. 즉 인간존엄을 공공질서(l\textquotesingle ordre public)의 한 구성요소로 취급할 수 있다고 보았다.\footnote{Conseil d\textquotesingle État, 27 octobre 1995, Commune de Morsang-sur-Orge.}

마뉴엘 배켄하임(Manuel Wackenheim)은 프랑스의 모르쌍-쒸르-오르쥐(Morsang-sur-Orge) 市에서 난쟁이 던지기 경기를 통해 돈을 벌고 있는 난쟁이였다. 난쟁이 던지기 경기는 난쟁이를 멀리 던지는 팀이 승리하는 경기이고, 여기서 난쟁이들은 충분한 보호구를 착용하고 있으며, 착지 지점에는 충격을 흡수하기에 충분한 매트리스가 깔려 있다. 그러나 모르쌍-쒸르-오르쥐의 시장은 이 경기를 금지하는 행정명령을 내렸고, 배켄하임은 이 명령에 불복하는 소송을 제기했다. 베르사유의 행정법원은 1992년 2월에 ``모르쌍-쒸르-오르쥐의 공공질서, 평화, 보건을 문란하게 하였음을 확인할 수 없고'', ``당해 행사가 인간존엄을 깎아내리는 모욕을 표현한다고 가정하더라도, 특정한 지역적 사정이 고려되지 않은 금지는 법적으로 명령될 수 없다''면서 시장의 명령을 무효화했다. 그러나 시장은 이에 불복하여 상급법원인 꽁세유데따(Conseil d\textquotesingle État)에 항소하였고, 꽁세유데따는 베르사유 행정법원의 결정을 뒤집었다.\footnote{Ibid.} 꽁세유데따는 `` `난쟁이 던지기'라는 오락은 신체적 장애가 있는 한 인격을 발사체로서 사용하도록'' 하기 때문에, ``이런 종류의 오락은 인간의 존엄을 침해하는 것으로 간주되고'', ``그 금지는 지역적 상황의 고려없이 합법적''이라고 설시하면서, 인간존엄을 공공질서(l\textquotesingle ordre public)의 한 구성요소로 취급할 수 있다고 보았다. ``공공질서는 순수하게 ``물질적 외적''방식으로만 정의될 수 없고, 이에는 공권력이 적절한 존중을 요구해야 하는 인간에 대한 이해방식을 포함한다''고 주장했다.\footnote{Ibid.}

유엔인권위원회도 비슷한 의견을 나타냈다. 이 사건은 여기서 끝나지 않고, 배켄하임이 유럽인권위원회(the European Commission on Human Rights)과 유엔인권위원회(the Human Rights Committee of the United Nations)에 각각 제소하여 기각됨으로써 끝났는데,\footnote{CEDH, \emph{WACKENHEIM c. la FRANCE}, 16 octobre 1996, 29961/96. \emph{Manuel Wackenheim v France}, Communication No 854/1999, U.N. Doc. CCPR/C/75/D/854/1999 (2002), Selected Decisions under the Optional Protocol, Seventy-Fifth to Eighty-Fourth Sessions (July 2002-March 2005) (New York: United Nations Publications, 2007). 유럽인권협약(the European Convention on Human Rights, ECHR)은 명시적으로 인간의 존엄을 언급하지는 않고 있다. 배켄하임은 유럽인권협약의 신체의 자유와 안전에 대한 권리, 사생활 및 가족생활을 존중받을 권리, 차별금지 규정에 주로 의존하여 제소하였고 이는 1996년 10월에 기각되었다.} 배켄하임의 마지막 소송은 유엔인권위원회에 시민적 정치적 권리에 관한 국제규약(ICCPR)을 근거로 프랑스를 제소하는 것이었다. 배켄하임은 난쟁이 던지기의 금지가 자신의 직업 선택을 가로막음으로써 그의 존엄을 오히려 침해했다고 주장했다. 그러나 위원회는 규약 26조의 차별금지 규정을 고려하여 배켄하임에 대한 처분은 객관적이고 타당한 근거들 아래 이루어져 그 목적이 차별적이지 않으며, 이러한 처분이 공공질서를 보호하기 위해 필요했음을 프랑스가 입증했다고 판단했다. 이 때 공공질서는 규약의 목적에 부합하는 인간존엄에 대해 고려한 것을 말한다.\footnote{\emph{Manuel Wackenheim v France}, p.114.}

다만 꽁세유데따는 이러한 입장을 지속적으로 고수하지 않았다. 난쟁이던지기 사건에서 인간존엄을 공공질서(l\textquotesingle ordre public)의 한 구성요소로 취급할 수 있다고 보면서 소위 존엄주의적 견해를 취했던 꽁세유데따는, 2010년 ``공공장소에서 니캅(niqab)착용금지입법''에 대한 의견에서는 인간존엄의 규범은 ``결정의 자유를 희생하여 인간존엄을 보호하는 집단도덕적 요청의 원리, 그리고 인격과 동체적 측면인 자기결정의 보호의 원리''라는 두 모순적인 이해 위에 존재한다는 점을 확인하면서,\footnote{Conseil d'Etat. 2010. ``Etude sur les possibilités juridiques d'interdiction du port du voile integral'\,', www.conseil-etat.fr/cde/media/document/RAPPORT\%20ETUDES/etude\_voile\_integral\_anglais.p

  df(in English) (accessed 22 April 2013), 21-2면.} 어떤 금지의 기초로 삼는 것을 유보하는 태도를 취했다.\footnote{Ibid.} ``많은 경우 니캅의 착용은 자발적이기 때문에''\footnote{Ibid., 22면.} 인간존엄은 이런 금지의 기초로서 부적절하고 사실상 위험한 기반이라고 자문했다.

꽁세유데따가 여러 모순적인 가치들 사이에서 인간존엄을 기초로 판단을 내리는 것을 위험하다고 판단한 것은, 인간존엄의 가치가 사법에 의한 월권적 법형성의 근거로 사용될 수 있다는 것을 염두에 둔 것으로 보인다.

\subparagraph{이익형량의 고려사항}

인간존엄은 헌법적 권리들을 제한하는데 있어 한계를 설정하는 문제에 관여한다. 대부분의 헌법적 권리들은 상대적 권리이기 때문에 행사에 제한이 따를 수 있다. 하지만 이러한 제한은 비례적이어야 한다. 이 비례성을 구체화하는데 있어 인간존엄은 중요한 역할을 하게 된다. 바락은 이 비례성을 권리제한의 목적, 목적의 실현가능성, 목적실현의 한계이익이 제한된 권리, 권리행사의 해악, 해악의 실현가능성과 균형을 이루어야 한다는 관점에서 이해한다.\footnote{Aharon Barak, \emph{Human Dignity}, 112-3면.} 이러한 균형 내지는 형량에 있어 인간존엄이 하나의 요소로서 고려되어야 한다는 것이다. 다만, 바락에게 있어 인간존엄은 어디까지나 여러 헌법적 가치들 중 하나일 뿐, 이러한 형량의 고려대상인 유일한 가치이거나 항상 최상위의 가치인 것은 아니다.\footnote{Ibid., 106-7면.}

그러나 인간존엄은 서열상 다른 가치들의 우위에 있는 유일한 최우선의 헌법적 가치로서, 모든 해석이 전체 법체계의 관점에서 지향해야 할 최종적 목적이 된다고 이야기되기도 한다.\footnote{이에 대한 반례는, 남아프리카 공화국 헌법 제 39조에 대한 바락의 해석이다. 이 조항은 ``권리장전을 해석할 때 법원, 심판위원회 또는 포럼은 인간의 존엄, 평등 및 자유에 기초한 열린 민주 사회의 근간이 되는 가치들을 증진해야''한다고 규정함으로써 인간의 존엄, 평등 및 자유를 동등하게 서술하고 있고, 바락은 이러한 서술형식을 통해, 남아프리카 공화국에서는 인간존엄의 가치가 평등과 자유와 동급의 가치로서 설시되고, 유일한 최고의 가치로서 규정되고 있지 않다고 보고 있다. Aharon Barak, \emph{Human Dignity}, 253면.} 이는 개별 가치를 증진시키고자 하는 개별 규정의 모호성을 해결하는 목적의 수준에서 한 단계 더 나아가 가치들 간의 충돌 상황에서 규정들 간의 최종적 조정자가 되고, 심지어 인간존엄의 가치를 보호하는 규정 자체가 누락된 상황에서는 문제를 해결하는 기준을 새로 창조하기도 하는 규범창조자의 지위까지 얻게 될 가능성을 함축한다. 이러한 견지에서, 인간존엄은 단지 이익형량의 여러 고려요소들 중 하나가 아니라 최종적 형량의 판단기준으로 작동하는 것처럼 보인다.

\subparagraph{헌법에 규정한 사례 -- 남아프리카공화국헌법}

인간존엄의 가치로서의 역할 그리고 그 해석적 역할을 의식하고 조문체계를 구성한 것으로 보이는 헌법은 남아프리카공화국의 헌법이라고 할 수 있다. 남아프리카공화국헌법은 법규범으로서의 인간존엄을 가치와 권리로서 세부적으로 분석, 분류하여 체계화하고 있다. 이 헌법은 인간존엄을 먼저 국가적 가치와 권리장전의 지향으로 인식하고(제1조, 제7조), 이를 보호하기 위한 권리를 요청하며(제10조), 또한 기본적인 헌법상의 권리들을 제한하거나 해석함에 있어서 고려하여야 할 규범적 가치로 확인하고 있다(제36조, 제39조).

남아프리카공화국은 헌법 제1조에서 인간의 존엄을 해당국가의 가치로 선언하고 제7조에서 권리장전이 이 가치를 확신한다고 규정하고 있다. 또한 제10조에서는 인간의 존엄에 대한 권리를 부여하고, 제36조에서는 권리 제한의 기초로서 인간의 존엄을 이야기하고, 제39조에서는 권리해석에 있어서 인간존엄에 기초한 가치를 증진하여야 한다고 규정하고 있다. 먼저 인간의 존엄을 남아프리카공화국과 그 권리장전의 가치로 선언한 제1조와 제7조를 보자.

\begin{displayquote}
남아프리카공화국 헌법

제1장 기본 규정

제1조 (남아프리카공화국)

남아프리카공화국은 \ul{다음의 가치}에 기초한 단일 주권 민주주의 국가이다.

a. \ul{인간의 존엄}, 평등의 실현, 인권과 자유의 증진 실현

b. 인종 평등과 성 평등

c. 헌법의 최고성과 법치주의

d. 책임, 대응성 및 개방성을 보장하기 위한, 민주주의 정부의 성인 보통 선거권, 일반 국민 유권자 명단, 정기적 선거 및 복수 정당제

제2장 권리장전

제7조 (권리)

1. 본 권리장전은 남아프리카공화국 민주주의의 초석을 이룬다. 이는 우리나라 모든 국민의 권리를 소중히 간직하며 \ul{인간의 존엄}, 평등, 자유의 민주적 가치를 확신한다.

2. 국가는 권리장전에 포함된 권리를 존중, 보호, 증진 및 충족해야 한다.

3. 권리장전에 포함된 권리는 본 헌법 제36조 또는 권리장전의 기타 조항에 수록 또는 언급된 제한 사항에 따른다.
\end{displayquote}

이러한 헌법에 터잡아 남아프리카공화국의 많은 판결들은 인간의 존엄을 `최고의 가치'\footnote{\emph{S v. Makwanyane}, at para.111. Irma J. Kroeze, ``Human Dignity in Constitutional Law in South Africa'', in \emph{The Principle of Respect for Human Dignity} (European Commission for Democracy Through Law, Council of Europe Publishing, 1999); Nazeem M. I. Goolam, ``Human Dignity - Our Supreme Constitutional Value'' 4 \emph{Potchefstroom Electronic Law Journal} 1 (2001) 참조.}, `최고질서의 가치',\footnote{Author Chaskalson, ``Human Dignity as a Constitutional Value''; Nazeem M. I. Goolam, ``Human Dignity''.} `초석',\footnote{\emph{S v. Makwanyane}, at para.329; \emph{Khosa v. Minister of Social Development}, at para.114; \emph{De Reuck v. Director of Public Prosecutions}, at para. 61. \emph{Christian Education South Africa v. Minister of Education}, 2000(4) SA 757 (CC), para.36; \emph{Daniels v. Campbell}, 2004 (5) SA 331 (CC), para. 54; Nicholas Haysom, ``Dignity'', in Halton Cheadle, Dennis Davis and Nicholas Haysom (eds.), \emph{South African Constitutional Law: The Bill of Rights} (Durban: Butterworths, 2002) 참조.} `영감을 주는 정초하는 그리고 정초적인 가치',\footnote{\emph{Minister of Home Affairs v. Watchenuka}, at para.26.} `권리장전의 권리들을 관통하는 모티프'\footnote{\emph{National Coalition for Gay and Lesbian Equality v. Minister of Justice}, 1999 (1) SA 6 (CC), para. 120. Henk Botha, `Equality, Dignity and the Politics of Interpretation' \emph{South African Public Law} Vol. 19 (2004), 171면 참조.}등으로 부르고 있다.\footnote{Aharon Barak, \emph{Human Dignity}, 252-253면 참조.} 아론 바락은 이러한 남아프리카공화국헌법 제1조의 인간존엄의 규정방식과 그 사법적 이해에 대하여 이는 인간존엄을 헌법적 가치로 규정한 것이라고 이해한다.\footnote{Ibid., 252-253면.}

남아프리카공화국헌법보다는 불분명한 서술형식을 취하고 있지만 우리나라의 헌법 제10조\footnote{대한민국 헌법 제10조: 모든 국민은 인간으로서의 존엄과 가치를 가지며, 행복을 추구할 권리를 가진다. 국가는 개인이 가지는 불가침의 기본적 인권을 확인하고 이를 보장할 의무를 진다.}의 경우도 인간존엄을 헌법의 가치로 보는 근거로 해석할 수 있다는 입장이 있다. 예를 들어, 헌법재판연구원의 보고서는 제10조의 ``인간존엄 규정은 최고의 헌법가치이자 모든 국가기관을 구속하는 최고의 객관적 헌법규범으로서 국가행위의 방향을 결정하는 행위지침이 되는 동시에 인간존엄성을 실현해야 할 국가의 과제를 설정하는 의무규정이 된다''고 말한다.\footnote{헌법재판연구원(최규환), 인간존엄의 형량가능성 (2017), 26면.}

그런데 가치는 존재 그 자체만으로 어떤 구체적 명령이나 구속력을 전달하지 못한다. 가치는 명령 그 자체는 아니고, 명령이 달성하고자 하는 목표, 원래 의도, 혹은 배경에 해당하기 때문이다. 가치는 법규정을 매개로 다른 구체적 명령의 기능을 보조한다. 예를 들어 다음의 남아프리카공화국헌법처럼 권리를 제한할 때 기초적 개념이 되기도 하고(제36조), 권리장전을 해석할 때 증진해야 할 대상이 된다(제39조).

\begin{displayquote}
남아프리카공화국헌법 제36조 (권리의 제한)

1. 권리장전에 포함된 권리는, 그러한 권리의 제한이 다음을 포함한 모든 관련 요소를 고려할 때 인간의 존엄, 평등 및 자유에 기초한 열린 민주 사회에서 적절하고 정당한 범위에서, 일반적으로 적용되는 법률에 의해서만 제한할 수 있다.

a. 권리의 본질

b. 제한의 목적의 중요성

c. 제한의 본질과 범위

d. 제한과 목적의 관련성

e. 목적을 달성하기 위한 덜 제한적인 수단

2 전항 또는 본 헌법의 기타 조항에 명시된 경우에 의하지 않고는, 어떠한 법률도 권리장전에서 보호하는 권리를 제한할 수 없다.

제39조 (권리장전의 해석)

1. 권리장전을 \ul{해석할 때} 법원, 심판위원회 또는 포럼은

a. 인간의 존엄, 평등 및 자유에 기초한 열린 민주 사회의 근간이 되는 가치들을 증진해야 하며

b. 국제법을 고려해야 하고

c. 외국법을 고려할 수 있다.

2. 법률을 해석할 때 그리고 커먼로(common law) 혹은 관습법을 개발할 때, 모든 법원, 심판위원회 또는 포럼은 권리장전의 정신, 취지 및 목적을 증진해야 한다.

3. 권리장전은 해당 법이 권리장전에 부합하는 범위에서, 커먼로(common law), 관습법 또는 법률에 의해 인정 또는 부여된 그 외의 모든 권리 또는 자유의 존재를 부인하지 않는다.
\end{displayquote}

\subparagraph{논쟁점}

인간존엄 해석의 영향력이 가지는 중대성에도 불구하고, `법해석의 근거' 개념은 그 법규범적 위상이 논쟁적이다. 해석의 근거로서의 가치나 법목적은 그 자체로 입법권력이 제정한 개별적인 실정법 문언에 드러나지 않는 경우가 많고, 실정법의 내용이 모호할 때 그 확정이나 제한을 위한 보충적 개념으로 주로 이해되고 있다. 법해석작용은 개념상으로는 법형성 작용과 구별되는 것으로서 그 해석의 결과가 입법작용이 부여한 재량범위를 넘어서서는 안 된다는 태도가 어느 정도 확립되어 있기 때문이다. 그런데 ``모든 해석이 전체 법체계의 관점에서 지향해야 할 최종적 목적''으로 이해되는 인간존엄은 모든 종류의 해석을 관장하면서 모든 규범을 재해석하고 재배열할 가능성을 가지게 된다. 법형성보다는 법의 해석에 관여하는 것으로서 이해되어야 할 법의 해석근거로서의 인간존엄의 관념을 통해, 인간존엄 관념의 지지자들은 사법에 의한 월권적 법형성에 몰래 조력하고자 하는 것은 아닌지, 즉 인간존엄의 해석적 기능의 확장이 트로이목마가 되는 것은 아닌지에 대한 끊임없는 의심을 받게 만든다.\footnote{Marcus Düwell, ``Human dignity: concepts, discussions, philosophical perspectives'', 25, 29면, Aharon Barak, \emph{Human Dignity: The Constitutional Value and the Constitutional Right}, 8-9면 등 참조.}

특히 이러한 명령규범의 목적이나 해석근거로서의 인간존엄이 의도주의나 목적적 해석을 전제할 경우, 사실상 법명령규범 그 자체보다 더 강력한 효력을 가질 수도 있다. 이는 권리들을 제한하는 근거로 인간존엄이 이용될 때 더욱 분명해진다. 앞서 제시한 남아프리카공화국헌법 제36조에서 권리장전에서 포함한 권리들을 입법에 의해 제한할 때, 반드시 인간의 존엄에 기초할 것을 명령한다. 또한 남아프리카공화국헌법은 제39조에서 보듯, 권리장전을 해석할 때 해당 기관은 인간의 존엄을 증진하도록 해석해야 한다.

헌법상 가치로서, 기본권 제한 및 권리장전 해석에 있어서 인간존엄을 고려해야 한다고 주장하는 이러한 태도는, 인간존엄의 내용이 보다 구체화되고 이에 대한 정당성을 확보할 때에만, 월권적 법형성의 의심을 물리치고 그 설득력을 얻을 수 있을 것이다.

\paragraph{권리로서의 인간존엄}

한 개인이 타인이나 기관에게 적극적 혹은 소극적 의무를 부과할 힘의 행사권한을 규정하고 있는 주관적 권리규범은 전체 법규범 중 일부를 차지하고 있을 뿐이다. 하지만 인간존엄의 도덕적 요청의 중대성을 생각할 때, 국가의 자의적이고 우연적인 규범적 판단을 기다리지 않고 이해당사자인 개인이 직접 행사하고 처분할 수 있는 권능을 부여한다는 측면에서 인간존엄을 주관적 권리의 형식으로 보호하는 것은 현대법질서의 이상에 비추어 반드시 필요한 요청으로 이해되고 있다.

각 법질서가 보호하고자 하는 인간존엄의 가치를 주관적 권리규범으로 보호하는 경우에는 인간존엄의 보장이 좀 더 구체화되고 현실화된다. 이 경우 권리의 주체와 이에 대응하는 의무의 상대방, 그리고 그 권리의무의 내용을 확정할 것이 전제되기 때문이다. 또한 주관적 권리는 그 권리주체의 권리행사에 의해 그 효력의 발생이 좌우되기 때문에 주관적 권리당사자의 이익과 인간존엄에 기인한 권리행사가 이해상반의 문제를 거의 발생시키지 않는다. 설사 인간존엄이 담지하고 있는 객관적 가치의 내용이 당사자의 주관적 이익과 상충된다고 하더라도, 주관적 권리당사자는 자신의 권리를 행사하지 않으면 그만이기 때문이다. 예를 들어, 인간존엄의 가치가 개인의 생명을 중요시하여 극심한 고통을 호소하는 말기암 환자의 경우에도 연명의료행위의 중단을 허용하지 않는 존엄주의적 논거로 사용된다고 하더라도, 법규범의 영역에서 인간존엄의 가치를 선택하는 것이 오로지 개인에게 귀속된 주관적 권리의 문제라면, 그는 이러한 존엄에 근거한 권리를 행사하지 않고 단지 연명의료행위의 중단을 선택하면 그만이기 때문이다. 이렇듯 인간존엄을 그로부터 도출된 권리에 의해 보호한다는 사유는 인간존엄이 당사자에게 주관적인 어떤 가치를 증진시키는 것과 관련되는 한, 매우 자연스럽고 일관적이며 매력적으로 보이는 법적인 기술이라고 할 수 있다.

그런데 인간존엄의 보호에 관하여 문언상 `권리'의 형식으로 명시적으로 보호하고 있는 헌법은 많지 않다. 독일기본법도 인간존엄을 ``감히 건드릴 수 없다''고 선언하고 있을 뿐, `권리'의 용어를 사용하지 않아 인간존엄의 권리성에 관해 논쟁을 불러일으키고 있고, 이는 우리나라의 헌법도 마찬가지다. 대한민국 헌법 문언은 `권리'라는 표현을 `인간존엄'에 직접적으로 결부시키고 있지 않다. 헌법 제10조에서는 ``모든 국민은 인간의 존엄과 가치를 가지며{[}\ldots{]}''라고 규정하고 있다. 이 구절의 인간의 존엄이 단지 헌법이 지향하여야 할 목표나 가치에 해당하는 것인지, 아니면 더 나아가 개인의 주관적 권리의 측면을 함께 담고 있는 것인지에 대해서는 견해가 나뉘어 있다. 학설은 인간존엄이 그 자체로 주관적 권리라고 하거나,\footnote{김철수, 헌법학신론, 420-424면.} 인간존엄의 가치로부터 권리들이 도출될 수 있다고 보거나,\footnote{정종섭, 헌법학원론, 412면.} 제10조 ``조항''이 기본권을 설시한 조항이라고 간주하는 경우\footnote{한수웅, 헌법학 제6판(법문사, 2016), 525-526면.} 등 그 권리성을 어느 정도 범위에서 인정하고 있는 편이지만 그 이해방식은 매우 다양하다.

그러나 이미 여러차례 헌법재판소는 기본권에 대한 침해가능성을 전제로 하는 권리구제형 헌법소원 사건을 통해 인간존엄 침해여부의 판단을 함으로써, 인간존엄을 헌법적 권리로 인정하는 듯한 판결을 내어놓은 바 있다.\footnote{헌재 2011.8.30. 2006헌마788 ``인간의 존엄성은 `국가권력의 한계'로서 국가에 의한 침해로부터 보호받을 개인의 방어권일 뿐 아니라, {[}\ldots{]}'', 헌재 2003.12.18. 2001헌마163 ``이 사건 계구사용행위는 {[}\ldots{]} 인간의 존엄성을 침해한 것으로 판단된다'' 등 참조.} 그러나, 그간의 판단들은 다른 기본권의 침해에 동반하여 공권력 행위의 위헌성을 확인하거나 인간존엄의 침해를 부정한 사안이 대부분이어서, 인간존엄이 헌법재판소에 의해 독자적 기본권으로 확인된 것인지 불분명한 경우가 많았다. 그러나 헌법재판소는 구치소 내 과밀수용행위 위헌확인(2016. 12. 29. 2013헌마142)을 통하여, 헌법 제10조의 인간존엄 규정의 기본권성을 명확히 한 것으로 보지 않을 수 없는 판결을 내어놓았다. 이 판결 역시 인간존엄에 `권리'의 용어를 결부시키고 있지는 않다. 하지만 본 사안은 헌법 제68조 1항에 의거한 기본권 침해를 전제로 하는 권리구제형 헌법소원 사건으로서, 행복추구권, 인격권, 인간다운 생활을 할 권리의 침해 가능성에도 불구하고, 명시적으로 인간의 존엄과 가치의 침해를 다투는 것으로 포섭될 수 있다고 하였고, 해당 사안의 침해의 위헌성을 최종 확인함으로써 인간의 존엄과 가치의 침해에 대한 판단만으로도 권리구제형 헌법소원사건이 인용될 수 있음을 보여주었다.\footnote{헌재 2016. 12. 29. 2013헌마142. ``가. 제한되는 기본권: 이 사건 수용행위로 인하여 헌법 제10조에 의하여 보장되는 청구인의 인간의 존엄과 가치가 침해되었는지 여부가 문제된다. 청구인은 이 사건 수용행위로 인하여 행복추구권, 인격권 및 인간다운 생활을 할 권리를 침해받았다는 주장도 하고 있으나, 위 기본권들의 침해를 다투는 청구인의 주장은 모두 인간의 존엄과 가치의 침해를 다투는 청구인의 주장에 포섭될 수 있으므로 별도로 판단하지 아니한다.''}

남아프리카공화국 헌법의 경우에는 앞서 보았듯 가치로서의 인간존엄을 규정하고 있을 뿐만 아니라, 제10조에서 다음과 같이 인간의 존엄을 존중 및 보호받을 권리를 별도로 설시함으로써 이러한 논란을 종결시키고 있다.

\begin{displayquote}
남아프리카공화국 헌법 제10조 (인간의 존엄) 모든 사람은 고유한 존엄을 지니고 있으며 각자의 \ul{존엄을 존중 및 보호받을 권리}를 가진다.
\end{displayquote}

앞서 살펴본 우리 헌법재판소의 태도에도 불구하고 인간존엄의 주관적 권리성을 보다 실질적으로 인정하기 위해서는 여전히 해결해야 할 많은 문제가 남아 있다. 또한 남아프리카공화국 헌법과 같이 인간존엄의 권리 혹은 인간존엄을 보호할 권리를 명문으로 규정하는 경우에도 인간존엄의 권리 혹은 인간존엄을 보호할 권리가 정확히 어떤 구체적인 보호내용을 가지는지, 권리의 상대방은 어떤 의무를 지는 것인지, 그리고 다른 권리들과는 어떤 위계적 관계를 가지는지 분명하지 않다.

\subparagraph{헌법적 권리로서의 인간존엄을 인정하기}

헌법적 권리를 인정하는 것은 단지 헌법이 지향하는 가치를 인정하는 것보다 상대적으로 까다로운 조건을 만족시켜야 한다. 어떤 권리가 적어도 법적 권리로서 자격을 갖추기 위해서는 적어도 세 가지 요건을 갖출 것이 요구된다고 보여지기 때문이다. 먼저 권리는 그 권리의 상대방에게 어떤 의무를 부담할 것을 요청하는 것이기에 그 권리를 행사하는 주체가 분명하게 존재해야 한다. 둘째로, 권리는 구체적인 상대방이 의무를 부담한다고 인정할 수 있는 것이어야 한다. 셋째로, 권리는 그 내용이 어느 정도 특정되어야 한다. 특정되지 않는 내용에 대하여 상대방에게 의무를 부담시키는 것은 과도하고 때로 불가능하다.

바락은 인간존엄의 헌법적 권리를 헌법의 ``해석''이라는 방법에 의하여 확인하려고 시도한다. 남아프리카공화국헌법과 같은 예외가 존재하기는 하지만, 많은 국가의 헌법에서는 인간존엄의 권리의 존재에 대해 ``침묵(silence)''한다.\footnote{Aharon Barak, \emph{Human Dignity}, 144면.} 그러나 이러한 침묵이 반드시 그러한 권리의 결여나 공백(lacuna)을 의미하는 것은 아니다. 따라서 어떤 헌법적 권리의 인정이 정당한 사법활동이 되기 위해서는 침묵이 권리의 존재에 대해서 ``말하고 있는'' 것으로 생각될 수만 있다면,\footnote{Ibid.} 이러한 침묵은 권리의 공백이 아니라 권리의 존재를 표현하고 있는 것이다. 바락에 의하면 하나의 법체계가 인간존엄의 헌법적 권리를 묵시적으로 인정한다고 해석할 수 있을 때, 인간존엄의 헌법적 권리를 인정할 수 있다는 것이다.

인간존엄 권리의 존재를 인정하는 또다른 설명은, 인간존엄의 권리를 개인이 가지는 고유한 인권으로 이해하는 방법이다. 대한민국 헌법 제 10조 후단은 ``국가는 개인이 가지는 불가침의 기본적 인권을 확인하고 이를 보장할 의무를 진다''고 규정하고, 제37조 제1항은 ``국민의 자유와 권리는 헌법에 열거되지 아니한 이유로 경시되지 아니한다''고 하고 있다. 이러한 입장에서는 헌법의 문언과 맥락을 고려하는 내적 해석을 뛰어넘어, 권리의 인정작업은 인간의 고유한 성질을 탐구하는 작업으로 그 자리를 옮겨갈 수 있다.

해석의 방법에 의해 인간존엄의 헌법적 권리를 인정하든, 고유한 인간성을 관찰함으로써 인권으로써 도출하든, 인간존엄의 헌법적 권리가 단지 추상적이고 막연한 권리의 이상을 제시하는 것을 넘어서 구체적인 법적 권리로 인정되기 위해서는 앞서 제시한 권리주체, 권리상대방, 권리내용이라는 권리의 3요소가 구체적으로 인정될 수 있을 만한 것인지에 대한 검토가 필요하다.

\subparagraph{인간존엄 권리의 내용}

인간존엄의 헌법적 권리의 존재를 인정할 수 있다면, 더 나아가 인간존엄의 권리, 혹은 인간존엄을 존중 및 보호받을 권리는 구체적으로 무엇을 의미하는가? 일단 형식적으로 보면 이는 ``인간존엄을 침해당하거나 보호받지 못하는 경우 침해의 배제나 적극적인 보호를 요구할 수 있는 주관적 권리''\footnote{헌법재판연구원(최규환), 인간존엄의 형량가능성 (2017), 29면, 한수웅, 헌법학 제5판, (법문사, 2015), 521면.}를 말한다고 할 수 있다. 바락에 의하면 ``권리의 내용은 그 권리의 배경에 깔린 목적에 의해 결정''되기 때문에, 인간존엄의 헌법적 권리의 내용은 그 배경에 깔린 목적인 인간존엄의 헌법적 가치의 실현이라고 한다.\footnote{Aharon Barak, \emph{Human Dignity}, 144면.} 그렇다면 인간존엄의 헌법적 가치는 도대체 무엇인가?

인간존엄을 보호하는 권리의 내용을 확정하는 것은 이 권리가 다른 어떤 권리보다 우선시되고 중대한 것으로 이해되고 있기 때문에 특히 중요하다. 예를 들어 생명권을 보호받을 권리란 생명이라는 가치의 침해에 대한 배제나 생명에 대한 적극적인 보호를 요구할 수 있는 국가에 대한 청구권을 말한다고 할 수 있다. 그런데, 인간존엄에 대해서는 이러한 침해배제나 보호에 대한 기술이 다소 불명확하게 다가온다. 도대체 인간존엄에 대한 침해는 무엇이고 그에 대한 보호란 또 어떤 것을 의미하는가? 이를 위해 우리는 인간존엄의 내용을 분명하게 할 필요가 있다. 다음에서는 인간존엄의 권리, 혹은 인간존엄을 보호할 권리의 내용이나 의미를 제공하는 여러 견해들을 살펴보고자 한다.

형식적 권리 관점

인간존엄의 권리는 단지 ``상대방에게 자신을 존중할 것을 명령하는 권리'', 즉 형식적 권리라고 이해하는 입장이 있다. 이러한 권리에서 존중의 구체적 내용은 정해져 있지 않다. 안톤 파간은 이러한 관점의 뿌리를 칸트적 형식주의에서 찾고 있다.\footnote{Anton Fagan, ``Human Dignity in South African law'', \emph{The Cambridge Handbook of Human Dignity} (Cambridge, 2014), 401-406면.}

파간은 이러한 형식적 권리에는 난점이 있다고 주장한다. 인간의 존엄이 이러한 권리의 확정과 해명에 도움을 주지 않는다는 것이다. 안톤 파간은 가상적인 베이비시터 사례를 든다.\footnote{Ibid.} 어머니가 아들에게 베이비시터에게 순종하라고 지시한 후 베이비시터는 이 아들에게 아기에게 가서 코를 세게 치라고 명령한다고 가정해 보자. 어머니의 지시가 없으면 베이비시터의 명령은 아무런 의무도 부과하지 않았을 것이지만, 어머니의 지시가 존재한 덕분에 베이비시터의 명령이 아들에게 의무를 부과한다. 그러나 어머니의 지시는 의무의 내용을 결정할 때 어떤 역할도하지 않고 그 의무의 구체적 내용을 부과하는 것은 베이비시터의 명령이라는 것이다.

매우 심한 체계적이고 구조적인, 특정한 내용의 권리침해를 비난하는 권리

인간존엄을 보호하는 권리는 고문 금지, 굴욕 모욕 비하 금지, 수단화 금지 등 특정한 내용을 가진 권리라고 보는 관점이 있다. 뒤벨은 이를 체계적이고 구조적인 형태의 권리들의 침해를 비난하는 권리로 이해하고 있다.\footnote{Marcus Düwell, ``Human dignity: concepts, discussions, philosophical perspectives'', 28면.} 이러한 침해는 매우 심각한 권리침해로서 그것은 단지 특정한 권리에 대한 부정일 뿐만 아니라 인간 그 자체에 대한 경멸의 형태라는 것이다. 이는 나치의 잔악성에 대한 반성이나 뒤리히의 객체공식에서 드러나는 개인의 도구화, 비인격화 등에 대한 비판의 배경이 되는 존엄이라는 권리에 대한 이해라고 할 수 있다.

그러나 이러한 이해의 문제는 인간존엄의 침해가 다른 근본적 권리들의 하위 범주로 생각되도록 만들 수 있다. 인간존엄으로부터 권리가 도출된다는, 인간존엄이 인권보다 보다 근본적이고 포괄적이라는 생각을 뒷받침하지 못한다.

권리를 가질 권리, 법질서에 편입될 권리

한나 아렌트는 국적이 없을 때, 한 인격이 가지는 권리는 그것이 양도불가능하다고 하더라도 거의 무의미해진다고 주장한다.\footnote{Hannah Arendt, \emph{The Origins of Totalitarianism} (New York, Harcourt, Brace, Jovanovich, 1973), 268면. ``인간의 권리들은\ldots{} `양도불가능하다'고 정의되어왔다. 왜냐하면 이들은 모든 정부들에 독립적이라고 가정되기 때문이다; 하지만 인간이 그들 자신의 정부를 잃고 그들의 최소한의 권리에 의지해야만 하는 순간, 어떤 권위도 그들을 보호하기 위해 남아있지 않고, 어떤 기관도 그들을 보장하려고 하지 않는다는 것이 판명된다.''} 국가 없는 인격은 일종의 법적 림보(중간계), 지속적으로 법을 위반해야만 하는 상태, 사실상 법의 지배보다는 필연적으로 경찰력에 의해 직접 지배당함의 상태에 있기 때문이다.\footnote{Ibid., 287-8면.} 인권은 오직 국가의 시민들인 사람들을 위해서만 작동해왔다.\footnote{Jeremy Waldron, ``Citizenship and Dignity'', 338면.}

여러나라의 다양한 시민의 권리들은 유형적인 법의 형태로 인간의 영원한 권리들을 구체화하고 설명하려고 했고, 이러한 영원한 권리들은 스스로 시민권과 국적과 독립되려고 해 왔다. 모든 인간들은 어떤 종류의 정치공동체의 시민들인데, 만약 그들의 국가의 법들이 인간의 권리들의 요구에 부응하지 않으면, 그들은 그들을 민주적 국가의 입법이나 독재정에서의 혁명적 행동을 통해 변경하려 시도할 수도 있다.\footnote{Hannah Arendt, \emph{The Origins of Totalitarianism}, 293면.} 이는 세계질서 안에서 시민권 없이도 인권만으로 살아갈 수 있을 것 같은 희망을 심어주기도 한다. 하지만 만약 누군가가 어떤 공동체의 시민이 아니라면, 그가 자신의 인권을 주장할 법적 구조 자체가 없어진다. 국가적 권리들의 상실은 인권의 상실과 동등하기 때문에, 가장 중요한 권리는 권리를 가질 권리라는 것, 한 사람의 권리의 청구가 중요시되는 조직된 공동체의 구성원일 권리하는 것이 금세 명백해진다. \footnote{Jeremy Waldron, ``Citizenship and Dignity'', 338면.}

이와 같은 한나 아렌트의 권리를 가질 권리의 필요성을 토대로 하여 인간존엄을 보호하는 권리를 이해하려는 시도들이 다수 존재한다. 크리스토프 엔더스의 `권리를 가질 권리'나 제러미 월드론의 `시민권과 존엄'에 대한 논의가 이에 해당하는 사례들이다.\footnote{Christoph Enders, \emph{Die Menschenwürde in der Veriassunesordnung Zur Dogmatik des Art. 1 GG} (Tübingen: Mohr Siebeck, 1997). Christoph Enders, ``A Right to Have Rights - The German Constitutional Concept of Human Dignity'', \emph{NUJS L. Rev.} vol.3 (2010), Jeremy Waldron, ``Citizenship and Dignity''.} 이런 류의 사상에서, 인간의 존엄은 ``각 인간은 그에게 시민으로서의 권리들을 부여하는 정치질서의 구성원이어야한다''는 (합)법성(legitimate) 주장, 즉 법질서에 편입될 권리가 된다. \footnote{Marcus Düwell, ``Human dignity: concepts, discussions, philosophical perspectives'', 29면.}

이러한 권리 설명의 문제점은 인간의 존엄이 그 권리의 내용을 결정하거나 새로운 권리를 창출하라는 현대적 요청을 수행하지 못한다는 것이다. 또한 이는 현대적 의미에서 어떤 권리라기보다는 권리보유자의 어떤 지위나 자격을 의미하는 것일 수 있다.

포괄적 권리

인간존엄의 권리가 기존의 다른 권리와는 다른 특정한 권리내용을 가지는 특수한 권리로 이해하는 앞의 세 견해와는 달리, 인간존엄이 모든 기본권들의 총체적 목적을 대변하는 포괄적 가치이므로, 인간존엄의 권리는 모든 포괄적 가치를 보호하는 포괄적 권리라는 견해가 있다. 즉 인간존엄에 대한 보호를 요청하는 권리는 다른 권리들을 도출하는 소위 ``포괄적 기본권''이라는 것이다. 뒤벨에 의하면 이는 다른 권리보다 ``덜 구체적인 권리''이고 다른 인권에 의해 보호되지 않는 인간들에 대한 비존중의 형태들을 보호하는 ``컨테이너 개념''으로서 ``인권 시스템의 불완전성을 보완하는 최후의 수단''이라고 한다.\footnote{Ibid., 28면.}

아론 바락의 권리로서의 인간존엄의 개념도 이와 유사하다. 그는 인간존엄이 틀-권리(framework right) 혹은 어머니-권리(mother-right)의 역할을 한다고 주장한다.\footnote{Aharon Barak, \emph{Human Dignity}, 156-169면 참조.} 그에 의하면 권리의 범위는 그 권리의 배경적 목적에 따라 결정되는데, 인간존엄 권리의 배경적 목적인 인간존엄, 즉 인간으로서 한 인격의 인간성이라는 헌법적 가치의 충족이다. 따라서 인간존엄의 헌법적 권리의 범위는 인간존엄의 헌법적 가치의 범위와 ``동일하다''.\footnote{다만, 아론 바락은 인간존엄의 권리를 절대적인 것으로 간주하는 독일기본법과 같은 각국의 특유한 헌법구조가 이러한 ``동일성'' 귀결을 방해할 수 있으며, 인간존엄 권리의 범위가 인간존엄 가치의 범위보다 좁다는 결론으로 이끌 수 있다고 말하고 있다. 인간존엄의 권리가 절대적이고 영원한 특수사례에서는 객체공식과 같이 그 권리가 좁고 제한적이며, 따라서 그 권리의 범위가 인간의 인간성의 모든 측면으로 확대될 수 없다고 말한다. 이 경우 역사적 사회적 배경을 고려하는 외적 해석적 맥락보다는 헌법의 아키텍쳐를 고려하는 내적 해석적 맥락이 이러한 결론으로 유도한다는 것이다. Ibid., 111-112면.} 김철수 교수도 유사한 견해를 취하고 있다. 그에 의하면 인간존엄은 다른 모든 세분화되어 파생된 기본권들을 아우르는 ``포괄적 주기본권''이라는 것이다.\footnote{김철수, 헌법학신론, 320-321, 423면.}

그렇다면 인간존엄의 권리만이 가지는 배타적인 고유한 영역으로 보였던 경우들, 예를 들어 매우 심한 체계적이고 구조적인 특정한 내용의 권리침해를 비난하는 권리로서의 인간존엄의 권리는 어떻게 해명되는가? 이는 다음과 같이 해명된다. 먼저 인간존엄을 포괄적 기본권으로 이해하는 경우 발생하는 고유한 문제는, 포괄적 기본권인 인간존엄을 보호할 권리와 개별기본권들이 중첩된다는 점이다. 아론 바락은 이를 어머니-권리와 딸-권리(daughter-right)와의 관계로 표현한다. 딸-권리는 어머니-권리에서 도출되고 그 목적을 충족한다. 그렇다면 ``인간존엄 권리의 배타적인 고유한 규범적 영역이 존재하는가?''에 대한 아마도 가장 일반적인 대답은 ``그 배타적 영역은 인간존엄의 헌법적 권리에 포함되지만 다른 헌법적 권리의 범위에 속하지 않는 영역에만 적용된다''는 것이다.\footnote{Aharon Barak, ``Human Dignity: The Constitutional Value and the Constitutional Right'', \emph{Understanding Human Dignity} (Oxford, 2014), 371면.}

인간존엄 혹은 인간존엄을 보호할 권리를 포괄적 권리로 이해할 경우 생기는 문제는 인간존엄의 가치의 범위를 넓히는 경우 인정해야 할 권리의 범위가 지나치게 넓어진다는 것이다. 어떤 가치를 권리로서 보호하게 되면, 국가와 사회의 다른 구성원들고 같은 그 권리에 대한 의무의 수범자들이 져야될 책임의 범위가 증가하게 된다. 앞서 가치로서의 인간존엄이 사법에 의한 월권적 법형성의 근거가 될 수 있다는 점을 설명한 바 있는데, 사법적 해석에 의해 인간존엄의 보충적 권리의 범위가 확대될 가능성이 커지게 되면 이러한 위험성은 더욱 증가하게 된다. 따라서 인간존엄을 보호할 권리의 내용은 보다 분명하게 정의되고 한정되어야 한다.

\subsubsection{효력론: 형량불가능성을 가진 인간존엄}

우리 헌법재판소 판결들을 살펴보면, 인간존엄의 제한이 인정된다 하더라도 비례원칙을 통해 인간존엄을 다른 가치나 권리들과 비교형량한 후에서야 인간존엄의 침해를 인정하는 설시를 찾아볼 수 없다. 애초에 인간존엄의 제한 자체를 부인하거나, 그 제한을 인정하는 경우에는 반드시 해당 사안을 헌법을 위반하는 인간존엄의 침해로 선언한다. 이는 독일기본법과 그에 대한 일반적 해석론, 그리고 이에 따르고 있는 독일연방헌법재판소 판결과도 유사한 입장이다.

인간존엄에 이러한 특별한 지위를 인정하는 것은 인간존엄의 가치가 특별히 중요하기에 어떠한 경우에도 절대 침해되어서는 안 된다는 확신에 근거하고 있는 것으로 보인다. 인간존엄의 형량불가능성을 주장하는 측에서는, 예를 들어 무고한 어린아이를 납치한 납치범을 설사 고문을 통해 아이를 구출할 수 있다고 하더라도 고문은 인간존엄의 침해로서 받아들일 수 없다는 입장을 취하게 된다. 이는 구출할 수 있는 사람의 수가 무한히 늘어난다고 해도 마찬가지다. 이러한 확신은 특히 독일에서 제2차 세계대전 당시의 홀로코스트의 경험과 반성으로부터 더 강화되었다.

그러나 인간존엄을 모든 국가에서 반드시 절대권으로만 규정하고 있는 것은 아닌 것으로 보인다. 바락은 앞에서 살펴 본 남아프리카공화국헌법 제10조는 인간존엄을 상대적 권리로 규정하고 있는 대표적 사례라고 주장한다.\footnote{Aharon Barak, \emph{Human Dignity}, 244면.} 이에 대한 근거로 기본권에 대한 일반적 법률유보조항인 제36조에 인간존엄의 권리가 배제된다는 명시적 규정이 없다는 점을 들고 있다.\footnote{Aharon Barak, \emph{Human Dignity}, 245면. 남아프리카공화국헌법 제36조 (권리의 제한) (1) 권리장전에 포함된 권리는, 그러한 권리의 제한이 다음을 포함한 모든 관련 요소를 고려할 때 인간의 존엄성, 평등 및 자유에 기초한 열린 민주 사회에서 적절하고 정당한 범위에서, 일반적으로 적용되는 법률에 의해서만 제한할 수 있다. a. 권리의 본질, b. 제한의 목적의 중요성, c. 제한의 본질과 범위, d. 제한과 목적의 관련성, e. 목적을 달성하기 위한 덜 제한적인 수단} 그의 견해는 인간존엄의 권리의 절대성 여부는 인간존엄 권리의 내재적 속성에 따라 판단하기보다는 헌법 문언의 규정태도와 그 해석에 달려있는 것으로 보는 입장이라고 할 수 있다.

그러나 여전히 인간존엄을 그 권리의 내재적 속성에 따라 절대적 권리, 형량불가능한 권리로 이해할 수 있는지에 대한 논의의 여지는 남아있다. 대한민국 헌법은 독일기본법과 달리 인간존엄의 헌법적 권리의 존재와 그 절대적 성질에 대한 명시적 표현이 없다고 볼 수 있다. 그럼에도 불구하고 헌법재판소는 인간존엄의 형량불가능성을 인정하는 듯한 태도를 보이고 있기에 인간존엄의 권리의 내재적 속성이 그 효력에 있어서 형량불가능성을 담지하고 있다고 볼 수 있는지 고민해 보는 것은 주요한 과제 중 하나라고 할 수 있다.

다음에서는 먼저 인간존엄의 형량불가능성을 말하고 있는 것으로 간주되고 있는 독일기본법과 그 판결의 태도를 살펴보고자 한다.

\paragraph{독일기본법: ``인간존엄은 감히 건드릴 수 없다''}

독일기본법은 다음과 같이 인간존엄을 제1조 제1항에서 감히 건드릴 수 없는 것으로 설시하고 있으며, 이 조항을 토대로 많은 학설들이 인간존엄은 기본권 규정 전체, 나아가 기본법 전체를 이끄는 최고의 가치 혹은 권리로 이해된다고 주장하고 있다.

독일기본법(1949)

제1조 (인간존엄의 보호)(1) 인간존엄은 감히 건드릴 수 없다(`unantastbar'). 이를 존중하고 보호하는 것은 모든 국가권력의 의무이다.

독일기본법에서 인간존엄이 이렇게 특별한 위상을 지니는 것은 제2차 세계대전을 전후한 나치시대의 잔악성을 다시는 반복하지 않기 위한 반성적 고려 때문이라는 것은 잘 알려져 있다. 그렇다면 독일기본법상에서 인간존엄 개념의 규범적 성질은 어떠한가? 이에 대하여 독일헌법이론에서는 상당한 논의가 진전되어 있다.

\subparagraph{\texorpdfstring{a. 독일기본법 질서에서 인간존엄의 특별한 지위 }{a. 독일기본법 질서에서 인간존엄의 특별한 지위 }}

일반적으로 권리는 다른 혹은 타인의 권리와 충돌할 수 있기 때문에 제한되어야 하며, 이 때 어느 경우에 권리제한이 가능한 지를 결정하는 것은 중요하다. 그런데 독일에서 인간존엄은 유일하게 기본법 제1조 1항의 규정형식상 제한이 불가능한 것으로 해석되기 때문에 문제가 생긴다. 인간존엄 규정을 인간존엄의 권리를 인정하는 것으로 해석하는 경우, 인간존엄의 절대적 권리는 상대적 권리와 달리 제한이나 형량의 대상이 되지 않게 된다. 이 권리는 일반적으로 상대적 권리가 침해되었을 때 이 제한이 타당한지 여부를 검토하는 방식인 이단계 접근방식---일단 침해를 받았는지 확인한 후 비례원칙에 따라 이러한 침해가 정당한지 검토하는 방식---을 따르지 않으며, 모든 침해는 바로 위반에 해당하게 된다.

이러한 절대성은 인간존엄을 보호하는 권리의 범위를 좁히기 위한 다양한 시도로 이어져 왔다. 만약 절대적 권리의 범위가 넓으면 대개의 권리가 정당하게 침해될 수 있다는 사실을 설명하기 어렵다는 점, 그리고 존엄에 기댄 정책들이 정치적 논의를 중단시킨다는 점 때문이다. 따라서 실제로 독일의 연방헌법재판소는 헌법 제1조 위반을 매우 드물게만 인정해 왔다.\footnote{예를 들어, BVerfGE 39, 1(제1차 낙태판결), BVerfGE 88, 203(제 2차 낙태판결), BVerfG, 1BvR357/05(항공안전법).}

하지만 이러한 드문 침해인정은 인간존엄 권리의 직접적 효력에 대한 것이고, 인간존엄이 헌법적 가치로서 다른 권리와 결합하여 이들을 해석 적용할 때 사용되는 경우 이들 권리의 범위를 확장하거나 존엄의 영향을 깊게 만드는 광범위한 역할을 하고 있기 때문에, 그 실천적 영향이 결코 적은 것이 아니다. 특히 모든 권리는 존엄핵심(dignity core)를 가지기 때문에 권리제한 등이 존엄핵심에 가까워질수록 형량과정에서 해당 권리의 비중이 더 높아지게 된다고 생각되고 있다.

\subparagraph{\texorpdfstring{b. 판결 }{b. 판결 }}

독일의 1, 2차 낙태판결에서 공히 독일연방헌법재판소는 태아의 생명은 임신기간 내내 보호되어야 하며 이는 생명이 인간존엄의 중대한 기초임을 근거로 임부의 자기결정권보다 우선한다는 입장을 고수한다.\footnote{BVerfGE 39, 1(제1차 낙태판결), BVerfGE 88, 203(제 2차 낙태판결).} 낙태가 위법하지 않은 것으로 허용되는 경우는 오로지 임부의 생명이나 건강이 위협받는 경우와 같이 임신의 지속이 기대가능하지 않은 경우 뿐이다. 법적 규제는 반드시 형사처벌이어야 할 필요는 없으나 실질적으로 태아의 생명보호에 적합한 방법으로 이루어져야 하고 이 사안의 경우 형사처벌이 불가피하다는 것을 설시하고 있다. 즉, 생명은 인간존엄의 핵심적 기초이며 따라서 인간존엄에 근거한 생명권은 다른 어떤 권리나 이익과도 형량할 수 없다는 것이다.

이와 유사한 태도가 독일항공안전법 제14조 제3항에 관한 독일연방헌법재판소 판결(BVerfG, 1BvR357/05)에서도 드러난다. 미국의 2001년 911테러 이후, 테러에 대한 공포는 전세계적으로 확산되었고, 이는 독일도 예외는 아니었다. 이러한 사태에 대비한 조치로 독일은 2005년 항공안전사무규율에관한법률(이하 항공안전법)을 제정하여 공포했다.\footnote{Gesetz zur Neuregelung von Luftsicherheitsaufgaben(LuftSiG) vom 11. 1. 2005(BGBl I, 78).} 특히 이 법 14조 3항은 다음과 같이 납치된 항공기에 대한 무기 사용을 허용했고, 이는 항공기에 대한 합법적인 격추가 이루어지지 않더라도 납치된 무고한 승객과 승무원들의 사망의 결과가 예견되는 경우에 한한다 할지라도, 어찌되었든 무고한 승객과 승무원에 대한 적극적 생명의 침해가 수반되는 것을 허용하는 것이었다.\footnote{정문식 , ``독일항공안전법 제14조 제3항에 관한 독일연방헌법재판소 판결(BVerfG, 1BvR357/05)에 대한 분석과 평가'', 법학논총 26 (2006).이하 판례와 법조문은 이 논문의 번역 인용.}

{[}14조 3항{]} 상황에 근거하여 항공기가 사람들의 생명을 해칠 목적이라고 판단할 수 있고, 무기사용이 현재 위험을 방지할 수 있는 유일한 수단인 경우에만, 직접적인 무기사용이 허용된다.

이 법률에 대한 논란은 제정과정에서부터 시작되었고, 결국 공포 후에 항공기 이용자들이 헌법소원심판을 제기하였다. 그 개략적인 근거는 ``동 법률이 국가로 하여금 의도적으로 사람들을 죽이고, 범죄행위의 희생자들을 양산하는 것을 허용하기 때문에, 자신들의 인간존엄과 생명권을 직접 침해하고 이로써 기본권의 본질적 내용을 침해한다'' \footnote{위의 논문, 78면.}는 것이었다. 연방헌법재판소는 이에 대해 항공안전법 제14조 제3항을 위헌무효로 선언했다. 주문과 요지는 다음과 같다.

주문:

1. 2005년 1월 11일자 독일항공안전법(연방법률공보 I 78면) 제14조 제3항은 헌법 제87a조 제2항, 제35조 제2항 제3항 및 제1조 제1항 등과 관련하여 제2조 제2항과 합치하지 않고 무효다.

2. 독일연방공화국은 헌법소원청구인들에게 필요한 비용을 지급해야 한다.

판결요지:

1. 연방은 직접 헌법 제35조 제2항 제2문과 제3항 제1문으로부터, 자연재해와 특히 중대한 사고(besonders schwerer Unglücksfall)에 대처할 경우 본 조항에 따라 관계 州들과 협력에 관한 상세한 내용을 규정할 입법권한을 갖는다. 특히 중대한 사고의 개념은 재해발생이 거의 확실하게(mit an Sicherheit grenzender Wahrscheinlichkeit) 예상되는 과정도 포함한다.

2. 헌법 제35조 제2항 제2문과 제3항 제1문은 연방에게 자연재해와 특히 중대한 사고의 대처에 전문적인 군사무기를 보유한 군대를 투입할 것을 허용하지 않는다.

3. 항공안전법 제14조 제3항에 따라 생명을 해칠 목적의 항공기에 대해 직접적으로 무기를 사용할 수 있도록 군대에 권한을 위임한 것은, 범행에 참여하지 않은 항공기내 탑승자들이 관계되는 한, \ul{헌법 제1조 제1항의 인간존엄보장과 관련된} 제2조 제2항 제1문의 생명권과 합치하지 않는다.

이 판결은 범행에 참여하지 않은 항공기내 탑승자에 대해 무기를 사용하는 것이 인간존엄과 관계된 헌법 제1조1항을 침해하는 근거를 다음과 같이 설시하고 있다.

이는 그들을 범죄가담자들의 대상으로 만들 뿐만 아니라, (\ldots) 국가가 그들을 타인 보호를 위한 구조활동의 단순한 대상으로 다루는 것이다. \ldots{} 그들이 통제할 수 없는 상황 때문에, 승무원과 승객들은 상황에서 벗어날 수 없고 무기력하고 방어할 수 없으며, 항공기와 함께 과녁이 되어 격추되어 거의 확실히 죽을 것이다. 이러한 취급은 존엄과 양도불가능한 권리를 부여받은 주체로서의 인격들의 지위를 무시하는 것이다. 타인을 구하기 위한 수단으로서 죽음으로써, 그들은 대상으로 다루어짐과 동시에 그들의 권리를 박탈당한다. (후략)\footnote{http://www.bundesverfassungsgericht.de/SharedDocs/Entscheidungen/DE/2006/02/rs20060

  215\_1bvr035705.html, 124단락. BVerfG, 1BvR357/05.}

이 판결은 전형적으로 뒤리히의 객체공식을 염두에 두고 있다. 무고한 승객들이, 항공기에 대한 합법적인 격추가 이루어지지 않더라도 납치된 무고한 승객과 승무원들의 사망의 결과가 예견되는 경우라 하더라도, 항공기 격추명령에 의해 인간존엄과 인권을 침해당한다고 판단하는 근거는 그들이 타인 보호의 단순한 수단으로 전락하였기 때문이라는 것이다. 이 사안의 생명권은 절대적 규범인 인간존엄침해금지와 관련되어 있고, 따라서 이러한 격추명령을 가능하게 하는 법률은 다른 가치나 권리와의 형량을 요구하지 않으며 위헌이라는 것이다.

\paragraph{우리나라 판례의 태도}

헌법재판소는 2016. 12. 29. 2013헌마142 사건에서 과밀수용행위가 ``인간으로서의 존엄과 가치를 침해하여 헌법에 위반된다''고 설시하면서, 비례원칙을 적용하거나 다른 가치들과 형량하려는 시도를 하지 않고 있다. 이에 따르면, ``어떠한 경우에도'' 인간존엄을 훼손할 수 없다.

{[}\ldots{]} 특히 수형자의 경우 형벌의 집행을 위하여 교정시설에 격리된 채 강제적인 공동생활을 하게 되는바, 그 과정에서 구금의 목적 달성을 위하여 \ul{필요최소한의 범위 내에서는 수형자의 기본권에 대한 제한이 불가피하다 하더라도}, 국가는 인간의 존엄과 가치에서 비롯되는 위와 같은 국가형벌권 행사의 한계를 준수하여야 하고, \ul{어떠한 경우에도 수형자가 인간으로서 가지는 존엄과 가치를 훼손할 수 없다}. {[}\ldots{]} 따라서 청구인이 인간으로서의 최소한의 품위를 유지할 수 없을 정도로 과밀한 공간에서 이루어진 이 사건 수용행위는 청구인의 인간으로서의 존엄과 가치를 침해하여 헌법에 위반된다.

이외의 국내 판결 대부분이 인간존엄의 침해를 인정하면서도 비례원칙이나 형량을 통해 인간존엄의 제한이 가능하다고 판단하는 경우는 찾아볼 수 없다. 이러한 태도로부터 우리나라의 사법적 판단의 경향은 인간존엄은 형량불가능한 것으로 인정하고 있다고 볼 수 있다.

\paragraph{형량불가능성이 불러일으키는 난제}

이와 같이 인간존엄의 형량불가능성을 인정하게 되면, 일반적으로 권리의 제한이나 권리들간의 충돌에서 수행되는 비례성 심사를 할 수 없게 된다. 테러범이나 아동납치범에 대한 고문이나, 죽을 것이 확실한 인질이 탑승하고 있는 비행기를 요격하는 것은, 인간을 수단으로 사용하지 말라는 뒤리히의 객체공식의 함축에 따라 절대적으로 금지되는 것이 정당화될 수도 있다.

그러나 때로 인간존엄과 관련되었다고 생각되는 일부 가치나 권리들은 비교가능하고 상대적인 가치만을 가진 것처럼 보이기도 한다. 사회에서 완전히 소외되고 핍박받고 있는 한 난쟁이가 있으며 그가 할 수 있는 직업은 오직 난쟁이 던지기 경기의 오락물이 되는 것 이외에 변변한 직업이 없을 때, 단지 그의 품위가 인간존엄과 밀접하다는 이유로 단순히 해당 직업을 박탈하는 것이 옳다고 볼 수 있을까? 이러한 명예는 매우 소중한 것이지만, 먹고 살아야 하는데 필요한 기본적인 직업선택의 자유를 제한하면서까지 절대적인 것으로 생각되기는 어려울 것이다.

인간존엄의 절대적 보호가 사소한 이익을 보호하는 것이 되지 않기 위해서는, 인간존엄의 개념범위나 그 법적 가치로서의 구체적 역할과 위상을 보다 명확히 할 필요가 있다. 또한, 인간존엄을 둘러싼 권리들에 있어서도 비례원칙의 적용과 형량가능성을 인정하는 방향을 모색하지 않을 수 없다. 본고는 제3부에서 이러한 난제를 해결하기 위한 이원적 해결방법을 제안할 것이다.

지금까지 헌법담론에서의 인간존엄을 기능적 차원과 효력 차원에서 살펴보았다. 인간존엄은 법적인 가치 또는 권리로서 기능하고 있으며, 많은 국가들에서 형량이 불가능하고 절대적인 효력을 가지는 것으로 해석되고 있다. 그런데 헌법적 차원에서 인간존엄이 이러한 기능과 효력을 가지는 것으로 파악하기 위해, 학자들은 이에 합당한 인간존엄에 대한 각자의 개념적 이해를 전제하지 않을 수 없다. 대표적인 것은 인간존엄을 1) 법적으로 보호가치 있는 특유한 인간성으로 파악하는 것이다. 그리고 최근 대안적으로 등장하고 있는 다른 하나는 2) 시민들이 가지는 권리들을 보유하는 시민적 지위로 인간존엄을 파악하는 것이다. 각각의 주장은 인간존엄의 위와 같은 기능과 효력을 저마다의 방식으로 설명한다. 다음에서는 인간존엄의 개념에 대한 헌법 담론에서의 대표적 설명인 `법적으로 보호가치 있는 특유한 인간성'으로의 이해방식을 살펴보고, 이에 덧붙여 최근 등장한 대안적 설명인 `시민적 지위'로서의 이해방식을 살펴보려고 한다.

\subsubsection{개념론: `법적으로 보호가치 있는 특유한 인간성'으로서의 인간존엄}

\paragraph{철학적 배경: ``타인으로부터 존중을 유발하는 인간의 가치 속성''}

앞서 본고는 ``타인으로부터 존중을 유발하는 인간 안에 있는 절대적 내적 가치''로 인간존엄을 이해하는 일련의 흐름이 존재한다는 것을 살펴보았다. 이러한 태도에 의하면 인간존엄은 일종의 ``무조건적이고 비교불가능한 가치''로서 인간존엄을 가진 개별자들은 ``특별히 가치 있고 존중받을 만하다''는 것이 따라나온다고 생각한다.\footnote{Robert Streiffer, ``Human/Non-Human Chimeras''.} 사람이 자신 혹은 타인을 매우 특별한 방식으로 존중하거나 다른 가치에 대한 존중에 비해 우월성을 가지고 존중하기 위한 어떤 근거를 발견하기 위해, 다시 말해 타인 혹은 자기 존중을 위한 특별한 요청을 설명하기 위해, 인간존엄을 인간이 지닌 어떤 가치속성으로 바라보는 것은 직관적이면서 명료한 설명인 것처럼 보인다.

이러한 가치속성으로서의 인간존엄은 앞에서 설명한 인간존엄이 법에서 수행하는 역할을 가장 쉽게 설명해 주는 것처럼 보인다. 특정한 권리나 법률은 이러한 가치속성을 증진시키기 위한 것으로서 이들을 해석할 때 이 가치속성이 해당 규범 해석의 근거가 된다. 또한 이러한 가치를 증진시킬 수 있는 제도가 미비할 때, 해당 가치를 증진시킬 수 있는 권리들을 정초하는 역할을 할 수 있다는 것이다. 이러한 장점 때문에 많은 이들은 인간존엄을 인간이 보유하는 어떤 가치속성으로 전제하고 있는 것으로 보인다.

\paragraph{`법적으로 보호가치 있는 특유한 인간성'의 후보자들}

그렇다면 이러한 가치에는 어떠한 것이 있을 수 있는가? 최소한의 생계보장, 평등한 대우, 품위, 생명의 고귀함, 인간종이 가지고 있는 정체성, 자율적 선택을 하면서 살아갈 능력과 같은 것이 그 대표적인 후보라고 할 수 있을 것이다. 여기서는 인간존엄의 내용적 측면을 중심으로 두 부류로 나누어, 높은 수준의 인간성으로서의 인간존엄과 그 후견주의적 함축과 기능, 그리고 자유의 기초로서의 인간존엄과 그 자유주의적 함축과 기능을 각각 살펴보려고 한다. 이 두 기능은 한편으로는 인간존엄의 중대한 두 가지 의미를 잘 전달하고 있지만, 다른 한편으로는 그 근원으로부터 서로 날카롭게 대립하고 있다는 점에서 문제가 된다.

세계 각국의 법질서에서 받아들여지고 있는 인간존엄 이해의 한 줄기는 인간이 성취해야하거나 이미 가지고 있는 높은 수준의 인간성이라는 것이다. 구체적인 법질서에의 편입은 이를 성취하는 것을 방해하거나 이를 존중하지 않는 행위를 규제하는 방식으로 나타나고 있다.

다만 이러한 인간성이 무엇인지에 대해서는 명백하게 확립된 견해가 없다. 다만 개략적으로 유사 사례들을 분류해 봄으로써 그 의미를 추정해 볼 수 있다. 대표적으로, 인간답게 살 수 있는 최소한의 생계를 보장할 의무를 국가에게 부과한 사례, 다른 사람들과 동등하게 대우받을 평등의 권리와 이념을 도출하는 사례, 사람이 갖추어야 할 어떤 품위에 해당하는 것을 침해하거나 스스로 품격을 낮추는 것을 금지한 사례, 의사조력자살, 낙태 등 생명을 단절시키는 행위를 금지하는 생명의 소중함 혹은 신성성을 도출하는 사례, 그리고 생명공학이나 의료기술을 통한 향상이나 DNA조작 등 인간종이 일반적으로 유지해야한다고 믿고 있는 인간종의 정체성---특히, 생물학적 정체성---의 변경을 경계하거나 금지하는 사례에서 인간존엄을 그 근거로 사용하고 있다.

뒤에서 살펴볼 자유의 기초로서 인간존엄이 사용될 때와는 달리, 이러한 높은 수준의 인간성으로서의 인간존엄은 이 인간성을 보호하기 위해 인간 스스로의 자율적 선택에 의한 행위조차 인간성 파괴로 간주되는 경우에는 간섭하고 금지하는 후견주의적 함축을 전달한다는 측면에서 많은 논쟁을 불러일으키고 있다.

\subparagraph{높은 수준의 인간성으로서의 인간존엄}

최소한의 생계보장의 기초

인간존엄의 존재는 국가가 모든 국민들에게 최소한의 생계를 보장해야 할 의무를 지우는 근거가 된다고 이해되는 경우가 있다. 인간존엄은 인간다운 최소한의 삶과 모종의 관계가 있는 개념이라는 것이다. 이러한 입장은 우리나라 헌법재판소 판결에서도 나타나고 있다. 최소한의 필요한 보장수준을 제공하지 않는 것을 인간존엄의 침해로 보는 헌법재판소 판결로는 국민기초생활보장최저생계비와 관련하여 ``그 내용상 최소한의 필요한 보장수준을 제시하지 아니하여 인간으로서의 인격이나 본질적 가치가 훼손된다고 볼 수 없으므로 인간의 존엄과 가치 및 행복추구권을 침해하는 규정이라고 할 수 없다.''고 설시한 사례,\footnote{헌재 2004. 10. 28. 2002헌마328, 공보 제98호, 1187(2002년도 국민기초생활보장최저생계비 위헌확인)} 의료급여와 관련하여 ``생활능력 없는 장애인의 의료보장과 관련하여 그 내용상 최소한의 필요한 보장수준을 제시하지 아니하여 인간으로서의 존엄이나 본질적 가치를 훼손할 정도에 이른다고는 볼 수 없으므로 헌법 제10조의 인간의 존엄과 가치를 침해한다고 할 수 없다'' 고 설시한 사례가 있다.\footnote{헌재 2009. 11. 26. 2007헌마734, 판례집 21-2하, 576(의료급여법 제10조 등 위헌확인)}

평등의 기초

다른 한편으로 인간존엄은 평등의 기초가 된다고 이해되기도 한다. 미국 판례에서는 인종, 성별, 장애에 기초한 차별을 금지하는 법률의 규범적 기초로 자주 인용된다. 여기서 존엄은 직접 구체적인 법적 기준으로 나서기 보다는 주로 평등 원리에 어떤 기초를 제공한다.\footnote{Carter Snead, ``Human Dignity in US law'', \emph{The Cambridge Handbook of Human Dignity} (Cambridge, 2014), 389면.} 대표적인 美연방대법원의 논리는 ``차별은 모든 인간이 인간가족의 구성원이기 때문에 보유하는 근본적 평등을 존중하지 못함으로써 본질적 존엄을 침해한다''는 것이다. 2차대전 당시 일본계 미국인들을 강제수용소에 구금한 루즈벨트 정부를 옹호한 악명높은 판결 중 하나인 \emph{Korematsu v. US}사건에서 머피 대법관은 반대의견을 제시하면서 다수의견이 ``개인의 존엄을 파괴하는 우리의 적들에 의해 사용되는 가장 조악한 정당화 근거를 채택''하였다고 비판했다.\footnote{\emph{Korematsu v. US}, 323 US 214,240 (1944) (Murphy J dissenting).} 숙박업소가 흑인에게 숙소를 제공하지 않는 등의 인종을 차별하는 행위를 금지하는 연방의회가 제정한 1964년 민권법이 합헌이라고 판시한 \emph{Heart of Atlanta Motel Inc. v. US} 사건에서도, 대법원은 ``{[}민권법{]}의 핵심 목적은 \ldots{} 공중시설에 평등하게 접근하지 못하게 하는 개인의 존엄의 박탈을 \ldots{} 해결하기 위한 것''이라고 설시했다.\footnote{\emph{Heart of Atlanta Motel Inc. v. US}, 379 US 241,250 (majority opinion) and 291-2 (1964) (Goldberg J concurring).} 대법원은 또한 성별에 기초한 차별이 ``인간들로부터 그들 개인의 존엄을 박탈한다''고 설시하기도 하였다.\footnote{\emph{Roberts v. US Jaycees}, 468 US 609, 626 (1984).} 대법원은 동성간 성행위를 범죄화한 주법이 주로 자유나 자율성 원리에 기반하여 수정헌법 제14조의 적법절차조항에 의해 보호되는 자유의 이익을 위반한다고 해 왔으나,\footnote{539 US 558 (2003).} 자유의 개념을 보완하기 위해 모든 이들의 평등한 존엄에 호소하기도 했다.\footnote{539 US 567 (2003).} 또한 미국 연방의회는 장애인과 노년층의 평등을 보장하기 위한 입법에서 본질적인 `인간존엄'의 용어를 채택해왔다.\footnote{예를 들어, 42 USC 15001(c)(4) (`Developmental Disabilities Assistance and Bill of Rights') directs that `services, supports, and other assistance {[}offered pursuant to this law{]} should be provided in a manner that demonstrates respect for individual dignity', Special Olympics Sport and Empowerment Act of 2004 § 2(a)(2) (30 October 2004) `dignity and value the giftedness of children and adults with an intellectual disability', 12 USC 1701q(a)(`Housing and Recovery Act of 2008') 본법의 목적의 하나로 `{[}to{]} enable elderly persons to live with dignity and independence by expanding the supply of supportive housing'. 등 참조.}

품위의 기초

앞서 높은 수준의 인간성으로서의 인간존엄의 개념은 그 후견주의적 함축으로 인해 많은 논란을 불러일으켜왔다고 한 바 있다. 이러한 후견주의적 함축은 그 기준이 단순히 생계보장이나 모든 이들에 대한 평등한 존중을 넘어, 인간에게만 특유한 특별한 부가적인 존중을 요청할 때 그 논란이 더 증가할 수 있다. 여기서 살펴볼 높은 수준의 인간성으로서의 `품위'가 그 대표적인 예이다.

인간존엄을 품위와 관련시키는 많은 사례는 법문서에서 `품위'의 용어를 사용하여 명시적으로 등장하지 않고, 다른 표현과 형식으로 등장하는 경우도 많다. 뒤리히의 객체공식과 같은 개인의 수단화 금지의 표현이 대표적인 예이다.\footnote{인간존엄을 개인의 수단화 금지로 이해하는 것은 문제가 있다. 이에 대한 설명과 비판은 후술하기로 한다.} 여기서는 표현과 형식에 불구하고 품위를 인간존엄에 해당하는 높은 수준의 인간성으로 파악한 것으로 보지않을 수 없는 사례들을 함께 소개하고자 한다. 이러한 사안들은 인간이 비하나 굴욕적인 상황에 처하지 않도록 함으로써 인간으로서의 높은 품위를 유지시키고자 하고 있다.

우리나라 헌법재판소 판결

먼저 우리 헌법재판소가 수단화 금지의 표현에 의존하고 있지만 품위의 상실을 인간존엄의 기초로 본 것으로 간주할 있는 사례는 다음과 같다. 먼저 성매매와 관련하여 ``또한 인간의 성을 고귀한 것으로 여기고, 물질로 취급하거나 도구화하지 않아야 한다는 것은 인간의 존엄과 가치를 위하여 우리 공동체가 포기할 수 없는 중요한 가치이자 기본적 토대라 할 수 있다. 설령 강압이 아닌 스스로의 자율적인 의사에 의하여 성매매를 선택한 경우라 하더라도, 자신의 신체를 경제적 대가 또는 성구매자의 성적 만족이나 쾌락의 수단 내지 도구로 전락시키는 행위를 허용하는 것은 단순히 사적인 영역의 문제를 넘어 인간존엄을 자본의 위력에 양보하는 것''이라고 설시하였다.\footnote{헌재 2016. 3. 31. 2013헌가2, 판례집 28-1상, 259(성매매알선 등 행위의 처벌에 관한 법률 제21조 제1항 위헌제청)} 또한 구치소 내의 과밀수용행위에 대하여, ``인간의 존엄과 가치는 모든 인간을 그 자체로서 목적으로 존중할 것을 요구하고, 인간을 다른 목적을 위한 단순한 수단으로 취급하는 것을 허용하지 아니하는바{[}\ldots{]}. 그러므로 인간의 존엄과 가치는 국가가 형벌권을 행사함에 있어 사람을 국가행위의 단순한 객체로 취급하거나 비인간적이고 잔혹한 형벌을 부과하는 것을 금지하고, 행형(行刑)에 있어 인간 생존의 기본조건이 박탈된 시설에 사람을 수용하는 것을 금지한다''고 설시한 바 있다.\footnote{헌재 2016. 12. 29. 2013헌마142, 판례집 28-2하, 652(구치소 내 과밀수용행위 위헌확인)} 이 판결들은 직접적으로 `품위'의 용어를 사용하고 있지 않지만, 성매매가 인간의 신체가 수단이 되는 노동과 다르다는 점, 과밀한 수용이 반드시 생존을 위협하지는 않는다는 점에서 인간의 높은 품위를 그 침해된 인간성으로 본 것으로 이해해야 할 것이다.

인간으로서의 기본적 품위를 유지할 수 없도록 하는 것을 인간존엄의 침해로 직접적으로 설시한 경우도 있다. 헌법재판소는 유치실 내 화장실 사용을 강제한 것과 관련하여, ``유치기간동안 위와 같은 구조의 화장실을 사용하도록 강제한 피청구인의 행위는 인간으로서의 기본적 품위를 유지할 수 없도록 하는 것으로서, 수인하기 어려운 정도라고 보여지므로 전체적으로 볼 때 비인도적·굴욕적일 뿐만 아니라 동시에 비록 건강을 침해할 정도는 아니라고 할지라도 헌법 제10조의 인간의 존엄과 가치로부터 유래하는 인격권을 침해하는 정도에 이르렀다고 판단된다''고 설시하였고,\footnote{헌재 2001. 7. 19. 2000헌마546, 판례집 13-2, 103(유치장내 화장실설치 및 관리행위 위헌확인)} 두 팔을 몸에 고정시키는 계구를 장기간 착용시킨 행위에 대하여, ``청구인은 가장 기본적인 생리현상을 해결할 때마다 인간으로서의 기본적인 품위유지조차 어려운 생활을 장기간 강요당했으므로 그 자체로 인간존엄을 훼손당한 것으로 볼 수 있다''고 설시하였다.\footnote{헌재 2003. 12. 18. 2001헌마163, 판례집 15-2하, 562(계구사용행위 위헌확인)}

프랑스의 난쟁이던지기 사건

품위를 인간존엄에 관련된 중요한 인간성으로 이해한 것으로 평가되는 대표적인 사례로서 최근 인간존엄의 법적 적용에 대한 가장 많은 논란을 불러일으킨 것은 앞서 살펴보았던 프랑스의 난쟁이던지기 사건이다. 프랑스 헌법재판소(Conseil Constitutionnel)가 ``헌법적 가치의 원리''라는 표현으로 인간존엄 원리를 재발견한 이후로,\footnote{Conseil Constitutionnel, 27 juillet 1994, 94-343-344 DC, Loi Bioéthique.} 프랑스에서 인간존엄의 헌법원리가 어떤 의미를 가지는지에 대한 해석이 둘로 나뉘었다는 점을 앞서 설명하였다. 하나는 수많은 인권들, 특히 사회권의 정초 혹은 확장의 방향성을 제공하는 기능을 한다는 해석이고, 다른 하나는 높은 수준의 인간성을 향한 개인의 책임을 강조하는 소위 `존엄주의적(dignitarian)' 해석이다. 난쟁이던지기 사건은 여기서 후자의 해석에 해당한다. 이 해석은 인간존엄의 원리를 개인의 이익증진이나 인권도출의 차원과는 구별되는 것으로 이해한다. 이 판결에서 꽁세유데따는, 추상적 목적으로서의 인간존엄의 보호가 유효하게 시민적 자유의 사용에 대한 행정적 제약의 우선적인 근거가 될 수 있다고 보았다.\footnote{Conseil d\textquotesingle État, 27 octobre 1995, Commune de Morsang-sur-Orge.}

기타

프랑스의 난쟁이던지기 사건에 대한 꽁세유데따의 이러한 태도는 새로운 것은 아니었다. 독일연방행정최고법원이 해당 여성의 동의하에 여성의 신체가 타인에 의해 몰래 훔쳐보는 형식으로 전시되는 핍쇼(Peep-Show)를 금지한 사례에서도 난쟁이 던지기 사건에 대한 꽁세유데따와 유사한 결론이 이미 내려진 바 있다.\footnote{BVerwGE 64, 274.}

이러한 사례에서, 인간존엄이란 인간이 가지고 있거나 가져야 할 어떤 높은 인간성, 즉 품위를 언급하고 있는 것으로 파악된다. 판결들이 `도구화 금지' 등의 용어를 사용하고 있는 경우가 발견되고 있지만, 이러한 사안들에서 인간존엄이 단지 인간 신체의 도구화에만 연관되었다고 보기는 어렵다. 금지되지 않는 일반적인 노무계약이나 스포츠나 예술활동을 포함한 인간의 수많은 활동들이 많은 경우에 인간 신체를 도구로 (심지어 전적으로) 이용하여 이루어지고 있기 때문이다. 이러한 사안에서, 각국의 법원은 인간성이 겪는 어떤 굴욕이나 비하, 즉 `품위'의 격하를 인간존엄의 침해로 판단하고 있는 것으로 보인다.

\subparagraph{자유의 기초로서의 인간존엄}

지금까지 인간존엄의 내용적 측면에서, 높은 수준의 인간성으로서의 인간존엄과 그 후견주의적 함축과 기능을 살펴보았다. 여기서는 내용적 측면의 다른 한쪽 측면인 자유의 기초로서의 인간존엄과 그 자유주의적 함축과 기능을 살펴보려고 한다. 이 두 기능은 그 운용의 측면에서 명백한 모순을 담지하고 있기에, 각각의 내용을 옹호하는 양측이 서로 날카롭게 대립하고 있다.

세계 각국의 법질서에서 받아들여지고 있는 인간존엄 이해의 다른 한 줄기는, 인간존엄은 인간이 스스로 원하는 바를 선택할 자유 그 자체에 있다는 것이다. 구체적인 법질서에의 편입은 개인의 선택행위를 존중하는 일반적 행동자유권과 자기운명결정권과 같은 형식으로 나타나거나, 선택할 수 있는 여건을 조성해 주는 제도(예를 들어 신체의 자유)를 도입하는 방식으로 나타나고 있다. 대표적으로 미국의 판례는 형사절차와 처벌과 관련하여 인간존엄을 자주 언급하고 있다.

미국 밖에서의 법제도에서는 주로 `인간존엄'이 나치 이데올로기에 대한 반성으로 인간종의 침해할 수 없는 가치를 정초하는 인격적 의미의 법개념으로 도입된 데 반하여, 미국 법문화에서 초기의 `존엄'은 주로 정부기관에 대한 적절한 존중을 표시하기 위해 처음 사용되었고, 본 논문에서 다루고자 하는 `인간존엄'의 개념은 적어도 미국헌법에서 공식적인 개념으로 도입되지는 않았다.\footnote{Carter Snead, ``Human Dignity in US law'', 387면 참조.} 하지만 미국법체계에서도 인간존엄의 관념은 간접적으로, 그리고 광범위하게 나타나고 있다. 이는 자유, 평등, 적법절차원리와 같은 다른 이미 확정된 법적 개념들의 규범적 힘을 더 강화시키는 수사적인 방식으로 이루어지고 있다. 다음에서는 인간존엄이 미국 형법과 형사절차의 맥락에서 어떻게 침투해 들어와 있는지 살펴보고자 한다.

미국법에서 대표적으로 인간존엄이 등장하고 있는 영역은 형사절차와 처벌과 관계된 것이다. 먼저 형사절차와 관련한 연방대법원의 인간존엄의 언급을 살펴보자. 미국 수정헌법 제4조는 부당한 압수와 수색, 영장발부로부터 국민의 사생활이 침해되지 않아야 한다는 것을 선언한 것으로 잘 알려져 있는데,\footnote{미국 수정헌법 제4조: ``불합리한 압수와 수색에 대하여 신체,주거,서류,물건의 안전을 확보할 국민의 권리는 침해되어서는 아니된다. 선서나 확약에 의하여 상당하다고 인정되는 이유가 있어 특별히 수색할 장소와 압수할 물건, 체포·구속할 사람을 특정한 경우를 제외하고는 영장은 발부되어서는 아니된다. (The right of the people to be secure in their persons, houses, papers, and effects, against unreasonable searches and seizures, shall not be violated, and no Warrants shall issue, but upon probable cause, supported by Oath or affirmation, and particularly describing the place to be searched, and the persons or things to be seized.)''} \emph{Schmerber v. California} 사건에서 미연방대법원은 제4조의 핵심적 기능이 ``국가의 정당화되지 않는 침해로부터 개인의 사생활과 존엄을 보호하는 것''이라고 설시함으로써, 존엄에 수정헌법 제4조에서 국가권력의 헌법적 한계로 기능하는 사생활을 보다 강조하는 역할을 부여하고 있다.\footnote{\emph{Schmerber v. California}, 384 US 757 (1966).: ``{[}t{]}he overriding function of the Fourth Amendment is to protect personal privacy and dignity against unwarranted intrusion by the State''.} 카터 스니드는 이러한 설시가 어떤 우선적 원리(a first-order principle)가 아니라 단지 사생활의 중요성을 강조하는 수사적 전략으로 사용된 것이라고 말한다.\footnote{Carter Snead, ``Human Dignity in US law'', 388면.} 또한 미국의 수정헌법 제5조는 불리한 증언을 강요받지 않을 권리를 부여하고 있는데, \emph{Miranda v. Arizona} 사건에서 연방대법원은 ``이 특권에 깔린 헌법적 기반은 정부가 그 시민의 존엄과 고결성에 부여해야 할 존중''이라고 설시했다.\footnote{\emph{Miranda v. Arizona}, 384 US 436, 460 (1966).}

인간존엄은 미국에서 처벌과 관련해서도 자주 인용된다. 수정헌법 제8조는 ``잔인하고 기이한 처벌''을 금지하고 있는데, 이에 관하여 \emph{Roper v. Simmons} 사건에서 연방대법원은 ``극악무도한 범죄로 처벌받는 자들을 보호함으로써, 수정헌법 제8조는 모든 인간의 존엄을 존중할 정부의 의무를 재확인한다''고 설시하였고,\footnote{\emph{Roper v. Simmons}, 534 US 551, 560 (2005).} 또한 \emph{Trop v. Dulles} 사건에서 미연방대법원은 ``제8조에 깔려있는 기본 개념은 인간의 존엄에 다름아니다''라고 설시하였다.\footnote{\emph{Trop v. Dulles}, 356 US 86, 100 (1958).} 이러한 입장을 통해 미연방대법원은 정신장애가 있는 피고인이나\footnote{\emph{Atkins v. Virginia}, 536 US 304 (2002).} 18세 미만의 피고인들에게\footnote{\emph{Roper v. Simmons}, 543 US 551 (2005).} 사형을 부과할 수 없다고 설시해 왔다. 또한 연방대법원은 비하적이고 비인간적인 조건에 죄수들을 구금하는 것은 ``잔인하고 기이한'' 처벌을 구성할 수 있다고 설시해왔다.\footnote{\emph{Estelle v. Gamble}, 429 US 97,104 (1976); \emph{Hope v. Pelzer}, 536 US 730 (2002).} 이와 같이 미연방대법원은 반복적으로 죄수를 포함한 모든 인간의 가치로서의 인간존엄의 개념을 주장해 왔다.

인간존엄에 대한 호소는 의회에 의해 제정된 범죄피해자 보호를 위한 미연방법률에도 반영되었는데, 18 USC 3771(a)(8) 조항은 ``범죄의 피해자는 공정하게, 그리고 피해자의 존엄과 사생활을 존중하여 대우받아야 할 권리를 {[}\ldots{]} 가진다''고 규정하고 있다

\paragraph{한계와 논쟁}

그러나 앞에서 나열한 어떤 후보도 특별한 보호를 요청하는 ``법적으로 보호가치 있는 특유한 인간성''으로서 충분히 정당화되고 있지 못하다. 서로 다른 문화적 배경을 가진 사회마다, 혹은 한 사회에서 서로 다른 가치관을 가진 사람들 사이에서 어떤 가치가 이러한 인간성에 해당하는지에 대하여 합의되기 어려운 논쟁을 유발해 왔다. 또한, 왜 그런 각각의 인간성 혹은 가치에 대한 이해방식으로부터 우리가 기대하고 있는 다양하고도 구체적인 규범과 권리들이 도출되는지, 그러고 그러한 규범과 권리들은 왜 그로부터 도출되지 않은 다른 규범에 특별히 우선해야 하는지 설명해야 하는데, 그런 정당화를 충분히 제공하고 있는 설명을 찾기 어렵다. 대표적으로 인간존엄에 대한 자율성 중심의 이해와 후견적 규제의 기초로서 높은 수준의 인간성 중심의 이해는 두 입장 사이의 격차를 줄이지 못하고 있다. 또한 이러한 이해는 일단 특별한 법적 위상을 가진 인간존엄으로 인정되면 사법부에 의한 해석이 입법부에 의한 규범을 압도할 수 있다는 점, 개인의 주관적 권리가 객관적 규범에 의해 쉽게 제한될 수 있다는 점에서 더욱 문제가 된다.

\paragraph{대안적 견해: 시민적 지위(citizenship)로서의 인간존엄}

우리는 앞에서 주관적 권리규범으로 인간존엄을 보호한다는 관념의 다양한 이해방식 중 권리를 가질 권리로 이해할 수 있다는 하나의 입장을 살펴본 바 있다. 한나 아렌트가 권리를 가질 수 있는 제도로서의 국가 구성원 자격을 유지하는 것의 중요성을 논증한 것으로부터 그 중요성은 분명해졌다.

그러나 이러한 한 인간이 그에게 시민으로서의 권리를 부여받을 수 있는 정치질서의 구성원어야 할 자격의 문제를, 일반적인 의미에서 ``권리''의 문제로 보기에는 다소 상이한 측면이 있다. 보통 ``개인 a는 G에 대한 권리를 가지고 있다''는 권리진술에 의해 설명되는 권리는, 권리보유자, 권리상대방, 권리내용의 삼가관계의 구조를 갖는데,\footnote{김도균, 권리의 문법, 2008, 박영사, 4-5면.} 권리를 가질 권리는 이 중 권리내용이 ``권리를 가짐''이라는 특유한 내용으로 구성되어 있기 때문이다. 이를 재산권이라는 권리에 비유하자면, 일반적으로 재산권은 재산이라는 권리내용을 갖는데, 지금 말하고자 하는 권리는 재산권 그 자체가 아니라, 재산권을 가질 권리에 가까운 것이다. 우리는 보통 이를 재산권자라는 `지위'나 `자격'으로 더 자주 설명한다. 따라서 이러한 권리를 가질 권리로서의 인간존엄을 다소 독자적인 개념인 ``지위''의 문제로 이해하는 견해가 있다.

\subparagraph{\texorpdfstring{\emph{Trop v. Dulles} 사건(시민권은 박탈할 수 없다)}{Trop v. Dulles 사건(시민권은 박탈할 수 없다)}}

\emph{Trop v. Dulles} 사건(1985)에서 미연방대법원은 어떤 공격에 대한 벌칙으로서 시민권을 상실시키는 조항은 잔인하고 기이한 형벌을 금지하는 수정헌법 제8조의 금지를 위반하는 것이라고 주장했다. 잔인하고 기이한 형법의 금지 조항의 근거에 관하여 미연방대법원은 다음과 같이 말했다.

수정헌법 제8조에 깔려있는 기본 개념은 인간의 존엄에 다름아니다.\footnote{\emph{Trop v. Dulles}, 356 US 86,100 (1958): ``The basic concept underlying the Eighth Amendment is nothing less than the dignity of man.''}

이는 시민권의 박탈을 잔인한 고문이나 사형죄 반대론에서 바라본 사형처럼 이해하는 것처럼 보인다. 미국 수정헌법8조의 잔인하고 기이한 처벌에 시민권의 박탈을 연계하는 논증은, 처벌이 너무 심해서 인간의 존엄을 강등시키는 정도에 이르러서는 안되고 무의미한 고통의 부과에 불과해서는 안 된다는 일반적인 논의를 떠올리게 만든다. 그러나 국적박탈은 일반적 의미에서 고통스럽거나 잔인하지는 않다. 우리는 어떤 육체적 학대나 고문도 하지 않고 특정인의 국적을 박탈할 수 있다.

다만 우리는 이러한 국적박탈로 인해 해당 인격이 앞으로 받게 될 고난을 충분히 예측할 수 있게 만드는 중대한 취약성을 창출하는데 그 잔인성이 있다고 설명할 수 있다. 워렌 대법관도 이러한 입장을 취했다. 워렌 대법관은 한 인격의 시민권을 취함은 ``조직된 사회에서 그 개인의 지위를 완전히 파괴하는 것''을 함축하는 것으로 이해한다.

이는 개인에게서 수세기동안 발전한 정치적 존재를 파괴하는 것이다. 이 처벌은 국가적이고 국제적인 정치 공동체에서 시민임을 그의 지위에서 빼앗는다. 그의 바로 그 존재는 그가 우연히 자신을 발견한 그 국가의 용인에 있다\ldots{} 그것은 개인을 영속적으로 증가하는 위험과 곤란의 운명에 드러내게 한다. 그는 어떤 차별이 그에게 생길수 있는지, 어떤 추방/박탈이 그에게 지시될 수 있는지, 그리고 언제 또는 무슨이유로 그의 출신지역에서 그의 존재가 끝날 수 있는지에 대해 알지 못한다 \ldots{} 그는 국적이 없다, 즉 민주주의의 국제 공동체에서 비통한 상황에 있다.\footnote{\emph{Trop v. Dulles}, 356 U.S. 86 (1958), 101-102 (Warren CJ, for the Court).}

이는 앞서 권리를 가질 권리에서 설명한 한나 아렌트의 설명, 즉 국가 없는 인격은 일종의 법적 림보(중간계), 지속적으로 법을 위반해야만 하는 상태, 사실상 법의 지배보다는 필연적으로 경찰력에 의해 직접 지배당함의 상태에 있다(여기서 위반들은 규칙이 아닌 예외임이 가정될 것이다)는 설명과 일맥상통하는 것이다.

인권과 시민권의 관계

그렇다면 인권의 개념이 보편화된 현대사회에서, 인권의 존재는 시민권을 불필요하게 만드는가? 그렇지 않다. 이러한 권리 보유의 권리, 혹은 국적 보유의 권리를 미연방대법원과 마찬가지로 제러미 월드론은 인간존엄을 보호하는 권리의 개념에 연결시킨다.

인권에 대한 위대한 선언들과 헌장들은 권리들을 인간존엄에 매우 가깝게 연관시키고, 이러한 연관이 단순한 시민권의 권리들을 무색케 하는 것처럼 보일수도 있다. 그러나 실천이나 이론 모두에 있어 인권이, 각각의 국가에서 시민권을 제공하는 법적 직조로 편입되지 않는다면, 일반적으로 안전하지 않다는 것이 판명된다. 그렇다면 실제 용어에 있어서, 시민의 존엄은 인간존엄의 필연적인 부수물이 될 것이다. 그리고 시민적 존엄의 윤곽을 탐색하는 것은 그것이 인간존엄 그 자체에 적용되기 때문에, 우리의 존엄 이론의 불가결한 부분이 될 것이다.\footnote{Jeremy Waldron, ``Citizenship and Dignity'', 338면.}

그에 의하면 인권의 존재가 시민권을 무용하게 만드는 것이 아니라 오히려 시민권의 존재 안에서만 인권이 법적 제도로서 활기를 띨 수 있다는 것이다. 여기서는 월드론의 논의를 통해 `존엄'이 여전히 어떤 높은 지위의 의미를 가질 수 있는 가능성을 살펴보고, 그럼에도 불구하고 월드론이 가지는 귀족주의적 지위를 모든 인간에게 평등하게 확장하려는 논의, 즉 시민권으로서의 인간존엄의 이해가 가지는 한계를 살펴보려고 한다.\footnote{앞에서 살펴보았지만, 이러한 높은 지위, 즉 우월성이 부착되는 대상은, 개별적 인간이 아니라 인간이 가지는 도덕성이고, 그 우월성의 의미는 도덕성이 가지는 객관적 중요성이라는 것이 본고의 주장이다.}

\subparagraph{\texorpdfstring{철학적 배경: 귀족적 지위의 평등주의적 확장론 }{철학적 배경: 귀족적 지위의 평등주의적 확장론 }}

우리는 제1장에서 법적으로 보호가치있는 특유한 인간성로서의 `인간존엄'의 이해방식을 살펴보면서 그 이해방식이 가지는 여러 한계도 동시에 살펴보았다. 가치로서의 `인간존엄'의 이해방식이 가지는 이러한 한계에 대하여, 그에 대한 대안 중 하나는 높은 지위로서의 고대적 의미의 존엄 개념을 복원하는 대안을 생각해 볼 수 있다. 귀족적 지위로서의 존엄에는 여러가지 구체적인 행위 가이드와 존중의 방식이 정해져 있었기 때문이다.

예를 들어 월드론이 들고 있는 판사의 사례를 들어보자.\footnote{Jeremy Waldron, \emph{Dignity, Rank, \& Rights}, p.18.} 판사들이 그들의 공무에 대한 존엄을 향유한다고 하자. 여기서 판사는 판사로서의 지위를 가지지만, 동시에 법정에서 구체적인 혐오로 인한 공격으로부터 보호받아야 한다. 그러나 이러한 구체적인 혐오공격으로부터의 보호가 판사로서의 지위의 모든 것은 아니다. 하지만 동시에 필수불가결한 것이다. 또한 추가적인 대우를 받아야 할 수도 있고, 판사용 가운이나 가발을 제공받으며, 연회장에서 좋은 좌석을 배치받는 등의 대우가 필요할 수 있다. 이들 모두 사법의 존엄을 위해 중요하다. 하지만 역시 이것이 판사의 지위의 전부는 아니고, 실제로 판사의 지위는 판사의 역할, 권능, 책임과 관계가 (어찌보면 더) 깊다.

현대사회에서 인권과 존엄의 관계도 마찬가지다. 노동자의 존엄을 보장하기 위한 적절한 보수를 보장할 적극적 권리라던가 굴욕적인 처우를 하지 않을 소극적 권리가 있을 수 있다. 이들은 매우 중요하고 필수불가결하지만, 이것이 인간존엄이라는 지위의 전부는 아니다. 오히려 노동자의 일반적인 삶이 존엄한 지위와 (어찌보면 더) 관계가 깊을지도 모른다.

제레미 월드론은, ``우리는 고귀한 계급과 높은 공무에 대한 그것{[}존엄{]}의 고대적 관련성과의 신의를 어떻게든 유지하기 위해 노력해야 한다''고 말한다.\footnote{Jeremy Waldron, \emph{Dignity, Rank, \& Rights}, 30면.} 문제는 높은 지위로서의 고대적 존엄 개념과 현대의 평등주의적 인간존엄 개념이 딜레마에 봉착한다는 사실이다. 월드론에 의하면 ``존엄 이론들은 두 개의 명백하게 다른 관념들 사이를 항해해야만 한다''.\footnote{Jeremy Waldron, ``Citizenship and Dignity'', 327면.} 그 두 관념은 바로 \emph{dignitas}로 대변되는 귀족적 지위의 관념과 인간존엄이라는 평등주의적 관념이다.

i) 라틴어 \emph{dignitas} : ``귀족체계와 계층적 업무체계에서 어떤 구체적 역할과 계급에 부착되는 지위''

ii) 인간존엄이라는 평등주의적 개념 human dignity : ``가장 높은 사람들로부터 가장 낮은 사람들까지, 도덕적 영웅으로부터 가장 비열한 범죄자까지 모든 사람에게 부여되는 것으로 이해되는 평등주의적 개념.'' \footnote{Ibid.}

그런데 월드론은 위의 두 관념 중 하나를 택하는 것이 아니라, 두 관념 모두를 받아들이는 쪽을 택한다. 월드론은 그 해결책의 일단을 칸트가 모든 인간이 가지는 국가시민의 지위를 존엄이라는 개념을 통해 서술하고 있다는 점에서 발견한다. 칸트는 ``그런데 아무런 작위{[}존엄{]} 없이는 어떠한 인간도 국가 안에 있을 수는 없을 것이다. 왜냐하면 적어도 그는 국가시민이라는 작위{[}존엄{]}는 가지고 있기 때문이다.''라고 말하고 있기 때문이다.\footnote{Jeremy Waldron, \emph{Dignity, Rank, \& Rights}, 30면.} 이와 같이 칸트의 존엄에서 발견된 인간에게 고루퍼진 어떤 지위의 측면을 강조하는 월드론의 이해방식은 ``귀족적 지위의 평등주의적 확장''을 시도하는 주요한 근거로 활용되고 있다.

이러한 이해를 토대로 월드론은 인간존엄이 ``사회구성원으로서의 정상적인 지위''라고 말한다. 이러한 지위인 ``존엄은 그 지위에 맞는 대우와 인정에 관한 요구를 낳''으며, ``사회적⋅법적 지위로서의 존엄은 사회와 법에 의해 확립되고, 옹호되고, 유지되고, 정당성이 입증되어야'' 하며, ``타인의 존엄을 침해한다고 생각되는 방법의 행위를 금지할 것을 요구받는다''고 주장한다. \footnote{Jeremy Waldron, \emph{The Harm in Hate Speech} (Havard University Press, 2012), 홍성수⋅이소영 역, 혐오표현, 자유는 어떻게 해악이 되는가? (도서출판 이후, 2017), 80-81면.}

가치와의 관련성 부정

그런데 월드론은, 존엄이라는 (그에 의하면 법적) 지위가 어떤 가치와 관련되어 있다는 사실은 부정한다. 그는 비록 법이라는 서식지 안에서이기는 하지만, 가치로서의 존엄의 이해방식이 부적절함을 설명하기 위해 다음과 같이 시민권의 유비를 든다.

시민권은 개인들이 가지는 어떤 지위를 지시하기 때문에 매우 중요하다. 이 지위는 그들이 대우받아야 하고 그들의 행위가 어떤 국가와 시민사회의 작동과정에서 부합해야 할 근본적 방식들의 일부를 지도한다. 누군가는 시민권을 각인의 특정한 가치가 있는 위치의 관점에서 분석할수도 있겠지만, 이는 이상하고 명확히해주는 게 없는 분석이다. 한 시민은 단지 국가의 가치 있는 자산도 아니고, 그렇다고 한 시민이 국가의 유용성과 분리된, 고유한(in its own right) 가치의 어떤 것도 아니다. 만약 우리가 여기서 어떤 가치에 대해 말하고 있기는 하다면, 그 자신의 가치를 인식하는 (그리고 그 가치에 대한 이러한 인식이 가장 중요한 특성중의 하나인), 그리고 그(가치)에 따라 행위하는, 어떤 것의 가치에 대해 말하는 것이다---이는, 공교롭게도, 우리가 일반적으로 가치라고 생각하는 그런 방식이 아니거나, 좁은 의미에서 가치이론이라고 불리는 어떤 것에 의해 가치라고 생각하는 방식이 아니다. 가치와 좋음에 관한 어떤 분석은, 혹은 시민권을 이러한 범주들로 환원하는 것은 우리의 이해를 왜곡하고 그 개념에서 중요했던 많은 것들에 대한 설명을 하지 못하게 할 것이다. 나는 시민권의 개념을 이해하려고 애쓰는 데 있어 우리는 재빨리 어떤 가치 분석이 부적절하다는 것을 발견하리라 생각한다.(강조-역자 주)\footnote{Jeremy Waldron, \emph{Dignity, Rank, \& Rights}, 138-139면.}

그는 시민권을 국가의 자산도 아니지만 국가와 유리된 어떤 것도 아니라고 주장한다. 시민권은 이러저러한 권리와 의무들과 관련되지만, 결코 그러한 권리와 의무의 목록으로 환원되는 것도 아니라고 말한다. 그러면서 그가 가치와 존엄, 혹은 가치와 시민권의 관계를 일정정도 이상으로 가깝게 하는 것을 거부하는 맥락은 분명하지 않다. 추정컨대, 일반적인 가치론에서 말하는 가치의 그 자신이 가지는 ``고유성''이 국가와 법률에 의해 필요적으로 등장하는 시민권과 존엄에서는 찾아보기 어렵다고 주장하는 것 같다.

월드론의 말대로, 또한 여러 인권문헌에서처럼, 시민권으로부터 어떤 권리와 의무들이 도출되는 것이라면, 시민권 그 자체를 어떤 권리와 의무의 목록으로 환원하거나, 시민권 그 자체를 가치로 보는 것은 옳지 않다. 특히 시민권은 국가와 밀접한 관련을 가진다는 의미에서 국가의 발생과 그 명운을 함께한다고 해도 과언이 아니며, 근대적 의미에서 이러한 국가는 권리의무를 발생시키는 법률의 발생과 그 명운을 같이한다고 해도 과언이 아닐 것이다. 따라서 시민권은 어떤 가치 그 자체는 아닌 것이다.

\subparagraph{시민적 지위로서의 인간존엄 이해의 한계와 시사점}

인간존엄을 ``이미 존재하는 권리들을 보유할 권리, 혹은 시민적 지위''로 이해하는 대안적 관점의 경우 이미 존재하는 권리들에 대한 인간존엄의 형식적 위상을 설명하는 데는 유용하지만, 이러한 지위는 명목적 의미만을 지닐 뿐이다. 현대의 인권이론과 인간존엄의 요청은 규범이 부족한 곳에서 새로운 권리를 창출할 것을 강력히 요청하고 있다. 많은 현대적 인간존엄의 담론은 인간존엄이 인권들을 실질적으로 정당화하고, 또한 이 과정에서 집단학살이나 모멸감과 비하를 유발하는 체계적이고 구조적인 형태의 심각한 침해의 사안에서 인간존엄을 보호할 권리를 추가적으로 발생시킬 것을 요청해 왔는데, 이러한 현대적 담론이 요청하는 권리산출이라는 보충적이고도 중대한 함축, 실질적 기능을 설명하지 못한다.

a. 한계: 권한 창출의 도덕적 정당화의 부재

``지위 S로부터 권리R이 도출된다''라는 표현방식은 자칫하면 지위의 사전적 정의가 직접 지칭하고 있는 내용들이 그 권리R 창출의 규범적 근거라고 오해할 수도 있다. 대표적인 법적지위인 계약자, 청약자, 수락자와 같이 규범적 근거를 직접 지위의 이름으로 삼은 경우들이 이러한 오해를 더욱 강화한다. `계약', `청약', `수락'과 같은 행위는 실제로 해당 권한을 근거짓는 규범적 근거이기도 하기 때문이다.

그러나 이는 우연에 불과하다. 정확히 말하면 계약자의 지위가 그의 권리R1을 창출하는 것이 아니라, 그가 계약이라는 행위를 했기 때문에 그의 권리 R1이 창출된 것이다. 명칭이 원인C를 직접 명기하지 않은 사례를 보자. 친권자가 친권이라는 지위의 권한들을 행사할 수 있는 규범적 근거는 그가 친권자라는 지위에 있다는 사실이 아니라, 때로는 그가 그 아이를 낳는 데 기여했기 때문이며, 아이가 연약하고 보호를 필요로 하며, 사회공동체가 그러한 의무를 부여하기 때문이다. 명칭이 원인C를 일부 담지하고 있는 사례를 보자. 아버지의 사전적 정의가 ``자녀를 낳는 데 생물학적으로 기여한 남성''이라는 의미라면, `아버지'라는 이름의 지위의 권한을 행사할 수 있는 규범적 근거는 자녀를 낳는 데 생물학적으로 기여함과 아이가 연약하고 보호를 필요로 하며, 사회공동체가 그러한 의무를 부여함등의 원인들이 복합적으로 작용하고 있기 때문이다.

따라서 우리는 인간존엄이라는 지위가 불러일으키는 권리들에 대해, 인간존엄이라는 이유만으로는 특별한 대우를 받아야 할 어떤 도덕적 정당화 근거도 제시할 수 없게 된다. 그러나 이러한 명목적 요약 기능 만으로는 현대 담론이 인간존엄에 기대하는 바를 충족시키지 못한다.

그런데 월드론은 존엄의 가치와의 모종의 관련성까지도 부정한다. 위의 의미에서 월드론이 ``좁은 의미에서 가치이론이라고 불리는 어떤 것에 의해 가치라고 생각하는 방식''으로 시민권을 이해할 수 없다고 한 것은 옳은 것으로 보인다. 본고의 지위를 분석한 단락들에서 설명하듯, 시민권 그 자체는 가치가 아니라 어떤 가치의 담지자성을 보여주기 위한 지위이기 때문이다. 그러나 이에 더하여 그가 가치관련성을 아예 배제한 표현, 즉 ``각인의 특정한 가치가 있는 위치(location)의 관점에서 분석할수도 있겠지만, 이는 이상하고 명확히해주는 게 없는 분석''이라고 말한 점은 수긍하기 어렵다. 먼저 시민권이 인간에게 고유한 특성과 무관하게 오로지 국가의 발생에 의해 비로소 인정되는 지위라고 한다면, (존엄을 제외하고) 시민권에 한정했을 때 이러한 발언을 일부 이해할 수는 있다. 시민권이 어차피 국가나 법률이 실정적으로 부여해 놓은 어떤 권리나 의무목록을 가진 위상에 불과하다면 여기에 어떤 가치가 끼어들 여지는 없다. 그러나 시민권이 구체적인 공동체, 혹은 국가나 법률의 발생 이전에 이미 잠재하는 인간의 권리이고, 다만 국가나 법률이 발생함으로써 비로소 현실화하는 것 뿐이라면, 시민권은 모종의 가치와 중요한 연결점을 가지게 된다. 이는 더 이상 이상하거나 명확히 해 주는 게 없는 분석이 아니다. 시민권이 국가가 대내영토나 대외영역에서 국가의 보호를 받을 권리가 있다거나, 일정한 조건 하에 투표권을 발생시키고 참정권을 발생시킨다고 한다면, 이것이 과연 각 개인이 가지고 있는 어떤 가치와 무관한가? 그의 말을 다시 돌려준다면, 한 국가의 시민임은 ``국가의 유용성과 분리된 그 자체의 관점에서 가치의 어떤 것''이 맞다고 충분히 볼 수 있다. 한 국가의 시민임은 국가와 그 법률의 발생과 동시에 시작하지만, 그것이 보호하고자 하는 가치는 국가 발생 이전에 이미 있었고, 국가의 발생과 동시에 보호의 필요성이 더욱 두드러지게 현실화된 것이라고 볼 여지가 충분하다.

b. 시사점: 법의 강제로부터 면제되는 인간의 고귀한 가치의 해명

모든 인간에게 평등주의적으로 확장되는 ``권리들을 가질 지위''의 존재는, 인간이라면 누구나 어떤 중요한 측면에서는 법의 강제로부터 면제되어야 하는 성질, 즉 ``법적존중을 받아야 함''의 속성의 보유에 대한 실정법에 선재하는 정당화 근거를 보유하고 있음을 보여주는 중요한 단서가 된다.

현대적 의미에서 인간존엄은 평등주의적 함축을 가져야 한다. 그런데 돈 헤르조그는 월드론의 귀족주의적 지위로서의 존엄 개념이 거만하다는 이유, 법적 책무의 면탈이 남용된다는 이유로 비판한다.\footnote{Don Herzog, ``Aristocratic Dignity?'', Comments in Jeremy Waldron, \emph{Dignity, Rank, \& Rights} (Oxford, 2012), 99-118면.} 헤르조그에 의하면 귀족주의적 존엄, 즉 ``모든 이의 집은 그의 성이고, 이는 그 안에서 그가 절대적 주권을 가지기 때문이다''라는 \textless 캐슬 독트린 I(Castle Doctrine I)\textgreater 은 계급적 억압을 정당화한다. 예를 들면 이 \textless 캐슬 독트린I\textgreater 은 한 남자의 성 안에서 그의 부인에 대한 부부강간을 범죄로 구성하기 어렵게 만든다. 또한 이러한 절대적 주권은 모든 이들에게 가능한 것이 아니다.

월드론은 헤르조그의 귀족주의적 존엄의 거만함에 대한 지적을, 루트랜드 백작 부인의 사례를 들어 재반박한다. 루트랜드 백작 부인은 형사적 범죄를 지은 것이 아니라 단지 채무불이행의 상태에 있는데, 백작부인이라는 고귀한 지위로 인해 신체구속을 면제받는다. 이러한 지위는 현대에 이르러 모든 시민들에게 확대되어 ``누구도 채무에 의해서는 구속되지 않는다''와 같이 법적 책무의 존중받는 형태로 고착되었고 이것이 귀족적 존엄의 인간존엄으로의 확장이라는 것이다.\footnote{Jeremy Waldron, ``Reply'' in \emph{Dignity, Rank, \& Rights} (Oxford, 2012), 146면.} 즉, 귀족주의적 존엄의 확장은 법적 책무를 면제시켜주는 것이 아니라 법적 책무의 존중받는 형태를 일반화하는 것이라는 것이다.

그러나 월드론의 존엄을 설명하는 전략이 귀족적 지위의 단순히 확장하는 원이라면, 루트랜드 백작 부인의 사례처럼 현대의 민주시민적 함축을 가진 사례만이 확장되고, 거만함을 상징하는 사례는 확장되지 않는지, 즉 특정한 귀족적 함축들은 왜 배제하는지 그 기준을 설명하는데는 한계가 있다. 월드론이 말하는 귀족성의 구체적 내용이야말로 존엄의 핵심을 설명해 줄 핵심 키워드이기 때문이다.

월드론이 존엄을 설명하는데 있어 높이, 자세의 개념을 중시하고 있는 이상, 헤르조그의 지적은 매우 적절하다. 다만, 헤르조그는 이러한 거만함 사례가 월드론이 지지하는 귀족성 함축의 부정적인 측면 ,즉 평등주의적 함축을 저해하는 사례라고 생각했지만 본고는 오히려 이러한 거만함 설명이 인간존엄을 잘 설명해 준다고 생각한다.

실정법 이전에 ``선재함''으로서의 거만한 지위

헤르조그 자신은 지적하고 있지 않지만 헤르조그의 귀족적 존엄의 사례들은 왜 존엄이 때로는 법을 강제하고 때로는 법의 제약으로부터 면제되는 것을 돕는지 보여준다. 그는 귀족적 지위개념으로서의 존엄이 ``나는 특별한 특권들을 향유하고, 너 같은 자들에게 답변할 필요가 없다'' \textless 캐슬 독트린 II, Castle Doctrine II\textgreater 는 함축을 가지고 있다고 말한다.\footnote{Don Herzog, ``Aristocratic Dignity?'', 101면.} 그런데 이러한 특권은 어떤 난제를 제기한다기 보다는 오히려 매우 소중한 함축을 전달해 준다. 이는 현대사회에서 실존하는 인간이 법이라는 구성적인 제도 이전에 가치 있게 존재한다는 선재성---이것이야말로 어떤 중요한 ``귀족성''을 표현하고 있는 것 같다---을 표현하고 있기 때문이다. \textless 캐슬독트린\textgreater 이 존엄을 정당화하는 데 있어 우리에게 주는 근거로서의 성격은, 그것이 보호하려는 권리의 내용이 아니라, 정당화의 방법에서 드러난다. 뒤이어 살펴보듯, \textless 캐슬 독트린\textgreater 의 권리를 보호하는 성벽은 거만함으로 구성된 카스트적 성벽을 의미하는 것이 아니라 인간의 가치로 구성된 도덕적 정당화 근거를 의미한다.

헤르조그는 귀족적 지위로서의 존엄이 평등주의적 함축을 전달하기 어렵다고 생각하면서 월드론의 단순 귀족-지위론적 구성을 거부하기 위해 \textless 캐슬 독트린\textgreater 의 I, II 사례들을 언급했다. 헤르조그에 의하면 귀족지위적 개념으로서의 존엄은 존엄의 양도불가능성과 연관되어 있고, 다소의 거만함과 연결되어 있다. 여기에는 다른 일반인들과 함께 법에 따라 심판받기를 거부하는 종류의 거만함도 숨어있다는 것이다. 국가의 피소면제권도 마찬가지로 이러한 국가의 존엄 개념, 국가주권성 개념 등으로 설명될 수 있다. \textless 캐슬 독트린\textgreater 을 문언 그대로 적용하면, 집의 주인인 남편이 부인을 강간하는 사례도 범죄로 구성하기 어려워진다. 다시 말해 그는 주로 법적 책무의 면제 측면에서 존엄을 바라보고, 이런 것이 모든 이들에게 평등하게 적용될 수 없다고 말한다.

표면상 월드론은 헤르조그의 비판의 지점을 인정한다. 그는 ``단순 보편화는 레벨업 된 존엄의 새로운 세계로 들어가야 하는 것을 특성화하기에 충분치 않다''고 말한다. 그러나 그는 헤르조그의 사례가 왜 보편화되어서는 안 되는지에 대한 충분한 논증없이, ``채무에 의해서는 구속되지 않는다''는 법언을 모든 이에게 보편화시킨 그의 \textless 루트랜드 백작부인 사례\textgreater 를 들면서, 이는 법적 책무를 면제시켜주는 것이 아니라 법적 책무성(accountability)의 존중받는 형태를 일반화하는 것으로 보는 것이 더 낫다고 설명할 뿐이다.

월드론의 말대로 이러한 ``채무로 인한 구속으로부터의 면제''라는 귀족적 지위의 귀결은 법적 책무성에 포함되는 것인가 법을 뛰어넘는 것인가? 월드론이 말하는 존중받는 책무성의 가치는 어떤 근본적 가치에서 온 것이라기보다는 법문언이라는 권위로 환원되는 것 같다. 그런데 우리는 누군가가 채무때문에 구속되지 않는다라고 했을 때, 그의 인간적 가치(존엄)가 그러한 대우를 유발했다고 생각하는가 아니면 단순히 특정 국가의 특정 법문언이 그런 규정을 입법화했기 때문이라고 생각하는가? 당연히 우리의 인간적 가치(존엄)가 채무로 인한 구속을 배제시킨다고 생각할 것이다. 결국 존엄을 해석하는 귀족적 지위의 핵심에는 법을 넘어선다는 사실이 존재하며, 이를 빠뜨려서는 안 된다. 어떤 국가의 강제력도 나의 존엄이라는 성벽을 넘어서서는 안 된다는 것이다.

드워킨은 ``공동체의 목적을 달성하려는 정치적 결정에 대한 정당화를 이기는 으뜸패로서 권리가 가장 잘 이해된다''고 말한다. 어떤 사람이 포르노그라피를 출판할 권리가 있다면, 공무원들이 이를 금지하는 것은 공동체 전체에 이익이 된다 하더라도 이는 권리 침해로서 뭔가 잘못되었다(wrong)는 것이다.\footnote{Ronald Dworkin, ``Rights as Trumps'' in Jeremy Waldron(ed.), \emph{Theories of Rights} (Oxford, 1984), 153면.} 실정법적 권리들은 대부분 또다른 실정법적 제한이 따르거나, 권리 내재적 제약이 따르거나, 헌법(적)규범상의 제한이 따르기에, 어떤 권리가 실정법적 권리로 이해되는 한 이러한 으뜸패로서의 권리는 그렇게 큰 반향을 가져오는 것은 아니다. 그러나 어떤 권리가 도덕적 정당화로서 이해된다면, 으뜸패로서의 권리의 특성은 실정법체계에서 구별되는 힘을 가지게 된다.

인권을 존엄으로부터 정초하고 불가침적인 것으로 이해하는 논의는 특히 인권의 이러한 으뜸패로서의 특성을 가장 분명히 하는 것이다. 나는 이러한 으뜸패로서의 특성이 헤르조그의 걱정과는 달리 \textless 캐슬 독트린 II\textgreater---``나는 특별한 특권들을 향유하고, 너 같은 자들에게 답변할 필요가 없다''---에 고스란히 드러난다고 생각한다. 다만 특권의 정당화에서 차이가 있다. 귀족적 지위의 정당화가 카스트적인 계급론에서 나왔다고 한다면, 현대적 존엄 담론에서의 \textless 캐슬 독트린\textgreater 의 정당화는 도덕가치론적 정당화이다. 인간이 특별한 특권들을 향유하고, 어떤 법적 구성물에 의한 침해로부터도 면제되는 이유는 그 특권이 모종의 인간성 혹은 우리가 `존엄'이라고 부르는 것으로부터 유래되기 때문이고 그러한 종류의 인간성이 그러한 보호를 받을만한 가치가 있기 때문이다. 따라서 우리는 그러한 가치로부터 정당화되지 않는 특권을 도덕적으로 인정할 이유가 없고, 이로써 \textless 캐슬 독트린\textgreater 의 남용의 위험은 사라진다.

월드론이 단순 보편화되기 어렵다고 말했던 종류의 기이한 특권들은 이와 같이 도덕적 가치의 평가에서 탈락하기 때문에 그러한 특권이 평등화된 인간의 존엄에서는 계수되지 않는다. 한 사람의 인격이 차지하는 고유한 성(castle)은 어디까지인가의 문제도 이 지점에서 해명될 수 있다. 부부간 강간이 성립되는 이유는 캐슬 독트린이 무너진 것이 아니라, 도덕적으로 보호되는 성(castle)의 영역이 한정되기 때문이다.

이러한 설명은 왜 특정한 권리들이 정치적 결정 혹은 실정법이라는 구성물을 때로 이기는 지 잘 설명해 준다. 그 중 중요한 하나의 소극적 측면은 권리가 법에서 온 것이 아니라는 사실이다. 이러한 특권적 권리는 법이전에 이미 정당화된다. 이로부터 우리는 법의 강제로부터의 면제될 권한을 획득할 수 있다. 이러한 지위는 헤르조그가 말한 법으로부터 상위의 지위에 있는 지위라고 할 수 있다. 현재 우리가 생각할 수 있는 법으로부터도 면제받는 개념은 이러한 권리 밖에 없다. 본고는 이러한 함축을 제3부에서 도덕성의 객관적 중요성의 개념을 통해 전달할 수 있다고 주장할 것이다

\subsection{생명윤리 담론에서 인간존엄}

제1장과 제2장에서는 인권담론와 각국 헌법이론에서 인간존엄의 다양한 역할을 살펴보았다. 그런데 최근 인간존엄의 논의가 가장 눈에 띄게 등장하는 분야는 생명공학 및 의료분야의 윤리학이다. 이는 생명윤리의 담론이 인권담론과 밀접한 관계를 맺고 있기 때문이다. 유전공학, 인간향상기술을 비롯한 생명공학기술의 발달은 인류의 건강에 기여하는 바가 크기도 하지만, 역설적으로 그 기술의 개발과정이나 적용을 통해 인권과 인간성, 즉 인간존엄을 침해할 여지가 크다는 것이다.\footnote{Michael J. Sandel, \emph{The Case against Perfection: Ethics in the Age of Genetic Engineering} (Cambridge, MA, Harvard University Press, 2009), 24면.} 이는 뒤벨이 분류한 생명윤리에서 인간존엄의 역사를 보더라도 분명하다.

인간존엄이 생명윤리의 영역에서 사용된 역사를 마커스 뒤벨은 세계 제2차 대전이 끝난 후부터 세 단계로 나누고 있다.\footnote{Marcus Düwell, ``On the border of life and death: human dignity and bioethics'', \emph{The Cambridge Handbook of Human Dignity} (Cambridge, 2014), 527면. Roberto Andorno \& Marcus Düwell, ``Menschenwürdebegriff in der Bioethik'', in J. C. Joerden, E. Hilgendorf and F. Thiele (eds.). \emph{Menschenwürde und Medizin: Ein interdisziplinäres Handbuch} (Berlin: Duncker \& Humblot, 2013), 465-481면 참조.} 첫째 단계는 2차대전 직후 인간대상실험에서 인간을 보호하는 단계이다. 뉘른베르크 강령(Nuremberg Code)이 이를 대표하고 있다. 둘째 단계는 1970년대에서 1990년대 사이에 낙태, 유전자 진단, 안락사와 같은 행위에 대하여 인간존엄을 명시적으로 언급하기 시작하는 단계이다. 이는 주로 개인의 존엄을 중요하게 다루고 있다. 셋째 단계는 1990년대 말 이후로, 이때부터 인간종(human species)의 특성들을 향상시키려는 시도나 인간의 중심적 특성을 변경할 가능성이 있는 인간게놈의 조작에 관한 우려와 같이, 인간종의 존엄에 대한 언급이 두드러졌다. 이와 같이 생명윤리의 역사는 인권의 역사와 그 맥을 함께하고 있다.

이런 상황으로 인해 생명윤리영역에서 인간존엄을 보호되어야 할 핵심원리로 확인하는 수많은 법문서와 공식 보고서들이 작성되기도 했다.\footnote{예를 들어, (United States) President's Council on Bioethics Washington, \emph{Human Dignity and Bioethics: Essays Commissioned by the Presidents Council on Bioethics} (The President's Council on Bioethics, Washington DC, 2008); Convention for the Protection of Human Rights and Dignity of the Human Being with regard to the Application of Biology and Medicine: Convention on Human Rights and Biomedicine, CETS No. 164 (Oviedo Convention, 1997).Christopher McCrudden,``In Pursuit of Human Dignity: An Introduction to Current Debates'', 3면 참조.} 유네스코의 생명윤리와 인권에 관한 보편선언(Universal Declaration on Bioethics and Human Rights)은 인간존엄을 다루는 조항에서 ``과학이나 사회만의 이익보다 개인의 이익과 복지가 우선해야 한다''고 선언한다.\footnote{Universal Declaration on Bioethics and Human Rights(19 October 2005), Article 3(2).} 그러나 문제는 인간존엄의 이러한 사용이 이러한 사용에 반대하는 사람들의 발언권을 막고 논의종결자(conversation stopper)로 쓰일 수 있다는 점이다.\footnote{Dieter Birnbacher, ``Ambiguities in the concept of Menschenwürde'', in K. Bayertz (ed.), \emph{Sanctity of Life and Human Dignity} (Kluwer, 1996), 107면.} 이 분야의 많은 과학자들은 인간존엄의 사용을 부정적으로 보는 경향이 있다. 루스 맥클린은 ``인간존엄에의 호소가 의료윤리의 영역으로 이주하고 있다''고 지적하면서, 존엄 개념은 의료연구나 실천이 인간존엄을 위반하거나 위협한다는 주장에 일관성과 유용성이 없다고 비판한다.\footnote{Ruth Macklin, ``Dignity is a Useless Concept: It means no more than respect for persons or their autonomy'' \emph{British Medical Journal 327},1419면 (2003).} 존엄에의 호소는 다른 더 상세한 개념들에 대한 모호한 재진술이거나 그 주제에 대한 이해에 아무런 더함이 없는 단순한 슬로건에 불과하다는 것이다.

이러한 문제들로부터 인간대상의 실험에서 인간을 보호해야 하는 근거, 각 개인의 의료적 행위에서 중대한 선택의 문제를 존중해야 하는 근거, 그리고 인간종의 중요한 특성들이 함부로 훼손되지 않도록 해야 하는 근거로서 인간존엄의 의미를 분석하는 것은 중요한 작업으로 여겨지고 있다. 다음에서 생명윤리의 영역에서 가정될 수 있는 인간존엄의 의미들을 살펴보고자 한다.

\subsubsection{자유의 기초로서의 인간존엄}

\paragraph{충분한 정보에 의한 동의(informed consent)}

뒤벨이 분류한 인간존엄이 생명윤리의 영역에서 사용된 역사의 첫 단계, 즉 생명윤리에서 인간존엄 보호의 첫째 단계는 2차대전 직후 인간대상실험에서 인간을 보호하는 단계였다. 뉘른베르크 강령(Nuremberg Code)의 핵심은, 인간은 존엄을 가지기 때문에 그가 어떤 인간실험에 관련되게 되면 반드시 그에게서 충분한 정보에 입각한 동의를 받아야 한다는 것(informed consent, 이하IC, 혹은 동의원칙)이었다. 나아가 이 원칙은 단순히 인간을 대상으로 하는 실험 뿐만 아니라 의료적 치료 전반에 관철되어야 할 원칙으로 간주되어 왔다. 이후 수십년간의 생명윤리 논쟁사는 이 IC의 요청을 구체화하는 정당화 방법, 범위, 역할, 적용규칙에 관한 것이었다고 해도 과언이 아니다.\footnote{참고 Manson, Neil and O'Neill, Onora, \emph{Rethinking Informed Consent in Bioethics} (Cambridge, 2007); Deryck Beyleveld and Roger Brownsword, \emph{Consent in the law} (Oxford, 2007).}

그러나 이 원칙이 의료적 치료 전반으로 확장됨에 따라, 최근에는 이 IC원칙이 현실적이고 바람직한가에 대한 논란이 생기기 시작했다. 이 원칙이 의사의 소통기술과 환자의 이해수준에 대해서 설정하는 기준은 너무 높아서 때로 충족시키기 어렵다. 이는 결국 매우 번거로운 절차만을 남기게 되고 이 원칙이 애초에 달성하고자 했던 환자의 진심어린 동의의 기준을 충족시키기 어려울 수 있다. 또한 전통적인 의료가 수행해 왔던 역할을 넘어서는 미용적 성형이나 생명연장술과 같은 영역에서, 동의는 환자의 선호를 어디까지 충족시켜야 하는가의 문제와 관련되는데, 이 때의 동의는 전통적 의료행위에서의 동의가 가지는 의미와 사뭇 달라지게 된다. IC원칙이 추구하는 바는 환자의 모든 선호-충족을 요청하는 적극적인 것인가 아니면, 단지 원치않는 치료로부터 환자를 보호하는 데 있는 다소 소극적인 것인가? 만약 IC원칙이 인간실험을 수행하는 국가나 권력의 자의적 행사로부터 개인을 보호하고자 하는 인권 체제의 산물이며 인권은 인간존엄으로부터 도출된다는 사유와 그 필요성에 입각하여 도출된 원칙이라면, 일반적인 인권이론은 개인의 선호-충족 사안에서 IC원칙의 필요성을 강화해 주지는 않을 것이다. 따라서 우리가 인간존엄의 대표적 귀결을 IC원칙에 두더라도, 이러한 원칙을 개별 사안에 적용하는데 있어 우리가 해야할 일은 결국 우리가 생명윤리의 사안에서 가지고 있는 권리, 또는 인권이 무엇인가 하는 것이다.

\paragraph{연명치료중단 등에 대한 자기결정권}

미국의 생명윤리 맥락에서 인간존엄의 사용은 주로 뒤에 살펴볼 생명의 소중함으로서 이해되어 왔지만, 이와 대립적으로 개인의 자기결정권으로서 인간존엄을 이해한 사례도 그보다는 적은 수기는 하지만 존재하고 있다. 예를 들어, 낙태의 문제에 있어 일부 법률가들은 여성의 존엄이 국가의 간섭없이 임신을 자유롭게 중단할 능력에 의해 보호된다고 주장해왔다.\footnote{\emph{Planned Parenthood v. Casey}, 505 US 833 (1992); \emph{Gonzales v. Carhart}, 550 US 124 (2007).} 존엄사를 다루는 데 있어서 미연방대법원은 (생명연장수단을 포함하여) 원치않는 의료적 치료를 거부할 보호되는 자유이익을 인정하면서도, \emph{Washington v. Glucksberg} 사건에서 의사조력자살을 할 헌법적 권리는 존재하지 않는다고 설시했다.\footnote{\emph{Washington v. Glucksberg}, 521 US 702 (1997).} 이 사안에서 결론에 있어서는 다수의견에 동조했던 O'Connor 대법관은 ``이 자유는 특정한 종류의 원치않는 치료를 거부할 개인의 권리뿐만 아니라, 존엄의 이익을 포함하고, 죽음 이후에도 남아있을 기억들의 성격을 결정하는 이익을 포함한다\ldots{} 참기어려운 고통과 무기력하고 고통스럽게 살아있는 마지막 날들의 비존엄을 피하는 것은 확실히 `그 사람 자신의 존재, 의미, 우주, 인간 삶의 미스테리의 개념을 정의하는 \ldots{} 자유의 중심에' 있는 것이다''라고 말했다.

이러한 자율성으로서의 인간존엄을 취하는 입장에서는 높은 수준의 인간성으로서의 인간존엄을 이해하는 입장과는 다른 결론을 끌어내고 있다. 여성의 존엄은 낙태선택의 자유에 달려있으며, 환자의 존엄은 타인에게 극단적으로 의존해야 하는 삶을 끝낼 수 있는 자유에 달려있다는 것이다. 이러한 입장에서는 자유롭게 선택할 수 없는 상황이야말로 인간존엄을 침해하게 된다.

\subsubsection{높은 수준의 인간성으로서의 인간존엄}

앞서 살펴본 동의원칙과 자기결정권으로서의 인간존엄은 몇 가지 실천적인 전제에 입각해 있다고 보여진다. 하나는 자기의 일은 자기가 가장 잘 안다는 인식적 전제이고, 다른 하나는 자기의 일을 결정할 권위를 가진 자, 가장 잘 결정할 자는 자신이라는 민주적 전제이다. 이러한 전제들은 자율성을 도덕적 논의의 핵심에 놓는 전통들과 맞닿아 있다. 그러나 인간존엄을 둘러싼 여타의 중요한 사례들은 이러한 전제에 의문을 제기한다. 인간존엄과 관련된 사안에서 과연 자기자신은 타인에 대하여 인식적 우위를 갖고, 그 결정에 대한 민주적 권위를 갖는가? 이와 관련하여 의사들에게는 환자에 의해서 동의된 안락사 혹은 조력자살을 지원할 것인지의 문제가 지속적으로 제기되어 왔다. 이 때 인간존엄은 때로 동의원칙의 한계로 제시되기도 한다. 즉, 이 때 인간존엄은 인간의 자율적 행위에 근본적인 한계를 설정하고 일종의 금기를 형성하는 것이다.\footnote{Emile Durkheim, \emph{Le suicide: Etude de sociologie} (Paris: Alcan.1897) (English trans. J.A. Spaulding and G. Simpson, Suicide: A Study in Sociology. Glencoe, IL: The Free Press, 1951).} 인간존엄은 자율적 행위선택보다는 생명의 신성함(sanctity of life)에 더 관계되고, 이는 인권 담론의 맥락보다는 특수한 종교적 교리에 더 토대를 두고 있는 것으로 보인다.\footnote{Kurt Bayertz (ed.), \emph{Sanctity of Life and Human Dignity} (Dordrecht: Springer, 1996); Marcus Düwell, \emph{Bioethics -- Methods, Theories, Domains} (London, New York: Routledge, 2012), 128-34면 참조.}

따라서 생명윤리와 관련된 법적 규제의 영역에서도 헌법담론에서와 마찬가지로 `인간존엄'을 인간이 가지고 있는 어떤 높은 수준의 인간성으로 이해하려는 태도들이 감지되는 것 같다. 로버트 스트라이퍼는 키메라 연구를 규제해야 할 철학적 근거들 중 하나인 ``인간존엄 논변''을 대표적으로 다음과 같이 소개한다. 그에 의하면, 인간존엄 논변은 `인간존엄'이 지칭하는 대상의 문제에 관해, 인간존엄이 일종의 ``무조건적이고 비교불가능한 가치''라고 설명하고, 따라서 존엄을 가진 개별자들은 ``특별히 가치 있고 존중받을 만하다''는 주장이다.\footnote{Robert Streiffer, ``Human/Non-Human Chimeras''; Philip Karpowicz, Cynthia B. Cohen, Derek van der Kooy, ``Developing Human-Nonhuman Chimeras in Human Stem Cell Research: Ethical Issues and Boundaries''~\emph{Kennedy Institute of Ethics Journal}, 15(2) (2005), 119-120면.} 나아가 이 논변은, 인간존엄은 어떤 존엄을 기초하는 능력들을 보유하기 때문에 존중받을 가치가 있다는 전제 하에서,\footnote{인간존엄을 어떤 능력들을 보유하는 문제와 연관시키는 이러한 전제까지도 현대적 패러다임에 포함시킬 것인지는 논란이 있을 수 있다. 이는 올리버 센슨이 제공하고 있는 현대적 패러다임과는 멀어진다. 그는 어떤 능력을 보유하기에 가지는 존엄을 실현적 존엄(realized dignity)와 구분하여 `시원적 존엄(initial dignity)'로 분류하고 이는 실현되거나 버려질 수 있는 것이 아니며, 이러한 이중적 관념(two-folder notion)을 가지는 것은 전통적 패러다임에서 나타나는 것이라고 주장하고 있기 때문이다. Oliver Sensen, \emph{Kant on Human Dignity}, 162-163면.} 그러한 존엄-기초 능력들의 발달, 유지, 행사를 제약해서는 안된다는 추정이 성립하고, 이러한 발달, 유지, 행사를 지원할 적극적 의무까지도 있다고 주장할 수도 있다고 말한다. 스트라이퍼는 인간-비인간 키메라를 창조하는 생명공학적 문제를 염두에 두고 있는데, 인간존엄 논변이 함축하는 것은, 비인간들에게 인간존엄과 관계된 능력 발달에 필요한 물리적 구성요소를 부여함으로써, 그리고 이러한 구성요소들을 비인간육체에 집어넣음으로써, 인간-비인간 키메라를 창조하는 사람들은 인간존엄을 손상시키기에 이러한 연구는 금지된다는 것이다.\footnote{Robert Streiffer, ``Human/Non-Human Chimeras''.}

아래에서는 생명윤리의 문제에서 인간존엄을 이상과 같이 높은 수준의 인간성으로 이해했을 때, 어떤 구체적인 의미를 가지는지 그 개념의 후보자들을 탐색해보고자 한다.

\paragraph{생명의 소중함의 기초}

대부분의 국가에서 살인죄를 처벌하고 있다는 점에서 생명의 소중함은 어느 사회에서나 인정되는 가치이다. 또한 대부분의 사회에서는 자살---특히 자살방조---도 금지함으로써 자신의 선택에 의한 생명단절조차 중대한 가치를 침해하는 것으로서 국가가 후견적으로 간섭할 필요가 있는 것으로 제도화하고 있다. 이에 더하여 극심한 통증을 호소하는 환자들의 의사조력자살의 선택이나, 임신부의 자기 신체에 대한 자기결정권에도 불구하고 태아의 생명을 단절시키는 행위를 일반적으로 금지하는 각국의 낙태금지의 경향에 비추어 볼 때, 생명은 자기결정권이나 복지를 누릴 삶의 가치를 넘어서는 특별한 가치로 이해되고 있는 것으로 보인다. 그리고 이러한 특별한 가치의 위상을 논증하기 위해 생명은 인간존엄과 밀접한 관계를 가지고 있는 것으로 흔히 이야기된다.

\subparagraph{\texorpdfstring{ 낙태 및 인간배아의 파괴조장 금지}{ 낙태 및 인간배아의 파괴조장 금지}}

생명의 소중함으로써의 인간존엄의 개념은 미국의 공공생명윤리의 맥락에서도 자주 등장하고 있다. 예를 들어, 줄기세포를 이용한 연구가 각광받기 시작한 이래로 이러한 줄기세포연구를 보조하기 위한 재정 정책이 마련되어왔는데, 2006년 부시행정부는 이러한 재정정책이 살아있는 인간배아의 파괴를 조장하지 않도록 엄격히 통제된 배아줄기세포 연구에 대해서만 재정지원을 하도록 정책을 이끌어 나갔다. 당시 의회는 이러한 엄격한 정책을 완화하고 자유로운 연구를 촉진하려는 법률안을 제시하였고, 부시행정부는 ``인간 생명을 조작하고 인간존엄을 침해하려는 경향''에 대해 경고하면서 이 법률안에 대한 거부권을 행사하기도 했다.\footnote{Veto Letter from President George W. Bush to Congress (19 July 2006).}

다능성세포의 비-배아적 대체원천과 관련된 연구 수행을 지시하는 국립보건원령에서도 당시 부시대통령은 ``최상의 윤리적 규준을 유지하고 인간 생명과 인간존엄을 존중하면서, 의료 연구와 함께 국가가 앞으로 나가도록 할 수 있는 도덕적 윤리적 한계선을 확립하는 것은 중요하다''고 선언했다.\footnote{George W. Bush, ``Expanding Approved Stem Cell Lines in Ethically Responsible Ways'', Exec. Order No. 13,435,72 Fed. Reg. 34951 (20 June 2007) 참고로 이는 2009년 버락 오바마 정부의 행정명령에 의해 폐기되었다. Barack Obama, ``Removing Barriers to Responsible Scientific Research Involving Human Stem Cells'', Exec. Order No. 13,505 (9 March 2009).} 낙태와 관련해서도 태아의 법적 보호를 강조하는 측에서는 인간존엄의 언어를 반복적으로 사용했다. 특히 2003년의 부분출산낙태금지법에서는 살아있는 태아를 부분적으로 출산시켜 낙태하는 특정한 낙태방법을 금지했는데,\footnote{The Partial Birth Abortion Act of 2003.} 연방대법원은 ``이 법이 인간 생명의 존엄을 존중을 표현한 것''이라고 설시했다.\footnote{\emph{Gonzales v. Carhart}, 550 US 124 (2007): ``the Act expresses respect for the dignity of human life''.}

\paragraph{인간종의 정체성 변경 금지의 기초}

인간존엄이 인간종이 가지고 있고 그 정체성을 형성하는 어떤 고유한 속성이며 이러한 정체성을 훼손하거나 다른 종으로 전이시키는 것을 금지하는 근거로 이해하려는 입장이 있다. 대표적인 것은 키메라 규제와 향상기술의 규제 문제라고 할 수 있다. 우리나라의 생명윤리 및 안전에 관한 법률도 키메라 규제(이종 간의 착상 금지)를 인간존엄의 표제 아래 다음과 같이 대부분 절대적 금지사유로 규제한다.

생명윤리 및 안전에 관한 법률(2016)

제4장 배아 등의 생성과 연구

제1절 인간존엄과 정체성 보호

제20조(인간복제의 금지) ① 누구든지 체세포복제배아 및 단성생식배아(이하 "체세포복제배아등"이라 한다)를 인간 또는 동물의 자궁에 착상시켜서는 아니 되며, 착상된 상태를 유지하거나 출산하여서는 아니 된다.

② 누구든지 제1항에 따른 행위를 유인하거나 알선하여서는 아니 된다.

제21조(이종 간의 착상 등의 금지) ① 누구든지 인간의 배아를 동물의 자궁에 착상시키거나 동물의 배아를 인간의 자궁에 착상시키는 행위를 하여서는 아니 된다.

② 누구든지 다음 각 호의 행위를 하여서는 아니 된다.

1. 인간의 난자를 동물의 정자로 수정시키거나 동물의 난자를 인간의 정자로 수정시키는 행위. 다만, 의학적으로 인간의 정자의 활동성을 시험하기 위한 경우는 제외한다.

2. 핵이 제거된 인간의 난자에 동물의 체세포 핵을 이식하거나 핵이 제거된 동물의 난자에 인간의 체세포 핵을 이식하는 행위

3. 인간의 배아와 동물의 배아를 융합하는 행위

4. 다른 유전정보를 가진 인간의 배아를 융합하는 행위

③ 누구든지 제2항 각 호의 어느 하나에 해당하는 행위로부터 생성된 것을 인간 또는 동물의 자궁에 착상시키는 행위를 하여서는 아니 된다.

앞서 삶과 죽음을 둘러싼 개인의 삶과 관련된 생명윤리에서의 인간존엄의 문제들을 살펴보았다면, 이와 같은 법률에서 다루고 있는 인간존엄의 문제는 보다 집단적인 차원에서 제기되는 인간종의 정체성을 보호하는 것과 관련된 문제라고 할 수 있다.

또한, 현대 생명공학의 발전은 인간종의 생물학적 한계를 초월하거나 소위 `향상(enhancement)'시킬 수 있는 기술을 개발하고 있다. 과연 인간은 인간종의 향상을 위한 기술을 개발해도 좋은가? 더 나아가 그러한 기술을 개발할 의무가 있다고 볼 수 있는가? 향상기술을 둘러싸고 이러한 기술이 인간존엄을 위반하는지에 대한 논쟁이 격렬히 전개되고 있다.

많은 생명공학-보수주의자들은 인간의 본성을 수정하는 기술을 사용하는 것에 대해 일반적으로 반대하고 있다.\footnote{대표적 학자로, 레온 카스(Leon Kass), 프란시스 후쿠야마(Francis Fukuyama), 조지 애나스(George Annas), 웨슬리 스미스(Wesley Smith), 제러미 리프킨(Jeremy Rifkin), 빌 매키븐(Bill McKibben) 등.} 이들은 미끄러운 경사길 논변을 사용하여 향상 기술의 사용이 오히려 인간존엄을 약화시킨다고 주장하기도 한다.

이와 반대로 닉 보스트롬(Nick Bostrom)은 이러한 생명공학-보수주의자들에 반대하여 트랜스휴머니즘 내지 포스트휴먼주의 운동에 의해 옹호된 인간향상과 관련한 기술이 널리 이용 가능해야 하며, 그 수단이나 목적이 인간존엄이나 그로부터 도출된 근본적 인권을 위협하지 않는다는 그의 입장을 ``포스트휴먼 존엄의 옹호''라는 제목의 논문에서 밝히고 있다.\footnote{Nick Bostrom, ``In Defense of Posthuman Dignity'', \emph{Bioethics} 19(3) (2005), 202-14면.}

인간 본성은 무엇이고 이는 인간존엄과 어떻게 관계되는가? 생명공학-보수주의자들의 주장대로 인간 본성의 수정, 즉 향상은 그 자체로 인간존엄을 침해하는 것인가? 아니면 생물학적 인간본성(biological human nature)을 변형하려는 의지야말로 진정한 인간의 본질본성(essence)인 자유를 표현하는 것은 아닌가?

향상기술을 둘러싼 논쟁을 통해 인간존엄에 대한 기존의 설명들은 포스트휴먼주의와의 대립구도에서 인격주의, 자연주의, 그리고 포스트휴먼주의의 세 가지 설명으로 정리될 수 있을 듯 하다.\footnote{Martin G. Weiss, ``Enhancement or Post Human Dignity'', \emph{The Cambridge Handbook of Human Dignity} (Cambridge, 2014).}

a. 인격주의: 인격에는 등급이 없다는 것을 의미하는 절대적 개념으로 인간존엄을 이해하는 이 입장에서는, 인격을 손상시키지 않는 한 인간존엄에는 아무런 영향을 미치지 않는다.

b. 자연주의: 이 입장은 인간의 자연적 특징---`genetic asset' (Fukuyama), `natural origin'(Habermas) or `openness to the unbidden' (Sandel)---으로 존엄을 확인한다. 향상기술을 인간본성의 거만한 조작으로 이는 소위 `신놀이(playing god)'라고 비난한다.

c. 포스트휴먼주의: 향상기술은 거만한 조작이 아니라 신과의 `공동-창조(Co-Creation)'로서 인간이 신을 닮았음을 표현하는 것이다. 이는 르네상스의 인문주의와 계몽주의를 계승하는 것이다.

자연주의와 같은 논의가 인간존엄에 관한 보수적 논의와 닮아 있다면 포스트휴먼주의의 논의는 생명공학기술의 발전을 옹호하는 보다 자유주의적인 입장을 지원한다. 이러한 입장의 뿌리를 보스트롬은 피코델라미란돌라로 대변되는 르네상스의 휴머니즘과 자연본성으로부터 인간을 해방시킨 칸트적 계몽주의에서 찾고 있다.\footnote{Nick Bostrom, ``In Defense of Posthuman Dignity'', 203면.} 과거에도 인류가 교육이나 종교를 통해 인간을 개발시키고 변화시키려는 노력은 지속되어 왔지만, 이때까지도 생물학적 근간이 흔들릴 수 있다는 생각은 하지 못했다. 적어도 생물학적 안정성은 인간의 영원한 본성 혹은 본질을 의미하는 것처럼 여겨졌다. 그러나 트랜스휴머니즘을 주장하는 학자들은 생명공학의 발전 이후, 이 마지막 안정성까지 사라졌고 영원한 본성에 대한 믿음은 사라졌으며 생물학적 객관성 역시 존재하지 않는다는 것을 받아들여야 할 시점이 왔다고 말한다.\footnote{Martin G. Weiss, ``Enhancement or Post Human Dignity''.} 인간임을 유지시켜주는 자연스러운 경계는 허물어졌고 생물학적 안정성은 결코 인간임(being human)을 담보하지 못한다는 것이다. 그러나 생명공학의 발전은 인간존엄의 위기를 말해주는 것이 아니라, 마침내 르네상스와 계몽주의가 꿈꾸던 자유로운 존재로 인간이라는 동물을 변모시킬 것이라는 희망을 준다는 것이다. 트랜스휴먼주의자들에게 있어, 이러한 기술의 발달은 건강하고 활동적인 삶을 연장시키고, 지적인 능력을 증진시키며, 다른 사람과 함께 살아갈 수 있는 정서적 능력을 고양시켜주며 이러한 중심적 특성의 향상은 바람직하다.\footnote{이는 마틴 바이스가 말한 세가지 인간의 중심적 능력들인 `healthspan', `cognition', `emotion'을 설명한 것이다. Ibid.}

결국 인간향상 기술에 있어 인간존엄의 의미를 해명하면서 우리가 해야할 질문은 ``과연 포스트휴먼, 트랜스휴먼이 되는 것이 좋은가?''라고 할 수 있다. 과연 포스트휴먼, 트랜스휴먼이 지향하는 가치는 진정한 인간의 가치이며 그렇게 되어야 할 우리의 의무를 유발하는 그러한 가치인가? 생명공학기술을 둘러싼 인간존엄의 논의는 이 문제를 해명하는 데 있다. 뒤벨이 잘 지적하였듯, 인간존엄은 나치강제수용소에서의 잔혹행위로부터 개별 인간들을 보호하는 개념이기도 하지만, 이제는 광범위한 기술발전에 관한 규범적 지향을 제공해야 하는 개념이 되었다.\footnote{Marcus Düwell, ``On the border of life and death: human dignity and bioethics'', 533면.} 인간종과 관련된 집단적 개념으로서 인간존엄이 어떻게 운영되는 것이 좋을지에 관하여 더 심도깊은 논의가 필요하며, 특히 이는 국제적 인권프레임 안에서 적절히 이해되어야 할 필요가 있다.

\subsection{소결: 법적 담론에서 인간존엄 이해의 문제점과 해결과제}

이상 제2부의 논의를 통해 기존의 법적 담론에서 인간존엄이 어떻게 사용되어 왔는지를 분석해 보았다. 이러한 논의들은 인간존엄에 대한 기존의 이해방식---인권담론에서 다양한 인권들을 정초하고, 또한 법적 가치와 권리로서 기능하며, 다른 가치나 권리에 대하여 형량불가능한 효력을 가지는, 법적으로 보호가치 있는 어떤 인간성으로서의 인간존엄의 이해방식---을 담고 있다. 그러나 이러한 이해방식은 여러가지 모순과 난점들을 수반하며, 또한 법담론에서 사용되는 인간존엄의 사용의 국면들을 모두 포착하지 못하고 있다. 다음에서는 기존의 이해방식의 난점과 인간존엄의 올바른 이해방식이 해결해야 할 과제를 제시하고자 한다.

\subsubsection{맥락-관련적 인권 또는 기본권 도출에 있어서의 문제}

현대 인권 담론이 인간존엄의 이해방식의 확립에 거는 기대는 아마도 인간존엄이 구체적인 인권주장들을 발생시키는 논리적 기초가 된다는 강건한 정초주의의 입장에 가깝다고 할 수 있을 것이다. 또한 사람들은 뒤벨이 우려한 바와 같은 맥락-관련적 위협에 대응할 수 있는 유연하고 다양한 권리들을 산출하는 근거가 되어주기를 바란다.

문제는 이러한 인간존엄의 인정이 기존에 확립되어왔던 인권과 기본권의 이론체계를 무색하게 만들 수 있다는 사실이다. 인간존엄 개념은 지나친 추상성과 개방성을 내재하고 있기에 권리의 개념을 형해화시키고 기존의 논의를 잠식시키는 논의종결자(conversation stopper)가 될 수 있다. 특히 ``모든 해석이 전체 법체계의 관점에서 지향해야 할 최종적 목적''으로 이해되는 추상적인 인간존엄의 개념이 사법기관의 법해석작용에 의해 남용될 경우 입법작용이 부여한 재량범위를 넘어서는 사법기관의 법형성이 무분별하게 발생할 가능성이 상존한다.

이러한 가능성을 의식한 듯, 월드론은 인간존엄의 인권정초기능을 이미 현존하고 있는 특정한 권리 조항들을 해석하는 것으로 축소시키고자 시도한다.\footnote{Jeremy Waldron, ``Is Dignity the Foundation of Human Rights?'', 131면 .} 그러나 인간존엄이 권리를 창조하는 역할을 하지 못하고, 단지 기존에 존재하는 규범의 이해를 돕는 해석적 기능만을 수행하게 된다면, 기존 권리의 해석만으로는 도출하기 어려운 인간성의 핵심적 부분들을 보호할 권리의 누락이 발생하는 것을 피할 수 없게 된다. 또한 이러한 해석은 현대사회가 인간존엄에 거는 기대에 한참 미치지 못하며, 인권의 `정초'라는 말에 부여된 역할을 수행하지 못하는 것으로 보아야 한다.\footnote{Christopher McCrudden, ``Human Dignity and Judicial Interpretation of Human Rights'', 679-680면 참조.}

따라서 인간존엄의 이해방식은 현대사회의 위협에 대응하는 맥락-관련적 인권을 풍부하게 산출하는 기초가 되면서도 인간존엄의 해석적 기능의 확장이 사법의 월권적 법형성을 방조하는 트로이목마가 되지 않도록 하는 그런 이해방식이 되어야 한다. 본고는 제3부에서 인간적으로 좋은 삶을 사는 것의 객관적 중요성으로서의 인간존엄과 이에 대한 믿음형성의 이익을 의미하는 인간존엄-이익의 이원적 구성을 통하여 이러한 이해방식을 수립하고자 한다.

\subsubsection{인간존엄은 단지 누락된 권리를 의미하는가?}

인간존엄은 그 자체로 권리이거나 인간존엄을 보호하는 특수한 권리들이 존재한다고 한다. 또한 이러한 인간존엄 관련적 권리들은 다른 권리들을 도출하는 포괄적 권리라는 견해도 존재한다. 이 견해들을 동시에 수용하는 것은 쉽지 않은 일이다. 한편으로는 인간 생활의 특수한 양태를 보호하는 근거로 인간존엄을 이해하면서, 동시에 다른 한편으로는 인간 생활의 모든 양상을 아우르는 것으로 인간존엄을 이해해야 하기 때문이다. 이를 수용하는 한 방법은 특수한 생활양태를 보호하는 것처럼 보이는 특수한 권리로서의 인간존엄이 사실 다른 모든 권리들에 의해 보호되고 남는 것들을 모두 아우르는 보충적 권리였다는 점을 인정하는 것이다.

그러나 단지 권리들을 보충하는 기능으로만 인간존엄을 이해하는데는 석연치 않은 부분이 있다. 인간존엄(으로부터 도출된 권리)은 다른 권리들에 비하여 일정 이상의 우선성을 가지는 것으로 인식되어 왔기 때문이다. 단지 누락된 권리로서의 인간존엄은 현대적 요청이 가지는 이러한 특별한 우선성의 함축을 수용하지 못한다.

그렇다면 인간존엄은 포괄적 권리로서 누락된 권리의 보충적 기능을 하기도 하지만, 한편에서는 추가적이고 독립적인 특수한 권리로서 이원화되어 사용되고 있다고 보아야 한다. 그렇다면 이러한 추가적 권리는 제2부에서 살펴본 권리를 가질 권리나 고문이나 홀로코스트와 같은 매우 심한 체계적이고 구조적인 침해와 같은 것을 의미하는 것인가? 앞서 살펴본 권리의 후보자들은 인간존엄에 의한 특별한 보호가 필요하다고 여겨져왔던 사안들을 모두 포괄하지 못하는 개념들이라고 할 수 있다.

제3부에서는 이러한 독립적 인간존엄의 개념은 메이어 댄-코헨이 제시한 의미의존성을 충족시켜야 한다는 점을 살펴보고, 이를 만족하는 이익으로 좋은 삶의 객관적 중요성에 관한 중심부 믿음형성의 이익(인간존엄-이익)을 다루게 될 것이다. 이 이익은 단지 누락된 권리가 아닌 추가적인 특수한 이익으로서 인간존엄성에 요청되는 특별한 우선성의 함축을 수용하는 개념이라는 것이 밝혀질 것이다.

\subsubsection{인간존엄은 형량가능한가?}

끊임없이 제기되는 문제중 하나는, 인간존엄의 우선적이고 중대한 함축이 사법적 판단에 있어서도 절대성이나 형량불가능성과 같은 구속력을 가져오느냐는 것이다. 독일기본법 제1조의 ``감히 건드릴 수 없다''는 어구를 해석하는 데 있어 많은 학자들은 인간존엄이 절대성을 가지고 있다는 점을 일단 인정하고 있다. 독일기본법에서 인간존엄과 그로부터 도출된 권리는 절대적이어서, 이는 형량불가능하다는 것이다. 다만 이러한 구속력에 대한 부담과의 현실적 타협으로 인간존엄에 의해 보호되는 영역의 의미범위를 최대한 좁게 해석해야 한다고 주장한다.\footnote{Dieter Grimm, ``Dignity in a Legal Context: Dignity as an Absolute Right'', \emph{Understanding Human Dignity} (Oxford, 2013) 참조.}

그러나 인간존엄의 불가침적이고 절대적인 함축은 비단 독일기본법의 규정방식에서만 비롯되는 것은 아닌 것으로 보인다. 인간존엄을 실정법 이전에 근현대를 아우르는 국가권력과 공동체를 지도하는 근본적 이념으로 보고,\footnote{더 나아가 인간존엄을 천부인권, 전국가적 자연권으로 이해하고 있는 견해로는 김철수, 헌법학신론, 424-425면.} 이 때 의존하는 칸트철학의 다른 어떤 가치와도 비교할 수 없는 인간존엄의 우월적 지위를 함께 받아들이는, 다시 말해 `불가침성 도그마'를 받아들이는 현상이 우리 헌법재판소와 여러 학설들에서 발견되고 있기 때문이다.\footnote{헌법재판연구원(최규환), 인간존엄의 형량가능성 (2017), 52면.} 그러나 독일 학설이 인간존엄의 의미범위를 축소하고, 독일연방헌법재판소가 구체적 사례에서 직접 인간존엄의 위반으로 위헌을 선언하는 것을 자제하고 있는 태도에서 짐작할 수 있는 바와 같이 인간존엄의 형량불가능성은 권리체계의 논의에서 감당하기 어려운 측면이 있다. 또한 앞서 살펴 본 인간존엄을 매개로 한 사법의 월권적 법형성의 위험을 더욱 증폭시키는 경향이 있다.

본 논문은 인간존엄의 절대성 함축에서 오는 형량불가능성에 대한 부담의 일부는 오도된 것으로서, 이는 `인간존엄'이 법적 맥락에서 사용되는 의미와 국면을 정확히 구분하지 못했기 때문이라고 파악한다. 인간존엄의 대표적인 법적 기능은 모든 인간들의 인간적으로 좋은 삶에 대한 평등한 존중을 정당화해 주는 역할이다. 이 역할로부터 여러 인권들이 도출되고 현존하는 권리들이 해석된다. `인간존엄'이 이러한 역할을 서술하기 위해 사용될 때, 인간존엄은 형량의 대상이 되는 어떤 실현적 존엄을 지칭하는 것이 아니라 법적 논증에서 이미 전제되어 있는 ``인간적으로 좋은 삶을 사는 것은 객관적으로 중요하다''는 사실을 지칭한다. 이러한 논증-독립적 사실은 형량해서는 안되는 것이 아니라 형량 자체가 불가능한 어떤 것이다. 이는 시원적 존엄(initial dignity)의 전통을 따른 용법에 해당한다. 이러한 이해로부터 인간존엄으로부터 도출된 여러 권리들은 위와 같은 논증독립적 사실로서의 인간존엄과는 구별되는 것으로써, 일반적으로 형량가능하다고 보아야 한다.

이러한 설명은 불만족스러울 수 있다. 나치의 홀로코스트 사례, 고문금지, 키메라 연구 금지의 사례와 같은 특수한 사안들에서, 우리는 인간존엄이 다른 이익에 비하여 특별한 중요성을 가지고 기능하리라는 직관과 소망을 가지고 있고, 이러한 소망을 투영하고 있는 것이 현대 인권담론이 인간존엄에 부여하고 있는 역할이라고 할 수 있기 때문이다.

이러한 불만족은, 절대성의 함축을 도덕성의 객관적 중요성이라는 법적 논증에서의 독립적인 전제사실로 이해하는 것과 특수한 사안들에서 제기되는 인간존엄에 대한 우선적 보호의 함축이 구별된다는 점을 파악함으로써 해소될 수 있다. 이러한 특수한 사안들은 좋은 삶의 객관적 중요성에 관한 믿음형성의 이익의 보호를 요청하는 사안으로서, 이러한 믿음이 단지 주변부 믿음이 아니라 중심부 믿음에 해당할 때 특별한 우선성을 가지게 된다. 다만 이러한 우선성이 법적 논증에서의 독립적인 전제사실과 같은 형량불가능한 절대적 불가침성을 의미하는 것은 아니다.

인간존엄 이해방식을 재구성하고 있는 제3부의 논의를 통해 인간존엄이 가진 형량불가능성 난제는 이와 같이 해소될 것이다.

\subsubsection{인간존엄은 법적으로 보호가치 있는 특유한 인간성인가?}

제2부에서 살펴본 법적 담론에서 인간존엄의 기능과 효력은, 인간존엄의 개념에 대한 어떤 이해방식을 전제하고 있다는 사실을 살펴보았다. 이러한 대표적 이해방식은 먼저 인간존엄을 법적으로 보호가치 있는 특유한 인간성으로 파악하는 것이었다. 그리고 다른 하나는, 최근 대안적으로 등장하고 있는 다른 하나는 시민들이 가지는 권리들을 보유하는 시민적 지위로 파악하는 것이었다. 이는 일반적으로 기대하고 있는 인간존엄의 기능과 효력을 설명하기에 좋은 개념적 기초가 되고 있는가?

위에서 이러한 인간성으로 제시된 여러 후보들---최소한의 생계보장, 평등, 품위, 그리고 자유 등---은 다양한 인간존엄의 기능을 포괄하면서도 우선적이고 특별한 보호를 요청한다는 것을 충분히 정당화하고 있지 못하다는 점을 살펴보았다. 이에 대안적 견해로 제시된 귀족적 지위의 평등주의적 확장으로 이해된, 현존하는 권리들을 보유할 시민적 지위로서의 인간존엄 역시 다양한 새로운 인권들을 정초하는 기능을 충분히 설명하지 못하고 있다.

다만 이러한 시민적 지위로서의 인간존엄을 이해해보려는 시도가 시사해주는 바가 있다. 과거에 귀족들이 향유했던 지위의 핵심에는 실정법 이전에 선재하는 어떤 고귀함이 있었고 이를 일컬어 `존엄'이라고 불렀다는 사실이다. 제3부는 이러한 고귀함의 함축이 도덕성의 객관적 중요성, 다시 말해 각 개인이 인간적으로 좋은 삶을 사는 것의 객관적 중요성을 의미한다는 점을 밝힐 것이다. 이러한 해명은 기존의 이해방식이 가졌던 앞서의 문제들을 해결하는 개념적 토대가 될 것이다.

\section{제3부 인간존엄 이해방식의 재구성}

제3부의 제1장과 제2장에서는, 제1부에서 제기된 인간존엄의 철학적 이해방식의 한계와 제2부에서 분석된 법적담론에서의 인간존엄 이해방식의 문제점과 해결과제를 염두에 두고, 이들 각각의 이해방식을 비판적으로 재구성해보고자 한다. 그리고 마지막 제3장에서 본고가 제시하고 있는 이원적 이해방식이 실제의 구체적 사안들에 어떻게 실현될 수 있는지 그 적용방안을 제시해보고자 한다.

\subsection{인간존엄의 철학적 이해의 재구성}

본고는 존엄의 고대적 어원과 현대적 법규범에 공통된 개개인에 대한 가치부여라는 이상에 호응하면서도, 근원적으로는 상대적인 문화에 좌우되지 않는 보편적인 방식으로 도덕적 정당화를 수행하는 반성적 정당화의 방법론을 채택하여 인간존엄을 정당화함으로써, 법담론에서의 인간존엄을 굳건한 도덕철학의 토대 위에 세우고자 시도한다. 이러한 정당화의 방식에 가장 부합하는 인간존엄의 이해방식은, 다음에서 살펴보고자 하는 드워킨의 좋은 삶에 대한 이론에서 출발하는 인간적으로 좋은 삶의 객관적 중요성 관념이라고 할 수 있다. 먼저 존엄의 어원과 이러한 관념이 어떻게 관계될 수 있는지 그 가능성을 살펴보고자 한다.

\subsubsection{존엄의 어원: 우월성으로서의 존엄}

각종 인권문서나 각국의 헌법과 사법적 결정에서 언급된 존엄에 대한 현대적 이해방식이 인간에게 부착된 어떤 평등한 가치속성을 지칭하고 있다는 유력한 해석에도 불구하고, ``고귀한 계급과 높은 공무에 대한 그것{[}존엄{]}의 고대적 관련성과의 신의를 어떻게든 유지하기 위해 노력해야 한다''는 월드론의 취지에는 공감할 바가 있다.\footnote{Jeremy Waldron, \emph{Dignity, Rank, \& Rights}, 30면.} 또한 본고는 이러한 신의유지가 인간존엄의 의미를 부여하는 데 모종의 열쇠가 될 수 있다고 생각한다.

또한 이러한 신의를 유지하려는 노력이 반드시 월드론의 여러 저작에서 나타나는 전략처럼 예전의 귀족이 향유했던 구체적인 법적 지위---예를 들어 시민권---의 확장으로서 인간존엄을 이해하는 방식으로 수행되어야 하는 것은 아니다. 지위는 법적 지위로서 그 법률효과를 수반하는 자의 법적 권한을 약술하는 단순한 명칭(abbreviation)으로도 사용되지만, 그 이전에 사전적 의미로서 어떤 상대적인 위치의 개념을 설명하기도 한다. 옥스포드의 영어사전은 `status'를 ``1. Relative social or professional position; standing''이라고 정의한다. 이 사전적 정의는 비록 사회적, 직업적 위치에 한정하고 있지만, 우리의 실제 용례로 볼 때 `지위'는 아마도 기준적 위치에 대하여 그와 이질적인 위치에 있음을 일반적으로 통칭하는 것으로 보인다. 법적 지위에 있어 보통사람(normal person)의 기준적 위치가, 예를 들어 범죄자의 지위에 대해 범죄를 저지르지 않은 자의 위치, 특정 물건에 대한 소유권을 가진자의 지위에 해당 소유권을 가지지 않은 자의 위치라면, 그에 대한 범죄자, 소유권자처럼 이질적인 위치에 있다는 것을 `지위'라는 용어가 표상한다는 것이다.

칸트에게 있어 이러한 지위로서의 존엄의 관념은, 앞서 올리버 센슨의 분석으로부터 살펴보았듯, 구체적인 개별 인간이나 그 인간에게 부착되는 어떤 귀족적인 법적 지위와 같은 것만이 아니라 인간성의 핵심을 이루는 도덕성에 부착되는, 다른 관념들에 대한 우월성으로 특징적으로 표현되고 있다. 따라서 현대적 의미의 존엄이 지위의 개념으로부터 계승하는 것은 개별 인격이 향유하는 어떤 법적 지위의 의미라기보다는, 칸트에게서 그 근거를 찾을 수 있는 도덕성의 다른 모든 가치에 대한 우월성의 의미라고 보아야 한다.

본고는 이러한 전통적 이해방식으로 바라본 칸트의 입장을 계승한다. 단락을 달리하여 도덕성의 우월성으로서의 존엄의 개념을 살펴보자.

\subsubsection{도덕성의 우월성}

본고는 일정한 지위적 함축이 현대의 `인간존엄'의 사용에도 유지되고 있다고 생각한다. 그러나 본고는, 월드론의 전략과는 달리, 이러한 우월성의 함축이 인간에게 부착된 계급적 속성, 구체적인 법적 지위라고 생각하지 않는다. 이러한 지위적 함축은 인간 개인이 아닌, 인간의 도덕성에 부착되는 우월성을 표현하고 있는 것이다. 앞서 칸트가 존엄을 가치와 관련시켰다고 생각되는 윤리형이상학 정초의 8가지 용례들을 분석하였듯이, 칸트는 윤리형이상학 정초에서 존엄은 도덕성의 높은 성질을 말한다는 것을 분명히 했다. 이 중 4번의 용례들을 다시 한 번 살펴보자.

{[}2 \& 3{]} 목적들의 나라에서 모든 것은 가격을 갖거나 존엄을 갖는다. 가격을 갖는 것은 같은 가격을 갖는 다른 것으로 대치될 수 있다. 이에 반해 모든 가격을 뛰어넘는, 그러니까 같은 가격을 갖기를 허용하지 않는 것은 존엄을 갖는다. (GMS 4:434.31-34)

이 구절에서 칸트는 존엄을 `뛰어넘음', `우월함' 등의 높음의 속성으로 표현했다. 여기서 가격을 뛰어넘고, 우월한 가치를 가지는 것은 바로 ``도덕성(윤리성)''이라는 점이 다음의 4, 5번째 사용에서 보다 분명하게 드러난다.

{[}4 \& 5{]} 보편적인 인간의 경향성 및 필요들과 관련되어 있는 것은 시장가격을 갖는다. {[}\ldots{]} 그러나 그 아래에서만 어떤 것이 목적 그 자체일 수 있는 그런 조건을 이루는 것은 한낱 상대적 가치, 다시 말해 가격을 갖는 것이 아니라 내적 가치, 다시 말해 존엄을 갖는다.

무릇 도덕성은 그 아래에서만 이성적 존재자가 목적 그 자체일 수 있는 조건이다. {[}\ldots{]} 그러므로 윤리성과, 윤리적일 수 있는 한에서의 인간성만이 존엄을 가지는 것이다.(GMS 4:434.35-435.09)

여기서 도덕성, 존엄, 가치의 개념은 상호 연결되는데, 어떤 내적 가치를 가지는 것, 즉 경향성이나 필요로부터도 독립적인 것은 바로 도덕성(윤리성)이다. 도덕성 추구의 명령은 이러한 경향성이나 필요와의 관련성으로부터 독립적이어야 하며, 이러한 관점에서 도덕적 가치는 다른 상대적·조건적 가치를 뛰어넘는다는 것이다. 칸트의 존엄을 단지 ``내적 가치''와 동일한 것을 지칭하는 것으로 정의하는 것은 칸트의 다른 대부분의 `존엄'의 사용들과 배치되며 존엄의 구체적 의미, 즉 높음이나 우월성의 의미를 상실시킨다. 칸트가 제공하는 설명은 도덕성은 다른 가치에 비해 더 중요하다, 즉 가치서열상 더 높고 우월하다는 것이며, 센슨에 따르면 이 구절은 ``도덕성은 단지 종속적·상대적 가치(가격)가 아니라, 우월한 내적 가치(가치에 있어 존엄)를 가진다.''는 것으로 해석된다.\footnote{Oliver Sensen, \emph{Kant on Human Dignity}, 184-186면.}

\subsubsection{드워킨의 `도덕성' 개념과 도덕성의 객관적 중요성}

\paragraph{\texorpdfstring{(1) 도덕성의 의미 : 인간적으로 좋은 삶을 사는 것 }{(1) 도덕성의 의미 : 인간적으로 좋은 삶을 사는 것 }}

현대 윤리학이나 각종 인권목록들에서 인간존엄의 개념을 사용하는 맥락을 잘 살펴보면 단지 고전적 완전주의의 의무들을 표현하고 있는 것이 아니라, 일종의 개인주의적 함축을 보다 중대하게 다루고 있음을 잘 알 수 있다. 인간존엄을 둘러싼 사용들은 대개 ``개개인의 인간에게 가치를 부여하고 책임을 부과''\footnote{Ronald Dworkin, \emph{Is Democracy Possible Here?: Principles for a New Political Debate} (Princeton, 2008). (홍한별 역, 민주주의는 가능한가? (문학과지성사, 2012), 23면.)}하는 것들이기 때문이다. 물론 여기서 개인주의는 단지 형식적인 차원에서 언급된 것이며, 개인이 속한 공동체나 전통의 성공과 관련시킬 수 있는 포괄적인 개념이다.\footnote{Ibid.} 이러한 개념들은 윤리학에서 발전시켜온 좋음이나 도덕성의 개념과 밀접히 관련되어 있다. 따라서 우리가 성공적인 인간존엄의 이론을 성취하기 위해서는 개개인의 좋은 삶의 이해가 그에 대한 의무를 발생시키도록 도덕의 이론을 구성할 것이 요청된다.

\subparagraph{결과주의 전략}

이러한 맥락에서 존엄, 즉 우월성의 대상인 도덕성은 자신을 포함한 인간의 삶을 좋은 삶으로 만드는 것과 밀접하게 관련된다. 도덕성을 인간의 좋은 삶과 연결시키는 전략은 크게 두 가지로 나누어 볼 수 있다. 하나는 결과주의적 전략이고 다른 하나는 칸트 도덕철학에 대한 최선의 이해를 통해 연결시킬 수 있다는 로널드 드워킨의 전략이다.

먼저, 가장 간편하고 직관적으로 보이는 설명방식은 결과주의 전략이라고 할 수 있다. 이 전략은 의무론적 이론이 도덕성을 좋은 삶에 연결시키는데 있어 노정하는 다음과 같은 한계로부터 출발한다. 인간존엄의 개념은 이 개념이 우리가 도덕적으로 살기 위해 어떤 행위를 해야하는지 지침을 제공해줄 것이 기대되기 때문에 실용적 의미를 지닌다. 그런데 의무론적 이론은 결과만으로는 도덕적 행위의 기준이 충분치 않다고 하거나, 결과와는 전혀 관계없는 기준을 제시한다. 의무론적 윤리이론에 따르면 의무를 따라 사는 것이 행복을 극대화하지 못하더라도 사람은 도덕적 의무를 이행해야 한다고 주장한다. 인간존엄을 의무론적 개념으로만 이해해야 한다면, 인간존엄에 기인한 타인존중은 타인 행복의 최대화라는 측면에서 개념화되기 어렵다. 이는, 인간존엄에 기인하여 우리가 국민의 권리를 존중하고 보장한다는 것이 그 나라 국민들의 행복의 증가를 뒷받침하는 것과는 무관하게 된다는 것을 함축한다.

이와 같이 인간존엄의 개념을 순전히 의무론적인 기준으로만 제시하게 되면, 자신과 타인의 좋은 삶에 기반하여 권리를 부여하고 이를 존중할 의무를 부과하는 인권적 논의에서 이 개념이 기여할 수 있는 폭이 매우 좁아질 것이다. 그런데 결과주의 이론은 도덕적 행위의 기준을 세우는 데 있어 행위의 결과를 관련시킨다. 따라서 결과주의 이론은 도덕성을 좋은 삶에 연결하는 데 유리하며 이와 완전히 상반되는 입장에서 인간존엄을 정의하는 것은 적어도 실용적으로 좋은 전략이 아니다. 결과주의 전략은 올바른 행위는 행위의 결과와 관련되는 측면이 존재하며, 우리는 항상 그 행동의 결과가 다른 사람의 권리를 존중하는 행동을 선택해야 하며, 그리고 그러한 측면에서 인간존엄의 규범적 이론을 정초해야 한다고 주장한다.\footnote{Marcus Düwell, ``Human dignity: concepts, discussions, philosophical perspectives'', 44면. Philip Pettit, ``The Consequentialist Perspective'', in M. W. Baron, P. Pettit and M. Slote (eds.), \emph{Three Methods of Ethics: A Debate} (Blackwell, 1977) 참고.}

\subparagraph{도덕이론의 난제와 드워킨의 도덕관}

도덕성을 자신과 타인의 좋은 삶과 연결시키는 두 번째 전략은 칸트를 드워킨의 ``최선의 설명''으로 이해해보려는 전략이다. 칸트의 도덕철학은 자신의 이익을 타인의 이익보다 중요하게 여기지 않는 ``자기-자제의 도덕(the morality of self-abnegation)''의 전형인 것으로 이해되고는 한다. 칸트에 따르면 ``행위자의 이익이나 경향성에 의해서만 동기를 부여받은 행위는 어떠한 것도 도덕적으로 선하지 않기 때문이다.'' 따라서 칸트의 이론에서 ``행위의 도덕적 동력이, 자신의 삶을 무언가 탁월한 것으로 만들고 삶이라는 과업을 잘 해내려는 행위자 자신의 포부에서 나올 수 있다는 생각이 들어설 자리가 없어 보인다.'' \footnote{Ronald Dworkin, \emph{Justice for Hedgehogs} (Havard, 2013). (박경신 역, 정의론 (민음사, 2015), 56-59면.)}

이러한 구성은 쉬운 일이 아니다. 대개의 도덕성 개념은 내가 타인으로부터 어떻게 대우받을지를 설명하는 권리의 이론이나 나의 행복으로 이끄는 좋음의 이론보다는 내가 타인에게 어떻게 행위해야 하는지를 설명하는 의무나 옳음의 이론과 관계된다고 생각되기 마련이다. 타인을 배려하라는 도덕의 정신은 도덕이 나에게 가져오는 혜택에 의존하는 것이 아니며, 따라서 우리는 대개 도덕적이어야 하는 이유가 우리의 장기적 이익이어서는 안 된다고 믿기 때문이다.\footnote{드워킨은 이를 엄숙주의(austere view)라고 설명하고 비판한다. Ibid., 한국어판(박경신 역) 314면.}

그러나 이미 많은 도덕의 이론들은 이러한 엄숙주의를 배격하고 좋음의 이론과 도덕의 이론을 통합하려는 시도로 바라볼 수 있다. 예를 들어 플라톤과 아리스토텔레스의 이론을 (주로 타인에 대한 배려와 의무를 유발하는) 도덕적 가치들에 대한 자신들의 이해방식이 올바르다고 주장하는 이론으로 이해할 수 있다면, 이들은 또한 이러한 도덕적 가치들을 추구하는 삶이 그들에게 행복(eudaimonia)의 존재상태를 가장 잘 가져다 준다고 주장한 이론으로도 동시에 이해되고 있다.\footnote{Ibid., 한국어판(박경신 역) 303-304면 참조.}

도덕의 이론을 좋음의 이론과 통합시키려는 시도는 좋음의 이론을 자연과학에 의해 검증가능한 인간의 쾌락과 욕망에만 관련시킴으로서 이루어질 수도 있다. 좋음이 인간의 욕구충족에만 관련된다고 가정하면, 도덕은 이러한 욕구충족을 증진시키는 것을 의미한다. 그리고 욕구간의 충돌은 현대화된 게임이론을 통해 조정될 수 있다. 이의 토대가 되는 대표적인 시도가 홉스의 도덕철학이라고 할 수 있다. 그러나 이러한 철학은 자기 자신의 좋은 삶을 어떤 책임을 유발하는 객관적으로 중요한 것으로 받아들이고자 하는, 그리고 과학과는 독립적인 ``고유한 탐구 기준과 정당화 기준을 갖춘 별개의 지식부문으로'' 간주하려는 도덕에 대한 우리의 숭고한 직관을 희생시킨다. 이러한 근대적 관점을 받아들이는 것보다는 차라리 플라톤과 아리스토텔레스와 같은 고대 철학자들의 행복의 이론으로 돌아가는 것이 우리의 희망적인 관점을 더 잘 좇는다고 할 수 있다.\footnote{Ibid., 한국어판(박경신 역) 52-57면 참조.}

로널드 드워킨은 이러한 우리의 희망을 놓지 않으면서도, 현대 인권이론의 개인주의적 함축과 윤리학의 도덕성 개념 사이의 난제를 해결하고 이들을 조화시키려고 시도한다. 이 시도는 ``도덕''과 ``윤리''의 통합을 통해 이루어진다. 그는 ``사람은 오직 그 모든 인간성의 형태들 속에 담긴 인간성 자체를 존중할 때만 자신의 성공적인 삶에 불가결한 존엄과 자기 존중을 성취할 수 있다''고 말한다. 그는 이러한 주장을 칸트에 대한 최선의 해석적 이해로부터 나올 수 있다는 의미에서 `칸트의 원칙'이라고 부르는데, 이러한 전제 하에서 좋은 삶의 이해방식---드워킨의 고유한 용법에서의 ``윤리''---은 ``도덕''---타인의 좋은 삶을 존중하는 이해방식---을 통합시킬 수 있는 기초가 될 수 있다고 주장한다.\footnote{Ibid., 한국어판(박경신 역) 58-59면.} 이 원칙은 참된 자기 존중이 모든 인간의 삶을 동시에 존중하는 것을 수반하도록 만드는 그런 원칙이기 때문이다.\footnote{Ibid., 한국어판(박경신 역) 403면.}

드워킨은 도덕의 숭고한 독립성을 유지하기 위해 도덕성과 가치의 중심에 ``잘 산다는 것''을 둔다.\footnote{드워킨의 원문에서는 `도덕성'과 `윤리성', `잘 살기'과 `좋은 삶'을 구분하고 있으나, 본고의 용례는 이러한 구분론을 따르고 있지 않다. 이러한 구분론은 드워킨의 관점을 서술하기 위해 필요할 때에만 드러내도록 하겠다. Ibid.,} 이 때, 이러한 가치는 단지, ``자신이 원하는 것을 갖는다''고 할 때 그 ``원하는 것''만을 의미하지 않으며,\footnote{Ibid., 한국어판(박경신 역) 318면.} 여기서 잘 사는 것, 좋은 삶을 사는 것은 ``우리의 비판적 이익, 즉 우리가 가져야 하는 이익의 문제다''.\footnote{Ibid.,} 이는 욕망의 단순한 존재사실로부터 도덕적 의무를 정당화 할 수 없다는 흄의 원리에서부터 따라나오는 것이다. 우리는 여기서 도덕적 의무를 정당화 할 수 있는 종류의 개념으로서 도덕성과 좋은 삶, 잘 사는 것을 이해하려고 시도하고 있다.

그런데 이러한 비판적 이익이 지나치게 강조되어 커다란 희생을 통해 사실상 끔찍한 불행을 사는 삶을 좋은 삶이라고 규정하는 이해방식을 가지게 된다면 그 또한 우리의 직관에 부합하지 않는다. ``비도덕적으로 살았다면 {[}\ldots{]} 번창하여 오랫동안 평안하게 살았을 사람이 그렇게 사는 대신 끔찍한 불행을 감수했다고 해서 더 좋은 삶을 살았다고 말하기는 어려울 것''이기 때문이다. 따라서 드워킨은 좋은 삶을 옳은 삶 혹은 잘 사는 삶과 구별해낸다. 그렇다면 좋은 삶은 도덕적 삶이나 올바른 삶과 구별되어 완전히 독립적으로 성취될 수 있는 욕망을 단순히 충족하는 삶을 말하는가? 그렇지는 않다. 좋은 삶이 중요한 이유는 좋은 삶을 창출하는 것이 잘 사는 것에 기여하기 때문이다. 따라서 우리는 좋은 삶을 단지 욕망을 충족하는 독립적 기반에서 구성하는 것이 아니라, 올바른 잘 사는 삶의 이해방식과 상호의존적으로 구성한다. 그리고 우리는 비판적인 의미에서의 좋은 삶을 원한다. 따라서 좋은 삶이 무엇인가는 ``판단하고 논쟁해야 할 문제다''.\footnote{Ibid., 한국어판(박경신 역) 318-319면.}

인간존엄은 개인의 좋은 삶과 분리되어 생각될 수 없다. 그런데 `좋은 삶'이 지칭하는 대상에 대해서는 깊은 논쟁이 있다. 예를 들어, 드워킨의 방식으로 말하면 잘 살기 위해서는 다음의 두 토대와 조건이 만족되어야 한다. 하나는 자기 존중의 원리, 혹은 본질적 가치의 원리라고 불리는 것으로써, 각 사람은 자기 자신의 삶을 (잘 사는 것을) 객관적으로 중요한 것으로서 진지하게 받아들여야 한다는 것이다. 자신의 삶이 낭비되지 않고 성공적인 수행이 되는 것이 중요한 문제임을 각 사람은 받아들여야 한다. 둘째는 개인적 책임의 원칙 혹은 진정성의 원리라 불리는 것으로써, 각 사람은 자신의 삶에서 성공의 요건들을 식별하고 일관된 서사를 통해 그 삶을 창조해야 할 개인적 책임이 있다는 것이다. 자기 자신이 아닌 누군가가 이러한 개인적 가치를 지시하거나 강요해서는 안 된다.\footnote{이 두 원리는 드워킨의 \emph{Is Democracy Possible Here?}와 \emph{Justice for Hedgehogs} 에서 뒤섞여 정의되고 있다. Ronald Dworkin, \emph{Is Democracy Possible Here?}, 한국어판(홍한별 역) 22-23면; Ronald Dworkin, \emph{Justice for Hedgehogs}, 한국어판(박경신 역) 332, 334면 등 참조.}

그런데 이러한 방식으로 좋은 삶과 잘 사는 것을 설명하는 것은, 오로지 자기 자신의 좋음에 관련된 것으로서, 타인에 대한 존중을 중심에 두는 전통적인 도덕의 이론과 거리가 먼 것처럼 보일지 모른다. 그러나 여기서 중요하게 살펴야 할 것은, 이러한 삶의 수행이 가져다 주는 결과적인 욕구의 만족이나 쾌락이 아니라, 이러한 삶의 수행이 자기 자신에게 있어 개인적인 책임으로 다가온다는 것이다. 즉, 각 사람은 자기 자신의 삶을 잘 사는 것을 객관적으로 중요한 것으로 받아들이면 자신에게 단지 좋은 것이 아니라, 그렇게 받아들여야 할 의무가 있다는 것이고, 좋은 삶을 창조하는 것이 그에게 단지 좋은 것이 아니라, 그렇게 창조해야 할 개인적인 책임이 있다는 것이다. 그러한 의미에서 인간적으로 좋은 삶을 사는 것은 하나의 욕구나 단지 그에게 좋음이 아닌, ``도덕성''을 성취할 자격을 일부 갖추게 된다.

\paragraph{우월성의 의미 : 객관적 중요성}

\subparagraph{도덕성의 우월성 또는 객관적 중요성의 의미}

앞서 인간존엄이 도덕성에 부착되는 우월성이라는 속성이라는 것으로 살펴본 바 있다. 그렇다면 도덕성이 우월하다는 것은 어떤 의미인가? 본고는 이 우월성이 어떤 객관적 중요함의 속성을 의미하고 있다고 파악한다. 이 때 말하는 객관성은 어떤 의미인가? 잘 살기와 도덕성을 둘러싼 두 가지 객관성의 후보자들을 생각해 볼 수 있다. 하나는 성공적인 좋은 삶의 기준과 관련된 객관성이고, 다른 하나는 잘 사는 것이 누구에게나 중요하다는 평가로서의 객관성이다. 먼저, 드워킨은 그동안 모호하게 쓰여왔던 `잘 사는 것'과 `좋은 삶'을 구분하면서, ``좋은 삶''에도 어느 정도 객관적인 기준이 있다는 것을 인정한다.

성공적인 삶에 대한 객관적인 기준이 있고, 잘 산다는 게 무엇인지 착각할 수도 있다는 것, 그런 실수를 하지 않는 것이 아주 중요하다는 생각을 저버리기는 무척 힘들고 거의 불가능해 보인다. 그런 가정조차 저버린다면, 성공적인 삶을 이루기 위해 상식적이고 중대한 결정을 내리기가 어려워질 것이다. 이런 결정은 단순히 무엇이 즐거울 것인가를 예측해서 내린다거나 할 수가 없다.\footnote{Ronald Dworkin, \emph{Is Democracy Possible Here?}, 한국어판(홍한별 역) 27면.}

즉, 개인적 책임의 원칙은 ``좋은 삶''의 기준에 (부차적인) 객관성이 존재함을 부정하지 않는다. 존엄은 물론 궁극적인 자기 결정에 의존하지만, 이러한 결정의 토대가 되는 ``좋은 삶''의 기준에도 여전히 객관성은 존재한다.

그런데, 이러한 ``좋은 삶''의 객관성은 도덕성을 관통하는 핵심적인 잘 살기를 구성하는 객관성은 아니다. 만약 이러한 객관성이 도덕성을 결정한다면, 드워킨이 이야기한 ``대규모 살상과 배신으로 점철된 경력''의 메디치가의 군주는 그럼에도 불구하도, ``성취, 정제, 교양, 쾌락''을 누렸기 때문에 잘 살았다, 도덕성에 가까웠다는 평가를 받아야 할 것이다.\footnote{Ronald Dworkin, \emph{Justice for Hedgehogs}, 한국어판(박경신 역) 326면.} 그러나 이러한 평가를 내리는 것은 부적절하다. 드워킨에 의하면 도덕적으로 잘 사는 것은 ``좋은 삶''을 살기 위해 노력하는 것을 포함하지만, 우리는 여전히 옳은 것과 좋은 것을 구별해야 하며, 우리가 올바로 살았다고 할 때, 혹은 인간적으로 좋은 삶을 살았다고 할 때에서야 비로소 도덕적으로 좋은 삶을 살았다, 잘 살았다고 말할 수 있는 것이다. 따라서 이러한 좋은 삶의 경험적이고 구체적인 기준과 관련된 객관성은 도덕성의 우월성을 표상하는 속성이라고 보기 어렵다.

인간존엄의 우월성의 의미를 가장 잘 포착하는 보다 유력한 두 번째 객관성은, 잘 사는 것이 누구에게나 중요하다는 평가라고 할 수 있다. 이러한 객관성 관념은 도덕성을 다른 어떤 가치나 관념보다 우월한 것으로 위치시킨다.

지위의 개념을 설명할 때, 사전적 정의에서는 그 최근류로 `위치'나 `자리'의 언어가 사용되고 있다는 점을 설명한바 있다. 그러나 이는 사실 지위를 설명하기에 불충분한 최근류이다. 위치라는 최근류는 주로 공간(space)과 관련된 속성들을 설명하기에 적합하다. 하지만 인간 간의 관계와 권한, 의무들을 설명해 주는 `지위'라는 용어를 위치를 최근류로 하여 설명하는 것은 ``A의 아버지임은 B의 자녀임보다 1\emph{m} 위에 있는 위치''라고 설명하는 것처럼 이상하다. 다만 이러한 정의는 유비로서만 가능하다. 가부장제 사회 아래서 A는 B보다 높은 위치에 있다라고 말하는 것은 좋은 비유라고 할 수 있을 것이다. 그러나 비유를 비유로서만 받아들이지 않는 한, 절대적 위치, 보통사람(normal person)과 같은 설명할 수 없는 공간적 위상을 점하는 기이한 지위가 탄생하는 것을 막을 수 없다.

이는 사람의 사회적 지위를 설명할 때 뿐만 아니라 도덕성의 우월성이라는 높음의 속성을 설명할 때에도 마찬가지다. 도덕성에 어떤 공간적인 속성을 부착한다는 것은 기이한 일이다. 도덕성이 어떤 특유한 우월성이나 높음의 지위를 가지고 있다는 것을 다시 표현하면 이 도덕성은 어떤 다른 가치나 평가에 의존하지 않고 단적으로 중요하다는 말로 대체할 수 있을 것이다. 드워킨은 이를 ``우리가 잘 사는 것은 나 자신이나 타인에게 중요한 것이 아니라 단적으로 중요하다''고 표현하고 있다.\footnote{Ronald Dworkin, \emph{Justice for Hedgehogs}, 한국어판(박경신 역) 320면(번역 일부 수정).} 이와 같이 도덕성의 객관적 중요성은 바로 도덕성의 높은 지위의 관념을 가장 잘 계승하고 있다.

따라서 우리의 인격안에 내재해 있다고 흔히 오해되어 왔던 인간존엄은 인간이 보유하는 가치속성으로 환원될 수 있는 성질의 것이 아니다. ``모든 인간이 존엄하다.''라고 하는 인간존엄명제 DP(Dignity Proposition)는 ``모든 인간은 존중받아야 할 VP라는 가치속성(value property)을 가진다.''라거나 ``모든 인간은 그를 존중받도록 해 주는 VP라는 가치속성을 가진다.''라는 내용으로 전환되는 것이 아니다. 인간존엄명제(DP)는 ``인간적으로 좋은 삶을 영위하는 것은 객관적으로 중요하다'' 혹은 ``모든 인간이 인간적으로 좋은 삶을 영위하는 것은 객관적 중요성의 속성을 가진다.''는 명제P\emph{dig}를 의미한다.

\begin{longtable}[]{@{}
  >{\raggedright\arraybackslash}p{(\linewidth - 0\tabcolsep) * \real{1.0000}}@{}}
\toprule\noalign{}
\begin{minipage}[b]{\linewidth}\centering
\textbf{P\emph{dig}}: ``모든 인간이 인간적으로 좋은 삶을 영위하는 것은 객관적으로 중요하다.''
\end{minipage} \\
\midrule\noalign{}
\endhead
\bottomrule\noalign{}
\endlastfoot
\end{longtable}

\paragraph{객관적 중요성과 타인존중의 요청}

이러한 인간적으로 좋은 삶을 사는 것의 객관적 중요성의 존재는 왜 내가 타인의 좋은 삶을 존중하는 행위를 해야하는지에 대한 규범적 근거를 제시해 준다. 인간존엄이 소위 ``헌법적 가치(constitutional value)''라고 말할 때, 이는 어떤 특정한 인권이나 그러한 인권을 정초하는 개별적 가치와 유사한 어떤 제3의 가치를 나열하는 것이 아니라 ``우리가 자신 뿐만 아니라 타인의 권리와 이익을 존중해야 한다.''는 명제의 근거로서 사용되는 것이다. 어떤 규범에 인간존엄 근거가 부재한다고 가정하면, 우리는 나 자신의 좋은 삶이 나에게만 중요한 주관적 중요성을 주장할 수 있을 지언정, 나 자신의 좋은 삶이 타인에게도 존중받아야 할 객관적으로 중요한 것임을 주장할 근거가 소실된다.

\subsection{법적 차원에서 인간존엄 이해의 재구성: 이원적 이해방식}

\subsubsection{평등한 존중의 정당화 근거로서의 인간존엄}

제2부에서 현대법질서에서 인간존엄이 인간이 추구해야 할 어떤 특유한 인간성이나 시민으로서 권리들을 가질 지위로 기능하고 있다는 이해방식에는 한계가 있다는 점을 각각 살펴보았다. 본고는, 인간적으로 좋은 삶의 객관적 중요성이라는 속성으로 이해된 인간존엄이 법적 질서 안으로 편입된다는 것은 사회구성원들에 대한 법질서 내에서의 평등한 존중을 정당화해주는 것으로 이해되어야 한다고 주장한다. 다음에서는 평등한 존중의 정당화 근거로서의 인간존엄의 법적 위상을 살펴보고, 이러한 이해방식이 가지는 일관성과 장점을 고찰해보고자 한다.

\paragraph{철학적 배경: 인간적으로 좋은 삶의 객관적 중요성으로서의 인간존엄}

그동안 ``인간은 존엄하다.''라는 인간존엄명제(Dignity Proposition, DP)에서 지칭되는 인간존엄은 인간이 보유한 어떤 능력과 같은 내재적 속성으로서 오해되어 왔다. 그리고 이 테제를 건전하게 만드는 방법은 때로 그러한 내재적 속성이 무엇인지, 그리고 실제로 그러한 속성을 인간이 보유하는지 구체적으로 밝히는 일이라고 생각되어 왔다. 그러나 본고는 제3부의 제1장에서 ``인간이 존엄하다.''는 명제의 정확한 의미는 ``인간적으로 좋은 삶을 사는 것은 객관적으로 중요하다.''(P\emph{dig})를 뜻하는 것으로 파악했다. 이는 ``윤리성과, 윤리적일 수 있는 한에서의 인간성만이 존엄을 가지는 것''(GMS 4:435)이라고 주장한 칸트의 존엄의 개념을 전통적 이해방식의 측면에서 계승하는 것이다. 이 때, 윤리성 혹은 도덕성은 나 자신 혹은 타인이 인간답게 잘 사는 것에 책임이 있음을 의미한다.

이러한 이해방식은 인간존엄명제가 법질서 내에서 구성원 사이의 존중 의무의 정당화 근거라는데는 찬성하지만, 그것이 인간이 가진 어떤 능력과 같은 속성의 보유를 인정하는 테제라는 해석에 대해서는 반대함으로써 지속적으로 제기되어왔던 현대적 이해방식의 난점들을 해소한다. 이 객관적 중요성의 관념은 특정한 가치를 지지하는 것이 아니라 그러한 가치들을 누리고 사는 인간적으로 좋은 삶에 부착되는 ``객관적 중요성''이라는 특유한 속성이다. 이러한 속성은 서로 충돌하거나 대립하지 않으며, 다만 우리가 자신과 타인의 좋은 삶을 존중해야 할 이유를 제공해 주는 속성이다.

\paragraph{구체적 기능}

\subparagraph{인권도출의 기초}

이렇게 이해된 인간존엄은 인간존엄을 보호하기 위한 규범을 창출하는 강건한 정초로서의 역할을 수행하는가? 본고는 제2부에서 인간존엄이 이를 보호하는 규범과 권리들을 도출하는 강건한 정초 역할을 하고 있다는 주장을 살펴보았다. 이러한 주장에 의하면 인간존엄이라는 법적 가치는 인권들을 도출하는 강건한 기초이며, 또한 동시에 각 권리와 규범들의 해석의 기초가 된다. 이러한 이해에 따라 남아프리카공화국헌법 제10조는 인간존엄을 보호하는 권리를 기본권의 형식으로 명시적으로 수용하고 있다는 점도 살펴보았다. 우리나라 헌법이 인간존엄의 보호를 권리의 형식으로도 수용하고 있는지에 대해서는 그 해석이 갈리지만 헌법 제10조는 ``모든 국민은 인간으로서의 존엄과 가치를 가지며 {[}\ldots{]}''라고 하면서 이를 국민의 권리와 의무를 선언하는 제2장의 첫번째 조항으로 삼고 있다는 점은 고려되어야 할 것이다. 여기서 헌법이 인간의 존엄보유를 선언하는 바의 규범적 위상, 법적 성격에 대해서는 여러 학설들이 있지만, 대개 다음의 두 가지 역할 사이에서 논의되고 있는 것 같다. 하나는 이 규범이 인간존엄을 1) 헌법의 기본원리로서 기본권 보장의 이념적 기초라고 말하고 있는 것인가라는 점, 다른 하나는 2) 구체적인 기본권들을 도출하는 근거로 이해하고 있는 것인가 하는 점이다.\footnote{정종섭, 헌법학원론 제11판(박영사, 2016), 410-412면.} 본고는 이에 더해 3) 여타 헌법 규범이나 기본권 규범들의 내용을 구체화하고 조정해 주는 규범해석원리로서 사용되는가 하는 점을 추가하고자 한다.

학설들은 인간존엄이 1)의 역할, 즉 단순히 법의 비규제적인 이념으로서의 역할을 수행하고 있다는 점에 대해서는 거의 의문을 품지 않는다. 그보다는 법이 전제하고 있는 인간존엄의 사실이 다른 기본권들과의 관계를 설정하고 열거되지 않은 인권까지도 기본권으로 인정해 주는 보충적 기능을 할 수 있는가 하는 2)의 문제와, 규범간 충돌시에 최종적 조정자 역할을 하는가에 관한 3)의 문제에 대해서는 논란이 있다.

인간존엄은 인권을 도출하는 것과 어떻게 관련되는가?

본고가 제시하는 인간존엄의 규범은 제1장에서 제시한 도덕성의 우월성 이념이다. 본고는 이를 P\emph{dig}라는 명제로 제시하였다. 우리의 인격안에 내재해 있다고 흔히 오해되어 왔던 인간존엄은 몇 가지 가치속성으로 환원될 수 있는 성질의 것이 아니며, ``모든 인간이 존엄하다''라고 하는 인간존엄명제(DP)는 ``모든 인간이 인간적으로 좋은 삶을 영위하는 것은 객관적으로 중요하다''는 명제P\emph{dig}를 의미한다. 우리가 어떤 가치속성이라고 생각해 왔던 `존엄'의 지칭대상은 인간안에 내재한 어떤 속성이 아니라 도덕성의 우월성, 다시 말해 도덕성에 부착되는 객관적 중요성이라는 평가적 속성이다. 인간은 어떤 침해에도 불구하고 존엄하다는 불가침성 명제는 P\emph{dig}로 적절히 전환하면 쉽게 이해될 수 있다. 즉, 개별 인간의 가치가 누군가에 의해 부정당하고, 권리가 침해당하더라도, 해당 인격의 가치와 권리가 객관적으로 중요하다는 사실에는 변함이 없다는 것이다.

인간존엄에는 불가침성과 함께 그에 대한 확립/보호필요성도 존재한다는 존엄의 이중성 문제도 해결된다. 존엄을 확립 또는 보호하기 위해 노력하라는 요청은 객관적으로 중요한 것으로 이해된 나의 인간적으로 좋은 삶(P\emph{gl}+)을 살기 위해, 그리고 시민과 타인의 P\emph{gl}+의 삶을 존중하기 위해 노력하라는 것으로 이해할 수 있다. 인간존엄을 보호하라는 규범으로부터 인권이 도출된다는 것은 이러한 의미에서 이해된다. 한 사람이 인간적으로 좋은 삶을 살기 위해 어떤 권리의 형식으로서 보호가 필요하다면 국가는 그러한 권리를 자명한 것으로 인식하고 보호해 주어야 한다는 것이다.

앞서 설명한 P\emph{gl}+가 함축하는 삶을 사는 것은 인간적으로 좋은 삶을 사는 것을 보장한다. 그런데 이 명제 자체는 그러한 삶을 사는 것이 가치있는지, 이를 보호하는 것이 규범적으로 정당화되는지에 대해서 말하는 명제는 아니다. 이러한 정당화는 ``인간이 존엄하다''라는 항상 참으로 가정되는 명제 즉, ``P\emph{gl}+가 의미하는 좋은 삶을 영위하는 것은 객관적으로 중요하다''라는 다음과 같은 인간존엄명제 P\emph{dig}로부터 가능해진다.

\begin{longtable}[]{@{}
  >{\raggedright\arraybackslash}p{(\linewidth - 0\tabcolsep) * \real{1.0000}}@{}}
\toprule\noalign{}
\begin{minipage}[b]{\linewidth}\centering
\textbf{P\emph{dig}}: ``\textbf{P\emph{gl}+}가 의미하는 좋은 삶을 영위하는 것은 객관적으로 중요하다.''
\end{minipage} \\
\midrule\noalign{}
\endhead
\bottomrule\noalign{}
\endlastfoot
\end{longtable}

인간이 존엄하다는 명제P\emph{dig}와 그 선언은 정당화된 P\emph{gl}+가 의미하는 인간적으로 좋은 삶을 사는 것이 객관적으로 중요하다는 것을 의미한다고 말하며, 이를 통해 이러한 객관적 중요성을 가진 좋은 삶을 사는 것을 방해하는 것은 그르다는 규범적 함축이 따라나온다.

P\emph{gl}+가 함축하는 삶을 사는 것은 인간적으로 좋은 삶을 사는 것을 보장하고, P\emph{dig}는 그러한 삶을 사는 것이 가치있게 만들고, 이를 보호하는 것을 규범적으로 정당화한다. 여기서 제임스 그리핀의 연역적 정당화에 대한 월드론의 분석의 틀을 다시 가져와 보자. 규범적 주체성의 보호를 인권의 목적으로 이해한 그리핀은 인권들을 단계적 연역에 의해 분석적으로 확립했다. 존엄으로부터 규범적 주체성이, 이로부터 자율성이, 이로부터 소극적 그리고 적극적 자유가 확립된다는 것이다.\footnote{Jeremy Waldron, ``Is Dignity the Foundation of Human Rights?'', 129면.} 본고는 인권의 도출기반이 인간적으로 좋은 삶을 사는 것(P\emph{gl}+)을 보장해 주는 데 있다고 보고, 이를 위해 자율성이, 복지가, 그리고 생명권이 확립된다. 그리고 이러한 확립의 중요성을 객관적인 것으로 정당화시켜주는 것은 바로P\emph{dig}라고 할 수 있다. 따라서P\emph{dig}는 진정한 논리적 인권 도출의 기초가 된다.

\subparagraph{권리해석의 기준}

맥크루든의 표현을 빌리면 ``존엄은 그 원리에 의해 생산되는 권리들의 목록을 더 깊게 설명해주는 해석적 원리가 된다 \ldots{} 그 권리들의 일부(혹은 전부)는 존엄의 렌즈를 통해 가장 잘 해석되는 것으로 여겨지게 된다.'' 이러한 태도는 타티아나 게더트-슈타이나허에 의해서 존엄의 도구적 기능으로 확인된 것과 유사하다.\footnote{Tatjana Geddert-Steinacher, \emph{Menschenwürde als Verfassungsbegriff} (Duncker \& Humblot, 1990), 136-154면 참조. 게더트-슈타이나허는 존엄의 도그마틱적 기능으로서 도구적 기능(기본권 해석의 원리로서의 인간존엄) 외에도, 구성적 기능(법원리의 뿌리로서의 인간존엄), 선언적 기능 등을 들고 있다.} 이를 분석한 것으로 보이는 김병곤의 설명을 보자.

독일의 基本法(Grundgesetz)에서 一般的 行動自由權(Allgemeine Handlungsfreiheit)의 範圍와 內容을 해석할 때 인간존엄이 해석의 기초가 된다. 또한 一般的 人格權(제2조), 良心과 宗敎의 自由(제4조), 生命權(제2조 제2항), 身體의 不可侵(제2조 제2항), 兵役拒否權(제4조 제3항), 親權(제6조 제2항), 亡命庇護權(제16조 제2항)의 해석에서 尊嚴論據(Würdeargument)는 基本權解釋의 구분적 단계로 이끌어 가며 衡量過程에서는 基本權상의 保護領域을 擴張하거나 制限하기도 하며 혹은 制限의 限界(Schranken-Schranke)로서 기능하기도 한다.

基本權이 지금까지 파악하지 못한 人間의 自律과 自主性(Eigenständigkeit)에 대하여 새로운 위험성이 등장하면 人間의 尊巖은 보호영역의 확장으로 나타나기도하고, 반대로 좁게 해석될 수도 있다. 人間의 尊嚴의 保護는 人權의 最高規範目的을 形成하고 그 점에서 基本權 領域의 目的論的 擴張을 필요로 한다. 또한 基本權의 制限的 해석에 있어서도 그 保護領域은 尊巖核心(Würdekern)을 내포한다. 한편 衡量過程에서 尊嚴論據는 制限의 限界로서 효력을 발휘할 때 가 있는데 이때는 相衝하는 基本權中의 한쪽 基本權의 보호영역의 일정한 의미 요소(Bedeutungselement)를 강화할 수 있다.\footnote{김병곤, ``人間의 尊嚴의 基本權解釋原理로서의 기능'', 東亞法學 제15호 (1993), 157-158면.}

여기서 인간존엄은 단지 존재하는 어떤 권리들의 기초가 아니라 그 권리들의 내용을 보완하거나 확장, 혹은 제한하고 권리들 간의 충돌이 생기면 이를 해소하기도 하며, 권리들의 제한을 다시 일정한 수준으로 한계짓는 기능을 한다. 게더트-슈타이나허는 이를, 기본권 보호영역의 확장기능(Die schutzbereichserweiternde Funktion), 기본권 보호영역 제한 기능(Die schutzbereichslimitierende Funktion), 기본권 제한의 한계 기능(regulative Schranken-Schranke)\footnote{김병곤은 이를 기본권 충돌의 해소기능으로 분류하고 있다. Ibid., 169-178면.} 등으로 분류하고 있다. 우리나라 대법원도 연명치료장치 제거에 관한 판결에서 생명권을 인간으로서의 존엄에 부합하게 보호하여야 한다고 판결하고 있는데, 해석적 원리로서 존엄을 파악하고 있는 것으로 보인다.

\ul{생명권이 가장 중요한 기본권이라고 하더라도 인간의 생명 역시 인간으로서의 존엄성이라는 인간 존재의 근원적인 가치에 부합하는 방식으로 보호되어야 할 것}이다. 따라서 이미 의식의 회복가능성을 상실하여 더 이상 인격체로서의 활동을 기대할 수 없고 자연적으로는 이미 죽음의 과정이 시작되었다고 볼 수 있는 회복불가능한 사망의 단계에 이른 후에는, 의학적으로 무의미한 신체 침해 행위에 해당하는 연명치료를 환자에게 강요하는 것이 오히려 인간의 존엄과 가치를 해하게 되므로, 이와 같은 예외적인 상황에서 죽음을 맞이하려는 환자의 의사결정을 존중하여 환자의 인간으로서의 존엄과 가치 및 행복추구권을 보호하는 것이 사회상규에 부합되고 헌법정신에도 어긋나지 아니한다고 할 것이다.

그러므로 회복불가능한 사망의 단계에 이른 후에 환자가 인간으로서의 존엄과 가치 및 행복추구권에 기초하여 자기결정권을 행사하는 것으로 인정되는 경우에는 특별한 사정이 없는 한 연명치료의 중단이 허용될 수 있다.\footnote{대법원 2009.05.21. 선고 2009다17417 전원합의체 판결{[}무의미한연명치료장치제거등{]}}

이 판결에서 생명권은 인간존엄 그 자체나 단순히 인간존엄으로부터 부수적으로 파생된 독자적인 권리가 아니라 인간존엄의 이념에 의해 그 내용이 한정되거나 다른 권리들과의 충돌시에 보호수준이 결정되는 그러한 권리로 위치지워진다.

이러한 사안에서 인간존엄의 규범이 생명권을 해석하고 확장, 제한하는 기능을 하기 위해서는 생명권의 규범보다 우선적이면서도 풍부하고 보충적이어야 한다. 제2부에서 살펴본 인간이 가지고 있는 특수한 가치속성으로서의 인간존엄의 설명은 이러한 기능을 하지 못하며, 인간성이 무엇인지 그 내용에 대한 당파적인 논쟁만을 촉발시킨다.

인간존엄을 권리들의 해석의 기준으로 만들어 주는 것은 단적인 인간의 속성에 대한 기술로서의 인간존엄이 아니라, P\emph{dig}와 P\emph{dig}가 지향하는 P\emph{gl}+가 가지고 있는 인간적으로 좋은 삶에 대한 깊은 통찰 때문이다. 이 명제들이 가지고 있는 좋은 삶에 대한 구체적 해명들은 단지 권리를 존중해야 할 것으로 정당화할 뿐만 아니라 그 권리의 내용과 한계가 무엇인지에 대한 구체적인 기준이 된다.

\paragraph{유용성}

\subparagraph{평등한 존중의 정당화 관념의 실질성}

P\emph{dig}가 함축하는 객관적 중요성은 지극히 형식적이어서, 구체적인 존중의 내용에 전혀 기여하는 바가 없다고 비판할 수도 있다. 뒤에 살펴보겠지만, 권리의 관점에서 인간존엄의 권리는 단지 ``상대방에게 자신을 존중할 것을 명령하는 권리''라고 보는 형식적 권리 입장이 있다. 이러한 권리에서 존중의 내용은 정해져 있지 않다. 안톤 파간은 이를 가상적인 베이비시터 사례에 비유한다.\footnote{Anton Fagan, ``Human Dignity in South African law'', 401-406면.} 어머니가 아들에게 베이비시터에게 순종하라고 지시한 후 베이비시터는 이 아들에게 아기에게 가서 코를 세게 치라고 명령한다고 가정해 보자. 어머니의 지시가 없으면 베이비 시터의 명령은 아무런 의무도 부과하지 않았을 것이지만, 어머니의 지시가 존재한 덕분에 베이비시터의 명령이 아들에게 의무를 부과한다. 그러나 어머니의 지시는 의무의 내용을 결정할 때 어떤 역할도하지 않고 그 의무의 구체적 내용을 부과하는 것은 베이비시터의 명령이라는 것이다. 안톤 파간은 이러한 관점의 뿌리를 칸트적 형식주의에서 찾으면서, 인간의 존엄이 이러한 권리의 확정과 해명에 도움을 주지 않기에, 이러한 형식적 권리에는 난점이 있다고 주장한다.\footnote{Ibid.}

만약 인간존엄을 평등한 존중의 정당화 근거로서 이해하게 되면, 이는 지나치게 형식적이어서 평등한 존중의 구체적인 내용과 법적 실현방법이 무엇인지를 확정하고 해명하는데 도움이 되지 않는다는 동일한 비판을 받을 수 있을 지도 모른다. 그러나 본고는 명제 ``인간적으로 좋은 삶을 사는 것은 객관적으로 중요하다.''(P\emph{dig})는 명제는 주어부인 ``인간적으로 좋은 삶을 사는 것''의 해석에 그 중요한 작업을 남겨두고 있는 것으로 파악한다. 이러한 이해는, 우리의 법적 가치와 권리가 자기 또는 타인이 존중할 의무의 대상이라는 점을 정초해 줌과 동시에, 우리가 인간존엄을 이상으로 삼고 있는 법을 건전하고 풍부한 것으로 만들기 위해 해야 할 보다 본격적인 작업을 `과연 인간적으로 좋은 삶을 사는 것은 어떻게 사는 것인가', `그리고 이를 위해 법의 수명자들은 어떤 권리와 의무를 지는가'라는 보다 풍부하고 생산적인 논의로 인도해 준다.

\subparagraph{시원적 존엄으로서 박탈불가능성의 해명}

도덕 실재론은 도덕적 진술을 참이게끔 하는 독립적인 사실(moral fact)이 실재한다고 주장한다. 이러한 사실의 존재는 도덕적 판단을 가능하게 해 준다. 예를 들어 \textless 무고한 어린아이를 재미삼아 고문하는 것은 그르다\textgreater 에 해당하는 도덕적 사실F가 도덕적 논증과 독립적으로 존재한다고 하면, 이 사실F의 존재는 명제P ``무고한 어린아이를 재미삼아 고문하는 것은 그르다.''를 참으로 만들어 준다. 이러한 도덕적 사실의 존부는 경험적 사건에 영향을 받지 않는다. 설사 어떤 사람이 무고한 어린아이를 재미삼아 고문하는 사건이 발생한다고 해도 명제P가 거짓이 되는 것은 아니다.

이와 유사하게, 어떤 법적 논증에 따른 사법적 결론이 건전하게 참인 것으로 이해되기 위해서는, 법규범을 정당화해 주는 일정한 논증-독립적인 사실이 있어야 한다. 대표적으로, 인간존엄명제(Dignity Proposition, 이하 DP)는 ``모든 인간은 존엄하다.''라고 말하는데, 이 명제를 참으로 만드는 사실은 법질서 내에서 가장 핵심적인 법적 사실이다. 그리고 이러한 법적 사실은 이와 관련된 어떤 경험적 위반상황에 그 존부가 영향을 받는 사실이 아니다. 설사 어떤 이가 존엄한 인간을 적절히 존중하지 못한다 하더라도, 이 명제DP가 거짓이 되는 것은 아니다.

독일기본법의 인간존엄의 ``건드릴 수 없음''이나, 우리나라 헌법의 ``인간존엄을 가진다''는 표현도 이러한 맥락에서 이해되어야 한다. 이 표현들은 원리적으로는 침해가 가능하지만 침해해서는 안 되는 어떤 가치를 묘사한 것으로 이해되어서는 안 된다. 이는 말 그대로 법질서 내에서 인간존엄명제를 참으로 만드는 인간존엄의 법적 사실이 실재하고 있는 상황을 기술한 것으로 풀이되어야 한다. 그러한 의미에서 인간존엄의 존재함은 성취되어야 할 어떤 것이 아니라 법적 논증에서 이미 전제되어 있는 ``인간적으로 좋은 삶을 사는 것은 객관적으로 중요하다''는 논증-독립적인 사실이다. 건드릴 수 없음의 의미는 이러한 논증-독립적으로 존재함을 의미한다고 보아야 한다. 여기서 인간존엄은 앞에서 언급한 시원적 존엄(initial dignity)의 전통을 잘 따르고 있는 것이다.

따라서 이 DP문장의 구체화가 실제로 상실할 수도 있고 성취를 위한 행위를 요구하는 인간의 일부 가치나 능력을 중심으로 정의된다면, 이는 인간존엄의 법적 사실을 정확히 기술한 것이라 하기 어렵다. 예를 들어, 인간은 자율적 행위능력을 보유하고 있다거나, 인간은 품위있어야 한다는 식으로 인간존엄의 법적 사실을 구성해서는 안 된다. 적절한 인간존엄의 규정은 이러한 기술을 피할 수 있는 것이어야 한다.

이러한 기술이 바로 ``인간적으로 좋은 삶의 객관적 중요성'' 관념을 통해 가능하다. 이 객관적 중요성의 관념은 특정한 가치를 지지하는 것이 아니라 그러한 가치들을 누리고 사는 인간적으로 좋은 삶에 부착되는 속성이다. 이러한 속성은 서로 충돌하거나 대립하지 않으며, 다만 우리가 자신과 타인의 좋은 삶을 존중해야 할 이유를 제공해 주는 속성이다.

\subsubsection{독립적 ``인간존엄-이익'': 인간적으로 좋은 삶을 사는 것의 객관적 중요성에 대한 믿음형성의 이익}

앞서 살펴본 인간적으로 좋은 삶을 사는 것의 객관적 중요성으로서의 개념적 이해와 이를 배경으로 하고 있는 평등한 존중의 정당화 근거로서의 인간존엄의 이해방식은, 후술하는 바와 같이 존엄의 어원에 충실하면서도 현대 인권담론이 요청하는 역할을 대부분 충실히 수행하고 있다. 하지만, 특정한 침해의 대상으로서의 사용되고 있는 인간존엄의 용례를 충분히 설명해주지 못하고 있는 것처럼 보일 수 있다. 예를 들어, 노예화(slavery)가 인간존엄을 침해한다고 하거나 잔인한 고문을 하는 것은 해당 인격의 존엄을 침해한다고 하는 경우와 같이 소위 ``인간존엄(-이익) 침해''의 사안이 존재하는 것처럼 보인다. 그리고 이러한 사안의 존재로부터 침해의 대상이 되는 어떤 인간존엄-이익이 존재하는 것이 추정된다.

존엄 침해의 가능성을 주장해 왔던 기존의 일부 시도들은, 인간존엄을 기존에 파악해 왔던 복지이익이나 자율성의 범주를 크게 벗어나지 않는 이익이나 그러한 이익을 향유하는 지위 정도로 파악한다. 이러한 시도들은 인간존엄을, 예를 들어 어떤 명예나 중세 및 근대에 귀족들이 누렸던 의전상의 혹은 법적인 지위, 심리적 혹은 정신과학적 이익, 배타적인 자기결정권 혹은 평등한 대우와 같은 인간의 특유한 이익을 지칭하는 것으로 파악한다. 이는 인간존엄의 범위를 그 어의와 현대적 맥락의 본의에 걸맞지 않게 축소한다. 인간존엄의 이익은 기존의 이익이 요청하였던 것보다 더 특별하고 우선적인 보호를 요청한다. 그러나 명예와 같은 법익침해나 정신적 침해, 혹은 몇몇 사실적인 편의를 취하는 귀족적 지위와 같은 것들이 우리가 그동안 다루어왔던 다른 중요한 법익들---예를 들어 생명이나 신체의 자유와 같은 것들---에 비하여 현대사회가 인간존엄에 요청하는 특별하고 우선적인 보호가 필요하다는 주장은 쉽게 납득하기 어렵다. 예를 들어, 생명이라는 법익은 명예훼손의 명예의 법익보다 보통 우월한 것으로 인정되고, 명예훼손 상황에서 타인의 생명침해는 정당방위나 긴급피난으로도 인정되지 않는다. 그러나 명예훼손을 존엄의 법익으로 구성해버리고, 존엄의 우선성을 인정한다면 우리의 법체계는 자신의 명예훼손을 이유로 타인을 살해하는 행위를 보호해야 할지도 모른다. 존엄의 법익을 이러한 일종의 정신적 이익이나 어떤 구체적 이익들을 향유하는 지위로 보는 이상, 기존의 법적 담론에서 받아들이기 부담스러운 원리가 되거나, 그 범위가 어의에 걸맞지 않게 축소되고 만다.

보다 유망한 시도는 메이어 댄-코헨의 시도이다. 댄-코헨은 이러한 ``인간존엄 침해'' 사안을 기존의 해악원리나 자율성 원리로는 설명되지 않는 우선성과 의미의존성을 가지는 독립적 인간존엄의 원리와 관계된 것으로 파악한다.\footnote{Meir Dan-Cohen, ``Defending Dignity'', \emph{Harmful thoughts: Essays on law, self, and morality}, (Princeton, 2002).} 본고는 이러한 댄-코헨의 의미의존성 아이디어에 착안하여, 기존의 복지이익이나, 자율성 중심의 이익으로부터 구별되는, 제3의 의미의존적 이익이 존재함을 구체적으로 밝혀내고자 시도한다. 이러한 이익은 인간적으로 좋은 삶의 의미에 대한 정당화된 믿음형성의 이익이다. 이러한 믿음은 중심부 믿음과 주변부 믿음으로 구성된다. 그 중 중심부 믿음을 형성하는 데 있어 인간존엄-이익이 존재하며, 이 중심부 믿음을 침해하는 경우 `인간존엄-이익의 침해'가 발생한다. 아래에서는 이를 차례로 설명하고자 한다.

메이어 댄-코헨은 인간존엄을 침해하는 대표적인 사안으로 자유롭고 행복한 노예의 사례를 들면서, 노예가 자유롭고 행복하다 하더라도 노예화가 여전히 인간존엄을 침해한다고 주장한다. 그런데 이러한 노예화가 침해하는 것은 기존에 우리의 좋은 삶을 구성한다고 여겨왔던 이익들을 침해하는 해악이나 자율성이 아니라고 말한다.\footnote{Ibid., 154-157면.} 그렇다면 소위 `인간존엄 침해'의 사안에서 침해된 이익은 무엇인가? 위에서 설명한 인간적으로 좋은 삶을 사는 것의 객관적 중요성으로서의 인간존엄은 어떤 이익이 아니라 평가적 개념이기에 침해해서는 안 되는 것이 아니라 침해가 불가능하다. 따라서 이러한 인간존엄은 현대사회에서 ``나의 혹은 타인의 인간존엄이 침해되었다''라고 말할 때의 `인간존엄'이 지칭하는 바를 적절히 포섭하고 있지 못하다. 이는 침해가 가능한 인간존엄이며, 어떤 중대한 침해에 대하여 그 침해상태를 해소시키거나 보상할 수 있도록 해 주는 소위 ``인간존엄의 항변''을 가능하게 만든다. 과연 이러한 이익으로서의 인간존엄, 즉 `인간존엄-이익'은 무엇을 지칭하고 있는가?

\paragraph{좋은 삶을 구성하는 제3의 이익의 존재}

복지이익이나 자율성이 아닌, 좋은 삶을 구성하는 제3의 이익이 있는 것처럼 보이는 사례들이 발견되고 있다. 그리고 우리는 이러한 사례들 중 일부 혹은 전부에서 `인간존엄의 침해'라고 불릴 만한 무언가를 경험하기도 한다. 몇 가지 예를 살펴보자. 1) 미국 \emph{People v. Minkowski} 사건\footnote{\emph{People v. Minkowski}, 204 Cal. App. 2d 832 (1962).}에서 산부인과 의사인 민코스키는 피해자가 전혀 인지하지 못하는 상황에서 여러차례의 강간을 저질렀다(rape by deception). 병원에 여러차례 반복적으로 방문하는 과정에서 대부분의 환자들은 진료에 수반된 마취 등으로 인해 언제 어떤 일이 발생하였는지 인지하지 못했다. 강간임에 분명한 이 상황에서 침해된 해(harm)는 정확히 무엇인가? 육체적 손상(physical injury)이 단언되기 어려운 상황에서 이러한 종류의 해는 심리적인(psychological) 것이라고 말할 수도 있을 것이다. 그러나 피해자들이 여전히 인지하지 못하고 있다는 것을 가정하면, 이 해가 정신적 악영향(adverse mental effects)이라고 이해되기는 어려운 점이 있다. 댄-코헨은 피해자들의 알지못함(ignorance)이 아픈 경험으로부터 이들을 단절시켜주고 있다는 사실에도 불구하고, 이러한 강간이 강간으로서 비난받는데 있어서 아무런 문제가 없도록 해 주는 무엇인가를 찾으려고 시도한다.\footnote{Meir Dan-Cohen, ``Defending Dignity'', 153-154면.} 강간으로부터 침해되는 본질적인 이익은 무엇인가? 2) 우리는 신체에 가해지는 형벌에 대한 거부감을 갖고 있다. 미국 \emph{State v. Braxton}사건에서 성범죄자가 30년의 유기징역과 외과적으로 거세하는 것 중에서 자신의 처벌을 선택하라는 하급법원의 선고에 대해 피고인들은 거세형을 선택하려고 했으나, 항소심은 피고인들을 보호한다는 목적아래 거세형 선택지를 배제해 버렸다.\footnote{\emph{State v. Braxton}, 326 S.E. 2d 410 (S.C. 1985).} 우리는 이를 피고인에 대한 사실적인 복지이익이나 선택의 자유의 확보의 차원에서 설명할 수가 없다. 이때 피고인은 어떤 이익을 보호받은 것인가? 3) 유사하게, 알코올중독을 치료하려는 목적 아래 부인의 동의하에 남편 브라운이 부인이 술을 마실 때마다 반복적으로 부인을 때렸고 이로 인해 기소된 미국 \emph{State v. Brown}사건에서, 법원은 부인이 동의했다는 항변을 거부했다.\footnote{\emph{State v. Brown}, 364 A.2d 27 (N.J. Super. Ct. Law Div. 1976), aff'd, 381 A.2d 1231 (N.J. Super. Ct. App. Div. 1977).} 치료목적이라는 점과 부인이 동의했다는 점을 감안할 때, 법원은 어떤 법익을 보호하려고 한 것인가? 4) 프랑스에서는 일반적인 신장을 가진 성인들이 난쟁이를 멀리 던지는 것을 겨루는 난쟁이던지기 행사를 금지시켰다.\footnote{Conseil d\textquotesingle État, 27 octobre 1995, Commune de Morsang-sur-Orge. Manuel Wackenheim v France, Communication No 854/1999, U.N. Doc. CCPR/C/75/D/854/1999 (2002).} 이 사례에서 난쟁이들은 어떤 스포츠보다도 확실한 안전장비를 착용하고 있었고, 스스로 이러한 직업을 선택하였으며, 나름대로의 자긍심을 가지고 있는 경우도 발견되었다. 5) 좀 더 가상적인 사고실험의 사례로는 복지이익이 충분히 만족되고, 선택의 자유가 보장된 혹은 자신이 스스로 선택한 노예의 사례가 있다. 그럼에도 불구하고 그 누구도 노예가 되는 것을 금지하는 것은 어떤 이익을 보호하는가?

이러한 사례들은 해악의 원리가 보호하는 법익과 자율성 원리가 보호하는 법익만으로는 해결되지 않는 제3의 영역에 있는 이익이 존재함을 추정케 한다. 어떤 이들은 이러한 사례가 기존 법익들에 의해 설명된다고 주장할 지도 모른다. 예를 들어 외과적 거세형은 신체를 심각하게 침해한다. 따라서 해악의 원리에 의해 해명된다고 주장할 수도 있다. 그러나 우리는 자발적 거세가 허용되는 사례를 알고 있다. 성전환을 원하는 사람이 의료적으로 거세하는 경우가 이에 해당된다. 그럼에도 불구하고 왜 형벌로서의 거세형은 선택할 수 없는가? 알코올중독을 이유로 동의하에 부인을 때린 경우도 마찬가지의 사례다. 해악의 원리로 설명될 수 있는 것처럼 보이지만, 사실 권투경기 같은 경우에는 동의된 폭력이 허용된다. 난쟁이 던지기 사례의 경우, 해리엇 바버는 원거리 해악 개념을 적용한다. 이 활동은 참가자들의 해악 없이 안전하게 수행될 수 있음을 인정하지만, 원치 않는 비-참가자들인 다른 난쟁이들에 원거리 해악(remote harms)을 유발한다는 근거 하에서 해악 원리 아래 금지되어야 한다고 주장한다.\footnote{Harriet Erica Baber, ``The Ethics of Dwarf Tossing'', \emph{International Journal of Applied Philosophy} Vol. 4, Issue 4 (1989).} 그러나 이렇게 해악의 개념을 확장시키는 것은 해악원리의 사실적 한계 원리로서의 기능을 반감시킨다.

노예의 경우 일반적으로는 지극히 궁핍하고 폭력에 시달리는 노예를 상상하면서 복지이익이 침해되는 경우라고 생각하기 쉽지만, 자비로운 주인에 의해 충분한 복지이익을 향유하는 노예를 가정할 수도 있다. 이 경우에도 자율성 이익의 침해는 여전히 남는다는 것이 일반적인 생각이겠지만, 노예와 자율성의 침해의 관계는 의외로 그렇게 분명하지 않다. 먼저 당사자 간의 노예계약의 합의에 의해 노예화 된 경우, 노예화단계에서는 자율성의 침해가 없다. 스스로 선택한 노예의 경우 노예에게서 침해되는 핵심적인 자율성은 지속적으로 노예로 살아가면서 겪는 선택의 자유의 제한이다. 그러나 선택의 자유는 대부분의 자유로운 성인에게도 무한정 인정되는 것이 아니다. 선택의 자유가 유의미한 정도로 제한된다고 하려면, 다른 제한들보다 더 제한되어야 한다. 심각한 중증장애인의 경우 평균적인 노예들에 비해 행위선택의 자유가 더 제한된다. 다른 이들은 사실상의 자율성과 법적 자율성을 구분하여, 노예는 사실상 자율성을 향유하지 못하는 것이 아니라 법적으로 자율성을 향유하지 못한다는 측면에서 노예화를 비판한다. 그렇다면 사실적으로 자율성을 향유하지만 법적으로는 향유하지 못한다는 데 있어 어떤 사실상의 악(evil)이 존재하는가? 왜 법적 자율성이 그토록 중요한가? 사실상 자유롭고 만족스러운 노예에게 여전히 남아있는 악은 무엇인가?

이들은 주로 해악의 원리와 같은 기존 자유주의적 원리들의 변용만으로는 설명하기 어렵다. 역사적으로 이 시점에서 이러한 문제가 표면화된 크게 두 가지의 계기가 있다. 하나는 2차 대전의 나치독일의 유태인 학살과 같이 해악의 수량적 증가 이상의 무언가의 침해를 드러낸 사건들이 발생했다는 역사 경험적 계기와, 다른 하나는 생명공학 내지 의료기술의 발전에 따라 인간 생명에 대한 세밀한 조작이 가능해지게 된 상황적 계기가 그것이다. 나치독일의 잔악성은 엄청난 충격을 국제사회에 던져주었는데, 이는 단순히 각인의 살해행위를 합한 의미로서의 대량학살 그 이상의 무언가를 내포하고 있고, 이는 ``다시는 일어나서는 안 됨(\emph{Never again})''이라는 구호아래 국제인권규약의 강화라는 결과를 가져왔다. 생명공학기술의 발달은, 예를 들어, 유전자 가위 등으로 인류의 특성을 의도적으로 조작하거나, 키메라와 같은 새로운 생명체의 탄생을 조작할 수 있다는 가능성을 열어주고 있는데, 이러한 가능성을 적절히 규제해야 한다는 공감대가 형성되어 있다. 그러나 기존의 자율성이나 해악금지원리와 같은 지도원리들은 왜 인간유전자를 함부로 조작해서는 안 되고 키메라의 탄생을 유도해서는 안 되는지에 대한 설득력 있는 설명을 제공하지 못하고 있다. 이러한 상황에서 위와 같은 사례에 침해되고 있는 어떤 제3의 독립적 보호이익의 규명이 요청된다.

\paragraph{제3의 이익의 의미의존성}

지위나 복지이익, 자율성 등으로 환원하는 기존 시도들의 실패를 반복하지 않으면서도, 이러한 새로운 사례들의 핵심적인 악으로부터 보호해야할 이익은 어떤 것인가? 이 이익의 성격은 댄-코헨의 의미의존성(meaning dependence) 개념에 의해 그 규명의 단초를 얻을 수 있다. 해당 사례들은 보통 일컬어지는 사실적인 이익이나 자율적 선택의 이익을 침해당한 것이 아니다. 그러나 기존의 원리, 특히 해악의 원리 역시 우리에게 가해지는 악을 남김없이 서술하려는 완전한 테스트를 제공하려는 시도의 일종이기도 했기에, 남은 악을 서술한다는 것은 쉬운 일이 아니다. 여전히 남아있는 악은 무엇일까? 댄-코헨은 이것을 행위의 의미와 관계시킨다. 노예화는 왜 나쁜가? 노예에게 가하는 해악이나, 단순히 노예의 자율성 행사가능성의 축소가 아니라면 무엇일까? `노예'가 지칭하는 의미에는 제도나 관행에 의해 모욕적이고 비하적인 의미가 축적되어 왔고, 이러한 `노예'에 붙여진 의미가 누군가를 노예화하는 제도에 대한 금지로 이어진 것이다. 성범죄자인 피고인에 대해 외과적거세형의 선택가능성을 박탈한 \emph{State v. Braxton} 사건에서 보았듯, 거세형이 나쁜 이유는 복지이익이나 자율성의 후퇴가 아니다. 사실적 차원에서 똑같은 외과적 거세의 의미를 부여받을 수도 있는 자율적 성전환자의 외과적 성전환 시술은 전혀 존엄 침해의 평가를 받지 않음에도 불구하고, 거세의 형벌이라는 의미를 갖는 거세형은 전혀 다른 평가를 받는데 이는 결국 해당 개념의 평가가 지극히 의미의존적이라는 것이다. \emph{State v. Brown} 사건에서 동의된 폭력은 `와이프 때리기'가 지칭하는 의미를 함축하지만, 권투시합에서의 동의된 폭력은 `스포츠'가 지칭하는 의미를 수반한다.\footnote{Meir Dan-Cohen, ``Defending Dignity'', 161-163면 참조.} 즉, 위 사안에서 침해된 이익은 그 행위가 가지는 의미에 의존되어 있다.

\paragraph{인간적으로 좋은 삶을 사는 것의 객관적 중요성에 대한 믿음}

그렇다면, 복지이익이나 자율성이 아닌, 좋은 삶을 구성하는 의미의존적인 제3의 이익은 구체적으로 어떤 이익을 말하는 것인가? 먼저 기존에 좋은 삶을 구성하고 있는 것으로 여겨왔던 이익들을 살펴보자.

\subparagraph{좋은 삶을 구성하는 이익들}

좋은 삶을 구성하는 요소들에 대하여 여러 규범이론과 도덕철학들이 상이한 분석을 내어놓을 수 있고, 공동체마다, 그리고 개인마다 다른 이해방식을 형성할 수 있겠지만, 우리는 무엇이 좋은 삶인지를 정의하는 보편적으로 정당화된 명제P\emph{gl}(Good Life Proposition)이 예를 들어 다음과 같은 것이라고 일단 가정해 볼 수 있다.

\begin{longtable}[]{@{}
  >{\raggedright\arraybackslash}p{(\linewidth - 0\tabcolsep) * \real{1.0000}}@{}}
\toprule\noalign{}
\begin{minipage}[b]{\linewidth}\raggedright
\textbf{P\emph{gl}}: ``인간으로서의 삶의 좋음이란 자신이 속한 공동체와 함께 어울려 개인으로서 풍부한 복지기반 안에서 자아실현을 하며 평등한 대우를 받고 평화로움을 만끽하며 수명을 다하여 살아가는 것이다.''
\end{minipage} \\
\midrule\noalign{}
\endhead
\bottomrule\noalign{}
\endlastfoot
\end{longtable}

이 예는 주로 현대 복지국가의 이상에서 법이 보호해야할 개인적 혹은 공동체적 이익들이라고 평가되어 온 것과 자율성의 법익, 평등의 요청 등을 종합해 본 것이다. 이런 보편적 명제는 개별 인간들의 취향이나 믿음이 반영된 주관적으로 좋은 삶과는 구별된다. 예를 들어, 이러한 보편적 P\emph{gl}은 ``수술시 위급한 경우 필요하다면 수혈을 받는 것이 좋다.''는 결론을 함축하고 있다고 하자. 그런데 이 P\emph{gl}이 객관적으로 참이라고 하더라도 모든 사람들이 이를 참으로 받아들이는 것은 아니다. 사람들은 이 명제에 대한 각 개인의 주관에 따라 참이라는 믿음태도(belief)를 형성할 수도 있고 거짓이라는 믿음태도를 형성할 수도 있다. 보편적 P\emph{gl} 명제와 대조적으로, 우리는 A라는 사람이 좋은 삶에 대한 의미를 부여하는 어떤 명제A-P\emph{gl}(Good Life Proposition for A)이 참이라는 믿음을 주관적으로 형성하게 되는 것을 가정할 수 있다. 예를 들어, A가 P\emph{gl}, ``인간으로서의 삶의 좋음이란 자신이 속한 공동체와 함께 어울려 개인으로서 풍부한 복지기반 안에서 자아실현을 하며 평등한 대우를 받고 평화로움을 만끽하며 수명을 다하여 살아가는 것이다.''을 대부분 받아들이지만, 동시에 그는 종교적인 이유로 ``위급상황에서도 수혈받는 것을 거부해야 한다''고도 생각한다고 가정하자. 그가 믿고있는 A-P\emph{gl}은 다음과 같을 것이다.

A-P\emph{gl} : ``인간으로서의 삶의 좋음이란 자신이 속한 공동체와 함께 어울려 개인으로서 풍부한 복지기반 안에서 자아실현을 하며 평등한 대우를 받고 평화로움을 만끽하며 수명을 다하여 살아가는 것이다. 다만, 설사 건강이나 생명에 치명적인 위협이 있다 하더라도 수술 등의 위급한 과정에서도 수혈을 받아서는 안 된다.''

그런데 수혈을 포함한 상황에서, A의 복지와 자율적 선택 등의 사실적 기반이 그에게 좋은 삶을 가져다 주는 것을 객관적으로 정당화시켜주는 근거가 되는 명제는 주관적으로 형성된 A-P\emph{gl}이 아니라 객관적으로 정당화된 P\emph{gl}이다. A-P\emph{gl}은 A만이 주관적으로 믿음태도를 형성한 것일 뿐이어서 오류의 가능성이 있기 때문이다. A가 위중한 상황에서 본인이 수혈받았다는 사실을 모른다면, A의 객관적 복지는 수혈받았을 때가 수혈을 받지 않았을 때보다 더 성취되기 쉽다. P\emph{gl}명제는 인간으로서의 삶의 좋음에는 복지나 자율성의 요소가 필요함을 보통 함축하고 있을 것이다. 따라서 우리는 P\emph{gl}로부터, A가 복지와 자율성을 누리고 있을 때 A가 인간적으로 좋은 삶을 살기 위한 중요한 기반들을 확보했다고 말할 수 있고, 반대로 A에 대하여 복지에 대한 침해나 자율성에 대한 침해가 발생했을 때 우리는 A의 인간적으로 좋은 삶이 침해되었다고 말할 수 있다.

\subparagraph{정당화된 믿음 형성의 이익}

그러나 A의 사례를 통해 우리는 좋은 삶에 필요한 추가적인 요소를 발견할 수 있다. A가 생명이 위중한 상황에서 수혈을 받았고 이를 알게되었다고 하자. 그의 수혈거부에 대한 믿음이 확고한 경우, 수혈로 인하여 A는 분명히 완전한 좋은 삶을 누리기 어렵게 될 것이다. A의 경우에서 사람들은 복지와 자율성과 같은 사실적 요소들의 충족만으로 인간적으로 행복한 삶을 누리고 있지 않다는 사실을 확인할 수 있다. 사람들은 좋은 삶을 누리기 위해서 복지이익과 자율성이익 등의 요소들로 구성된 삶(P\emph{gl})을 단지 살고 있는 것이 아니라, 이러한 삶을 사는 것의 의미를 이해하고 이러한 이해를 기반으로 그러한 삶이 객관적으로 중요한 것이라는 믿음을 형성하고 유지하면서 산다. A와 사람들은 위의 P\emph{gl}요소들을 완벽하게 누리더라도 자신의 믿음에 위배되는 (수혈을 받는) 상황에 처하면 충분히 행복을 누리기 어렵다. 그러한 삶이 진정으로 행복한 삶이 아니라는 단지 인지적 태도만을 가지는 것이 아니라 객관적으로 중요한 것이 붕괴되었다고 생각하기 때문이다. 일반적인 사람들은 수술시에 수혈을 받는 것이 좋은 삶을 증진시키는 요소이고 필요적절한 의료행위를 받고 사는 삶이 객관적으로 중요하다고 생각하겠지만, 수혈을 거부하는 신념을 가진 A에게는 수혈을 받는 사태가 (객관적으로는 P\emph{gl}에 부합할 지 모르지만) A의 완전한 좋은 삶을 형성하는 것을 어떤 특유한 방식으로 방해한다.

만약 A가 어떤 위급한 수술상황에서 원치않지만 필요한 수혈을 받게 되었다면, 이 때 A에게서 좋은 삶을 사는 것을 방해한 요소는 결국P\emph{gl}에 포함되었던 사실적 요소가 아니라, 이러한 명제들에 대한 태도, 즉 믿음과 관련되어 있다. A는 수혈을 받을 때가 아니라 자신이 수술과정에서 수혈을 받았다는 것을 알게되는 순간, A는 수혈받았다는 사실 그 자체 때문이 아니라, P\emph{gl}의 의미에 기반한 삶의 객관적 중요성을 믿지 않고 A-P\emph{gl}의 의미에 기반한 삶의 객관적 중요성을 믿고 있다는 점 때문에 그의 주관적으로 좋은 삶은 저하될 수 있다. A가 A-P\emph{gl}을 믿건 아니면 P\emph{gl}을 믿건 상관없이, 수혈에 의해 이 때 그의 좋은 삶이 저하되거나 증진된다는 사실로부터 알 수 있는 것은 A가 해당 명제의 의미를 이해하고 그 의미가 객관적으로 중요하다고 생각한다는 점이 그의 좋은 삶의 핵심부분을 구성한다는 점이다.

그렇다면 이 때 침해된 이익은 무엇인가? 믿음 그 자체는 주관적으로 형성한 개인의 태도에 불과하다. 개개인이 형성하고 있는 주관적 믿음 그 자체는 억측에 기반할 수도 있고, 개개인의 성급한 성격의 산물일 수도 있다. 수혈거부의 사례에서 수혈을 받는 경우 A의 좋은 삶을 침해하는 원인제공자는 수혈행위를 수행하는 의사가 아니라 억측을 형성시킨 자신일 수 있다. 따라서 개인이 억견에 기초하여 형성한 믿음태도 그 자체는 개인의 주관적으로 좋은 삶을 결정할 수는 있어도 어떤 보호나 존중을 요청하는 이익은 아니다.

그러나 각 구성원이 정당화된 믿음을 형성하고 유지하는 것에는 중대한 이익이 있다. 이러한 믿음을 정상적으로 형성하지 않으면 개개인은 그러한 믿음에 기반하여 객관적으로 좋은 삶을 구성하여 살아갈 수 있는 토대를 잃고, 그러한 구성원들이 속해있는 공동체 역시 목표와 방향을 상실하게 되기 때문이다. 잘못된 믿음형성을 유도하는 것은 원래 형성되었어야 할 정당화된 P\emph{gl}에 기반한 삶의 객관적 중요성에 대한 믿음형성의 이익을 침해한다. 이러한 정당화가 주로 공동체 안에서 이루어진다는 점에서 이를 공동체의 믿음형성의 이익으로 이해할 수도 있다.\footnote{이러한 이익은 2차 침해(정신적 침해)에 의해 훼손되는 신체적 혹은 정신적 이익과는 구별해야 한다. 어떤 주관적 믿음의 침해가 2차적으로 불러일으키는 어떤 충격이나 공포, 허탈감과 같은 주관적인 심리적 정신적 상태(2차 침해)가 존재할 수 있다. 예를 들어, 국정농단의 주범으로 지목된 최서원 및 전 이화여대 총장 등이 연루된 정유라의 이화여대 입학을 둘러싼 이대학사비리 사건에서, 법원은 ``최씨의 범행으로 인해 국민과 사회 전체에 준 충격과 허탈감은 그 크기를 헤아리기 어렵고, 누구든 공평한 기회를 부여받고 열심히 배우고 노력하면 그에 상응하는 결과를 얻으리란 믿음 대신 `빽도 능력'이란 냉소가 사실일지 모른다는 의구심마저 생기게 했다''고 설시했다. 이 때 ``국민과 사회 전체에 준 충격과 허탈감''이나 ``\,`빽도 능력'이란 냉소가 사실일지 모른다는 의구심''을 통해 알 수 있는 침해된 것은 개별 국민들이 가지게 된 어떤 충격감이나 허탈감각과 같은 구체적이고 주관적인 정신적 상태이다. 그러나 이것이 이 사안에서 보다 본질적으로 침해된 것으로 보이는 공동체의 믿음형성의 이익 침해 그 자체를 기술하고 있는 것은 아니다. 앞서 수혈거부의 신념을 가진 A의 A-P\emph{gl}에 대한 믿음이 가져온 수혈에 의한 정신적 충격 사례에서 보듯, 주관적 믿음태도로 인한 충격은 국가 간섭을 바로 불러일으키지 않는다. 이 때 침해된 진정한 이익은 국민 혹은 공동체 구성원들이 정당하게 형성해 왔던 어떤 공동체의 이상이 객관적으로 중요하다는 것에 대한 믿음형성의 이익이라고 보아야 한다. 이 믿음의 유지가 건강한 공동체를 유지하는 데 핵심이 되기 때문이다. 서울중앙지방법원2017.6.23 선고2017고합76.연합뉴스 ``\,`국정농단' 최순실 첫 유죄 징역 3년\ldots 이대비리 9명 모두 유죄(종합)''(http://www.yonhapnews.co.kr/bulleti

  n/2017/06/23/0200000000AKR20170623066351004.HTML?input=1195m)등 참조.}

이렇게 이해된, 인간적으로 좋은 삶에 대한 ``정당화된 믿음 형성의 이익''은 인간적으로 좋은 삶을 사는데 있어 핵심이 되는 구성요소가 된다. 이로부터 우리는 좋은 삶은 복지와 같은 사실적 구성요소뿐만 아니라 정당화된 P\emph{gl}에 기반한 삶의 객관적 중요성에 대한 믿음의 형성에도 좌우된다는 점을 알 수 있다. 따라서 이를P\emph{gl}에 추가하여 반영한 P\emph{gl}+를 보다 구체적으로 기술하면 다음과 같다.

% 필요 시 한글 각주와 박스가 어울리도록 조정 가능
\newtcolorbox{highlightbox}{
  colback=gray!5,
  colframe=black!40,
  boxrule=0.4pt,
  arc=2pt,
  left=6pt,
  right=6pt,
  top=6pt,
  bottom=6pt,
  fonttitle=\bfseries,
}

\begin{highlightbox}
\textbf{P\emph{gl+}}: ``인간적으로 좋은 삶을 산다는 것은 자신이 속한 공동체와 함께 어울려 개인으로서 풍부한 복지기반 안에서 자아실현을 하며 평등한 대우를 받고 평화로움을 만끽하며 수명을 다하여 살아가는 것이며, 또한 동시에 이러한 인간적으로 좋은 삶을 사는 것이 객관적으로 중요하다는 정당화된 믿음을 형성하며 살아가는 것이다.''\footnote{자신의 삶을 살 만한 것으로 가치있게 여겨야 한다는 것이 좋은 삶의 기술에 반드시 포함되어야 한다는 P\emph{gl}+의 기술은 다음의 두 학자의 의견에 영향을 받은 것이다. 하나는 크리스틴 코스가드의 실천적 정체성 이해방식(conception of practical identity)이고, 다른 하나는 드워킨의 좋은 삶의 두 원칙이 말하려는 바이다. 코스가드의 실천적 정체성 이해방식은 ``우리 자신에 대한 특정한 기술을 통해서 우리의 삶을 살 만한 가치가 있는 것으로 생각하고 우리의 행위를 수행할 만한 가치가 있는 것으로 생각하는 그런 기술''을 말한다. 그에 의하면 반성적 존재가 행위의 근거를 찾기 위해서는 이러한 이해방식을 가지고 있어야 한다. (그리고 이러한 실천적 정체성은 여럿으로 이루어져 있으며, 각 정체성마다 그러한 정체성을 드러내는 행위법칙으로 구성된다. 그리고 스스로 그러한 행위법칙을 위반할 때마다 이 실천적 정체성은 훼손된다. 그러나 여기서는 타인에 의한 훼손가능성을 염두에 둔다.) Christine Korsgaard, \emph{The Sources of Normativity} (Cambridge, 1996). (김양현⋅강현정 역, 규범성의 원천 (철학과 현실사, 2011), 168-171면 참조), 168-171 참조. 드워킨의 좋은 삶의 두 원칙은 Ronald Dworkin, \emph{Justice for Hedgehogs} 참조.}
\end{highlightbox}

정당화된 믿음형성-이익은 제3의 이익이 갖추어야 할 의미의존성 요건을 잘 충족시킨다. 믿음형성-이익은 그 이익의 내용이 좋은 삶에 대한 의미를 기반으로 한 그에 대한 믿음형성이라는 측면에서 의미의존적이다. 난쟁이 던지기 사례나 나치독일의 인종대량학살 사례, 소위 해악없는 강간(harmless rape)사례,\footnote{앞서 미국 \emph{Minkowski}사건에서 보았듯, 의식 없는 희생자가 비밀리에 타인에 의해 어떤 육체적 손실(damage)없이 침해(violate)되고 사건이 밝혀지지 않은 사례를 뜻한다. John Gardner and Stephen Shute, ``The Wrongness of Rape'' in Horder J (ed.) \emph{Oxford Essays in Jurisprudence 4th Series} (2000), 195-199면. 해악에 대한 자율성 우위의 이해방식에서, 이러한 강간은 해당 희생자의 자율성을 위한 미래의 능력에 영향을 미치지 않기 때문에 해악이 없는 것으로 간주된다. Julia Davis, ``Forbidding Dwarf Tossing: Defending Dignity or Discrimination Based on Size?'', \emph{Yearbook of New Zealand Jurisprudence}, Vol. 9 (2006), 247면.} 그리고 행복한 자율적 노예가 제3의 이익을 요청하는 이유는 이러한 사안에서 침해된 이익들이 인간적으로 좋은 삶을 구성하는 명제가 가진 의미에 의존하고 있기 때문이다. 먼저 나치독일의 인종대량학살 사례의 경우는 사실적 원리들인 해악의 원리나 자율성 원리만으로도 이미 우리는 충분히 그 해악과 위법성을 논증할 수 있다. 이는 생명권의 과도한 침해이고, 이것이 대량으로 일어났다는 점에서 그 해악의 크기는 지나치게 크다. 그럼에도 불구하고 2차 대전 후의 국제정세는 대량학살에 의한 생명과 신체 훼손에 의한 해악의 총합 이상의 무엇, 즉 인간존엄의 심대한 침해라는 별도의 의미침해적 요소를 나치독일의 행위에 부여하려고 애썼다. ``특정인종을 지목해서 조직적으로 인종청소를 단행한 행위''가 인간의 좋은 삶의 의미P\emph{gl}이 의미하는 좋은 삶을 사는 것의 객관적 중요성에 대한 인류 공동체 구성원의 믿음을 형성하고 유지하는 데 심각한 위협을 가했기 때문이었다.

지금까지 의미의존성을 가지는 이익은 좋은 삶의 객관적 중요성에 대한 정당화된 믿음형성의 이익이라는 것을 살펴보았다. 이는 기존의 이익과 독립적인 성격을 갖지만, 이러한 이익이 다른 기존의 이익들에 비하여 항상 우선성을 가지지는 않는다. 그런데 때로는 믿음형성의 이익이 복지이익이나 자율성과 같은 기존의 이익에 우선하는 경우가 있는 것으로 보인다. 믿음형성-이익은 다른 해악이나 자율성의 이익에 대하여 언제 어느 정도로 우선성을 가지는가?

\paragraph{좋은 삶에 관한 중심부 믿음형성의 이익의 우선성}

좋은 삶에 대한 모든 종류의 믿음형성의 이익이 자율성이나 복지이익에 우선하는가? 그렇지 않다. 믿음형성의 대상이 되는 믿음들 중에서, 주변부 믿음과 중심부 믿음이 구별될 수 있으며, 우선성을 보유할 수 있는 것은 중심부 믿음이라고 할 수 있다.

\subparagraph{믿음체계의 구조: 인식론적 전체론과 중심부 믿음}

우리가 형성하고 있는 믿음들 중 많은 것은 좋은 삶을 살아가는데 큰 영향을 미치지 않는 주변적인 믿음도 있고, 좋은 삶에 큰 변화를 일으키는 중심적인 믿음도 있다. 그런데 이러한 믿음은 개별 명제에 대한 각각의 믿음의 총합으로 구성되는 단순한 구조를 띠고 있지는 않다.

먼저 믿음의 대상이 되는 명제, P\emph{gl}+는 어떤 원자적인 문장들로 구성된 단순한 집합이 아니라 문장들 상호 간에 영향을 주고받는 복합적 이해방식(conception), 혹은 다수의 문장으로 이루어진 하나의 이론체계라고 할 수 있다. 예를 들어, 대표적인 P\emph{gl}+의 사례로 코스가드의 실천적 정체성 이해방식을 들 수 있다. 코스가드에 의하면 실천적 정체성은 ``우리 자신에 대한 특정한 기술을 통해서 우리의 삶을 살 만한 가치가 있는 것으로 생각하고 우리의 행위를 수행할 만한 가치가 있는 것으로 생각하는 그런 기술이다.'' 그런데, 그에 따르면 이 ``실천적 정체성은 복잡한 문제이며 보통의 사람에게는 그런 이해방식이 복합적으로 되어있을 것''이라고 말한다.\footnote{Christine Korsgaard, \emph{The Sources of Normativity} (Cambridge, 1996). (김양현⋅강현정 역, 규범성의 원천 (철학과 현실사, 2011), 170면).} 실제로 한 사람이 자신의 좋은 삶을 이해하는 방식 혹은 한 공동체의 대표적인 구성원이 정당화된 인간적으로 좋은 삶을 이해하는 방식, 그리고 그 객관적 중요성을 이해하는 방식은 매우 복잡한 구조를 띤다.

콰인은 그의 인식론적 전체론(Epistemological Holism)을 설명하면서 검증이나 반증의 대상이 되는 이론의 단위가 하나의 개별문장이 아니라 이론들 전체에 해당하는 문장들 전체의 집합이라고 이야기한다. 우리의 ``지식 또는 믿음들의 총체는 오직 그 언저리에서만 경험과 부딪치는 인공적 제조물이다.''\footnote{Willard Van Orman Quine, ``Main Trends in Recent Philosophy: Two Dogmas of Empiricism'', \emph{The Philosophical Review}, Vol. 60, No. 1 (1951). (허라금 역, ``경험주의의 두 가지 도그마'', 논리적 관점에서 (서광사, 1993), 61면).} 마찬가지로 우리의 좋은 삶의 구성요소와 그 객관적 중요성을 말하는P\emph{gl}+과 P\emph{dig}도 매우 복합적인 양상을 띠는 이해방식이며, 따라서 그 검증과 반증의 대상이 되는 것은 이들을 구성하는 개별문장이나 단순한 문장들의 모음인 집합이 아니라 이들을 구성하는 문장들이 형성하는 장(field)으로서의 전체라고 할 것이다.

인식론적 관점에서 세계에 대한 어떤 진술도, 예를 들어 심지어 배중률과 같은 ``순수 수학과 논리학의 심오한 진술조차'' 그와 상반되는 경험과 마주치면 포기할 수 있다.\footnote{Ibid.} 경험은 주로 주변부 경계의 진술들과 맞닿아 있을 뿐이지만, 과학 전체라는 장을 통해 논리적으로 상호 연결된 다른 진술들에 영향을 미치고 결국은 핵심부의 논리법칙에 영향을 미칠 수도 있다는 것이다. 그러나 배중률과 같은 중심부의 중대한 논리법칙들은 그 진술을 포기하는 경우 전체 과학체계에 심대한 영향을 끼치기 때문에, 그보다는 주변부의 진술들을 적절히 조정하는 수준에서 마무리되기 마련이다. 예를 들어, 하나의 과학이론체계 S가 ``빛보다 빠른 것은 없다.''라는 명제에 대한 중심부 믿음을 가운뎃고리로 전제하고 있고, 이 명제에 대한 믿음을 흔들어놓는 과학적 증거가 발견되면, 해당 과학이론 전체가 위태로워진다고 하자. 대개의 과학 실험은 체계S의 주변적 진술들을 강화하거나 때로 약화시키는 것과 관계되어 있다. 특별한 흠 없이 수행된 과학실험이 체계S의 주변적 진술을 약화시키는 경우, 보통은 ``빛보다 빠른 것은 없다.''는 중심부 진술을 변경하기 보다는, 주변에 있는 다른 진술을 변경하기 마련이다. 심지어 한 과학자가 잘 통제된 실험을 통해 빛보다 빠른 물질을 발견했다고 주장하여 중심부 믿음을 구성하는 명제를 직접 반증할 때조차, 기존의 이론체계 S를 믿고 있는 구성원들은 자신들의 중심부 믿음을 수정하기보다는 일단 실험의 통제 실패나 수행방법을 의심하는 방식으로 반응한다. 이러한 방어적 반응은 한편으로 해당 중심부 믿음을 유지하는 것이 과학체계S에서 차지하고 있는 중요성을 나타내준다.

\subparagraph{좋은 삶에 관한 중심부 믿음과 그 형성의 이익}

우리의 좋은 삶의 구성요소를 진술하는 명제 P\emph{gl}+에 대한 믿음체계S\emph{pgl+}와 관련해서도, 그 반증의 대상이 되는 것은 믿음의 대상인 진술들을 구성하는 개별문장이나 단순한 문장들의 모음인 집합이 아니라 이들을 구성하는 문장들이 형성하는 장(field)으로서의 전체라고 할 것이다. 그런데 과학적 진술에 대한 믿음체계와는 달리 도덕적 진술에 대한 믿음체계는 그 침해, 그리고 그로 인한 포기와 변경에 의한 사회적 영향이 다르다. 과학적 믿음체계의 포기와 변경은 자연세계에 대한 인식의 진보와 과학기술의 발전을 의미할 수 있지만, 도덕적 믿음체계의 포기와 변경은 그러한 믿음에 기반한 개인적 삶의 정체성 형성을 방해하거나 이를 기반으로 하는 공동체의 결속을 붕괴시킬 가능성이 있다는 점에서 그 중요성이 구별될 수 있다.

먼저 과학적 진술의 경우와 마찬가지로, 우리의 좋은 삶의 구성요소의 이해방식에 대한 주변부 믿음도 주변부의 경험, 즉 일상적인 사소한 침해들에 의해 변경되고 포기될 가능성이 있지만, 그 중심부의 믿음이 쉽게 포기되지는 않는다. 대개의 경미한 침해들은 예외적인 상황으로서 우리의 인식을 통해 조정되게 마련이고, 드물지 않게 발생하는 일상적 범죄행위나 굴욕감이나 불쾌감을 불러일으키는 많은 상황들은 국가 공권력의 간섭과 구제에 의해 조정된다. 우리는 사회에 살인행위가 일부 존재한다고 하여 생명이라는 가치가 중요하다는 믿음 그 자체가 붕괴된다고 생각하지는 않는다. 이러한 행위는 다만 일탈행위일 뿐이며 생명은 여전히 존귀하다고 생각한다. 그리고 이러한 가치는 비록 사후적 수단이기는 하지만 형벌이라는 국가간섭에 의해 그 중요성에 대한 믿음이 강화된다.

그러나P\emph{gl}+를 구성하는 핵심적 문장들, 체계 전체를 긴밀히 연결하는 고리가 되는 중심적 진술들을 재고하지 않을 수 없게 만드는 침해의 경험에 부딪히게 되면 그 믿음체계S\emph{pgl+} 전체를 포기해야 할 수도 있는 상황에 직면하게 된다. 이러한 중심부 믿음에 해당하는 것은 ``모든 인간이 인간적으로 좋은 삶을 영위하는 것은 객관적으로(누구에게나) 중요하다''라는 인간존엄명제P\emph{dig}에 대한 믿음(B\emph{pdig})이라고 할 수 있다. 이 믿음은 모든 다른 윤리적 믿음들을 연결해주는 가운뎃고리이기 때문이다.

이러한 믿음을 형성할 구성원들의 이익은 대표적으로 P\emph{gl}+를 구성하는 핵심적 문장들이 지시하는 가치에 대한 공동체 구성원들의 무관심이나 냉소로 인하여 약화되기도 하고, 국가가 적극적으로 해당 가치를 평가절하하는 행위에 동조함으로써 또한 침해되기도 한다. 사회에 살인행위가 존재한다는 사실만으로는 생명이 중요하다는 믿음체계가 붕괴되지 않지만, 살인행위가 당연시되고 만연하며, 심지어 국가에 의해서도 살인행위로부터의 보호나 구제가 전혀 이루어지지 않고 심지어 국가가 살인행위를 적극적으로 일삼거나 방조한다면 이러한 객관성에 대한 믿음체계는 붕괴의 위험에 직면하게 된다. 2차세계대전 당시의 나치에 의한 유태인 학살이 `인간존엄(-이익)의 침해'로 이해되는 이유를 이 맥락에서 이해할 수 있다. 유태인에 대한 혐오와 그 인간성에 대한 침해가 일상적으로 만연한 가운데, 이에 대한 나치정부의 공식적 동조는 모든 인간이 좋은 삶을 살아야 할 객관적 중요성을 가진다는 중심부 믿음B\emph{pdig}을 붕괴시켰기 때문이다. 따라서 우리가 인간적으로 좋은 삶을 살기 위해서는 좋은 삶에 관한 중심부 믿음B\emph{pdig}을 형성하고 유지할 이익을 가진다.

\subparagraph{좋은 삶에 관한 중심부 믿음 형성 이익의 우선성}

이러한 중심부 믿음과 관계된 이익은 다른 이익인 복지-이익이나 자율성-이익에 일응 우선하는 것으로 여겨진다. 한 공동체 구성원들의 믿음의 대상이 되는 건전하고 정당화된 P\emph{gl}+의 명제의 중심부에, ``아버지와 딸 사이에 (설사 하늘이 무너진다 하더라도) 성관계를 해서는 안 된다.''는 절대적 근친성교금지(incest taboo)의 강한 함축이 있을 수 있다는 것을 가정하는 것은 어렵지 않을 것이다. 그리고 이 공동체에서 근친성교가 인정된다는 것은 공동체 가치체계의 종말이나 재앙의 의미를 가지고 있을 수 있다. 그런데 어떤 범죄자가 이 공동체의 평화로운 한 가정집에 침입하여 강도를 행하던 중에 두 부녀(父女)를 발견하고 두 부녀 간에 성교를 하지 않으면 아버지를 죽이겠다고 강요한다고 하자. 강도가 1) 여성을 직접 강간하는 행위와, 이 사례처럼 2) 두 부녀 간에 성교를 하지 않으면 아버지를 죽이겠다고 강요하는 행위 중 어떤 것이 더 악(evil)의 정도가 큰가? 아마도 2)의 경우가 더 중대한 악을 수반한다고 가정하는 입장을 쉽게 무시할 수 없을 것이다. 이 때, 사랑하는 가족 간의 강요된 성관계 그 자체의 해(physical damage)는 0에 수렴할 수도 있고, 오히려 범죄자에 의한 직접적인 강간행위의 해의 정도가 훨씬 높을 수 있다. 자율성을 강조하는 입장에서는 이 사안에서 강간과의 해악의 비교대상은 부녀간의 성관계가 아닌 강요로 인한 자율적 선택의 좌절이며, 강요행위의 해악이 강간의 해악을 상회한다고 주장할 수도 있다. 그러나 하나의 의미체계를 공유하는 이 공동체의 경우에 근친성교의 강요가 주는 보다 본질적인 악(evil)은, 강요가 가져오는 결과적 해(physical damage)나 단순한 자율성의 침해 그 자체라기보다는 ``사랑하는 두 부녀간의 있어서는 안 될 성관계''의 비극적 의미가 실현된다면 유발될 수 있는, 인간적으로 좋은 삶을 사는 것이 객관적으로 중요하다는 믿음형성의 좌절, 공동체의 객관적 믿음체계로부터 소외라고 할 수 있다. 이러한 상황에 맞서 아버지가 딸과의 성관계(비극적 의미의 실현) 대신 죽음(해악)을 선택한다고 하더라도 이러한 선택은 불합리할 지언정 쉽게 비난하기 어려울 것이다. 이러한 중심부 믿음형성과 유지의 이익은 때로 복지-이익이나 자율성-이익을 뛰어넘는다.

알파고의 개발이나, 자신이 특정인으로부터 한두 차례 사기를 당했다는 사실, 최근들어 일이 유난히 잘 안 풀린다는 사실과 같이 주변부의 믿음에만 관여하는 사례들과 달리, 부녀간의 성행위 강요, 나치의 유태인 대량학살, 통제되지 않은 특정 공동체 안에서의 집단 강간의 만연화, 노예제의 인정, 끔찍한 전쟁상태의 지속, 국가에 의해 방조되는 고문의 당연시함, 이종간 교배를 통한 인간-비인간 잡종개체의 개발을 방관하는 것과 같은 사태는 한 사회의 믿음체계를 붕괴시킬 가능성, 즉 인간이 존엄하다는 공동체 구성원의 중심부 믿음 자체를 변경시킬 가능성을 제공한다.

이와 같이 좋은 삶의 구성요소와 그 객관적 중요성을 말하는P\emph{gl}+의 중심적 진술인 P\emph{dig}를 재고하지 않을 수 없게 만드는 중대한 침해의 경험에 부딪히게 될 때, 다른 어떤 법익보다도 우선해서 고려되어야 할 인간존엄-이익, 즉 좋은 삶에 관한 중심부 믿음(B\emph{pdig})을 형성하고 유지할 이익의 존재를 확인하게 된다. 그리고 이러한 이익은 댄-코헨이 제시하였고, 현대 법이론이 추구하고 있는 것으로 보이는 ``인간존엄 원리''가 충족해야 할 세 가지 조건, 즉 기존의 해악 원리나 자율성 원리로부터의 독립성(independence), 이 원리가 보호하는 가치 혹은 이익이 다른 원리가 보호하는 가치들보다 우선해야 한다는 우선성(priority), 그리고 이렇게 보호되는 가치는 어떤 명백한 혹은 관행적이거나 사회적인 의미에 의존된다는 의미의존성(meaning dependence) 조건을 잘 만족시킨다.\footnote{Meir Dan-Cohen, ``Defending Dignity'', \emph{Harmful thoughts: Essays on law, self, and morality}, (Princeton, 2002), 157-161. 댄-코헨은 이 저작을 통해 이러한 세 가지 조건들을 만족하는 새로운 개념을 구체적으로 발견하지는 않았다. 그가 염두에 두고 있는 인간존엄 원리라고 불리는 것들이 가져야 할 어렴풋한 세 가지 특성을 서술하고 있을 뿐이다.}

\paragraph{좋은 삶을 구성하는 이익들의 체계}

앞서 무엇이 좋은 삶인지를 정의하는 보편적으로 정당화된 명제P\emph{gl}+(added Good Life Proposition)를 대략적으로 가정해 보았다. 먼저 기존에 복지이익들이라고 평가되어 온 것과 자율성의 법익을 종합한 P\emph{gl}은 ``인간으로서의 삶의 좋음이란 자신이 속한 공동체와 함께 어울려 개인으로서 풍부한 복지기반 안에서 자아실현을 하며 평등한 대우를 받고 평화로움을 만끽하며 수명을 다하여 살아가는 것이다.''와 같은 방식으로 정의될 수 있는데, 이는 원래 형성되었어야 할 정당화된 P\emph{gl}에 기반한 삶의 객관적 중요성에 대한 믿음이라는 제3의 법익을 빠뜨린 것이다. 이러한 `공동체 구성원의 믿음', 인간적으로 좋은 삶의 객관적 중요성에 대한 ``믿음'' 을 추가한 P\emph{gl}+는 ``인간적으로 좋은 삶을 산다는 것은 자신이 속한 공동체와 함께 어울려 개인으로서 풍부한 복지기반 안에서 자아실현을 하며 평등한 대우를 받고 평화로움을 만끽하며 수명을 다하여 살아가는 것이며, 또한 동시에 이러한 인간적으로 좋은 삶을 사는 것이 객관적으로 중요하다는 믿음을 유지하며 살아가는 것이다.''라고 정의될 수 있다. 또한 P\emph{gl}+가 함축하는 삶을 사는 것은 인간적으로 좋은 삶을 사는 것을 보장하는데, 이 명제 자체는 그러한 삶을 사는 것이 가치있는지, 이를 보호하는 것이 규범적으로 정당화되는지에 대해서 말하는 명제는 아니어서, 이러한 정당화는 ``인간이 존엄하다''라는 항상 참으로 가정되는 명제 즉, ``P\emph{gl}+가 의미하는 좋은 삶을 영위하는 것은 객관적으로 중요하다''라는 인간존엄명제 P\emph{dig}(Dignity Proposition), ``P\emph{gl}+가 의미하는 좋은 삶을 영위하는 것은 객관적으로 중요하다.'' 로부터 가능해진다는 점을 이미 설명하였다.

인간이 존엄하다는 명제P\emph{dig}와 그 선언은 정당화된 P\emph{gl}+가 의미하는 인간적으로 좋은 삶을 사는 것이 객관적으로 중요하다는 것을 의미한다고 말하며, 이를 통해 이러한 객관적 중요성을 가진 좋은 삶을 사는 것을 방해하는 것은 그르다는 규범적 함축이 따라나온다. 이를 표로 정리해보면 다음과 같다.

\renewcommand{\arraystretch}{1.5} % 표 내부 줄간격

\begin{table}[H]
\centering
\caption*{표 1: 좋은 삶을 구성하는 이익들의 체계}

\begin{tabular}{|p{4.2cm}|p{4.2cm}|p{6.6cm}|}
\hline
\multicolumn{3}{|c|}{\textbf{P\textsubscript{dig}}: ``P\textsubscript{gl}+에 해당하는 삶을 사는 것은 객관적으로 중요하다.''} \\
\hline
\multicolumn{3}{|c|}{\textbf{P\textsubscript{gl}+}: ``인간적으로 좋은 삶은 다음과 같은 요소들로 구성된다.''} \\
\hline
\multicolumn{2}{|c|}{\textbf{사실적 기반 (from P\textsubscript{gl})}} & \textbf{의미 기반} \\
\hline
\textbf{복지} & \textbf{자율성} 

\vspace{0.5em}

스스로 자신의 삶의 행위에 대해 통제를 행사 & 
\textbf{P\textsubscript{gl}}에 기반한 삶을 사는 것의 객관적 중요성 
(\textbf{P\textsubscript{dig}})에 대한 믿음의 형성

\vspace{0.5em}

→ 이 믿음은 ``정당화된 좋은 삶의 의미내용에 맞게 살아가는 것이 단지 나에게만이 아니라, 객관적으로 중요하다''는 믿음임 \\
\hline
예: 육체적, 감정적, 정신적 건강, 소유와 부, 평판, 공동체 서비스와 기관들, 사회적 지원, 안전과 안전한 육체적 환경
& 예: 언제 어디서 어떻게 행위할 지 선택하기 
& 예: 육체적 정신적 건강을 누리고, 자신이 선택한 삶을 살 수 있는 사실적 기반을 가지면서 사는 것이 객관적으로 중요하다는 믿음을 형성하고 유지하기 \\
\hline
\end{tabular}
\end{table}

이 표는 사실적 기반만큼이나 의미 기반의 이익이 인간으로서의 좋은 삶을 영위하는데 독립적이고 필수적인 요건이라는 것을 잘 보여주고 있다.

이 지점에서 앞서 좋은 삶을 구성하는 제3의 이익을 침해한 것으로 추정했던 사례들을 다시 한 번 살펴볼 필요가 있다. 우리는 보통 ``인간적인 삶의 좋음에 비추어 형사절차상의 형벌의 목적으로 태형, 거세형과 같은 신체형을 수인하지 않는 것이 객관적으로 중요하다''는 믿음을 가지고 있다고 하자. 그런데 \emph{State v. Braxton} 사건에서 성범죄자가 30년의 유기징역과 외과적으로 거세하는 것을 선택하라는 하급법원의 선고\footnote{\emph{State v. Braxton}, 326 S.E. 2d 410 (S.C. 1985).}는 선택지를 확장하여 자율성을 증대함과 동시에 위 명제의 객관적 중요성에 대한 우리의 믿음의 견고함을 어느 정도 감소시킨다고 볼 수 있다. 또한 우리는 ``인간적인 좋은 삶의 이상에 비추어 볼 때, 부인을 때리지 않는 것이 객관적으로 중요하다''는 강한 믿음을 가지고 있다고 볼 때, 부인의 동의하에 부인을 때린 \emph{State v. Brown} 사건에서,\footnote{\emph{State v. Brown}, 364 A.2d 27 (N.J. Super. Ct. Law Div. 1976), aff'd, 381 A.2d 1231 (N.J. Super. Ct. App. Div. 1977).} ``부인 때리기''의 금지를 통해 우리는 부인의 자율성은 감소됨과 동시에 이러한 믿음을 강화한다. 반면 우리는 ``인간적인 좋은 삶의 이상에 비추어 스포츠를 자유롭게 즐기는 것은 객관적으로 중요하다''는 믿음을 가지고 있고, 이 때 복싱경기를 금지하지 않음으로써 우리의 신뢰는 유지, 강화된다. 노예화의 경우, ``노예로서 사는 삶은 인간으로서 좋은 삶이 아니다''라는 명제의 객관적 중요성에 대한 우리의 확신은 매우 강고한 편이다. 그러나 성전환자의 자발적 의료적 거세나 복싱경기에서 보듯, 어떤 의미부여 이전에 사실로서의 거세나 물리력의 행사가 그 자체로 부정적인 것은 아니다. 따라서 그러한 행위나 결과가 인간적으로 좋은 삶에 있어 어떤 의미를 지니느냐, 어떤 의미로서 이해되느냐가 좋은 삶의 형성에 더 근본적인 위치에 있는 것으로 보인다.

물론 이러한 명제P\emph{gl}+의 구체화된 내용의 객관성에 대해서는 다양한 반론이 가능하다. 보조적인 차원에서의 형벌 목적의 신체형이 반드시 좋은 삶에 위배되지 않으며, 알코올중독 치료 목적으로 동의하에 부인을 때리는 행위 역시 충분히 용인 가능하며, 행복한 선택한 노예 역시 충분히 인간으로서 좋은 삶을 누린다고 말하는 것은 역시 좋은 삶의 정의가 어떠한 정당화에 기반해 있느냐에 따라 가능한 주장들이다. 여기서 강조하고자 하는 점은 이러한 사례들이 중대하게 거론되는 이유가 어떤 복지이익이나 자율성의 후퇴와 무관하다고 볼 여지가 존재하며, 사실적 차원에서 동일한 행위인 성전환자의 외과적 거세나 복싱경기와 다른 평가를 내린 점에서 보듯, 보호의 대상으로 거론된 것은 어떤 사실적 기반이 아니라 좋은 삶을 규정하는 명제가 표상하는 의미의 객관적 중요성에 대한 믿음의 형성 및 유지와 관계되어 있다는 사실이다.

\paragraph{믿음형성 이익 침해의 양태}

이러한 믿음 형성의 이익은 그 사실적 기반에 관련된 이익 못지않게 중요해서 때로 그 침해에 대한 보호가 필요하다. 예를 들어보자. 어떤 7살짜리 아이 A가 나름의 진지한 고민을 통해, ``나의 인간적인 삶의 좋음은 어른이 되어 좋은 외과의사가 되어 의술을 베푸는 것으로만 달성된다.''라는 명제P는 객관적으로 중요하다는 믿음을 확립했다고 하자. 이러한 P의 중요성에 대한 믿음은 A가 20대가 되어서까지 계속되었고, 의대에 진학하여 열심히 의사면허를 취득하기 위해 노력하고 있다고 하자. 이 때 그의 믿음형성과 유지를 방해하는 세 가지 사례를 생각해보자. 1) 이 사례에서 A의 아버지인 C가 가정형편 때문에 더 이상 A의 학비를 지원할 수 없으며, 학업을 중단하고 취업을 해야 한다고 진지하게 설득하는 경우, A가 이에 응해 학업을 포기하고 그의 믿음에 반하는 삶을 살아가거나 기존의 믿음을 변경시키는 경우를 생각해 보자. 이러한 믿음의 상실은 나아가 삶의 전반적인 목표를 상실시킬 수도 있다. 2) 이 사례에서 A에게 악의를 품은 B가 A의 좋은 삶의 의미를 훼손시킬 목적과 고의를 가지고 수술에 필수적인 A의 손을 일상생활에는 전혀 문제가 없고 수술집도가 불가능할 정도만 망가뜨렸다고 하자. 이러한 행위는 그 믿음을 유지하기 위해 필요한 사실적 기반들을 침해한다. 이러한 사실적 기반이 심각하게 훼손되면, A는 자신의 P의 중요성에 대한 확신을 더 이상 유지하기 어려울 것이다. 3) 불의의 사고로 손을 다쳐 다른 생활에는 큰 지장이 없으나, 외과의사가 되는 꿈을 유지하는 것은 불가능하게 된 경우, 심지어 다른 직업을 가지게 되어 더 풍족한 삶을 살게 되었다 하더라도 그의 P에 대한 믿음이 강고한 이상 그의 삶 전체의 만족도는 중대한 수준으로 하락할 수 있다.

마지막 3)의 사례는 타인에 의한 침해의 사례는 아니다. 따라서 제3의 믿음형성 이익의 침해형태를 두 가지로 구분할 수 있을 것이다. 하나는 명제의 전제들에 대한 논리적 또는 심리적 공격을 통해 전체 논증을 약화시키는 것이고, 다른 하나는 좋은 삶의 사실적 기반에 대한 심각한 침해다.

침해형태1 : ``객관적 중요성 평가에 사용된 전제들을 (사실적 혹은 논리적으로) 공격하여 전체 논증을 약화시키는 행위''

침해형태2 : ``객관적으로 중요한 것으로 평가된 좋은 삶의 사실적 기반을 심각하게 박탈하는 행위''

이를 위의 사실적 기반의 이익들의 침해형태와 비교하여 \textless 표1\textgreater 을 활용해 다음과 같이 정리할 수 있을 것이다.

\textless 표2: 좋은 삶을 구성하는 이익들의 침해 양태\textgreater{}

\begin{longtable}[]{@{}
  >{\centering\arraybackslash}p{(\linewidth - 4\tabcolsep) * \real{0.2516}}
  >{\centering\arraybackslash}p{(\linewidth - 4\tabcolsep) * \real{0.2521}}
  >{\centering\arraybackslash}p{(\linewidth - 4\tabcolsep) * \real{0.4963}}@{}}
\toprule\noalign{}
\multicolumn{3}{@{}>{\centering\arraybackslash}p{(\linewidth - 4\tabcolsep) * \real{1.0000} + 4\tabcolsep}@{}}{%
\begin{minipage}[b]{\linewidth}\centering
P\emph{gl}+: ``인간적으로 좋은 삶은 {[}아래의 요소들{]}로 구성된다.''
\end{minipage}} \\
\midrule\noalign{}
\endhead
\bottomrule\noalign{}
\endlastfoot
\multicolumn{2}{@{}>{\centering\arraybackslash}p{(\linewidth - 4\tabcolsep) * \real{0.5037} + 2\tabcolsep}}{%
사실적 기반(from P\emph{gl})} & 의미 기반 \\
복지 & 자율성 & 인간적으로 좋은 삶의 사실적 기반을 규정하는 명제(P\emph{gl})에 기반한 삶을 사는 것의 객관적 중요성(P\emph{dig})에 대한 믿음의 형성 \\
\multirow{2}{=}{침해형태:

복지이익의 직접적 후퇴} & \multirow{2}{=}{침해형태:

자율적 선택행위를 좌절시킴} & 침해형태1: 객관적 중요성 평가에 사용된 전제들을 (사실적 혹은 논리적으로) 공격하여 전체 논증을 약화시키는 행위 {[}순수한 믿음침해{]} \\
& & 침해형태2: 객관적으로 중요한 것으로 평가된 좋은 삶의 의미의 전제가 되는 사실적 기반을 심각하게 박탈하는 행위 {[}사실기반 침해가 동반된 믿음침해{]} \\
\end{longtable}

\renewcommand{\arraystretch}{1.5} % 표 내부 줄간격

\begin{table}[H]
\centering
\caption*{표 2: 좋은 삶을 구성하는 이익들의 침해 양태}

\begin{tabular}{|p{4.2cm}|p{4.2cm}|p{6.6cm}|}
\hline
\multicolumn{3}{|c|}{\textbf{P\textsubscript{gl}+}: ``인간적으로 좋은 삶은 다음과 같은 요소들로 구성된다.''} \\
\hline
\multicolumn{2}{|c|}{\textbf{사실적 기반 (from P\textsubscript{gl})}} & \textbf{의미 기반} \\
\hline
\textbf{복지} & \textbf{자율성} 

자신의 삶의 행위에 대해 통제할 수 있는 능력의 상실
& 
\textbf{P\textsubscript{gl}}에 기반한 삶을 사는 것의 객관적 중요성 
(\textbf{P\textsubscript{dig}})에 대한 믿음을 형성할 수 없거나 약화시키는 상태.

→ ``나의 삶이 객관적으로 중요하다''는 믿음 자체가 구성되지 않거나 붕괴됨 \\
\hline
침해형태: 

\vspace{0.5em}

복지이익의 직접적 후퇴
& 침해형태: 

\vspace{0.5em}

자율적 선택행위의 좌절
& 침해형태 1: 객관적 중요성 평가에 사용된 전제들을 (사실적 또는 논리적으로) 공격하여 믿음 자체를 약화시키는 행위. 즉, 순수한 믿음 침해.

\vspace{0.5em}

침해형태 2: 좋은 삶의 의미의 전제가 되는 사실적 기반을 실질적으로 박탈함으로써 믿음을 동시에 침해하는 행위. 즉, 사실 기반 침해가 수반된 믿음 침해. \\
\hline
\end{tabular}
\end{table}

믿음의 형성과 유지를 무너뜨리는 두 가지 방식은 크게 두 가지라고 볼 수 있다. 아버지의 진지한 설득에 근거한 1)의 사례와 같이 믿음에 대한 공격은 기본적으로 반증에 의한 것이라고 볼 수 있다. 그러나 믿음은 억측을 포함하는 것으로 반드시 합리적 논변에 의해 형성되는 것만은 아니다. 따라서 믿음의 상실도 반드시 논리적 공격에 의해서만 침해되지 않는다. 첫째로 명제의 전제들에 대한 자신이나 타인의 논리적 공격의 사례를 보자. 예를 들어, P\emph{gl}이 ``인간의 삶의 좋음은 특별한 학습능력과 이를 통한 전략적 판단능력이 배타적으로 인간만이 보유한다는 전제하에서만 성취가능하다''라는 명제를 함축하고 있고, 어떤 사람이 이러한 명제에 대한 확신을 가지고 있다고 한다면, 이 사람의 확신을 무너뜨리는 방법 중 하나는 이러한 명제의 전제인 학습능력과 판단능력이 인간 이외의 존재에도 있음을 증명하는 것이다. 최근의 바둑 경기에서 기계인 알파고가 인간 이세돌을 이겼다는 사실이 이러한 학습능력과 전략적 판단이 기계를 통해서도 가능하다는 것을 함축한다면, 학습이 가능한 기계가 가능함을 증명하는 이론은 인간의 존엄을 침해하는 것처럼 보일 수 있다. 이는 지동설이 (설사 참이라고 하더라도) 신의 존엄을 침해한다고 생각하는 구조와 유사하다.

둘째로, 고의의 상해행위에 의한 2)의 사례와 같이, 믿음은 좋은 삶의 의미내용을 규정하는 사실적 기반을 박탈함으로써 침해할 수도 있다. 강간을 수행하는 행위는 피해자가 생각한 좋은 삶의 사실적 기반을 침해함으로써 이를 규정하는 명제의 전제 부분을 상실시키고, 결과적으로 그 명제에 대한 믿음의 유지를 불가능하게 할 수 있다.

\paragraph{중심부-믿음형성 이익의 침해와 그 형량의 방법}

그러나 모든 믿음형성이 보호의 이익, 즉 법익을 가지는 것은 아니다. 먼저 잘못된 믿음의 형성과 유지는 보호이익을 가지지 않는다. 코페르니쿠스가 당대의 주변 사람들에게 지구가 태양 주위를 돌고 있다는 지동설을 과학적 근거를 들어가며 진지하게 설득하고 있다고 하자. 코페르니쿠스는 주변 사람들의 중대한 믿음-이익을 침해하는 것이고 이러한 행위를 중단해야 하는가? 과학적 진리를 탐구하는 행위는 설사 기존의 믿음을 침해한다고 하더라도 오히려 정당화된 믿음 형성에 기여하기 때문에 규제의 대상이 되기 어렵다.

또한 본고는 중심부 믿음과 주변부 믿음을 구분함으로써, 좋은 삶과 관계된 중심부 믿음 침해여부가 인간존엄-이익의 침해와 관련지을 수 있는 최소기준(threshold) 역할을 한다고 주장하였다. 주변부 믿음의 침해는 좋은 삶의 객관적 중요성에 대한 본질적 믿음의 형성을 침해하지 않는다. 앞서 외과의사가 되고자 했던 A의 아버지인 C가 학업을 중단하고 취업을 해야 한다고 진지하게 설득한 사례의 경우, A가 이에 응해 학업을 포기한다 하더라도 그는 또다른 좋은 삶에 대한 계획을 가질 수 있고, 이는 인간적으로 좋은 삶을 사는 것이 객관적으로 중요하다는 믿음의 형성을 방해하지 않는다.

나아가 설사 중심부 믿음과 관계된 침해가 있다고 하더라고, 정당화된 중심부 믿음은 쉽게 흔들리지 않는다. 국가와 공동체가 대개의 중심부 믿음과 관계된 침해행위를 범죄로 규정하거나 구제나 회복의 제도를 준비해두고 있기 때문이다. A라는 사람이 상해나 강간의 피해자가 된다 하더라도, 이러한 행위는 매우 부당한 것으로서 국가 공권력에 의해 범죄로 규정되며, 공동체 구성원들은 이로 인한 내 삶의 좋음의 객관적 중요성 그 자체가 상실되었다고 생각하기 보다는 A가 끔찍한 범죄의 피해자가 되었다고 생각하기 때문이다. 따라서 일반적 범죄행위에 의해 자신의 좋은 삶에 대한 믿음의 형성과 유지가 어느 정도 방해받는다 하더라도 먼저 국가의 제도에 의해 그 피해를 구제받게 된다.

그러나 국가나 공동체가 이에 대해 무관심하거나, 오히려 피해자에게 부당한 낙인을 찍거나, 구제에 미온적이라면, 그 때부터 중심부 믿음은 흔들릴 수 있다. 나치정부가 직접 홀로코스트에 관여하는 행위는 대표적으로 이러한 중심부 믿음의 형성과 유지를 방해하는 행위라고 할 수 있다. 생명공학기술의 발달과 관련하여 정부가 사람과 동물 사이의 이종간 교배 행위를 전혀 규제하지 않고 방치한다면, 사람들의 인간존엄에 대한 믿음은 흔들릴 수 있을 것이다. 이러한 침해는 다른 사실적 침해에 우선하여 보호될 필요성이 있다.

따라서, 이러한 보호를 수행하는 구체적인 법적 실천은 다음과 같이 이루어져야 할 것이다.

1) 먼저 침해행위가 정당화된 믿음을 침해하는지, 잘못된 믿음을 침해하는지 구별해야 한다. 문제가 되는 침해행위는 정당화된 믿음을 침해하는 것이어야 한다.

2) 둘째, 침해행위의 대상이 인간적으로 좋은 삶과 관계된 중심부 믿음과 관계된 것인지 주변부 믿음과 관계된 것인지 구별해야 한다. 문제가 되는 침해행위는 중심부 믿음과 관계된 것이어야 한다.

3) 셋째, 침해행위가 중심부 믿음을 위협하는 수준에 이르렀는지 확인해야 한다. 문제가 되는 침해행위가 이미 다른 제도에 의해 충분히 보호받고 있거나, 기존의 강고한 믿음을 붕괴하는 수준에 이르지 않았다면 이는 중심부 믿음을 위협하고 있다고 보기 어렵다.

이러한 세 단계를 거친다 하더라도 중심부 믿음을 위협하고 있는 행위가 침해하는 중심부 믿음형성과 유지의 이익을 다른 사실적 이익과 비교형량해야 하는 것인지에 대한 문제가 여전히 남아있는 것처럼 보일 수도 있다. 그러나 이론적으로는 어떤 행위가 해당 공동체의 정당화된 중심부 믿음의 형성과 유지를 위협할 정도에 이르렀다고 판단되면 그 행위는 공동체 존속 목적을 무의미하게 만들기 때문에 다른 어떤 사실적 이익에도 불구하고 금지되어야 한다. 다만 법적 실천에 있어서는 중심부 믿음의 침해여부의 판단이 이미 일종의 비교형량의 단계를 충분히 수행하고 있다고 볼 수 있다. 중심부 믿음의 침해가 확정되기 위해서는 다른 이익들과의 비교형량에도 불구하고, 공동체 구성원이 그들의 믿음을 희생하기 위해 그러한 침해를 더 이상 감수할 의사가 없음을 전제로 하기 때문이다.

결론적으로, 인간존엄-이익 보호의 사법적 실천에 있어서 인간적으로 좋은 삶의 객관적 중요성에 관계된 믿음형성의 이익이 다른 이익과 형량의 문제에 부딪히는 경우 그 믿음형성의 이익과 사실적 이익 가운데에서 사법기관은 적절한 비교형량의 단계를 거치게 될 것이다. 이 때의 이익형량 과정은 바로 해당 침해가 정당화된 중심부 믿음을 심각하게 위협할 수준에 이르렀는지를 판단하는 위에서 제시한 세 단계를 말한다.

\subsection{이원적 이해방식의 실천적 함의}

\subsubsection{\texorpdfstring{1) 평등한 존중의 정당화로서의 인간존엄의 법질서에서의 수용: }{1) 평등한 존중의 정당화로서의 인간존엄의 법질서에서의 수용: }}

\paragraph{\texorpdfstring{ (1) 누락된 권리를 보충하거나 권리를 해석하는 근거}{ (1) 누락된 권리를 보충하거나 권리를 해석하는 근거}}

\subparagraph{구치소 내 과밀수용행위 위헌확인 사안의 재구성}

이원적 이해방식 중 첫번째 이해방식인 평등한 존중의 정당화로서의 인간존엄이 사법적 판단에서 수행하는 역할은, 기존의 사법질서가 갖추어 왔던 모든 이들의 평등한 존중을 위한 제도와 좋은 삶을 살도록 하기 위해 보장된 권리들이 해왔던 역할과 크게 다르지 않다. 다만 인간적으로 좋은 삶에 대한 공동체의 이해가 깊어지고 기존 질서가 충분히 예상하지 못한 상황에 처했을 때, 이러한 좋은 삶을 충분히 확보하지 못하는 상황에 대한 치유책으로서, 누락된 권리들을 보충하는 근거로서 인간존엄이 사용될 수 있다.

예를 들어 구치소 내의 과밀수용행위에 대하여, 국가가 충분히 여유있는 구금시설을 마련할 수 있음에도 불구하고 피수용자에 대하여 생존의 기본조건이 박탈된 시설에 구금하는 경우, 우리는 구체적인 권리가 법률에 의해 마련되어 있지 않더라도, 기본적 생존여건을 갖춘 시설을 요구할 권리를 인간존엄의 첫번째 이해방식으로부터 도출해낼 수 있다. \footnote{헌재 2016. 12. 29. 2013헌마142, 판례집 28-2하, 652(구치소 내 과밀수용행위 위헌확인) 참고.} 다만 이러한 도출은 다른 개별적 법적 구제수단을 모두 거친 이후에 이루어지는 것이 바람직할 것이다.

\paragraph{형량불가능성의 적용의 자제}

그러나 이러한 권리가 형량불가능하거나 절대적인 것은 아니다. 헌법재판소는 2016. 12. 29. 2013헌마142 사건에서 과밀수용행위가 ``인간으로서의 존엄과 가치를 침해하여 헌법에 위반된다''고 설시하면서 비례원칙을 적용하거나 다른 가치들과 형량하려는 시도를 하지 않고 있다. 그러나 비록 과밀수용행위에 의해 침해되는 기본권이 인간존엄과 밀접한 연관을 가진다 하더라도 다른 기본권과 충돌하는 경우나 헌법 제37조 제2항의 기본권 제한 규정에 의한 요건에 해당하는 경우 그 제한이 가능하다고 보아야 한다. 이러한 권리는 `인간적으로 좋은 삶의 객관적 중요성'이라는 침해불가능한 인간존엄의 사실 그 자체가 아니기 때문이다. 따라서, ``교정시설의 1인당 수용면적이 수형자의 인간으로서의 기본 욕구에 따른 생활조차 어렵게 할 만큼 지나치게 협소하다'' 하더라도 ``이는 그 자체로 국가형벌권 행사의 한계를 넘어 수형자의 인간의 존엄과 가치를 침해하는 것''은 아니라고 보아야 하고, 여전히 ``수형자 수와 수용거실 현황 등 수용시설 전반의 운영 실태와 수형자들의 생활여건, 수용기간, 접견 및 운동 기타 편의제공 여부, 수용에 소요되는 비용, 국가 예산의 문제 등 제반 사정을 종합적으로 고려''하여 결정할 수 있다.\footnote{헌재 2016. 12. 29. 2013헌마142.} 예를 들어, 특정 지역이 테러와 폭동으로 인하여 구금자의 수가 급증한 경우, 비록 해당 구금시설이 현저히 비좁다 하더라도, 국가안전보장이나 사회질서를 지킬 목적으로 이러한 권리를 제한하는 것은 가능하다고 보아야 한다.

\paragraph{후견적 규제의 기초와 자유의 기초 사이의 갈등 해결}

\subparagraph{프랑스의 난쟁이던지기 사건}

앞서 살펴보았던 꽁세유데따(Conseil d\textquotesingle État)의 프랑스의 난쟁이던지기 금지 판결은 인간존엄을 높은 수준의 인간성이라고 이해하면서 후견적 규제의 기초로 이해하는 입장과 자유의 기초로 이해하는 입장 사이의 첨예한 대립을 불러일으켰다. 높은 수준의 인간성을 향한 개인의 책임을 강조하는 소위 `존엄주의적(dignitarian)' 해석은 인간존엄의 보호가 유효하게 시민적 자유의 사용에 대한 행정적 제약의 우선적인 근거가 될 수 있다고 보았다.\footnote{Conseil d\textquotesingle État, 27 octobre 1995, Commune de Morsang-sur-Orge.}

그러나 이러한 입장은 높은 수준의 인간성이 무엇인지에 대한 논란과, 국가가 개인의 자유영역에 지나치게 간섭할 수 있는 빌미를 제공하는 계기가 된다는 비판 아래, 사법질서에서의 존엄 개념 남용의 우려를 불러일으켰다. 이는 2008년 니콜라스 사르코지 대통령이 1946년 프랑스헌법전문을 보완하고자 했던 노력의 일환으로 전문가들의 의견을 구했던 사안에서도 나타났는데, 사르코지는 인간존엄의 원리가 명시적으로 포함되기를 원했지만,\footnote{Marianne Gomez, ``La constitution pourrait s'ouvrir à la bioéthique'', \emph{La Croix}, (9 January 2008).} 전문가위원회는 두 가지 설명을 통해 그의 의견을 따르지 않았다.\footnote{Simone Veil, \emph{Redécouvrir le Préambule de la Constitution: Rapport au President de la République}.} 그 첫 번째 설명은 인간존엄 원리가 가진 강한 애매성, 심지어는 모순적이기까지 한 이중성 문제였다. 전문가위원회는 인간존엄 원리가 한편으로는 타인에 의한 지배나 침해로부터의 보호를 돕는 원리이지만, 다른 한편으로는 개인의 자유를 제약하는 원리이기도 하다는 점을 지적했다.\footnote{Ibid., 94면.} 두 번째는 평등주의적 이해와 존엄주의적 이해의 명확한 구분의 필요성을 지적했다. 전문가위원회의 의견은 `존엄'을 계속 사용해야 한다면, 그것은 `모든 이의 평등한 존엄'의 원리과 `인간존엄 원리' 사이의 엄격한 구별에 기반하여 헌법 문언은 오직 `모든 이의 평등한 존엄'을 포함하는 것만을 추천할 수 있다는 것이었다. 이러한 태도는 2010년 ``공공장소에서 니캅(niqab)착용금지입법''에 대한 꽁세유데따의 의견에도 영향을 주었는데, 여기서 꽁세유데따는 인간존엄원리의 이중성으로 인한 난점을 인정하고 어떤 금지의 기초로 삼는 것을 유보하는 태도를 취했다.\footnote{Conseil d'Etat. 2010. ``Etude sur les possibilités juridiques d'interdiction du port du voile integral'\,'.} 여기서 꽁세유데따는 인간존엄의 원리의 양가성을 지적하면서, 인간존엄의 원리는 ``결정의 자유를 희생하여 인간존엄을 보호하는 집단도덕적 요청의 원리, 그리고 인격과 동체적 측면인 자기결정의 보호의 원리''라는 두 모순적인 이해 위에 존재한다는 점을 확인했다.\footnote{Ibid., 21-2면.} 그런데 ``많은 경우 니캅의 착용은 자발적이기 때문에''\footnote{Ibid., 22면.} 이런 금지의 기초로서 부적절하고 사실상 위험한 기반이라고 자문했다.

이원적 이해방식은 이러한 위험을 회피하면서도, 중대한 사안에서 인간존엄 항변에 의한 보호를 주저하지 않는다. 난쟁이 던지기 사안의 경우, 단지 난쟁이의 품위와 같은 높은 수준의 인간성으로서의 인간존엄 개념을 적용하여 이를 어떤 비교형량의 고려없이 절대적으로 금지해서는 안 된다. 평등한 존중의 정당화로서의 인간존엄은 품위를 절대적으로 보호하는 규범이 아니라, 난쟁이를 비롯한 모든 이들의 좋은 삶을 평등하게 존중하라는 규범이기 때문이다.

그러나 이원적 이해방식의 두번째 `독립적 인간존엄-이익'은 여기서 판단을 멈추지 않는다. 난쟁이가 선택하는 직업을 국가가 보호하거나 방치하는 것과 같은 중요한 가치의 침해에 대한 국가의 방조행위가 우리 공동체가 정당하게 형성한 인간적으로 좋은 삶을 사는 것이 객관적으로 중요하다는 믿음의 중심부를 침해하는 수준에 이르면, 이는 별도의 판단을 거쳐 인간존엄의 침해로 인정받을 수 있으며, 이는 금지된다. 예를 들어, 그 직업이 성매매, 혹은 가학적 성행위를 방조하는 성매매, 핍쇼, 혹은 동물을 매우 잔인하게 학대하는 것을 공연히 전시하는 일이라면, 그 사안의 경중에 따라 각각 해당 공동체는 중심적 믿음 침해의 여부를 달리 판단할 수 있을 것이다. 그리고 이러한 믿음-침해의 경우 이를 방지하는 보호수단과 정도도 다양하게 나타날 수 있다. 이에 대해서는 다음에서 설명하고자 한다.

\subsubsection{독립적 인간존엄-이익의 적용}

\paragraph{의미의존적 사안의 포섭}

이원적 이해방식 중 두 번째 이해방식인 독립적 인간존엄-이익이 수행하는 역할은, 공동체의 인간존엄에 대한 믿음을 중대하게 침해할 수 있는 상황에 국가가 적극적으로 개입할 의무를 부과한다. 나치독일의 유태인 학살과 같은 잔인성 이상의 무언가를 드러낸 사건들이 발생하거나, 생명공학기술에 의해 유전자 가위 등으로 인류의 특성을 조작하거나, 키메라와 같은 새로운 생명체의 탄생을 조작하는 것이 매우 쉽게 이루어지는 경우, 인간이 존엄하다는 우리의 믿음은 위험에 처할 수 있다. 포태 후 9개월의 정상적이고 건강한 태아를 재미삼아 죽이는 경우에도 낙태를 처벌하지 않거나, 회복가능성이 높은 환자의 연명치료를 쉽고 간단하게 중단할 수 있도록 하는 경우에도 우리의 믿음은 위험에 처할 수 있다. 강간을 경미한 폭행이나 강요의 경우와 유사한 수준에서 처벌하는 것도 마찬가지다.

\paragraph{중대한 침해 여부의 확인}

다만 어떤 사안은 매우 논쟁적일 수 있다. 난쟁이던지기 놀이나 핍쇼, 혹은 성매매를 금지하지 않는 것이 우리의 인간존엄에 대한 믿음을 중대하게 침해하는지는 공동체의 믿음체계를 면밀히 분석함으로써 이루어져야 할 것이다.

\paragraph{의미를 회복시키는 수단의 고려}

대개의 믿음-법익도 그 믿음을 붕괴시키는 구체적 행위를 금지하거나 특별한 조력을 지원함으로써 그 법익을 보호할 수도 있다. 예를 들어 잔인한 고문이나, 학살, 비하적 행위에 대해서는 그러한 행위를 시효를 두지않고 금지하고 처벌하며 이에 대한 강력한 손해배상을 부과함으로써 인간존엄에 대한 믿음을 유지하도록 국가가 조력할 수 있다.

그런데 믿음-법익인 파생적-인간존엄의 침해는 인간적으로 좋은 삶에 대한 의미와 관련되어 있다. 따라서 일정부분 국가가 이에 대한 의미만을 회복시키는 수단을 통해서도 그 믿음체계의 붕괴를 막을 수 있다. 독일의 낙태판결에서, 일부의 낙태를 여전히 위법한 것으로 간주하면서도 처벌하지 않는 태도가 대표적이다. 이런 태도는 비록 어떤 반가치적인 상황에 대하여 물리적이고 신체적인 강제를 부과하지 않지만, 여전히 그러한 상황이 위법하다는 의미를 국가가 전달하고자 하는 노력을 유지한다. 인간존엄의 믿음체계를 유지하는 데 있어 이러한 상대적으로 경한 수단만으로 충분하고 적합하다면, 국가가 이를 넘어서서 과한 처벌과 같은 월권적 간섭을 수행하는 것은 자유의 침해로서 비례적이지 못한 것으로 간주되어야 할 것이다. 다음에서 이러한 사례로 파악될 수 있는 독일의 제2차 낙태판결과 한국의 연명치료중단 관련 판례를 살펴보고자 한다.

\paragraph{사안에의 적용}

\subparagraph{독일의 제2차 낙태판결과 상담모델}

낙태(인공임신중절)의 허용여부와 범위는 두 기본권 주체---태아와 임부---간의 기본권 충돌을 예정하고 있기 때문에, 각국에서 인간존엄의 개념을 둘러싸고 논쟁이 벌어지는 대표적인 사안이다. 여기서는 통일 후 독일의 제2차 낙태판결\footnote{BVerfGE 88, 203.}이 가지는 (각각 논란을 가지는) 두 가지 중대한 태도---불법이지만 처벌하지 않는 태도로의 보호의무범위의 변경, 엄격한 상담의무 규정의 구체적 도출---가 본 논문에서 주목하려는 인간존엄 개념의 실정법적 개입의 의미를 잘 보여주는 것이라 생각하여 독일의 사례를 소개하려고 한다.

먼저 이 판결이 있기 이전에 있었던 제1차 낙태판결의 경과를 살펴보자. 통일전 서독의 1974년 제5차 개정형법 218조 등은 낙태에 관하여 다음과 같은 규정을 두었다: 원칙상 수태 이후 13일 이상의 태아의 낙태는 처벌된다. 하지만 임산부 스스로의 결정으로 의사에 의해 행해진 수태 후 12주 까지의 낙태는 의사와의 상담이 있다면 형사처벌되지 않는다. 12주를 경과한 경우 의학적, 우생학적 사유가 있는 경우의 낙태는 처벌되지 않으며, 우생학적 사유는 22주까지만 허용된다. 보통 이러한 모델을 `기한모델'(Fristenlösung)이라고 한다.

이에 대하여 연방 의회 의원 193명이 임신 12주 내에서 의사의 상담만으로 처벌하지 않는 규정은 위헌이라는 취지로 소를 제기했고, 독일연방헌법재판소는 1975년 이 법을 위헌이라고 판단했다{[}제 1차 낙태판결{]}.\footnote{BVerfGE 39, 1.} 태아의 생명은 임신기간 내내 보호되어야 하며 이는 생명이 인간존엄의 중대한 기초임을 근거로 임부의 자기결정권보다 우선한다는 것을 근거로 한다는 점과, 법적 규제는 반드시 형사처벌이어야 할 필요는 없으나 실질적으로 태아의 생명보호에 적합한 방법으로 이루어져야 하고 이 사안의 경우 형사처벌이 불가피하다는 것을 설시하였다. 다만 임부의 생명이나 건강이 위협받는 경우에는 형법의 최후수단성과 기대불가능성이 고려되어야 하며, 낙태를 처벌하지 않을 수 있다고 보았다. 기대불가능성의 기준으로는 임부의 생명과 건강의 위협, 우생학적 사유, 윤리적 사유, 위급상황 등을 들었다. 이 결정을 통해 헌법재판소는 제5차 개정형법의 기한모델이 위헌임을 확인했고, 1976년의 수정입법을 통해, 독일은 원칙적으로 낙태를 금지하고 예외적인 사유---의학적 사유, 우생학적 사유, 성범죄로 인한 임신, 사회 경제적 사유---를 규정하는 전통적인 `사유규정방식' 혹은 `적응규정방식' (Indikationslösung)''으로 회귀하였다.

이후 독일은 통일되었고, 통일 당시 서독과 동독의 낙태 규정은 많은 차이를 보이고 있었다. 이에 1990년 독일통일조약 제31조는 낙태에 관한 새로운 법규정을 만들 의무를 부과했다. 이에 따라 논란끝에 통과된 1992년 임산부와 가족보조법(Schwangeren- und Familienhilfegesetz)에 의해 형법 218조 등이 개정되었다. 이는 낙태행위는 원칙적으로 처벌하지만, 예외사유가 존재하는 경우에는 위법하지 않으며, 추가적으로 임신 후 12주까지의 낙태의 경우 낙태 3일 전까지 상담소에서 상담을 받는 것을 전제로 의사에 의한 낙태를 허용하였다.

이에 대하여 의원 249명은 이 법에 대한 위헌법률심사를 청구하였고, 1992년 독일연방헌법재판소는 상담의무를 지킨 수태 후 12주까지의 낙태를 위법하지 않다고 선언한 218조a의 1조를 비롯한 관련규정들을 위헌으로 판단하였다. 이 판결은 태아의 생명권을 수태 후 전 기간동안 중시하고, 이 생명권이 임부의 권리보다 일반적으로 우선권을 가지며, 국가의 생명보호의무가 있음을 확인하였다는 점에서 제1차 낙태판결과 기본적인 동일성을 보인다. 그러나 이 판결로 귀결된 독일의 낙태에 대한 입법과 정책이 독일을 오히려 낙태 우호국가로 분류하게 만드는 것만 보더라도, 이 2차 판례의 설시에는 본 논문이 주목하고자 하는 몇 가지 중대한 차이점이 있다.

첫번째 국가의 보호의무의 범위와 그 근거이다. 제1차 낙태판결에서는 태어나지 않은 생명, 즉 태아에 대한 보호의무를 단적으로 자기결정권에 우선하는 인간존엄으로부터 이끌어 내었다고 한다면, 제2차 낙태판결에서는 법익의 의미와 보호필요성, 보호의 적절성과 실효성을 고려하여 국가의 보호의무를 정하게 된다.\footnote{제2차 낙태판결의 다섯번째 결정요지는 다음과 같이 설시한다. ``아직 태어나지 않은 인간의 생명에 대한 보호의무의 범위는 일방에서 보호되어야 할 법익과 타방에서 이와 충돌하는 법익의 의미와 보호필요성을 살펴서 결정되어야 한다. 이때 태아의 생명권에 의해 침해되는 법익으로서는 - 인간존엄에 대한 존중과 보호를 받을 임신부의 권리(기본권 제1조 제1항)로부터 출발하여 - 무엇보다도 먼저 임신부의 인격권(기본법 제2조 제1항) 및 임신부의 생명권과 신체의 불훼손성의 권리(기본법 제2조 제2항)가 문제된다. 이에 반해서 임신부는 임신중절에 수반되는 태아살해에 대하여 기본법상 제4조 제1항에서 보장한 법적 지위를 주장할 수는 없다.''(독일통일관련 독일연방헌법재판소판례번역집, 1997, 헌법재판소), 고봉진, ``생명윤리에서 인간존엄 ``개념''의 총체성'', 법철학연구 11권 1호 (2008), 89면 이하 참고.} 고봉진 교수는 이러한 연방헌법재판소의 태도를 적응사유모델에서 상담모델로 교체된 것으로서, ``\,`태어나지 않은 생명의 우월한 권리'가 아니라, `태어나지 않은 생명에 대한 더 나은 보호효과'가 그 배후에 있''는 ``정당화 도그마틱의 후퇴''(하쎄머의 표현)로 평가한다.

제2차 낙태 판결에 의하면 적절한 상담을 거친 낙태는 `위법하지만 처벌할 수 없다'는 규범적 평가를 받으며, 동시에 이와 같이 국가가 이를 위법하다고 평가하면서도 처벌하지 않는 태도가 국가의 보호의무를 이행하고 있는 것으로 평가받는다.\footnote{독일연방헌법재판소는 4번째 결정요지에서 ``임신중절은 임신의 전 기간동안 원칙적으로 불법인 것으로 간주되어야 하고 따라서 법적으로 금지되어야만 한다''고 주장하면서도 11번째 결정요지에서는 ``임신부에게 임신을 지속하도록 하기 위하여 임신의 초기단계에 있어서는 임신갈등을 겪고 있는 임신부의 상담에 중점을 두고 이때에 정당화사유와 결부된 형벌적 위하를 하지 않고,제3자에 의해 낙태의 정당화요건의 확인을 받는 것을 요구하지도 않는 태아의 생명보호에 관한 입법초안을 입법자가 마련하는 것이 헌법상 원칙적으로 금지되지 아니한다''고 말한다.(BVerfGE 88, 203).} 예를 들어 이러한 논리의 귀결로, 처벌할 수 없는 낙태에 대해서도 의료보험수급권이 수반되지 않는다.\footnote{16번째 결정요지 참조 : ``합법성이 확인되지 않은 낙태행위에 대해서 기본법은 법률상의 의료보험 제공청구권을 보장해 주지는 아니한다. 이에 반하여 경제적인 궁박의 경우에 상담규정에 의해서 형벌로 위하되지 않는 낙태에 대하여 社會扶助를 제공하는 것은 질병시 제공되는 임금계속지불제도와 마찬가지로 헌법상 이의가 제기될 수 없을 것이다.'' (BVerfGE 88, 203).} 이와 같이 낙태에 대한 입법방식과 보호의무의 정도에 대한 모델이 적응사유모델에서 사실상의 상담모델로 교체되기 위해서는, 그 이전에 기본권 충돌에 대한 입장이 생명의 자기결정권에 대한 우선성 모델에서 두 가치 혹은 권리 간의 형량가능 모델 혹은 효율성 모델로 전환되어야 한다. 이러한 관점에서 독일연방헌법재판소의 전체적인 입장은 이를 통해, ``상담을 통한 보호효과가 제재를 통한 억압적인 보호효과보다 더 효과적이고 효율적이라고 주장''\footnote{고봉진, ``생명윤리에서 인간존엄 ``개념''의 총체성'', 90면.}하는 것으로 보인다. 이러한 태도---형량가능모델을 전제한 상담모델을 취하는 태도---는 연방헌재가 인간존엄으로부터 도출된 태아 생명권의 절대성 논변을 유지하는 한 모순이라는 비판을 받고 있다.\footnote{고봉진, 위의 논문, 90면 참조. 상담모델을 옹호하면서 인간존엄의 비교형량 가능성을 인정하는 입장으로는 마렌홀쯔(Mahrenholz)와 좀머(Sommer)의 소수의견 참조(BVerfGE 88, 203 -- NJW 1993, 1776면). 상담모델을 사실상 기한모델이라고 비판하면서 이것이 보호의무를 이행하지 못한다는 입장으로는 Herbert Tröndle, ``vor § 218'', Tröndle/Fischer, Kommentar zum Strafgesetzbuch(49.Aufl., 1999), 1164면 참조.} 다만 이러한 비판은 이른바 적응사유 모델을 취했던 제1차 낙태판결도 동일하게 받는다.

두번째 차이는 상담모델이 지켜야하는 상담에 관한 규정이 매우 구체화되고 엄격해졌다는 것이다. 제2차 낙태판결의 12번째 판결요지는 ``그러한 상담초안은 태아의 생명보호를 위해서 임신부의 행위에 대한 적극적 요건을 설정하는 골격을 갖추어야 한다''고 설시한다. 이러한 골격을 갖추지 못한 상담규정은 상담이라는 형식적 절차를 갖추었다 할 지라도 위헌적이다. 판결의 결정주문 II의 3, 4의 명령을 통해 상담소의 기능과 기본적으로 갖추어야 할 요건들을 상세히 설시한다. 상담소는 기본적으로 임신부가 아니라 태아의 보호를 위해 기능하며, 임신부에게 책임있고 양심적인 결정을 하도록 조력해야만 한다{[}3의 (1){]}. 이외에도 자격있는 전문인들을 갖출것{[}3의 (3){]}과 임신부및가족부조법상의 인허와는 별도로 반드시 국가의 공인을 얻을 것을 요구하고 있으며{[}4의 (1){]}, 낙태를 시술하는 의사가 상담인이 되지 못할 것 등을 규정한다{[}4의 (2){]}. 연방헌재의 이러한 구체적 설시는 사법기관의 권능을 넘어 입법자의 입법재량범위를 침해하는 것이 아닌가 하는 의문을 갖게 할 수도 있다.

이러한 제2차 낙태판결의 두 가지 중대한 차별점, 즉 1) 제2차 낙태 판결에 의하면 적절한 상담을 거친 낙태는 `위법하지만 처벌할 수 없다'는 규범적 평가를 받으며, 동시에 이와 같이 국가가 이를 위법하다고 평가하면서도 처벌하지 않는 태도가 국가의 보호의무를 이행하고 있는 것으로 평가받는다는 점과, 2) 상담모델이 지켜야하는 상담에 관한 규정이 매우 구체화되고 엄격해져서 임신부가 아니라 태아의 보호를 위해 기능하는 골격을 갖추지 못한 상담규정은 상담이라는 형식적 절차를 갖추었다 할 지라도 위헌이며 입법자의 입법재량범위를 침해하는 것이 아닌가 하는 의문을 갖게 할 정도의 구체적인 상담기관 설치에 관한 기준을 제공하는 판결을 했다는 점은 본고가 제시하는 인간존엄의 믿음 기반 법익모델의 전형을 보여준다고 볼 수 있다.

먼저 적절한 상담을 받을 의무부과의 목적은 좋은 삶의 사실적 기반에 대한 간섭이 아니라 의미적 간섭만을 의도한다. 이는 어떤 처벌이나 이익의 손상, 혹은 낙태행위의 근본적인 좌절을 의도하는 것이 아니라 오로지 상담이라는 언어교환 행위를 통해 태아의 생명이 존엄하다는 의미론적 평가만을 고취시키는 것이기 때문이다. 둘째로 이렇게 상담을 거쳐 행해진 낙태가 처벌할 수는 없지만 여전히 위법하다는 평가를 받는 것 역시, 사실적 기반에 대한 간섭이 아니라 ``인간 생명의 보호는 객관적으로 중요하다''는 의미적인 평가에 의한 간섭만을 의도하고 있는 것으로 평가되어야 한다.

이러한 점에서 볼 때, 독일의 제2차 낙태판결은 현대적 인간존엄의 법익이 바로 믿음에 기반한 법익임을 포착하고 있는 것으로 보인다.

\subparagraph{한국의 연명치료중단 판결}

독일의 2차 낙태판결만큼 명확하지는 않지만, 대한민국의 대법원 판례 중에도 인간존엄의 법익이 믿음에 기반한 법익임을 포착한 것으로 보이는 소수의견의 설시가 있다.

일명 `김할머니 사건'으로 불리는, 회복불가능 사망의 단계에 진입한 환자에 대한 연명치료의 중단이 허용되는지 구체적으로 언제 어떤 경우에 허용되는지 여부에 관한 사건은 생명윤리영역에서 인간의 존엄이 실제 사법영역에서 가장 구체적으로 고려된 국내의 유일한 대표적 사건이라 해도 과언이 아니다. 김할머니라고 우리가 통칭하는 67세의 여성이 실제로 회복불가능 사망의 단계에 진입하였는지 등의 의학적 사실에 대한 판단은 본 논문에서 고려하려는 대상이 아니다. 다만 이러한 사안에서, 절대적이라고 생각되는 인간존엄이라는 규범적 판단준거가 사법적 판단에 어떤 방식으로 고려되고 개입되어야 하는지가 우리의 관심사다.

해당 판결의 판결요지는 다음과 같이 판단한다.

{[}다수의견{]} (가) 의학적으로 환자가 의식의 회복가능성이 없고 생명과 관련된 중요한 생체기능의 상실을 회복할 수 없으며 환자의 신체상태에 비추어 짧은 시간 내에 사망에 이를 수 있음이 명백한 경우(이하 `회복불가능한 사망의 단계'라 한다)에 이루어지는 진료행위(이하 `연명치료'라 한다)는, 원인이 되는 질병의 호전을 목적으로 하는 것이 아니라 질병의 호전을 사실상 포기한 상태에서 오로지 현 상태를 유지하기 위하여 이루어지는 치료에 불과하므로, 그에 이르지 아니한 경우와는 다른 기준으로 진료중단 허용 가능성을 판단하여야 한다. \ul{이미 의식의 회복가능성을 상실하여 더 이상 인격체로서의 활동을 기대할 수 없고 자연적으로는 이미 죽음의 과정이 시작되었다고 볼 수 있는 회복불가능한 사망의 단계에 이른 후에는, 의학적으로 무의미한 신체 침해 행위에 해당하는 연명치료를 환자에게 강요하는 것이 오히려 인간의 존엄과 가치를 해하게 되므로}, 이와 같은 예외적인 상황에서 죽음을 맞이하려는 환자의 의사결정을 존중하여 환자의 인간으로서의 존엄과 가치 및 행복추구권을 보호하는 것이 사회상규에 부합되고 헌법정신에도 어긋나지 아니한다. 그러므로 회복불가능한 사망의 단계에 이른 후에 \ul{환자가 인간으로서의 존엄과 가치 및 행복추구권에 기초하여 자기결정권을 행사하는 것으로 인정되는 경우에는 특별한 사정이 없는 한 연명치료의 중단이 허용될 수 있다}. 한편, 환자가 회복불가능한 사망의 단계에 이르렀는지 여부는 주치의의 소견뿐 아니라 사실조회, 진료기록 감정 등에 나타난 다른 전문의사의 의학적 소견을 종합하여 신중하게 판단하여야 한다.

(나) 환자가 회복불가능한 사망의 단계에 이르렀을 경우에 대비하여 \ul{미리 의료인에게 자신의 연명치료 거부 내지 중단에 관한 의사를 밝힌 경우(이하 `사전의료지시'라 한다)에는}, 비록 진료 중단 시점에서 자기결정권을 행사한 것은 아니지만 사전의료지시를 한 후 환자의 의사가 바뀌었다고 볼 만한 특별한 사정이 없는 한 사전의료지시에 의하여 \ul{자기결정권을 행사한 것으로 인정할 수 있다}. 다만, 이러한 사전의료지시는 진정한 자기결정권 행사로 볼 수 있을 정도의 요건을 갖추어야 하므로 의사결정능력이 있는 환자가 의료인으로부터 \ul{직접 충분한 의학적 정보를 제공받은 후 그 의학적 정보를 바탕으로 자신의 고유한 가치관에 따라 진지하게 구체적인 진료행위에 관한 의사를 결정하여야} 하며, 이와 같은 의사결정 과정이 환자 자신이 직접 의료인을 상대방으로 하여 작성한 서면이나 의료인이 환자를 진료하는 과정에서 위와 같은 의사결정 내용을 기재한 진료기록 등에 의하여 진료 중단 시점에서 명확하게 입증될 수 있어야 비로소 사전의료지시로서의 효력을 인정할 수 있다.

(다) 한편, 환자의 사전의료지시가 없는 상태에서 회복불가능한 사망의 단계에 진입한 경우에는 환자에게 의식의 회복가능성이 없으므로 더 이상 환자 자신이 자기결정권을 행사하여 진료행위의 내용 변경이나 중단을 요구하는 의사를 표시할 것을 기대할 수 없다. 그러나 환자의 평소 가치관이나 신념 등에 비추어 연명치료를 중단하는 것이 객관적으로 환자의 최선의 이익에 부합한다고 인정되어 환자에게 자기결정권을 행사할 수 있는 기회가 주어지더라도 연명치료의 중단을 선택하였을 것이라고 볼 수 있는 경우에는, 그 연명치료 중단에 관한 환자의 의사를 추정할 수 있다고 인정하는 것이 합리적이고 사회상규에 부합된다. 이러한 환자의 의사 추정은 객관적으로 이루어져야 한다. 따라서 환자의 의사를 확인할 수 있는 객관적인 자료가 있는 경우에는 반드시 이를 참고하여야 하고, 환자가 평소 일상생활을 통하여 가족, 친구 등에 대하여 한 의사표현, 타인에 대한 치료를 보고 환자가 보인 반응, 환자의 종교, 평소의 생활 태도 등을 환자의 나이, 치료의 부작용, 환자가 고통을 겪을 가능성, 회복불가능한 사망의 단계에 이르기까지의 치료 과정, 질병의 정도, 현재의 환자 상태 등 객관적인 사정과 종합하여, 환자가 현재의 신체상태에서 의학적으로 충분한 정보를 제공받는 경우 연명치료 중단을 선택하였을 것이라고 인정되는 경우라야 그 의사를 추정할 수 있다.

판례를 해석해 보면 다음과 같다. 먼저, 회복불가능 사망의 단계에 진입한 환자에 대한 연명치료의 중단은 특정한 요건을 충족시키면 허용된다고 본다. 즉, 적어도 생명권은 독일식 법도그마틱의 의미에서 절대권은 아니며, 처분불가능한 권리도 아니라는 것을 함축한다. 또한 이러한 생명의 침해가 인간의 존엄을 해치는 것은 아니라는 점도 분명하다. 둘째로, 환자의 충분한 정보를 토대로 한 진지한 사전의료지시가 있는 경우에는 환자의 자기결정권이 가장 우선시 된다. 즉, 인간존엄의 존중에 있어 최우선적으로 고려되는 것은 생명유지 그 자체의 가치보다는 자기결정권의 가치이다. 셋째로, 사전의료지시가 없는 상태에서 의식회복가능성이 없는 경우에는 환자의 의사를 추정할 수도 있다. 이 때 추정은 객관적인 방식으로 이루어져야 한다고 말한다. 즉, 다수의견에서 인간존엄은 환자의 구체적 자기결정행위 뿐만 아니라, 결정의사의 잠재성까지 고려하는 매우 개인적인 것으로 해석될 여지가 있다. 이러한 다수의견의 사법적 판단은 여러 비판의 여지를 남기지만 인간존엄의 존중이 근본적으로는 `자율적 선택의 존중'에 있다는 해석론을 적용한 것으로 볼 수 있다. 이러한 해석론에 대해서는 이미 비판한 바 있다.

이 지점에서 보다 주목하려고 하는 것은 이 판결의 일부 반대의견이다. 연명치료의 중단에 관한 대법원 2009.05.21. 선고 2009다17417 전원합의체 판결\footnote{이 사안에서 다수의견은 ``(전략) 회복 불가능한 사망의 단계에 이른 후에는, 의학적으로 무의미한 신체 침해 행위에 해당하는 연명치료를 환자에게 강요하는 것이 오히려 인간의 존엄과 가치를 해하게 되므로 이와 같은 예외적인 상황에서 죽음을 맞이하려는 환자의 의사결정을 존중하여 환자의 인간으로서의 존엄과 가치 및 행복추구권을 보호하는 것이 사회상규에 부합되고 헌법정신에도 어긋나지 아니한다''고 주장한다. 인간의 존엄이 ``일반적으로는 자율성보다 생명존중이 우선하나, 사실상의 해악이 극대화된 경우에만 예외적으로 자율성이 우선한다''는 함축을 가지고 있다는 것이다. 이는 존엄원리의 보호영역을 사실상의 기반으로서의 해악과 자율성으로 한정하고 변용하는 입장이라고 볼 수 있다.} 중 대법관 안대희, 대법관 양창수의 반대의견을 살펴보자. 이들은 언제 연명치료의 강요가 인간의 존엄과 가치를 해하게 되는지의 판단이 단순히 자율적 선택의 존중에 있는 것이 아니라 보다 더 다양한 사항을 고려해야 하는 것이라고 말한다. 다음을 보자.

(대법관 안대희, 대법관 양창수의 반대의견 중) ``구체적으로 어떠한 경우에 연명치료를 환자에게 강요하는 것이 오히려 환자의 인간으로서의 존엄과 가치를 해하게 되는가의 판단은 일률적으로 말할 수 없고, 그 가족을 포함한 환자 측 및 의료기관의 제반 사정을 합리적으로 고려하여 정할 수밖에 없다. 구체적으로는 환자의 나이·직업이나 경력, 평소의 종교·신념이나 생활태도, 질환의 경과와 현재 상태, 생명의 연장이 가능한 기간의 장단, 이미 지출한 또는 앞으로 지출하게 될 비용, 가족들의 상황, 환자로 인한 가족들의 정신적 고통, 그들의 경제적 지출을 포함한 생활상의 희생 등 환자 측의 사정은 물론이고, 의료기관의 성격이나 설비, 그 진료의 내용과 결과, 의료진의 견해 등과 같은 의료기관 측의 사정이 문제될 것이다. 그리고 연명치료의 중단에 관한 환자 가족들의 동의 여부도 그러한 판단에 있어서 고려되어야 할 중요한 요소의 하나로서, 오히려 이는 독자적인 요건에 해당한다고 할 것이다(가족의 동의 요건 및 그 내용에 관하여는 `장기 등 이식에 관한 법률' 제18조 제3항 제2호 등이 유추적용될 수 있다). 이렇게 보면 이 소수의견과 다수의견의 차이는 법논리적인 것 또는 소송내용 등은 별론으로 하고 그 결론에 있어서는 아주 큰 것이 아니라고 해도 좋을지 모른다.

그러나 중요한 것은 여기서 환자의 `가정적 의사'는 연명치료 중단의 허용 여부를 판단하는 유일한 또는 결정적인 요소는 아니라는 점이다. 즉, 환자의 가정적 의사가 연명치료의 중단에 찬성하지 않는 것으로 밝혀지더라도, 그 의사를 존중하여 연명치료를 계속하는 것이 환자의 인간으로서의 존엄과 가치에 반한다고 말할 수 있는 경우가 상정될 수 있다. 요컨대, 이 단계에서 연명치료의 중단 여부는 법질서 일반의 관점에서 행하여지는 당해 사안에 대한 객관적인 이익형량 내지 가치평가의 문제인 것이다.''

이 반대의견은 그 설시 중 언제 환자의 존엄과 가치를 침해하는가의 판단에 있어 제반사정을 고려해야 한다고 말한다. 환자의 가정적 의사도 유일한 요소가 될 수는 없다고 말한다.\footnote{이에 더하여 추정적 의사에 대해서도 같은 입장을 취할 수 있어야 한다고 생각한다.} 나아가 최종적인 중단여부를 결정하는 것은 ``법질서 일반의 관점에서 행하여지는 {[}\ldots{]} 가치평가의 문제''라고 주장한다.

결국 연명치료중단의 문제는 궁극적으로 오직 환자 본인의 이익, 사실적 기반의 문제가 아니라 그러한 결정이 법질서의 가치해석에 부합하느냐 하는 의미의존적 문제로서 판단되어야 한다는 것으로 해석될 수 있다. 즉 이 설시는 그러한 중단행위가 공동체 구성원과 당사자의 인간존엄의 믿음기반 법익을 침해하는 경우, 즉 ``인간적으로 좋은 삶을 영위하는 것이 객관적으로 중요하다''는 믿음의 이익을 침해한다면 이는 허용되어서는 안 된다는 것으로 이해해 볼 수 있다.

\subsection{소결}

이상의 제3부에서는, 인간존엄의 철학적 이해방식의 한계와 법적담론에서의 인간존엄 이해방식의 문제점과 해결과제를 염두에 두고, 이들 각각의 이해방식을 비판적으로 재구성하고 실제의 구체적 사안들에 어떻게 실현될 수 있는지 그 적용방안을 제시하고자 시도하였다. 먼저 인간존엄을 ``인간이 어떤 가치속성으로서 보유하는 내적 초월적 핵심''이라고 주장한 현대적 이해방식의 한계를 극복하고, 위상적 개념으로서의 그 역사적 어원을 존중하는 것으로 구성하고자 시도했다. 또한, 현대적 요청이 제기하는 맥락-관련적 인권 또는 기본권을 도출하고, 의미의존적 이익을 보호하며, 형량불가능성 도그마에 대한 적절한 설명과 그 한계와 기준을 제공하고자 하였다.

이는 우리가 법담론에서 혼동하면서 쓰고 있는 두 개의 `인간존엄' 사용을 확인하고 이를 구분함으로써 가능하다. 하나는 인간적으로 좋은 삶을 사는 것의 객관적 중요성이며, 다른 하나는 국가가 적극적으로 보호해야할 제3의 법익인 인간적으로 좋은 삶을 사는 것의 객관적 중요성에 대한 중심부 믿음을 형성하고 유지할 이익인 독립적 인간존엄-이익이다.

먼저 ``인간은 존엄하다''는 명제는 사실 도덕성의 높음을 나타내는 ``인간적으로 좋은 삶을 영위하는 것은 객관적으로 중요하다''(\emph{Pdig})라는 명제라는 점을 통해, 우리가 사용하는 위상적 개념으로서의 존엄의 어의의 함축을 살리면서도, 현대인권적 맥락이 제기하는 권리 정초의 역할을 수용할 수 있는 개념을 제공하였다.

그러나 이 명제는홀로코스트에 대한 반성, 노예금지, 키메라연구의 규제 등 특수한 사안에서 `인간존엄'의 현대적 사용을 모두 포괄하지 못하고 있다. 이러한 새로운 이익은 인간적으로 좋은 삶을 사는 것의 객관적 중요성에 대한 정당화된 믿음형성의 이익이라고 할 수 있다. 사람들은 복지이익의 향유나 자율적 선택만으로는 좋은 삶을 완전히 누리지 못하며 자신이 추구하는 삶이 객관적으로 중요한 것이라는 믿음을 형성하고 유지하며 살아갈 때에만 완전한 좋은 삶을 누릴 수 있기 때문이다.

이러한 믿음형성의 이익은 무분별하게 보호받아야 하는 것은 아니며 본고는 이러한 보호를 요청하는 기준을 제공하고자 시도하였다. 먼저 인간존엄-이익으로서 특별한 보호를 요청하는 이익은 잘못된 믿음이 아닌 정당화된 믿음이어야 한다. 또한 인간적으로 좋은 삶에 대한 믿음은 개별적 믿음의 모음이 아니라 믿음을 구성하는 문장들이 형성하는 장으로서의 전체이며, 또한 그 중 중심부 믿음은 쉽게 포기되는 성격의 것이 아니다. 따라서, 인간존엄-이익의 침해라고 할 수 있기 위해서는 중심부 믿음과 관계된 것이어야 하고 그 정도는 중심부 믿음이 포기될 수 있을 정도로 위협적인 것이어야 한다.

이상과 같이 제3장에서는 위와 같이 인간존엄을 한편으로 모든 이들의 인간적으로 좋은 삶에 대한 평등한 존중의 의무를 정당화시키는 관념으로, 다른 한편으로 인간적으로 좋은 삶을 사는 것의 객관적 중요성에 대한 정당화된 믿음형성의 이익으로 이원적으로 구성하는 전략을 통해 앞서 제기된 인간존엄의 유용성을 확립하고자 하였다. 인간존엄은 타인과 자신의 가치와 권리를 존중해야 할 의무를 굳건히 정초하는 개념이면서, 동시에 현대 사회에서 특별한 우선적 보호를 요청해 왔던 좋은 삶에 대한 중심부 믿음형성의 이익을 보호하는 구체적인 효용을 제공하는 개념이라고 할 수 있다.

\section{결론}

``존엄 개념은 의료연구나 실천이 인간존엄을 위반하거나 위협한다고 주장하는 사태의 해결에 있어 일관성과 유용성이 없''으며 ``존엄에의 호소는 다른 더 상세한 개념들에 대한 모호한 재진술이거나 그 주제에 대한 이해에 아무런 더함이 없는 단순한 슬로건에 불과하다''는 존엄 회의주의의 비판은 매우 강력하다.\footnote{Ruth Macklin, ``Dignity is a Useless Concept: It means no more than respect for persons or their autonomy''.} 이러한 회의주의의 의심은 존엄 개념에 대한 대표적 이해방식, 즉 ``한 인간이 특정한 능력이나 속성으로 환원될 수 있는 무조건적이고 비교불가능한 가치를 가지고 있고, 이 가치가 가지는 도덕적 함축을 통해 타인이나 국가공권력에 대하여 이에 대한 간섭이나 침해에 대한 무제한적인 면제를 요구할 수 있다''는 이해로부터 비롯된 것으로 보인다. 이러한 이해를 토대로 인간존엄의 개념을 운용하고 있는 사법적 현실이 명확한 침해의 요건과 합리적인 법적 효력의 수준을 제시하지 못하고, 상호 모순되는 기준을 제공한다면 이러한 회의주의의 중요한 의심은 더욱 강화되지 않을 수 없다.

이러한 문제들을 해결하기 위해 본고는 제1부에서 인간존엄의 철학적 이해방식을 규명하고 제2부에서 법적 담론에서의 이해방식을 살펴보았으며, 제3부에서 현대적 이해방식의 문제점에 대한 각각의 대안을 제시하였다. 먼저 본고는 제1부에서 인간존엄의 철학적 이해방식이 갖추어야 할 개념적 요건들을 제시하고 현대적 이해방식이 가지고 있는 난점들을 제기했다. 이는 첫째, 인간존엄은 권리산출의 기초가 되어야 하며, 둘째, 인간존엄은 인간에게서 박탈불가능한 것이어야 하며, 셋째, 인간존엄은 우리가 타인을 존중해야 할 기초를 제공하는 것이어야 하고, 넷째, 인간존엄은 값을 매길 수 없는 것으로서 다른 어떤 가치와도 형량할 수 없는 어떤 것이어야 하고, 다섯째, 인간존엄은 모든 이들이게 적용되는 평등주의적 개념이어야 하며, 여섯째, 인간존엄은 어느 문화권에서나 보편적으로 통용될 수 있는 개념이어야 한다는 것이다.

그런데 대표적인 현대적인 이해방식은 인간존엄을 ``타인으로부터 존중을 유발하는 인간의 가치 속성'', 예를 들어 자율성이나 품위, 생명의 신성성과 같은 개별 인간 안에 내재하는 것으로 이해하는 것으로 보인다. 그리고 이러한 이해방식은 칸트의 문헌에서 그 근거를 얻을 수 있는 것으로 간주되어 왔다. 그러나 이러한 현대적 이해방식은 위의 개념적 요건을 갖추기 어렵고, 또한 칸트의 인간존엄 개념은 이러한 이해방식을 대변하지 않는다.

첫째, 인간존엄은 권리산출의 기초가 되어야 하는데, 언뜻 현대적 이해방식은 권리, 혹은 인권의 당사자가 자신이 인간존엄이라는 가치를 보유했다는 것을 지적함으로써, 해당 가치를 실현하는데 필요한 권리를 가짐을 정당화할 수 있는 것처럼 보인다. 그러나 가치의 보유는 권리와 그에 기반한 타인의 행위나 감수를 수반하는 존중의무를 도출하는 논리적 기초가 될 수 없다.

둘째, 인간존엄은 박탈불가능한 속성을 가져야 하는데, 현대적 이해방식은 이와 동시에 인간존엄을 여전히 추구되어야 할 어떤 것, 혹은 침해가능한 어떤 것으로 표현하고 있다는 점에서 문제가 된다.

셋째, 현대적 의미에서의 인간존엄은 타인에 대한 일정한 침해를 금지하고 타인의 삶의 완전성을 기하는 것을 돕기 위해 주로 사용되는데, 어떤 능력에 대한 인간에게 부여된 특별한 능력이나 소질을 기초로, 인간의 존엄함을 파악하는 입장을 채택할 경우, 이러한 능력은 이미 선물받은 자로서, 혹은 어떤 목적을 가진 능력을 부여받은 자로서 주로 자신이 실현해야 할 어떤 의무를 부과하는 근거이지, 존중받을 권리를 창출하는 것이 아니라는 점에서, (특히 능력을 전제로 한) 어떤 가치로부터 타인존중의 의무를 논증하는 것은 생각보다 쉽지 않다.

넷째, 인간존엄의 사법적 요청은 국가기관 혹은 사인에게 어떤 절대적 비교우위의 부담을 지우는 것으로 판단된다. 우리의 사법현실은 대부분의 가치간의 충돌 문제에서 이익형량을 하고 있으며, 실제 사안들에서 인간존엄과 관련되어 있다고 보지 않을 수 없는 상이한 가치들 간에도 충돌의 문제는 발생한다. 그런데 인간존엄을 비교불가능한 인간의 가치 속성으로 이해한다면, 현실의 사법적 사안에서 이익형량의 문제를 해결하기 어렵다.

다섯째, 인간존엄은 인간 모두에게 적용되는 평등주의적 함축을 지녀야 한다. 현대적 이해방식은 언뜻 `존엄'의 우월한 지위로서의 고대적 사용에 비하여 이러한 함축을 더 잘 설명하는 것처럼 보일 수 있다. 그러나 현대적 이해방식에 의하면 이러한 가치속성을 보유하지 않는 자나 적게 보유하는 자---예를 들어 태아나 정신장애를 가진 자---에게는 인간존엄에 의한 존중을 전부 혹은 일부 인정하기 어렵게 만든다.

여섯째, 인간존엄은 문화를 초월한 보편적으로 가치있는 인간성을 전제로 해야 한다. 그러나 각 문화마다 중요하게 생각하는 인간성은 다양할 수 있다. 자율적 선택능력과 같은 특정한 가치속성을 전제로 하는 현대적 이해방식은 다양한 이해 중에서 모든 국가의 법제도가 인정할 수 있는 인간존엄의 이해방식으로 받아들이는 좋은 전략이 되기 어렵다.

또한 이러한 이해방식은 칸트의 타인존중의무를 도출하는 전략과 인간존엄 개념의 사용과도 일치하지 않는다. 칸트의 도덕철학은 정언명령을 토대로 한 강력한 형식주의를 토대로 하고 있는데, 만약 칸트의 철학이 인간존엄이라는 모종의 특수한 이익을 전제하고 있다면, 칸트 도덕철학의 논증구조는 무너질 수 있다. 칸트의 강력한 형식주의의 요청은 비도덕적 전제들로부터 도덕적 의무를 도출하는 것을 타율성을 발생시킨다는 이유로 거부하기 때문이다. 그러나 칸트는 분명 타인존중을 유발하는 인간 안에 있는 어떤 절대적 내적 가치로서 `인간존엄'을 사용하고 있지 않다. 칸트에게서 `존엄'은 여전히 스토아적인 지위로서의 개념, 즉 다른 어떤 것보다 위로 들어올려지는 어떤 `높음'의 개념을 표현하는데 사용되고 있다. 올리버 센슨의 표현대로, ``존엄은 가치에 대한 어떤 정의(definition)가 아니라, 도덕성이 우월하거나 특별하다는 것을 말하는 한 방식이다.''

본고는 제2부에서 법적 개념으로서의 인간존엄 개념이 수행하고 있는 역할들을 제시한다. 먼저 인권담론에서의 인간존엄의 역할은 홀로코스트라는 참혹한 사건으로 대표되는 제2차 세계대전시기의 나치 이데올로기에 대한 반성으로 형성된 국제사회의 이해방식에서 발견된다. 이들은 인간존엄 개념을 어떤 권리들, 특히 인권들을 도출하는 기초 혹은 정초라고 생각한다.

대표적인 정초 개념은 인간존엄이 모든 인권들의 텔로스와 같은 것이기에, 우리가 이미 가지고 있는 권리들을 넘어서서 다른 권리들을 도출하여 권리목록을 확장할 수 있는 가능성을 열어준다는 것을 의미한다. 그러나 한편에서는 인간존엄의 인권 정초의 의미를 축소하고자 노력한다. 인간존엄은 새로운 권리들을 도출하는 기초가 아니며, 여기서 정초의 의미는 이미 존재하는 권리들을 더 자세히 이해하기 위한 해석적 근거를 지칭할 뿐이라는 것이다.

그러나 경제질서나 자연환경의 변화, 생명공학기술의 발전 등에 의한 위협은 계속해서 새로운 보호방법을 강구할 것을 요청한다. 그럼에도 불구하고 이에 대한 새로운 대응을 하지 않는 것이야말로 부도덕한 일이 된다. 따라서 인권에 대한 이해와 구체화는 가능한 또는 있음직한 맥락-관련적 위협에 대한 대응으로 이루어져야 한다. 인권은 시간이 지남에 따라 변할 수 있고, 변할 것이고, 변해야만 하는 개념으로 이해되어야 하는데 인간존엄의 인권의 정초 개념은 새로운 형식의 인권을 창조하는 역할을 수행하는 개념으로 해석되어야 한다. 다만 인간존엄 개념의 지나친 추상성과 개방성이 권리의 개념을 형해화시키고 기존의 논의를 잠식시키는 논의종결자(conversation stopper)로 이용되는 것을 방지할 수 있는 인간존엄과 관련된 권리의 개념을 보다 구체적으로 형성할 필요가 있다.

세계인권선언을 비롯한 국제문서들을 통해 관찰할 수 있는20세기 후반의 국제사회의 인간존엄에 대한 이해방식은 또한 각국의 법문언, 특히 헌법에 영향을 끼쳤다. 학설들은 헌법에서의 인간존엄의 위상과 기능을 해석하면서, 인간이 가지는 어떤 인식가능한 고유한 속성으로, 또한 어떤 침해가능한 권리들이 도출되는 기초로 파악하는 현대적 이해방식을 그 토대로 삼고 있는 것으로 보인다.

먼저 인간존엄이 헌법담론에서 수행하는 기능에 있어, 많은 학자들은 인간존엄이 법적 가치와 권리로서 기능하고 있다고 말한다. 실제로 남아프리카공화국헌법은 이러한 구별개념 각각을 직접 헌법문언에 반영하고 있다. 먼저 가치로서의 인간존엄을 살펴보면, 학자들은 이를 인간존엄이 ``최고규범으로서 모든 법령의 효력과 내용을 해석하는 기준이 되는 근본원리''라고 하거나, ``기본권 보장의 이념적 기초'' 혹은 ``헌법상의 이념 또는 원리'' 등으로 표현하고 있다. 인간존엄이 헌법적 가치라고 한다면 헌법적 가치로서의 인간존엄은 구체적으로 어떻게 기능하는가?

우선, 헌법적 가치로서의 인간존엄은 권리규범의 내용을 명확히 하기 위한 해석의 근거로서 사용될 수 있다. 이 때 법적 가치는 주로 법적 권리가 증진시켜야 할 대상을 분명히 하는 개념이다. 한편으로 인간존엄은 권리를 제한하는 규범을 해석하는 근거로서도 사용된다. 또한, 인간존엄을 개별규범의 내용을 구체화하는 가치를 넘어서 헌법에서 보다 특별한 의미를 가진 가치로 파악하는 경우, 이는 서열상 다른 가치들의 우위에 있는 유일한 최우선의 헌법적 가치로서, 모든 해석이 전체 법체계의 관점에서 지향해야 할 최종적 목적이 된다고 일반적으로 이야기되기도 한다. 이는 개별 가치를 증진시키고자 하는 개별 규정의 모호성을 해결하는 목적의 수준에서 한 단계 더 나아가 가치들 간의 충돌 상황에서 규정들 간의 최종적 조정자가 되고, 심지어 인간존엄의 가치를 보호하는 규정 자체가 누락된 상황에서는 문제를 해결하는 기준을 새로 창조하기도 하는 규범창조자의 지위까지 얻게 될 가능성을 함축한다.

해석의 근거로서의 가치나 법목적은 그 자체로 입법권력이 제정한 개별적인 실정법 문언에 드러나지 않는 경우가 많고, 실정법의 내용이 모호할 때 그 확정이나 제한을 위한 보충적 개념으로 주로 이해되고 있다. 문제는 인간존엄이 의도주의나 목적적 해석을 경유하여, 사실상 법명령규범 그 자체보다 더 강력한 효력을 가질 수도 있다는 점이다. 이는 권리들을 제한하는 근거로 인간존엄이 이용될 때 더욱 분명해진다. 프랑스의 꽁세유데따(Conseil d\textquotesingle État)는 유명한 난쟁이 던지기 사건에 대한 판결을 통해 난쟁이던지기 오락을 금지함으로써 이러한 논란을 불러일으킨 바 있다. 인간존엄의 해석을 토대로 한 사법권력의 월권적 법형성을 방지하기 위해서는 법해석의 기초로서의 인간존엄의 개념에 관하여 보다 설득력있고 구체적인 기준을 제시할 필요가 있다.

각 법질서가 보호하고자 하는 인간존엄의 내용이 명확하고, 또한 이를 오로지 주관적 권리규범으로 보호하는 경우에는, 주관적 권리당사자의 이익과 인간존엄에 기인한 권리행사는 이해상반의 문제를 거의 발생시키지 않기에, 인간존엄을 그로부터 도출된 권리에 의해 보호한다는 사유는 인간존엄이 당사자에게 주관적인 어떤 가치를 증진시키는 것과 관련되는 한, 매우 자연스럽고 일관적이며 매력적으로 보이는 법적인 기술이라고 할 수 있다. 다만, 권리를 인정하는 문제는 여타의 법규범을 해석하는 문제와는 다른 차원의 것이어서, 권리의 주체와 상대방, 그리고 그 권리내용을 구체적으로 확정할 수 있어야 한다.

인간존엄의 권리, 즉 ``인간존엄을 침해당하거나 보호받지 못하는 경우 침해의 배제나 적극적인 보호를 요구할 수 있는 주관적 권리''의 구체적 내용에 대해서는 다양한 논의가 있다. 먼저 인간존엄의 권리는 단지 ``상대방에게 자신을 존중할 것을 명령하는 권리'', 즉 형식적 권리라고 이해하는 입장이 있다. 또한 인간존엄을 보호하는 권리는 고문 금지, 굴욕 모욕 비하 금지, 수단화 금지 등 특정한 내용을 가진 권리, 즉 체계적이고 구조적인 형태의 권리들의 침해를 비난하는 권리로 이해하는 입장이 있다. 이러한 이해의 문제는 인간존엄으로부터 권리가 도출된다는, 인간존엄이 인권보다 보다 근본적이고 포괄적이라는 생각을 뒷받침하지 못한다. 또 하나의 유력한 견해는 `권리를 가질 권리'나 권리를 가질 지위 내지는 `시민권(citizenship)'이라는 주장이 있다. 이런 류의 사상에서, 인간의 존엄은 법질서에 편입될 권리가 된다. 그러나 이러한 권리 설명의 문제점은 인간의 존엄이 그 권리의 내용을 결정하거나 새로운 권리를 창출하라는 현대적 요청을 수행하지 못한다.

이와는 달리, 인간존엄이 모든 기본권들의 총체적 목적을 대변하는 포괄적 가치이므로, 인간존엄의 권리는 모든 포괄적 가치를 보호하는 포괄적 권리라는 견해가 있다. 이는 ``인권 시스템의 불완전성을 보완하는 최후의 수단''이다. 아론 바락은 인간존엄이 틀-권리(framework right) 혹은 어머니-권리(mother-right)의 역할을 한다고 주장한다. 인간존엄은 다른 모든 세분화되어 파생된 기본권들을 아우르는 ``포괄적 주기본권''이라는 것이다. 인간존엄의 권리만이 가지는 배타적인 고유한 영역으로 보였던 경우들, 예를 들어 매우 심한 체계적이고 구조적인 특정한 내용의 권리침해를 비난하는 권리로서의 인간존엄의 권리는 딸-권리(daughter-right)로서 ``그 배타적 영역은 인간존엄의 헌법적 권리에 포함되지만 다른 헌법적 권리의 범위에 속하지 않는 영역에만 적용된다.'' 인간존엄의 권리를 포괄적 권리로 이해할 경우 생기는 문제는 인간존엄의 가치의 범위를 넓히는 경우 인정해야 할 권리의 범위가 지나치게 넓어진다는 것이다. 따라서 인간존엄을 보호할 권리의 내용은 보다 분명하게 정의되고 한정되어야 한다.

헌법담론에서 인간존엄 논의의 또 하나의 축은 그 효력의 절대성 문제라고 할 수 있다.학자들은 인간존엄을 토대로 하고 있는 규범이 매우 강력한 효력을 가진다고 생각한다. 이는 주로 형량불가능성의 도그마로 이해되곤 하는데 예를 들어 우리 헌법재판소 판결들을 살펴보면, 인간존엄의 제한이 인정된다 하더라도 비례원칙을 통해 인간존엄을 다른 가치나 권리들과 비교형량한 후에서야 인간존엄의 침해를 인정하는 설시를 찾아볼 수 없다. 애초에 사안이 인간존엄을 제한한다는 사실을 부인하거나, 그 제한을 인정하는 경우에는 반드시 해당 사안을 헌법을 위반하는 인간존엄의 침해로 선언한다. 이러한 태도는 인간존엄을 실제 사안에 적용하기에 매우 부담스러운 개념으로 만들고, 이에 따라 학자들은 그 개념범위를 축소하려는 움직임을 보이고, 사법기관은 인간존엄의 위반을 매우 소극적으로만 인정하는 경향이 있다.

본 논문은 제3부에서 인간존엄의 법적 개념 앞에 놓여진 여러 난점을 해결하는 이해방식을 구성하기 위해 노력하였다. 이러한 구성은 존엄이 ``인간이 어떤 가치속성으로서 보유하는 내적 초월적 핵심''이라고 주장한 현대적 이해방식의 한계를 극복하는 것이어야 하면서도, 도덕철학에서 그 정당성을 인정받을 수 있는 개념이고, 또한 ``고귀한 계급과 높은 공무에 대한 그것{[}존엄{]}의 고대적 관련성과의 신의를 어떻게든 유지하기 위해 노력''하는 그 역사적 어원을 존중하는 것이 되어야 했다. 또한, 이 구성은 현대적 요청이 제기하는 1) 맥락-관련적 인권 또는 기본권을 도출하고, 2) 의미의존적 이익을 보호하여 인간존엄에 기대하는 국제적 인권담론의 요청을 충족하면서도, 3) 형량불가능성 도그마에 대한 적절한 설명과 그 기준을 제공함으로써 기본권 체계에 있어 인간존엄이 사법에 의한 월권적 법형성을 획책하는 트로이 목마가 아니라는 점을 보여줄 수 있는 것이 되어야 했다.

본고는 우리가 법담론에서 혼동하면서 쓰고 있는 두 개의 `인간존엄' 사용의 국면, 즉 1) 법적 사실로서의 인간적으로 좋은 삶을 사는 것의 객관적 중요성과, 2) 국가가 적극적으로 보호해야할 제3의 법익인 인간적으로 좋은 삶을 사는 것의 객관적 중요성에 대한 중심부 믿음을 형성하고 유지할 이익인 독립적 인간존엄-이익을 구별함으로써 이러한 과제들을 해결하도자 시도했다.

먼저 우리가 침해불가능한 절대적 진리로 여기는 ``인간은 존엄하다''는 명제는 사실 도덕성의 높음을 나타내는 ``인간적으로 좋은 삶을 영위하는 것은 객관적으로 중요하다''(P\emph{dig})라는 명제라는 점을 통해, 고대적 \emph{dignitas}와의 관련성과 우리가 사용하는 어의의 함축을 살리면서도, 현대인권적 맥락이 제기하는 권리 정초의 역할을 수용할 수 있는 개념을 제공하였다.

그러나 인간존엄이 단지 모든 이들에 대한 평등한 존중을 정당화시키는 근거일 뿐이라는 주장은, `인간존엄'의 현대적 사용을 모두 포괄하지 못하고 있다. 홀로코스트에 대한 반성, 노예금지, 키메라연구의 규제 등 특수한 사안에서 인간존엄은 단지 법질서 안에서의 평등한 존중만을 요청하는 것이 아니라 독립적이고 우선적인 특유하고 새로운 이익의 보호를 요청하는 것처럼 보이기 때문이다.

이러한 새로운 이익은 인간적으로 좋은 삶을 사는 것의 객관적 중요성에 대한 정당화된 믿음형성의 이익이라고 할 수 있다. 사람들은 복지이익의 향유나 자율적 선택만으로는 좋은 삶을 완전히 누리지 못한다. 사람들은 자신이 추구하며 살고 있는 삶이 객관적으로 중요한 것이라는 믿음을 형성하고 유지하며 살아갈 때에만 완전한 좋은 삶을 누릴 수 있다. 그런데 이러한 믿음의 형성과 유지는 노예제를 인정한다거나 국가권력에 의해 자행된 홀로코스트와 같은 충격적 사건을 통해 위태로워질 수 있다. 또한 키메라 연구와 같은 인간종의 정체성을 불분명하게 만드는 생명공학기술의 사용도 이러한 믿음의 형성과 유지를 약화시킬 수 있다.

다만 좋은 삶과 관계된 믿음형성의 이익은 무분별하게 보호받아야 하는 것은 아니다. 먼저 인간존엄-이익으로서 특별한 보호를 요청하는 이익은 잘못된 믿음이 아닌 정당화된 믿음이어야 한다. 또한 인간적으로 좋은 삶에 대한 믿음은 개별적 믿음의 모음이 아니라 믿음을 구성하는 문장들이 형성하는 장으로서의 전체이며, 또한 그 중 중심부 믿음은 쉽게 포기되는 성격의 것이 아니다. 따라서, 인간존엄-이익의 침해라고 할 수 있기 위해서는 중심부 믿음과 관계된 것이어야 하고 그 정도는 중심부 믿음이 포기될 수 있을 정도로 위협적인 것이어야 한다.

본고는 위와 같이 인간존엄을 한편으로 모든 이들의 인간적으로 좋은 삶에 대한 평등한 존중의 의무를 정당화시키는 관념으로, 다른 한편으로 인간적으로 좋은 삶을 사는 것의 객관적 중요성에 대한 정당화된 믿음형성의 이익으로 이원적으로 구성하는 전략을 통해 앞서 제기된 인간존엄의 유용성에 대한 의심을 불식시킨다. 인간존엄은 더 이상 개별 인간에게 내재된 애매하고 모호하며 모순투성이인 가치속성들이 아니라, 타인과 자신의 가치와 권리를 존중해야 할 의무를 굳건히 정초하는 개념이면서, 동시에 현대 사회에서 특별한 우선적 보호를 요청해 왔던 좋은 삶에 대한 중심부 믿음형성의 이익을 보호하는 구체적인 효용을 제공하는 개념이기 때문이다.

본 논문이 제공한 이원적 해결방식과 각각의 두 이해방식의 함축들, 그리고 이를 뒷받침하기 위해 제공했던 논변들을 통하여, 인간존엄이 그간의 회의주의의 의심을 물리치고 인권의 문제들과 급격히 발전하는 생명공학의 시대에 발생하는 생명윤리의 문제들에서 굳건한 토대를 제공하는 유용한 개념이 되기를 기대한다.